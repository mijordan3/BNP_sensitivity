\documentclass[a0,plainsections,30pt]{sciposter}\usepackage[]{graphicx}\usepackage[]{color}
%% maxwidth is the original width if it is less than linewidth
%% otherwise use linewidth (to make sure the graphics do not exceed the margin)
\makeatletter
\def\maxwidth{ %
  \ifdim\Gin@nat@width>\linewidth
    \linewidth
  \else
    \Gin@nat@width
  \fi
}
\makeatother

\definecolor{fgcolor}{rgb}{0.345, 0.345, 0.345}
\newcommand{\hlnum}[1]{\textcolor[rgb]{0.686,0.059,0.569}{#1}}%
\newcommand{\hlstr}[1]{\textcolor[rgb]{0.192,0.494,0.8}{#1}}%
\newcommand{\hlcom}[1]{\textcolor[rgb]{0.678,0.584,0.686}{\textit{#1}}}%
\newcommand{\hlopt}[1]{\textcolor[rgb]{0,0,0}{#1}}%
\newcommand{\hlstd}[1]{\textcolor[rgb]{0.345,0.345,0.345}{#1}}%
\newcommand{\hlkwa}[1]{\textcolor[rgb]{0.161,0.373,0.58}{\textbf{#1}}}%
\newcommand{\hlkwb}[1]{\textcolor[rgb]{0.69,0.353,0.396}{#1}}%
\newcommand{\hlkwc}[1]{\textcolor[rgb]{0.333,0.667,0.333}{#1}}%
\newcommand{\hlkwd}[1]{\textcolor[rgb]{0.737,0.353,0.396}{\textbf{#1}}}%
\let\hlipl\hlkwb

\usepackage{framed}
\makeatletter
\newenvironment{kframe}{%
 \def\at@end@of@kframe{}%
 \ifinner\ifhmode%
  \def\at@end@of@kframe{\end{minipage}}%
  \begin{minipage}{\columnwidth}%
 \fi\fi%
 \def\FrameCommand##1{\hskip\@totalleftmargin \hskip-\fboxsep
 \colorbox{shadecolor}{##1}\hskip-\fboxsep
     % There is no \\@totalrightmargin, so:
     \hskip-\linewidth \hskip-\@totalleftmargin \hskip\columnwidth}%
 \MakeFramed {\advance\hsize-\width
   \@totalleftmargin\z@ \linewidth\hsize
   \@setminipage}}%
 {\par\unskip\endMakeFramed%
 \at@end@of@kframe}
\makeatother

\definecolor{shadecolor}{rgb}{.97, .97, .97}
\definecolor{messagecolor}{rgb}{0, 0, 0}
\definecolor{warningcolor}{rgb}{1, 0, 1}
\definecolor{errorcolor}{rgb}{1, 0, 0}
\newenvironment{knitrout}{}{} % an empty environment to be redefined in TeX

\usepackage{alltt}

\usepackage{graphicx}
% \usepackage{subcaption}
%\usepackage{caption}

\usepackage{epstopdf, graphicx}
\usepackage{booktabs,dcolumn}
\usepackage{amsmath}
\usepackage{amsthm}
\usepackage{amssymb}
\usepackage{bbm}
\usepackage{hhline}
\usepackage{multicol}
\setlength{\columnseprule}{0pt}

\usepackage{boxedminipage}

\newcommand{\widgraph}[2]{\includegraphics[keepaspectratio,width=#1]{#2}}

\newcommand{\Rbb}{\mathbb{R}}
\newcommand{\Expect}{\mathbb{E}}
\newcommand{\Expecthat}{\hat{\mathbb{E}}}
\newcommand{\Var}{\text{Var}}
\newcommand{\Cov}{\text{Cov}}
\newcommand{\vbfamily}{\mathcal{Q}}
\newcommand{\etaopt}{\eta^{*}}
\newcommand{\etazopt}{\eta_z^{*}}
\newcommand{\etathetaopt}{\eta_\theta^{*}}
%\newcommand{\qopt}{q^{*}}
\newcommand{\targethat}{\hat{g}}
\newcommand{\QExpect}
{\Expect_{q\left(\theta, z \vert \eta_\theta, \etazopt(\eta_\theta)\right)}}
\newcommand{\atzero}{\Big\rvert_{\eta_\theta = \etathetaopt, \epsilon = 0}}
\newcommand{\etathetalin}{\eta_\theta^{LIN}}

\usepackage[framemethod=TikZ, xcolor=RGB]{mdframed}

\definecolor{mydarkblue}{rgb}{0,.06,.5}
\definecolor{mydarkred}{rgb}{.5,0,.1}
\definecolor{myroyalblue}{rgb}{0,.1,.8}

\usepackage{sectsty}
\sectionfont{\color{mydarkblue}\centering\LARGE\bf}

\mdfdefinestyle{MyFrame}{%
    linecolor=mydarkblue,
    outerlinewidth=2pt,
    roundcorner=20pt,
    innertopmargin=10pt,
    innerbottommargin=10pt,
    innerrightmargin=10pt,
    innerleftmargin=10pt,
    backgroundcolor=blue!10}


\title{\textcolor{mydarkblue}{Evaluating Sensitivity to the Stick Breaking Prior in Bayesian Nonparametrics
}}

\author{Ryan Giordano\textsuperscript{1*} \quad
Runjing Liu\textsuperscript{1*} \quad
Michael I. Jordan\textsuperscript{1} \quad
Tamara Broderick\textsuperscript{2} \\
{\large\normalfont\textsuperscript{*} These authors contributed equally}\\
 {\large\normalfont\textsuperscript{1} Department of Statistics, UC Berkeley \quad \textsuperscript{2} Department of EECS, MIT}
 }

\leftlogo[1]{static_images/logo_left.png}
\rightlogo[1]{static_images/logo_right2.png}

% Set the color used for the section headings here
\definecolor{SectionCol}{rgb}{0,.06,.5}
\definecolor{lightblue}{rgb}{0.8,0.8,1}

% Set some fbox commands line width and the color we use in the f boxes
\setlength{\fboxrule}{.09cm}
\definecolor{boxcolor}{rgb}{1,1,1}
\definecolor{innerboxcolor}{rgb}{.9,.94,.98}

% Math macros
\newcommand{\eq}[1]{Eq.~(\ref{eq:#1})}

\newcommand{\kl}{\textrm{KL}}
\newcommand{\mbe}{\mathbb{E}}
\newcommand{\mbeq}{\mathbb{E}_{q}}
\newcommand{\var}{\textrm{Var}}
\newcommand{\cov}{\textrm{Cov}}
\newcommand{\iid}{\stackrel{iid}{\sim}}
\newcommand{\indep}{\stackrel{indep}{\sim}}
\newcommand{\gauss}{\mathcal{N}} % Gaussian distribution

\DeclareMathOperator*{\argmin}{arg\,min}

% my added commands
\usepackage{etoolbox}
\BeforeBeginEnvironment{figure}{\vskip-2ex}
\AfterEndEnvironment{figure}{\vskip-1ex}

% Fiddle with the margin
\addtolength{\topmargin}{-0.5in}
% \addtolength{\topmargin}{-0.875in}
\addtolength{\textheight}{1in}
\IfFileExists{upquote.sty}{\usepackage{upquote}}{}
\begin{document}
\conference{BNP 2019}

\setlength{\parskip}{0.25em}

\maketitle

\vspace{-1in}





%%%%%%%%%%%%%%%%%%%%%%%%%%%%%%%%%
%%%%%%%%%%%%%%%%%%%%%%%%%%%%%%%%%
% FIRST COLUMN
%%%%%%%%%%%%%%%%%%%%%%%%%%%%%%%%%
%%%%%%%%%%%%%%%%%%%%%%%%%%%%%%%%%
%\columnbreak

\begin{minipage}[t]{0.45\textwidth}

\begin{mdframed}[style=MyFrame]
%%%%%%%%%%%%%%%%%%%%%%%%%%%%%%%%%
%%%%%%%%%%%%%%%%%%%
\section*{Overview}
\vspace{-0.3in}
%%%%%%%%%%%%%%%%%%%
%%%%%%%%%%%%%%%%%%%%%%%%%%%%%%%%%
\begin{itemize}
\item Researchers often want to estimate the number of clusters present in a dataset.

\item A Bayesian nonparametric (BNP) model makes inferring the number of clusters amenable to Bayesian inference.

\item We approximate the exact posterior with variational Bayes.

\item \textbf{Question}: how sensitive is the
VB approximation, and the resulting inferences, to BNP model choices?

\item \textbf{Problem}: re-running VB for multiple model choices is expensive.

\item \textbf{We propose}: a linear approximation to efficiently
estimate BNP sensitivity from a single run of VB (to avoid
expensive refitting).

\item We evaluate sensitivity to parametric and functional perturbations.

\end{itemize}
\end{mdframed}
\vspace{-0.7in}

%%%%%%%%%%%%%%%%%%%%%%%%%%%%%%%%%
%%%%%%%%%%%%%%%%%%%
\section*{Model and inference }
\vspace{-0.3in}
%%%%%%%%%%%%%%%%%%%
%%%%%%%%%%%%%%%%%%%%%%%%%%%%%%%%%

The \textbf{Dirichlet process} is a popular Bayesian nonparametric
(BNP) model used for clustering.

\begin{figure}[!h]
\centering
\includegraphics[width = 0.95\textwidth]{./static_images/DP_stick_breaking.png}
\caption{The Dirichlet stick-breaking process. We start with a stick of
length 1, and recursively break off proportions $\nu_1$, $\nu_2$, $\nu_3$, ...
The weights on the components are the lengths broken off at each step.}
\setlength{\textfloatsep}{-10pt}
\end{figure}
%
We approximate the true posterior using \textbf{variational Bayes} (VB). Let $\eta^*$ denote the optimal variational parameters.
%
\begin{mdframed}[style=MyFrame]
- How does varying the distribution $\nu_k \sim \text{Beta}(1, \alpha)$ affect $\eta^*$?

- While VB is fast, refitting for multiple choices of the DP prior would be
computationally prohibitive.

- {\bf Hence, we propose a linear approximation, derived from local sensitivity measures. }
\end{mdframed}

%%%%%%%%%%%%%%%%%%%%%%%%%%%%%%%%%
%%%%%%%%%%%%%%%%%%
\vspace{-0.6in}
\section*{Hyperparameter sensitivity}
\vspace{-0.3in}
%%%%%%%%%%%%%%%%%%
%%%%%%%%%%%%%%%%%%%%%%%%%%%%%%%%%

\begin{itemize}

\item Let $\epsilon$ be a real-valued hyperparameter for the stick-breaking distribution
(e.g., this could be the $\alpha$ concentration parameter).

\item The optimal VB parameters $\eta^*$ depend on $\epsilon$ through optimizing the KL objective.

\item The expected number of clusters depend on $\eta^*$. Can use either an in-sample or a predictive quantity,
\vspace{-0.1in}
\begin{align}
\Expect_{q_{\eta^*}} \left[ \#\{\text{distinct clusters}\} \right]
\quad \text{ or } \quad
\Expect_{q_{\eta^*}}
\left[\#\{\substack{\text{distinct clusters}\\\text{in new dataset}}\} \right].
\end{align}
\end{itemize}

%%%%%%%%%%%%%%%%%%%%%%%%%%%%%%%%%
%%%%%%%%%%%%%%%%%%%%%%%%%
\vspace{-0.9in}
\subsection*{Linear approximation}
\vspace{-0.2in}
%%%%%%%%%%%%%%%%%%%%%%%%%
%%%%%%%%%%%%%%%%%%%%%%%%%%%%%%%%%
\begin{itemize}
\item WOLOG, let $\epsilon=0$ represent the original prior at which we fit a VB approximation. Then for any other $\epsilon$ we approximate
\begin{align}
\eta^*(\epsilon)  &\approx  \eta^*(0) +
\frac{d \eta^*(\epsilon)}{d\epsilon^T}\Big|_{\epsilon=0} \epsilon
\label{eq:linear_approx}
\end{align}

\item Evaluation of the derivative is done efficiently using formulas from \cite{giordano:2017:covariances} and auto-differentiation tools \cite{maclaurin:2015:autograd}.

\item We only use a linear approximation for the dependence of $\eta^*$ on $\epsilon$. We retain the non-linearities in the map from $\eta^*$ to the expected number of clusters.

\end{itemize}

%%%%%%%%%%%%%%%%%%%%%%%%%%%%%%%%%
%%%%%%%%%%%%%%%%%%
\begin{center}
\noindent\rule{0.6\textwidth}{1pt}
\end{center}
{\bf References}
%%%%%%%%%%%%%%%%%%
%%%%%%%%%%%%%%%%%%%%%%%%%%%%%%%%%
\renewcommand{\section}[2]{}%
\footnotesize{
  \bibliographystyle{abbrv}
  \bibliography{./references}
}

\end{minipage}
\hfill \vrule \hfill
\begin{minipage}[t]{0.45\textwidth}

%%%%%%%%%%%%%%%%%%%%%%%%%%%%%%%%%
%%%%%%%%%%%%%%%%%%%%%%%%%%%%%%%%%
% SECOND COLUMN
%%%%%%%%%%%%%%%%%%%%%%%%%%%%%%%%%
%%%%%%%%%%%%%%%%%%%%%%%%%%%%%%%%%



%%%%%%%%%%%%%%%%%%%%%%%%%%%%%%%%%
%%%%%%%%%%%%%%%%%%
\section*{Results}
\vspace{-0.3in}

\begin{minipage}[t]{0.49\textwidth}
    %% iris figure
    \subsection*{Fisher Iris Dataset}
    \begin{figure}[!h]
    \centering
    \includegraphics[width = 0.95\textwidth]{./static_images/iris_data.png}
    \caption{The iris data projected onto the first two principal components. Color corresponds to the true iris species.}
    \setlength{\textfloatsep}{-10pt}
    \end{figure}
\end{minipage}
%
\begin{minipage}[t]{0.49\textwidth}
    \subsection*{Mouse Genes}
    \begin{figure}[!h]
    \centering
    \includegraphics[width = 0.95\textwidth]{./static_images/mouse_genes.png}
    \caption{Mouse gene expression time series.}
    \setlength{\textfloatsep}{-10pt}
    \end{figure}
\end{minipage}
%%%%%%%%%%%%%%%%%%
%%%%%%%%%%%%%%%%%%%%%%%%%%%%%%%%%
We infer the number of distinct species in the iris
dataset \cite{iris_data_anderson}, and evaluate sensitivity to the Dirichlet process prior.
\vspace{0.1in}

{\bf \large Sensitivity to $\alpha$.}
We evaluate the dependence of the expected number of clusters to the stick-breaking concentration parameter $\alpha$.

\vspace{0.05in}
%

\begin{knitrout}
\definecolor{shadecolor}{rgb}{0.969, 0.969, 0.969}\color{fgcolor}

{\centering \includegraphics[width=0.98\linewidth,height=0.294\linewidth]{figure/param_sens_plot-1} 

}



\end{knitrout}

\begin{knitrout}
\definecolor{shadecolor}{rgb}{0.969, 0.969, 0.969}\color{fgcolor}

{\centering \includegraphics[width=0.98\linewidth,height=0.294\linewidth]{figure/gene_param_sens_plot-1} 

}



\end{knitrout}

{\bf \large Sensitivity to functional perturbations. }
Suppose we replace the original beta prior $p_0$ with another distribution $p_1$. Then we set our perturbation to be
\vspace{-0.3in}
\begin{align*}
p_c(\nu_k \vert \delta) \propto p_{0}(\nu_k)\left(\frac{p_1(\nu_k)}{p_0(\nu_k)}\right)^\delta
\end{align*}
\vspace{-0.3in}

and evaluate sensitivity to $\delta$:
%
\vspace{0.1in}
% \begin{figure}
% \centering

\begin{knitrout}
\definecolor{shadecolor}{rgb}{0.969, 0.969, 0.969}\color{fgcolor}

{\centering \includegraphics[width=0.98\linewidth,height=0.588\linewidth]{figure/functional_sens_plot-1} 

}



\end{knitrout}

% \begin{mdframed}[style=MyFrame]
% %%%%%%%%%%%%%%%%%%%%%%%%%%%%%%%%%
% %%%%%%%%%%%%%%%%%%
% \vspace{-0.6in}
% \section*{Conclusion}
% \vspace{-0.3in}
% %%%%%%%%%%%%%%%%%%
% %%%%%%%%%%%%%%%%%%%%%%%%%%%%%%%%%
%
% \begin{itemize}
%
% \item {\bf Our linear approximation provides a fast alternative to re-evaluating the full model after changing the BNP prior. }
%
% \item We applied our approximation to both parametric and functional perturbations of the stick-breaking prior.
%
% \item We attempted to improve the linearity by approximating only the dependence of the optimal variational parameters on prior parameters.
%
% \end{itemize}
% \end{mdframed}
%
% {\bf Our code: }\newline
% Paragami: general library for sensitivity analysis in optimization problems\newline
% {\color{blue} https://github.com/rgiordan/paragami}
%
% Code to evaluate BNP sensitivity as shown here: \newline
% {\color{blue} https://github.com/Runjing-Liu120/sensitivity\_to\_stick\_breaking\_in\_BNP}


\small{
%%%%%%%%%%%%%%%%%%%%%%%%%%%%%%%
%%%%%%%%%%%%%%%%%
{\bf Acknowledgments}:
% %%%%%%%%%%%%%%%%%%
%%%%%%%%%%%%%%%%%%%%%%%%%%%%%%%%%
Ryan Giordano's research was funded in full by the
Gordon and Betty Moore Foundation through Grant GBMF3834 and by the Alfred P. Sloan Foundation through Grant 2013-10-27 to the University of California, Berkeley. Runjing Liu's research was funded by the NSF graduate research fellowship. Tamara Broderick's research is supported by an NSF CAREER Award and an ARO YIP
Award. This research is supported in part by the DARPA program on
Lifelong Learning Machines.
}


\end{minipage}\\

\begin{center}

{\bf Contact: } rgiordano@berkeley.edu, runjing\_liu@berkeley.edu

\end{center}

\end{document}
