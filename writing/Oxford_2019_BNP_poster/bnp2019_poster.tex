\documentclass[a0,plainsections,30pt]{sciposter}\usepackage[]{graphicx}\usepackage[]{color}
%% maxwidth is the original width if it is less than linewidth
%% otherwise use linewidth (to make sure the graphics do not exceed the margin)
\makeatletter
\def\maxwidth{ %
  \ifdim\Gin@nat@width>\linewidth
    \linewidth
  \else
    \Gin@nat@width
  \fi
}
\makeatother

\definecolor{fgcolor}{rgb}{0.345, 0.345, 0.345}
\newcommand{\hlnum}[1]{\textcolor[rgb]{0.686,0.059,0.569}{#1}}%
\newcommand{\hlstr}[1]{\textcolor[rgb]{0.192,0.494,0.8}{#1}}%
\newcommand{\hlcom}[1]{\textcolor[rgb]{0.678,0.584,0.686}{\textit{#1}}}%
\newcommand{\hlopt}[1]{\textcolor[rgb]{0,0,0}{#1}}%
\newcommand{\hlstd}[1]{\textcolor[rgb]{0.345,0.345,0.345}{#1}}%
\newcommand{\hlkwa}[1]{\textcolor[rgb]{0.161,0.373,0.58}{\textbf{#1}}}%
\newcommand{\hlkwb}[1]{\textcolor[rgb]{0.69,0.353,0.396}{#1}}%
\newcommand{\hlkwc}[1]{\textcolor[rgb]{0.333,0.667,0.333}{#1}}%
\newcommand{\hlkwd}[1]{\textcolor[rgb]{0.737,0.353,0.396}{\textbf{#1}}}%
\let\hlipl\hlkwb

\usepackage{framed}
\makeatletter
\newenvironment{kframe}{%
 \def\at@end@of@kframe{}%
 \ifinner\ifhmode%
  \def\at@end@of@kframe{\end{minipage}}%
  \begin{minipage}{\columnwidth}%
 \fi\fi%
 \def\FrameCommand##1{\hskip\@totalleftmargin \hskip-\fboxsep
 \colorbox{shadecolor}{##1}\hskip-\fboxsep
     % There is no \\@totalrightmargin, so:
     \hskip-\linewidth \hskip-\@totalleftmargin \hskip\columnwidth}%
 \MakeFramed {\advance\hsize-\width
   \@totalleftmargin\z@ \linewidth\hsize
   \@setminipage}}%
 {\par\unskip\endMakeFramed%
 \at@end@of@kframe}
\makeatother

\definecolor{shadecolor}{rgb}{.97, .97, .97}
\definecolor{messagecolor}{rgb}{0, 0, 0}
\definecolor{warningcolor}{rgb}{1, 0, 1}
\definecolor{errorcolor}{rgb}{1, 0, 0}
\newenvironment{knitrout}{}{} % an empty environment to be redefined in TeX

\usepackage{alltt}

\usepackage{graphicx}
%\usepackage{subfig}
%\usepackage{subfigure}
%\usepackage{caption}

%\usepackage{epstopdf, graphicx}
%\usepackage{booktabs,dcolumn}
\usepackage{amsmath}
\usepackage{amsthm}
\usepackage{amssymb}
\usepackage{bbm}
%\usepackage{hhline}
%\usepackage{multicol}
\usepackage[authoryear]{natbib}
%\usepackage{hyperref} % WTF, this causes a "caption outside of float" error

\setlength{\columnseprule}{0pt}
%\usepackage{boxedminipage}

\newcommand{\widgraph}[2]{\includegraphics[keepaspectratio,width=#1]{#2}}

\newcommand{\Rbb}{\mathbb{R}}
\newcommand{\Expect}{\mathbb{E}}
\newcommand{\Expecthat}{\hat{\mathbb{E}}}
\newcommand{\Var}{\text{Var}}
\newcommand{\Cov}{\text{Cov}}
\newcommand{\vbfamily}{\mathcal{Q}}
\newcommand{\etaopt}{\eta^{*}}
\newcommand{\etazopt}{\eta_z^{*}}
\newcommand{\etathetaopt}{\eta_\theta^{*}}
%\newcommand{\qopt}{q^{*}}
\newcommand{\targethat}{\hat{g}}
\newcommand{\QExpect}
{\Expect_{q\left(\theta, z \vert \eta_\theta, \etazopt(\eta_\theta)\right)}}
\newcommand{\atzero}{\Big\rvert_{\eta_\theta = \etathetaopt, \epsilon = 0}}
\newcommand{\etathetalin}{\eta_\theta^{LIN}}

\newcommand{\targetexpectation}{\Expect_{q_{\eta^*}}
\left[\#\{\substack{\text{distinct clusters}\\\text{in new dataset}}\} \right]}

\usepackage[framemethod=TikZ, xcolor=RGB]{mdframed}

\definecolor{mydarkblue}{rgb}{0,.06,.5}
\definecolor{mydarkred}{rgb}{.5,0,.1}
\definecolor{myroyalblue}{rgb}{0,.1,.8}

\usepackage{sectsty}
\sectionfont{\color{mydarkblue}\centering\LARGE\bf}

\mdfdefinestyle{MyFrame}{%
    linecolor=mydarkblue,
    outerlinewidth=2pt,
    roundcorner=20pt,
    innertopmargin=10pt,
    innerbottommargin=10pt,
    innerrightmargin=10pt,
    innerleftmargin=10pt,
    backgroundcolor=blue!10}


\title{\textcolor{mydarkblue}{
Evaluating Sensitivity to the Stick Breaking Prior in Bayesian Nonparametrics
}}

\author{Ryan Giordano\textsuperscript{2*} \quad
Runjing Liu\textsuperscript{1*} \quad
Michael I. Jordan\textsuperscript{1} \quad
Tamara Broderick\textsuperscript{2} \\
{\large\normalfont\textsuperscript{*}
These authors contributed equally}\quad
 {\large\normalfont\textsuperscript{1}
 Department of Statistics, UC Berkeley \quad \textsuperscript{2} CSAIL, MIT}
 }

\leftlogo[1]{static_images/logo_left.png}
\rightlogo[1]{static_images/logo_right2.png}

% Set the color used for the section headings here
\definecolor{SectionCol}{rgb}{0,.06,.5}
\definecolor{lightblue}{rgb}{0.8,0.8,1}

% Set some fbox commands line width and the color we use in the f boxes
\setlength{\fboxrule}{.09cm}
\definecolor{boxcolor}{rgb}{1,1,1}
\definecolor{innerboxcolor}{rgb}{.9,.94,.98}

% Math macros
\newcommand{\eq}[1]{Eq.~(\ref{eq:#1})}

\newcommand{\kl}{\textrm{KL}}
\newcommand{\mbe}{\mathbb{E}}
\newcommand{\mbeq}{\mathbb{E}_{q}}
\newcommand{\var}{\textrm{Var}}
\newcommand{\cov}{\textrm{Cov}}
\newcommand{\iid}{\stackrel{iid}{\sim}}
\newcommand{\indep}{\stackrel{indep}{\sim}}
\newcommand{\gauss}{\mathcal{N}} % Gaussian distribution

\DeclareMathOperator*{\argmin}{arg\,min}

% my added commands
\usepackage{etoolbox}
% \BeforeBeginEnvironment{figure}{\vskip-2ex}
% \AfterEndEnvironment{figure}{\vskip-1ex}

% Fiddle with the margin
\addtolength{\topmargin}{-0.5in}
% \addtolength{\topmargin}{-0.875in}
\addtolength{\textheight}{1in}
\IfFileExists{upquote.sty}{\usepackage{upquote}}{}
\begin{document}
\conference{BNP 2019}

\setlength{\parskip}{0.25em}

\maketitle

\vspace{-1in}





%%%%%%%%%%%%%%%%%%%%%%%%%%%%%%%%%
%%%%%%%%%%%%%%%%%%%%%%%%%%%%%%%%%
% FIRST COLUMN
%%%%%%%%%%%%%%%%%%%%%%%%%%%%%%%%%
%%%%%%%%%%%%%%%%%%%%%%%%%%%%%%%%%
%\columnbreak

\begin{minipage}[t]{0.45\textwidth}

\begin{mdframed}[style=MyFrame]
%%%%%%%%%%%%%%%%%%%%%%%%%%%%%%%%%
%%%%%%%%%%%%%%%%%%%
\section*{Overview}
\vspace{-0.3in}
%%%%%%%%%%%%%%%%%%%
%%%%%%%%%%%%%%%%%%%%%%%%%%%%%%%%%
\begin{itemize}
\item Researchers often want to estimate the number of distinct clusters
that would be present in a new dataset.

\item A variational Bayes (VB) approximation to a Bayesian nonparametric (BNP)
model makes inferring the number of clusters amenable to Bayesian inference.
%We approximate the exact posterior with variational Bayes (VB).

\item \textbf{Question}: How sensitive are the
VB approximation and the resulting inferences to BNP model choices?

\item \textbf{Problem}: Re-running VB for multiple model choices is expensive.

\item \textbf{We propose}: A local approximation to efficiently
estimate BNP sensitivity from a single run of VB, avoiding
expensive refitting.

%\item We evaluate sensitivity to parametric and functional perturbations.
\end{itemize}
\end{mdframed}
\vspace{-0.7in}

%%%%%%%%%%%%%%%%%%%%%%%%%%%%%%%%%
%%%%%%%%%%%%%%%%%%%%%%%%%%%%%%%%%
\section*{Model and inference }
\vspace{-0.3in}
%%%%%%%%%%%%%%%%%%%%%%%%%%%%%%%%%
%%%%%%%%%%%%%%%%%%%%%%%%%%%%%%%%%

% In the setting of \textbf{unsupervised clustering}, suppose we wish to know how
% many distinct clusters we'd find in a new dataset with the same distribution as
% observed data.  We will answer with BNP.

The \textbf{Dirichlet process} is a popular Bayesian nonparametric
(BNP) model used for clustering.  A stick-breaking representation uses
a distribution on sticks $\{\nu_k\}$ to induce a prior on component
probabilities $\{\pi_k\}$.

\begin{figure}
\caption{The Dirichlet stick-breaking process with $\nu_k \sim \mathrm{Beta}(1, \alpha)$.}
\centering
\includegraphics[width = 0.95\textwidth]{./static_images/DP_stick_breaking.png}
\end{figure}
%
%We approximate the true posterior using \textbf{variational Bayes} (VB).
%
\begin{itemize}
\item $\epsilon :=$ Any real-valued hyperparameter for the stick-breaking prior
    $p(\nu_k | \epsilon)$.
\item $\etaopt :=$ The optimal variational parameters.
\item $\targetexpectation :=$ The quantity we want to infer.
\end{itemize}

\begin{mdframed}[style=MyFrame]
VB inference defines a map:
%
\begin{gather*}
\textrm{Stick-breaking prior}
    \mapsto \textrm{VB approximation}
    \mapsto \textrm{Posterior expected \# clusters},\\
\textrm{i.e.,}\\
\epsilon
    \mapsto \etaopt
    \mapsto \targetexpectation.
\end{gather*}
\end{mdframed}

%%%%%%%%%%%%%%%%%%%%%%%%%%%%%%%%%
%%%%%%%%%%%%%%%%%%%%%%%%%

\textbf{Parametric perturbation.} To stay within the class of Dirichlet
processes, we might take $\epsilon := \alpha - \alpha_0$ and $p(\nu_k | \alpha) :=
\mathrm{Beta}(\nu_k | 1, \alpha)$.

\textbf{Functional perturbation.} Suppose we are interested in the effect of
replacing the original beta prior $p_0(\nu_k) := \mathrm{Beta}(\nu_k | 1,
\alpha)$ with another distribution $p_1(\nu_k)$. Then we take $\epsilon :=
\delta$ in
%
\vspace{-0.3in}
\begin{align*}
p(\nu_k \vert \delta) \propto
    p_{0}(\nu_k)\left(\frac{p_1(\nu_k)}{p_0(\nu_k)}\right)^\delta.
\end{align*}
\vspace{-0.3in}

By using a multiplicative perturbation, the map $p_1 \mapsto \etaopt$ is
Fr\'{e}chet differentiable in the norm $\left\Vert p_1 / p_0
\right\Vert_\infty$.

%%%%%%%%%%%%%%%%%%%%%%%%%%%%%%%%%
%%%%%%%%%%%%%%%%%%%%%%%%%
\vspace{-0.3in}
\subsection*{Linear approximation}
\vspace{-0.2in}
%%%%%%%%%%%%%%%%%%%%%%%%%
%%%%%%%%%%%%%%%%%%%%%%%%%%%%%%%%%
Let $\epsilon=0$ represent the original prior at which we fit a VB
approximation.

\begin{mdframed}[style=MyFrame]
For any other $\epsilon \ne 0$ we approximate $\epsilon \mapsto \etaopt$ with
\begin{align*}
\eta^*(\epsilon)  &\approx  \eta^*(0) +
\frac{d \eta^*(\epsilon)}{d\epsilon^T}\Big|_{\epsilon=0} \epsilon.
\label{eq:linear_approx}
\end{align*}
We retain the non-linearities in $\etaopt \mapsto \targetexpectation$.
\end{mdframed}

\begin{itemize}
\item We can \textbf{automatically compute}
    the approximation
    using automatic differentiation \cite{giordano:2017:covariances}.
    %\cite{giordano:2017:covariances, maclaurin:2015:autograd}.
\item The approximation requires the
    \textbf{inverse Hessian of the VB objective}.
\item This is an application of \textbf{local Bayesian robustness}
\citep{gustafson:1996:localposterior}.
\end{itemize}

%%%%%%%%%%%%%%%%%%%%%%%%%%%%%%%%%
%%%%%%%%%%%%%%%%%%
% \begin{center}
% \noindent\rule{0.95\textwidth}{1pt}
% \end{center}
% {\bf References}
%%%%%%%%%%%%%%%%%%
%%%%%%%%%%%%%%%%%%%%%%%%%%%%%%%%%
% \renewcommand{\section}[2]{}%
% \footnotesize{
%   \bibliographystyle{abbrv}
%   \bibliography{./references}
% }

\end{minipage}
\hfill \vrule \hfill
\begin{minipage}[t]{0.45\textwidth}

%%%%%%%%%%%%%%%%%%%%%%%%%%%%%%%%%
%%%%%%%%%%%%%%%%%%%%%%%%%%%%%%%%%
% SECOND COLUMN
%%%%%%%%%%%%%%%%%%%%%%%%%%%%%%%%%
%%%%%%%%%%%%%%%%%%%%%%%%%%%%%%%%%



%%%%%%%%%%%%%%%%%%%%%%%%%%%%%%%%%
%%%%%%%%%%%%%%%%%%
%\vspace{-0.3in}
\section*{Data}
\vspace{-0.3in}
%
\begin{minipage}[t]{0.49\textwidth}
%
\subsection*{Iris data}
For our first dataset, we discard species
information and cluster the 150 four-dimensional observations of the Fisher iris
dataset \citep{iris_data_anderson}.
The loadings on the first two principle components is shown
to the right.
%
\end{minipage}
%
\begin{minipage}[t]{0.49\textwidth}


\begin{knitrout}
\definecolor{shadecolor}{rgb}{0.969, 0.969, 0.969}\color{fgcolor}

{\centering \includegraphics[width=0.98\linewidth,height=0.588\linewidth]{figure/iris_pca-1} 

}



\end{knitrout}
\end{minipage}
%

\vspace{1em}

\begin{minipage}[t]{0.49\textwidth}
%
\subsection*{Mouse data} For our second dataset, we use a publicly available
dataset of gene expression in mice \citep{shoemaker:2015:ultrasensitive} . We
smooth and center the gene expression time series and cluster the resulting
patterns.  A typical smoothed time series with uncertainty is shown to the
right.
%
\end{minipage}
%
\begin{minipage}[t]{0.49\textwidth}
\begin{figure}[!h]
\centering
\includegraphics[width = 0.95\textwidth]{./static_images/mouse_genes.png}
\end{figure}
\end{minipage}
%

%%%%%%%%%%%%%%%%%%%%%%%%%%%%%%%%%
%%%%%%%%%%%%%%%%%%
\vspace{-0.6in}
\section*{Results}
\vspace{-0.3in}

\textbf{Parametric perturbation.}
Vertical lines show the location of $\alpha_0$.

\vspace{0.05in}
%

\begin{knitrout}
\definecolor{shadecolor}{rgb}{0.969, 0.969, 0.969}\color{fgcolor}

{\centering \includegraphics[width=0.98\linewidth,height=0.294\linewidth]{figure/param_sens_plot-1} 

}



\end{knitrout}

\begin{knitrout}
\definecolor{shadecolor}{rgb}{0.969, 0.969, 0.969}\color{fgcolor}

{\centering \includegraphics[width=0.98\linewidth,height=0.294\linewidth]{figure/gene_param_sens_plot-1} 

}



\end{knitrout}

\textbf{Functional perturbation.} The first row shows the original prior
$p_0(\nu_k)$ and the functional perturbation $p_1(\nu_k)$.

\vspace{0.1in}

\begin{knitrout}
\definecolor{shadecolor}{rgb}{0.969, 0.969, 0.969}\color{fgcolor}

{\centering \includegraphics[width=0.98\linewidth,height=0.588\linewidth]{figure/functional_sens_plot-1} 

}



\end{knitrout}

The approximation consistently \textbf{performs better when extraploating to
fewer clusters}.

\noindent\rule{0.95\textwidth}{1pt}

{\bf Contact and code: } rgiordano@berkeley.edu, runjing\_liu@berkeley.edu
%{\bf Code: } \newline
% \url{https://github.com/Runjing-Liu120/sensitivity_to_stick_breaking_in_BNP}{a}
% \url{https://github.com/rgiordan/vittles}{b}
{\color{blue} https://github.com/Runjing-Liu120/sensitivity\_to\_stick\_breaking\_in\_BNP}
{\color{blue} https://github.com/rgiordan/vittles}

% \begin{center}
% \noindent\rule{0.95\textwidth}{1pt}
% \end{center}
% 
\small{
%%%%%%%%%%%%%%%%%%%%%%%%%%%%%%%
%%%%%%%%%%%%%%%%%
{\bf Acknowledgments}:
% %%%%%%%%%%%%%%%%%%
%%%%%%%%%%%%%%%%%%%%%%%%%%%%%%%%%
Ryan Giordano's research was funded in full by the
Gordon and Betty Moore Foundation through Grant GBMF3834 and by the Alfred P. Sloan Foundation through Grant 2013-10-27 to the University of California, Berkeley. Runjing Liu's research was funded by the NSF graduate research fellowship. Tamara Broderick's research is supported by an NSF CAREER Award and an ARO YIP
Award. This research is supported in part by the DARPA program on
Lifelong Learning Machines.
}


\end{minipage}\\

\begin{center}
\noindent\rule{0.95\textwidth}{1pt}
\end{center}

\small{
%%%%%%%%%%%%%%%%%%%%%%%%%%%%%%%
%%%%%%%%%%%%%%%%%
{\bf Acknowledgments}:
% %%%%%%%%%%%%%%%%%%
%%%%%%%%%%%%%%%%%%%%%%%%%%%%%%%%%
Ryan Giordano's research was funded in full by the
Gordon and Betty Moore Foundation through Grant GBMF3834 and by the Alfred P. Sloan Foundation through Grant 2013-10-27 to the University of California, Berkeley. Runjing Liu's research was funded by the NSF graduate research fellowship. Tamara Broderick's research is supported by an NSF CAREER Award and an ARO YIP
Award. This research is supported in part by the DARPA program on
Lifelong Learning Machines.
}


\renewcommand{\section}[2]{}%
\footnotesize{
  \bibliographystyle{abbrv}
  \bibliography{./references}
}

\end{document}
