Two central goals of many clustering problems are inferring how many distinct
clusters are present in a particular dataset and which observations cluster
together. A Bayesian nonparametric (BNP) approach to clustering assumes an
infinite number of \textit{components}, of which a finite number are present in
the data as \textit{clusters}. Like all Bayesian approaches, BNP requires the
specification of a prior, and this prior may favor a greater or fewer number of
distinct clusters. In practice, it is important to establish that the prior is
not too informative, particularly when---as is often the case in BNP---the
particular form of the prior is chosen for mathematical convenience rather than
because of a considered subjective belief.

The posterior in a BNP model cannot be calculated analytically, and thus
approximate methods are required in practice. In the present work, we employ a
truncated variational Bayes (VB) approximation, which posits a constrained
family of distributions, and uses optimization to find the member of the family
that is closest to the true posterior in Kullback-Leibler ($\mathrm{KL}$)
divergence \citep{blei:2017:vi_review, blei:2006:vi_for_dp}. Concretely, the
output of a VB approximation is a vector of optimal approximating parameters,
$\etaopt$ which, n general, depend on the prior.  For example, when using the
Dirichlet process (DP) prior with concentration parameter $\alpha$, we can
denote this dependence by $\etaopt(\alpha)$.

In principle, one can always assess the robustness of of a VB approximation by
refitting the VB posterior for several different plausible prior choices, though
this can be computationally prohibitive. We circumvent this computational
difficulty using a first-order Taylor series approximation to the optimum, as
given by the implicit function theorem applied to the first-order condition of
the optimization problem. For example, for the DP prior, we choose some
``initial value'' $\alpha_0$ of the concentration parameter, solve the
optimization problem to compute $\etaopt(\alpha_0)$, and use $\etaopt(\alpha_0)$
to approximate
%
\begin{align}\eqlabel{taylor}
%
\etalin(\alpha) :=
    \etaopt(\alpha_0) +
    \fracat{d\etaopt(\alpha)}{d\alpha}{\alpha_0} (\alpha - \alpha_0).
%
\end{align}
%
If the map $\alpha \mapsto \etaopt(\alpha)$ is continuously differentiable, then
we expect $\etaopt(\alpha) \approx \etalin(\alpha)$ when $\abs{\alpha -
\alpha_0}$ is small.  And if the derivative $d\etaopt(\alpha) / d\alpha$ can be
computed much more efficiently than the cost of optimizing directly,
$\etalin(\alpha)$ can be much faster to compute than $\etaopt(\alpha)$.

We study Taylor series approximations like \eqref{taylor}, both theoretically
and practically, for both parametric perturbations (such as the $\alpha$
parameter of a DP prior) and nonparametric perturbations (such as changing
the density of a stick-breaking representation of a random measure).
We state sufficient conditions for differentiability of VB approximations
in general, from which our BNP applications follow as a special case.
Among a class of nonparametric perturbations considered by
\citet{gustafson:1996:local}, we prove that VB optima are Fr{\'e}chet
differentiable only for multiplicative perturbations to the prior density.

We validate our theoretical results on a series of BNP models applied to real
datasets, both for parametric and nonparametric perturbations to the BNP prior.
We show that the approximation can be easily and efficiently computed even in
high-dimensional problems like BNP models, using modern automatic
differentiation tools \citep{baydin:2018:automatic, jax2018github}. We
demonstrate the usefulness of the approximation in practice on a series of
increasingly complex real-world datasets, validating our approximation against
the much more expensive process of refitting the variational posterior.   In
most cases, the local approximation provides qualitatively accurate results many
times faster re-optimizing.
