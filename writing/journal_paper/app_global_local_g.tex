
Given a posterior quantity $\g$,
we again take advantage of the fact that the optimal
local parameters can be found in closed form given global parameters.
In general, $\g$ will be a function of the entire vector of variational parameters.
However, in the same way that $\KLglobal$ implicitly sets the local parameters at their optimum
and is a function of only global parameters and the prior parameter $\t$,
we can construct an analogous mapping for $\g$,
\begin{align}\eqlabel{g_as_global}
(\t, \etaglob) \mapsto g\Big(\big(\etaglob, \etaoptz(\etaglob, \t))\Big).
\end{align}

We illustrate this mapping
when our quantity of interest is the in-sample expected posterior number of clusters.

\begin{ex}\exlabel{vb_insample_nclusters_globallocal}
%
Let
$\gclustersabbr(\eta)$ denote our variational approximation to
$\expect{\p(\z\vert\x)}{\nclusters(\z)}$.   Using the fact that
$\q(\z_\n \vert \etaopt_{\z_\n}) = \p(\z_\n\vert \beta, \nu, \x)$ is available in closed form, we can then take
%
\begin{align*}
%
\gclustersabbr(\etaopt) :={}&
    \expect{\q(\beta, \nu \vert\etaopt)}{
        \expect{\p(\z \vert \beta, \nu, \x)}{\nclusters(\z)}
    }
\\\approx{}&
    \expect{\p(\beta, \nu \vert \x)}{
        \expect{\p(\z \vert \beta, \nu, \x)}{\nclusters(\z)}
    }
    = \expect{\p(\z\vert\x)}{\nclusters(\z)} \Rightarrow \\
%
\gclustersabbr(\eta) ={}&
    \sumkm \left(1 -  \prod_{\n=1}^\N
        \left(1 - \expect{\q(\beta, \nu \vert \eta_\beta, \eta_\nu)}
                    {\expect{\p(\z_{\n} \vert \beta, \nu, \x)}{\z_{\n\k}}}
                    \right)\right).
%
\end{align*}
%
In this way, $\gclustersabbr(\eta)$ depends only on $\eta_\beta$ and $\eta_\nu$,
which are much lower-dimensional than $\eta_\z$, and retains nonlinearities in
the map
%
\begin{align*}
%
\eta_\beta, \eta_\nu \mapsto \expect{\q(\beta, \nu \vert \eta_\beta,
\eta_\nu)} {\expect{\p(\z_{\n} \vert \beta, \nu, \x)}{\z_{\n\k}}}.
%
\end{align*}
%
\end{ex}


The mapping (\eqref{g_as_global}) can be constructed for any posterior quantity $\g$.
Therefore, linearizing the global parameters using \eqref{global_sens, global_lin_approx} is sufficient:
we do not need to invert the full Hessian
and linearize the entire set of variational parameters, global and local.
