We now state conditions under which $\t \mapsto \etaopt(\t)$, as defined by
\defref{prior_t}, is continuously differentiable.  Our key theoretical tool will
be the implicit function theorem (e.g., \citet{krantz:2012:implicit}), applied
to the first-order conditions for the VB optimization problem, and the dominated
converence theorem (e.g., \citet[Theorem 16.8]{billingsley:1986:probability}),
which will allow us to express derivatives of variational expectations in terms
of properties  of other variational expectations.

The derivative can expressed in terms of unnormalized densities, which can
simplify some computation.  To that end, let $\qtil$ and $\ptil$ refer to
potentially unnormalized (but normalizable) versions of the respectively
corresponding $\q$ and $\p$ given in \defref{prior_t}, so that
%
\begin{align*}
%
\q(\theta \vert \eta) :={}
    \frac{\qtil(\theta \vert \eta)}
    {\int \qtil(\theta' \vert \eta) \mu(d\theta')} \mathand
\p(\theta \vert \t) :={}
    \frac{\ptil(\theta \vert \t)}
    {\int \ptil(\theta' \vert \t) \mu(d\theta')}.
%
\end{align*}

For \assuref{kl_opt_ok}, we state some regularity conditions on the ``base
problem'', $\KL{\eta}$.

%%%%%%%%%%%%%%%%%%%%%%%%%%%%%%%%%%%%%%%%%%%%%%%%%%%%%%%%%%%%%%%%%%%%%%%%%%%%
%%%%%%%%%%%%%%%%%%%%%%%%%%%%%%%%%%%%%%%%%%%%%%%%%%%%%%%%%%%%%%%%%%%%%%%%%%%%
\begin{assu}\assulabel{kl_opt_ok}
%
Let the following conditions on the variational approximation hold.
%
%%%%%%%%%%%%%%%%%%%%%%%%%%%%%%%%%%%%%%%%%%%%%%%%%%%%%%%%%%%%%%%%%%%%%%
\begin{enumerate}
%
    \item \itemlabel{kl_diffable} The map $\eta \mapsto \KL{\eta}$ is twice
    continuously differentiable at $\etaopt$.

    \item \itemlabel{kl_opt_interior} The optimal $\etaopt$ is interior
    to $\etadom$.

    \item\itemlabel{kl_hess} The Hessian matrix $\fracat{\partial^2 \KL{\eta}}
    {\partial \eta \partial \eta^T} {\etaopt}$ is positive definite.

    % \item\itemlabel{kl_dct} The unnormalized variational densities $\qtil(\theta
    % \vert \eta)$ satisfy \assuref{dist_fun_nice} with $\psi(\theta, \t) \equiv
    % 1$ (no $\theta$ dependence).
%
\end{enumerate}
%
\end{assu}
%%%%%%%%%%%%%%%%%%%%%%%%%%%%%%%%%%%%%%%%%%%%%%%%%%%%%%%%%%%%%%%%%%%%%%%%%%%%

Next, we assume that we can exchange the order of integration and
differentiation in variational expectations.

%%%%%%%%%%%%%%%%%%%%%%%%%%%%%%%%%%%%%%%%%%%%%%%%%%%%%%%%%%%%%%%%%%%%%%%%%%%%
%%%%%%%%%%%%%%%%%%%%%%%%%%%%%%%%%%%%%%%%%%%%%%%%%%%%%%%%%%%%%%%%%%%%%%%%%%%%
%
\begin{assu}\assulabel{exchange_order}
%
Assume that we can exchange the order of $\q$-integration and differentiation in
the expression $\expect{\q(\theta \vert \eta)}{\log \ptil(\theta \vert \t)}$ for
the derivatives $\partial / \partial \eta$, $\partial^2 / \partial \eta^2$, and
$\partial^2 / \partial \eta \partial \t$, at $\eta = \etaopt$ and $\t = 0$.
% %
% \begin{align*}
% %
% \fracat{\partial\expect{\q(\theta \vert \eta)}
%                        {\log \ptil(\theta \vert \t)}}
%        {\partial\eta}{\etaopt, \t=0}
% \textrm{, }
% \fracat{\partial^2\expect{\q(\theta \vert \eta)}
%                          {\log \ptil(\theta \vert \t)}}
%        {\partial\eta \partial\eta}{\etaopt, \t=0}
% \textrm{, and }
% \fracat{\partial^2\expect{\q(\theta \vert \eta)}
%                          {\log \ptil(\theta \vert \t)}}
%        {\partial\eta \partial \t}{\etaopt, \t=0}.
% %
% \end{align*}
%
\end{assu}
%%%%%%%%%%%%%%%%%%%%%%%%%%%%%%%%%%%%%%%%%%%%%%%%%%%%%%%%%%%%%%%%%%%%%%%%%%%%

In \appref{cont_lemmas}, we state easy-to-verify (but wordy) sufficient
conditions that allow us to establish \assuref{exchange_order} using the
dominated convergence theorem.

We are now in a position to define the quantities that occur in the derivative.


%%%%%%%%%%%%%%%%%%%%%%%%%%%%%%%%%%%%%%%%%%%%%%%%%%%%%%%%%%%%%%%%%%%%%%%%%%%%
%%%%%%%%%%%%%%%%%%%%%%%%%%%%%%%%%%%%%%%%%%%%%%%%%%%%%%%%%%%%%%%%%%%%%%%%%%%%
\begin{defn}\deflabel{deriv_quantities}
%
Under the conditions of \defref{prior_t}, when \assuref{kl_opt_ok,
exchange_order} hold, define
%
\begin{align*}
%
\hessopt :={}& \fracat{\partial^2 \KL{\eta}}
                      {\partial \eta \partial \eta^T}
                      {\etaopt} \mathand \\
%
\lqgradbar{\theta \vert \etaopt} :={}&
    \lqgrad{\theta \vert \etaopt} -
    \expect{\q(\theta \vert \etaopt)}{\lqgrad{\theta \vert \etaopt}}.
%
\end{align*}

% Note that if $\qtil(\theta \vert \eta)$ is already normalized ($\qtil = \q$),
% then $\expect{\q(\theta \vert \eta)}{\lqgrad{\theta \vert \eta}} = 0$ for all
% $\eta$ and $\lqgradbar{\theta \vert \etaopt} = \lqgrad{\theta \vert \etaopt}$.

Further, define

\begin{align*}
%
\crosshessian :={}&
    \fracat{\partial
            \expect{\q(\theta \vert \etaopt)}
                   {\fracat{\partial \log \ptil(\theta \vert \t)}
                           {\partial \t}{\t=0} }
            }
        {\partial \eta}{\etaopt}
={}
    \expect{\q(\theta \vert \etaopt)}{
          \lqgradbar{\theta \vert \etaopt}
          \fracat{\partial \log \ptil(\theta \vert \t)}
                 {\partial \t}{\t=0}},
%
\end{align*}
%
where the final equality follows from differentiating under the integral using
\assuref{exchange_order} (see \lemref{logq_derivs} in \appref{cont_lemmas} for
more details).
%
\end{defn}
%%%%%%%%%%%%%%%%%%%%%%%%%%%%%%%%%%%%%%%%%%%%%%%%%%%%%%%%%%%%%%%%%%%%%%%%%%%%


%%%%%%%%%%%%%%%%%%%%%%%%%%%%%%%%%%%%%%%%%%%%%%%%%%%%%%%%%%%%%%%%%%%%%%%%%%%%
%%%%%%%%%%%%%%%%%%%%%%%%%%%%%%%%%%%%%%%%%%%%%%%%%%%%%%%%%%%%%%%%%%%%%%%%%%%%
\begin{thm}\thmlabel{etat_deriv}
%
Under the conditions of \defref{prior_t, deriv_quantities}, let
\assuref{kl_opt_ok, exchange_order} hold.   Then the map $\t \mapsto
\etaopt(\t)$ is continuously differentiable at $\t=0$ with derivative
%
\begin{align}\eqlabel{vb_eta_sens}
%
\fracat{d \etaopt(\t)}{d \t}{0} ={}&
    - \hessopt^{-1} \crosshessian.
%
\end{align}
%
(For a proof, see \appref{proofs} \proofref{etat_deriv}.)
%
\end{thm}
%%%%%%%%%%%%%%%%%%%%%%%%%%%%%%%%%%%%%%%%%%%%%%%%%%%%%%%%%%%%%%%%%%%%%%%%%%%%
