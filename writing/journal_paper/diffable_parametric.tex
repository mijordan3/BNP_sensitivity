We now state conditions under which $\t \mapsto \etaopt(\t)$, as defined by
\defref{prior_t}, is continuously differentiable.  Our key theoretical tool will
be the implicit function theorem (e.g., \citet{krantz:2012:implicit}), applied
to the first-order conditions for the VB optimization problem, and the dominated
converence theorem (e.g., \citet[Theorem 16.8]{billingsley:1986:probability}),
which will allow us to express derivatives of variational expectations in terms
of properties  of other variational expectations.

For \assuref{kl_opt_ok}, we state some regularity conditions on the ``base
problem'', $\KL{\eta}$.

%%%%%%%%%%%%%%%%%%%%%%%%%%%%%%%%%%%%%%%%%%%%%%%%%%%%%%%%%%%%%%%%%%%%%%%%%%%%
%%%%%%%%%%%%%%%%%%%%%%%%%%%%%%%%%%%%%%%%%%%%%%%%%%%%%%%%%%%%%%%%%%%%%%%%%%%%
\begin{assu}\assulabel{kl_opt_ok}
%
Let the following conditions on the variational approximation hold.
%
%%%%%%%%%%%%%%%%%%%%%%%%%%%%%%%%%%%%%%%%%%%%%%%%%%%%%%%%%%%%%%%%%%%%%%
\begin{enumerate}
%
    \item \itemlabel{kl_diffable} The map $\eta \mapsto \KL{\eta}$ is twice
    continuously differentiable at $\etaopt$.

    \item \itemlabel{kl_opt_interior} The optimal $\etaopt$ is interior
    to $\etadom$.

    \item\itemlabel{kl_hess} The Hessian matrix $\fracat{\partial^2 \KL{\eta}}
    {\partial \eta \partial \eta^T} {\etaopt}$ is positive definite.

    % \item\itemlabel{kl_dct} The unnormalized variational densities $\qtil(\theta
    % \vert \eta)$ satisfy \assuref{dist_fun_nice} with $\psi(\theta, \t) \equiv
    % 1$ (no $\theta$ dependence).
%
\end{enumerate}
%
\end{assu}
%%%%%%%%%%%%%%%%%%%%%%%%%%%%%%%%%%%%%%%%%%%%%%%%%%%%%%%%%%%%%%%%%%%%%%%%%%%%

Next, we assume that we can exchange the order of integration and
differentiation in variational expectations.

%%%%%%%%%%%%%%%%%%%%%%%%%%%%%%%%%%%%%%%%%%%%%%%%%%%%%%%%%%%%%%%%%%%%%%%%%%%%
%%%%%%%%%%%%%%%%%%%%%%%%%%%%%%%%%%%%%%%%%%%%%%%%%%%%%%%%%%%%%%%%%%%%%%%%%%%%
%
\begin{assu}\assulabel{exchange_order}
%
Assume that we can exchange the order of $\q$-integration and differentiation in
the following expressions:
%
\begin{align*}
%
\fracat{\partial\expect{\q(\theta \vert \eta)}{\psi(\theta, \t)}}
       {\partial\eta}{\etaopt, \t=0}
\textrm{, }
\fracat{\partial^2\expect{\q(\theta \vert \eta)}{\psi(\theta, \t)}}
       {\partial\eta \partial\eta}{\etaopt, \t=0}
\textrm{, and }
\fracat{\partial^2\expect{\q(\theta \vert \eta)}{\psi(\theta, \t)}}
       {\partial\eta \partial \t}{\etaopt, \t=0}.
%
\end{align*}
%
\end{assu}
%%%%%%%%%%%%%%%%%%%%%%%%%%%%%%%%%%%%%%%%%%%%%%%%%%%%%%%%%%%%%%%%%%%%%%%%%%%%

In \appref{cont_lemmas}, we state easy-to-verify (but wordy) sufficient
conditions that allow us to verify \assuref{exchange_order} using the
dominated convergence theorem.

Before stating our theorem, we define one more additional notation.  The
derivative can expressed in terms of unnormalized densities, which can simplify
some computation.  Let $\qtil$ and $\ptil$ refer to potentially unnormalized
(but normalizable) versions of the respectively corresponding $\q$ and $\p$, so
that,
%
\begin{align*}
%
\q(\theta \vert \eta) :={}
    \frac{\qtil(\theta \vert \eta)}
    {\int \qtil(\theta' \vert \eta) \mu(d\theta')} \mathand
\p(\theta \vert \t) :={}
    \frac{\ptil(\theta \vert \t)}
    {\int \ptil(\theta' \vert \t) \mu(d\theta')}.
%
\end{align*}

%%%%%%%%%%%%%%%%%%%%%%%%%%%%%%%%%%%%%%%%%%%%%%%%%%%%%%%%%%%%%%%%%%%%%%%%%%%%
%%%%%%%%%%%%%%%%%%%%%%%%%%%%%%%%%%%%%%%%%%%%%%%%%%%%%%%%%%%%%%%%%%%%%%%%%%%%
\begin{thm}\thmlabel{etat_deriv}
%
Under the conditions of \defref{prior_t}, let \assuref{kl_opt_ok,
exchange_order} hold
%
Define\footnote{Note that if $\qtil(\theta \vert \eta)$ is already normalized
($\qtil = \q$), then $\expect{\q(\theta \vert \eta)}{\lqgrad{\theta \vert \eta}} =
0$ for all $\eta$ and $\lqgradbar{\theta \vert \etaopt} = \lqgrad{\theta \vert
\etaopt}$.}
%
\begin{align*}
%
\hessopt :={}& \fracat{\partial^2 \KL{\eta, 0}}
                      {\partial \eta \partial \eta^T}
                      {\etaopt},\\
%
\lqgradbar{\theta \vert \etaopt} :={}&
    \lqgrad{\theta \vert \etaopt} -
    \expect{\q(\theta \vert \etaopt)}{\lqgrad{\theta \vert \etaopt}},
    \mathand \\
%
\crosshessian :={}&   \expect{\q(\theta \vert \etaopt)}{
      \lqgradbar{\theta \vert \etaopt}
      \fracat{\partial \log \ptil(\theta \vert \t)}
             {\partial \t}{\t=0}}.
%
\end{align*}
%
Then the map $\t \mapsto \etaopt(\t)$ is continuously differentiable at $\t=0$
with derivative
%
\begin{align}\eqlabel{vb_eta_sens}
%
\fracat{d \etaopt(\t)}{d \t}{0} ={}&
    - \hessopt^{-1}
    \crosshessian.
    % \expect{\q(\theta \vert \etaopt)}{
    %     \lqgradbar{\theta \vert \etaopt}
    %     \fracat{\partial \log \p(\theta \vert \t)}{\partial \t}{\t=0}
    % }.
    % \fracat{\partial^2 \KL{\eta, \t}}
    %        {\partial \eta \partial \t}
    %        {\etaopt, 0}.
%
\end{align}
%
(For a proof, see \appref{proofs} \proofref{etat_deriv}.)
%
\end{thm}
%%%%%%%%%%%%%%%%%%%%%%%%%%%%%%%%%%%%%%%%%%%%%%%%%%%%%%%%%%%%%%%%%%%%%%%%%%%%
