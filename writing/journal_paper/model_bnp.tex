We will consider discrete Bayesian nonparametric (BNP) generative models which
draw data points $\x_n$ from one of a certain number of clusters indexed by $\k =
1, \ldots, \kmax$, where in principle $\kmax$ might be infinity.  Each cluster
is characterized by a vector $\theta_\k \in \thetadom \subseteq
\mathbb{R}^{\thetadim}$, with $\p(\x_n \vert \theta_\k)$ denoting the
distribution of data arising from  cluster $\k$. We will model the $\theta_\k$
as arising IID from a known prior $\theta_\k \iid \p(\theta_\k)$, and write
$\theta = (\theta_1, \theta_2, \ldots)$.

The cluster assignments are represented by indicator variables $\z_{\n\k} \in
\{0, 1\}$, where $\z_{\n\k} = 1$ for exactly one $\k$ and $0$ otherwise.  Let
$\z_\n$ with a single index denote all the cluster indicators for a given
datapoint, $\z_\n = (\z_{\n1}, \ldots, \z_{\n\kmax})$, and let $\z$ with no
index denote the collection $\z = (\z_1, \ldots, \z_\N)$.  With $\z$ defined in
this way, we can write
%
\begin{align*}
%
\p(\x_n \vert \z_\n, \theta) =
    \prod_{k=1}^\kmax \p(\x_n \vert \theta_\k)^{\z_{\n\k}}.
%
\end{align*}


The cluster assignments $\z_{\n}$ will be drawn according to the following
``stick-breaking process.''  Fix a density $\pstick(\cdot)$, with respect to the
Lebesgue measure, over stick lengths $\nuk \in (0, 1)$ and draw $\nuk \iid
\pstick(\nuk)$ for $\k=1,\ldots,\kmax - 1$.  If $\kmax = \infty$, there are an
infinite number of IID sticks and the $\kmax - 1$ can be ignored; if $\kmax <
\infty$, then we set $\nu_{\kmax} = 1$ for notational convenience below.  Given
these stick lengths, we compute indicator probabilities by
%
\begin{align*}
%
% \pi_\k := \begin{cases}
% \nuk \prod_{\k' < \k} (1 - \nu_{\k'}) & \textrm{For }k < \kmax
%     \textrm{ (all }k\textrm{ when }\kmax = \infty\textrm{)}\\
% \prod_{\k' < \k} (1 - \nu_{\k'}). & \textrm{For }k = \kmax \\
% \end{cases}
\pi_\k := \nuk \prod_{\k' < \k} (1 - \nu_{\k'})
%
\end{align*}
%
where the empty product is taken to be equal to $1$. The convention $\nu_{\kmax} =
1$ allows us to use the same formula for all $\pi_\k$ when $\kmax < \infty$.
Write $\nu := (\nu_1, \ldots, \nu_{\kmax})$ for the vector of all stick lengths
and $\pi := (\pi_1, \ldots, \pi_\kmax)$ for the corresponding vector of
proabilities. By construction, $\sum_{\k=1}^{\kmax} \pi_\k = 1$ even when $\kmax =
\infty$, and we draw $\z_\n$ according to
%
\begin{align*}
%
\p(\z_{\n\k} = 1 \vert \pi) ={}& \pi_\k \\
\p(\z_\n \vert \pi) ={}&
    \ind{\sum_{\k=1}^{\kmax} \z_{\n\k} = 1}
    \prod_{k=1}^{\kmax} \pi_\k^{\z_{\n\k}}.
%
\end{align*}
%
Since $\pi$ is a deterministic function of $\nu$, we can also write
$\p(\z_\n \vert \nu)$ with no ambiguity.

The stick-breaking process can be though of as a map from  the stick
distibution, $\pstick(\cdot)$, to a vector of probabilities, $\pi$. Different
stick distributions will give different indicator probabilities, with different
implied degrees of concentration.  A particularly common choice for
$\pstick(\cdot)$ is the $\mathrm{Beta}(\nuk \vert 1, \alpha)$ distribution,
%
\begin{align*}
%
\mathrm{Beta}(\nuk \vert 1, \alpha) =
    \frac{\Gamma(1 + \alpha) (1 - \nuk)^{\alpha - 1}}
         {\Gamma(\alpha)}
%
\end{align*}
%
since this choice is equivalent to drawing $\pi$ from a Dirichlet process.
In this case we can write $\pi \sim \mathrm{GEM}(\alpha)$.  TODO: probably
this is not precisely correct.

For a fixed $\pstick(\cdot)$, we can thus write the joint distribution of
the data and parameters as:
%
\begin{align}\eqlabel{bnp_model}
%
% \logp(\x, \theta, \z, \nu) ={}&
%     \sum_{n=1}^N \sum_{k=1}^{\kmax}
%         \z_{\n\k} \left(
%             \logp(\x_n \vert \theta_\k) + \log \pi_\k
%         \right) +
% \nonumber \\ {}&
%     \sum_{k=1}^{\kmax} \left(
%         \log \pstick(\nuk) + \logp(\theta_\k)
%     \right).
\MoveEqLeft
\logp(\x, \theta, \z, \nu) ={}
\nonumber\\&
    \sum_{n=1}^N \sum_{k=1}^{\kmax}
        \z_{\n\k} \left(
            \logp(\x_n \vert \theta_\k) + \log \pi_\k
        \right) +
    \sum_{k=1}^{\kmax} \left(
        \log \pstick(\nuk) + \logp(\theta_\k)
    \right).
%
\end{align}
%

%%%%%%%%%%%%%%%%%%%%%%%%%%%%%%%%%%%%%%%%%%%%%%%%%%%%%%%%%%%%%%%%%%%%%%%%%%%%%%%
%%%%%%%%%%%%%%%%%%%%%%%%%%%%%%%%%%%%%%%%%%%%%%%%%%%%%%%%%%%%%%%%%%%%%%%%%%%%%%%
\begin{ex}[Generative process]\exlabel{iris_bnp_process}
%
For the iris data, $\x_\n \in \mathbb{R}^4$, and we will model the clusters as
four dimensional Gaussians.  In this case, $\theta_\k = (\mu_k, \Sigma_\k)$,
where $\mu_\k \in \mathbb{R}^4$, $\Sigma_\k$ is a a four by four positive
definite covariance matrix, and
%
\begin{align*}
%
\p(\x_\n \vert \theta_\k) ={}& \normdist{\x_n \vert \mu_\k, \Sigma_\k} \\
\logp(\x_\n \vert \theta_\k) ={}&
    -\frac{1}{2}(\x_n - \mu_k)^T \Sigma_\k^{-1} (\x_n - \mu_k)
    -\frac{1}{2} \log |\Sigma_\k| + \const.\\
    & \constdesc{\theta_\k}
%
\end{align*}

For the iris data, we might imagine that each cluster corresponds to a different
species with a different distribution of flower dimensions.  The BNP model
implies that there are a potentially infinite number of differet iris species
that we might observe.  Then $\z_{\n\k} = 1$ would mean that observation $\n$
was a member of species $\k$, and $\sum_{k=1}^\kmax \ind{ \sum_{n=1}^{N}
\z_{\n\k} > 1}$ is the number of distinct species observed in our particular
dataset.
%
\end{ex}
%%%%%%%%%%%%%%%%%%%%%%%%%%%%%%%%%%%%%%%%%%%%%%%%%%%%%%%%%%%%%%%%%%%%%%%%%%%%%%%
