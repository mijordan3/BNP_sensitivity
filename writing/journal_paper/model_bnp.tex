A discrete Bayesian nonparametric (BNP) generative model
draws data points $\x_n$ from one of an infinite number of components indexed by
$\k = 1, 2, \ldots \infty$.
Each component is characterized by a vector $\beta_\k \in \betadom \subseteq
\mathbb{R}^{\betadim}$, with $\p(\x_n \vert \beta_\k)$ denoting the
distribution of data arising from  component $\k$.
We model the $\beta_\k$
as arising IID from a known prior, or \textit{base distribution}, denoted
$\pbetaprior(\beta_\k)$, and write
$\beta = (\beta_1, \beta_2, \ldots)$.

Assignment of data point $\n$ to a mixture component is represented by an
(infinite dimensional) vector $\z_\n = (\z_{\n1}, \z_{\n2}, \ldots)$
whose elements $\z_{\n\k} = 1$ for exactly one $\k$ and $0$ otherwise.
With $\z_\n$ defined in this way, we can write
%
\begin{align*}
%
\p(\x_n \vert \z_\n, \beta) =
    \prod_{k=1}^\infty \p(\x_n \vert \beta_\k)^{\z_{\n\k}}.
%
\end{align*}


The prior probabilities of assignments $\z_{\n}$ are generated according
to the following ``stick-breaking" process.
Fix a density $\pstick(\cdot)$, with respect to the
Lebesgue measure, over stick-breaking proportions $\nuk \in (0, 1)$ and
draw $\nuk\iid\pstick(\nuk)$ for $\k=1,2,\ldots\infty$.
Given these stick lengths, construct probabilities using the following formula:
%
\begin{align}\eqlabel{stick_breaking}
%
% \pi_\k := \begin{cases}
% \nuk \prod_{\k' < \k} (1 - \nu_{\k'}) & \textrm{For }k < \infty
%     \textrm{ (all }k\textrm{ when }\infty = \infty\textrm{)}\\
% \prod_{\k' < \k} (1 - \nu_{\k'}). & \textrm{For }k = \infty \\
% \end{cases}
\pi_\k := \nuk \prod_{\k' < \k} (1 - \nu_{\k'}),
%
\end{align}
%
where the empty product is taken to be equal to $1$.
By construction, $\sum_{\k=1}^{\infty} \pi_\k = 1$.
Given the probability vector $\pi := (\pi_1, \pi_2, \ldots)$,
the $\z_\n$ are drawn according to
%
\begin{align*}
%
% \p(\z_{\n\k} = 1 \vert \pi) ={}& \pi_\k \\
% \p(\z_\n \vert \pi) ={}&
%     \ind{\sum_{\k=1}^{\kmax} \z_{\n\k} = 1}
%     \prod_{k=1}^{\kmax} \pi_\k^{\z_{\n\k}}.
\p(\z_\n \vert \pi) ={}&
   \prod_{k=1}^{\infty} \pi_\k^{\z_{\n\k}}.
%
\end{align*}
%
Since $\pi$ is a deterministic function of the stick-breaking proportions
$\nu := (\nu_1, \nu_2, \ldots)$,
we can also write
$\p(\z_\n \vert \nu)$ with no ambiguity.

The stick-breaking distribution $\pstick$ can be thought of as inducing a
a distribution on the vector of probabilities $\pi$. Different
stick-breaking distributions will different favor assignment probabilities, each
with different implied degrees of concentration.
A particularly common choice for
$\pstick(\nuk)$ is the $\mathrm{Beta}(\nuk \vert 1, \alpha)$ distribution,
%
\begin{align*}
%
\mathrm{Beta}(\nuk \vert 1, \alpha) =
    \frac{\Gamma(1 + \alpha) (1 - \nuk)^{\alpha - 1}}
         {\Gamma(\alpha)}.
%
\end{align*}
%
When $\pstick$ is $\mathrm{Beta}(\nuk \vert 1, \alpha)$, the resulting distribution on $\pi$ is known as the
$\textit{GEM distribution}$, and we write $\pi \sim \mathrm{GEM}(\alpha)$.

The GEM distribution is closely related to the Dirichlet process (DP).
Define a measure on $\betadom$ as
\begin{align*}
  \mathcal{M} = \sum_{\k = 1}^\infty \pi_\k\delta_{\beta_\k},
\end{align*}
which places atoms at points $\beta_k$ with weight $\pi_\k$.
When $\pi \sim \mathrm{GEM}(\alpha)$ and
$\beta_\k \iid \pbetaprior(\beta_\k)$, $\mathcal{M}$ is a random measure
is distributed according to Dirichlet process with concentration parameter
$\alpha$ and base measure $\pbetaprior$
\citep{ferguson:1973:bayesian, sethuraman:1994:constructivedp}.

We keep the generic notation $\pstick$ for stick-breaking distributions
because in our sensitivity analysis,
we will consider stick-breaking distributions that are outside the
family of Beta distributions.

In the generative process we have just outlined,
the joint distribution of
the observed data and latent variables in a basic BNP mixture model is
%
\begin{align}\eqlabel{bnp_model}
%
% \logp(\x, \beta, \z, \nu) ={}&
%     \sum_{n=1}^N \sum_{k=1}^{\kmax}
%         \z_{\n\k} \left(
%             \logp(\x_n \vert \beta_\k) + \log \pi_\k
%         \right) +
% \nonumber \\ {}&
%     \sum_{k=1}^{\kmax} \left(
%         \log \pstick(\nuk) + \logp(\beta_\k)
%     \right).
\logp(\x, \beta, \z, \nu) =&
\sum_{n=1}^N \sum_{k=1}^{\infty}
    \z_{\n\k} \left(
        \logp(\x_n \vert \beta_\k) + \log \pi_\k
    \right)
\nonumber\\
   & +
    \sum_{k=1}^{\infty} \left(
        \log \pstick(\nuk) + \log \pbetaprior(\beta_\k)
    \right).
%
\end{align}
%

%%%%%%%%%%%%%%%%%%%%%%%%%%%%%%%%%%%%%%%%%%%%%%%%%%%%%%%%%%%%%%%%%%%%%%%%%%%%%%%
%%%%%%%%%%%%%%%%%%%%%%%%%%%%%%%%%%%%%%%%%%%%%%%%%%%%%%%%%%%%%%%%%%%%%%%%%%%%%%%
\begin{ex}[Gaussian mixture model]\exlabel{iris_bnp_process}
%
The observations are vectors $\x_\n \in \mathbb{R}^\d$,
and we model each component with a multivariate Gaussian.
In this model, $\beta_\k = (\mu_k, \Lambda_\k)$,
where $\mu_\k \in \mathbb{R}^\d$, $\Lambda_\k$ is a $\d\times\d$ positive
definite information matrix, and
%
\begin{align*}
%
\p(\x_\n \vert \beta_\k) ={}& \normdist{\x_n \vert \mu_\k, \Lambda_\k^{-1}} \\
\logp(\x_\n \vert \beta_\k) ={}&
    -\frac{1}{2}(\x_n - \mu_k)^T \Lambda_\k (\x_n - \mu_k)
    + \frac{1}{2} \log |\Lambda_\k| + \const.\\
    & \constdesc{\beta_\k}
%
\end{align*}
We let $\pbetaprior(\beta_\k)$ be the conjugate prior, which in this case is normal-Wishart:
\begin{align*}
  \pbetaprior(\beta_\k) &= \normalwishart{\beta_\k \vert \tau_0, n_0, p_0, V_0}\\
  \log\pbetaprior(\beta_\k) &=
      -\frac{\tau_0}{2}(\mu_\k - \mu_0)^T \Lambda_\k (\mu_\k - \mu_0)\\
      &{} + \frac{n_0 - p_0 - 1}{2} \log |\Lambda_\k| -
      \frac{1}{2} \textrm{Tr}(V_0 \Lambda_\k) + \const,
\end{align*}
where $(\tau_0, n_0, p_0, V_0)$ are fixed prior parameters.
%
% Below, we fit a
% Gaussian mixture model (GMM) to Fisher's iris data set CITE.
% Each observation is an iris flower with
% four measurements:
% sepal length, sepal width, petal length, and petal width.
% The components in this model can be interpreted as latent iris species;
% the inferential goal is to estimate $\z_\n$ and assign each
% observed flower to a latent species.

% For the iris data, we might imagine that each cluster corresponds to a different
% species with a different distribution of flower dimensions.  The BNP model
% implies that there are a potentially infinite number of differet iris species
% that we might observe.  Then $\z_{\n\k} = 1$ would mean that observation $\n$
% was a member of species $\k$, and $\sum_{k=1}^\kmax \ind{ \sum_{n=1}^{N}
% \z_{\n\k} > 1}$ is the number of distinct species observed in our particular
% dataset.
%
\end{ex}

We start with a Gaussian mixture model (GMM)
because it conforms cleanly to the
generative process culminating in \eqref{bnp_model}.
In \secref{results}, we fit a GMM to Fisher's iris data set
\citep{anderson:1936:iris, fisher:1936:iris} and
cluster irises into latent species based on morphological measurements.
The next two examples, which we will apply to real data sets,
require more careful modeling considerations,
and we adjust the factorization in \eqref{bnp_model} to suit our purposes.

%%%%%%%%%%%%%%%%%%%%%%%%%%%%%%%%%%%%%%%%%%%%%%%%%%%%%%%%%%%%%%%%%%%%%%%%%%%%%%%
%%%%%%%%%%%%%%%%%%%%%%%%%%%%%%%%%%%%%%%%%%%%%%%%%%%%%%%%%%%%%%%%%%%%%%%%%%%%%%%

\begin{ex}[Regression mixture model]\exlabel{mice_bnp_process}

We cluster time-course gene expression data.
An observation $\x_\n\in\mathbb{R}^\ntimepoints$ is a vector of
expression levels at $\ntimepoints$
time points.
Let $\regmatrix$ be an $\ntimepoints \times \d$ regressor matrix.
In our case, we will use cubic B-splines to smooth the time-course observations,
so the $ij$-th entry of $\regmatrix$
will be the $j$-th B-spline basis vector evaluated at the
$i$-th time point (\secref{results_mice}).

Each component is characterized by a vector of regression coefficients
$\mu_\k$ and a variance $\tau^{-1}_\k$, so
in this model, $\beta_k = (\mu_\k, \tau_\k)$.
The distribution of the data arising from component $k$ is
\begin{align*}
\p(\x_\n | \beta_\k, \b_\n) =
\normdist{\x_\n | \regmatrix\mu_\k + \b_\n,
\tau_\k^{-1}I_{\ntimepoints \times \ntimepoints}},
\end{align*}
%
where $\b_{n}$ is a gene-specific additive offset and $I$ is the identity matrix.
We include the additive offset because we
are interested in clustering gene expressions based on their patterns over time,
not their absolute level.

The joint distribution can be written in the same form as~\eqref{bnp_model},
except that the conditional data likelihood now depends on $\b_\n$ as well as $\beta_\k$,
and we include an additional prior term for $\b_\n$.
% \begin{align*}
% \logp(\x, \beta, \z, \nu, \b) =&
%     \sum_{n=1}^N \sum_{k=1}^{\infty}
%         \z_{\n\k} \left(
%             \logp(\x_n \vert \beta_\k, \b_n) + \log \pshift(\b_n) + \log \pi_\k
%         \right)  \\
%     &{} + \sum_{k=1}^{\infty} \left(
%         \log \pstick(\nuk) + \log \pbetaprior(\beta_\k)
%     \right).
% \end{align*}
%
\end{ex}

%%%%%%%%%%%%%%%%%%%%%%%%%%%%%%%%%%%%%%%%%%%%%%%%%%%%%%%%%%%%%%%%%%%%%%%%%%%%%%%
%%%%%%%%%%%%%%%%%%%%%%%%%%%%%%%%%%%%%%%%%%%%%%%%%%%%%%%%%%%%%%%%%%%%%%%%%%%%%%%

Our last example is a Bayesian topic model applied to genetic data.
Genotypes at genetic markers take the place of
words in a document; in lieu of inferring ``topics," we infer latent populations.

\begin{ex}[A topic model for population structure]\exlabel{structure_bnp_process}

We consider genetic data where the
data set consists of $\nindiv$ individuals genotyped at $\nloci$ loci.
For diploid organisms, there are two observations at each loci, one at each chromosome.
Let $\x_{\n\l\i}\in\{1, \ldots, J_\l\}$ be the observed genotype for
individual $\n$ at locus $\l$ and chromosome $\i$;
$J_\l$ is the number of possible genotypes at locus $\l$.
For example, if the measurements are single nucleotides (A, T, C or G)
then $J_\l = 4$ for all $\l$.

A latent population is characterized by the collection
$\beta_k = (\latentpop_{\k1}, \ldots, \latentpop_{\k\nloci})$ where
$\latentpop_{\k\l}\in\Delta^{J_\l - 1}$ are the latent frequencies for the $J_l$
possible genotypes at locus $\l$.
Let $\z_{\n\l\i}$ be the assignment of observation $\x_{\n\l\i}$ to a latent population.
Notice that for a given individual $\n$,
different loci, or even different chromosomes at a given locus,
may have different population assignments.
The distribution of $\x_{\n\l\i}\in\{1, \ldots, J_\l\}$ arising from population $\k$ is
\begin{align*}
\p(\x_{\n\l\i} \vert \latentpop_{\k}) =
\categoricaldist{\x_{\n\l\i}\vert \latentpop_{\k\l}}.
\end{align*}


Unlike the previous models, we now have a stick-breaking process for each individual.
Draw sticks
\begin{align*}
\nu_{\n\k} \iid \pstick(\nu_{\n\k}) \quad \forall \n = 1, \ldots, \nindiv; \k = 1, 2, \ldots \infty.
\end{align*}
The prior assignment probability vector
$\latentadmix_{\n} = (\latentadmix_{\n1}, \latentadmix_{\n2}, \ldots)$,
now unique to each individual,
is formed by the same stick-breaking construction as before,
%
\begin{align*}
\latentadmix_{\n\k} = \nu_{\n\k} \prod_{\k' < \k} (1 - \nu_{\n\k'}).
\end{align*}
%
The population assignment $\z_{\n\l\i}$
is drawn from the
usual multinomial distribution
%
\begin{align*}
p(\z_{\n\l\i} | \latentadmix_\n) = \prod_{k=1}^{\infty} \latentadmix_{\n\k}^{\z_{\n\l\i\k}}.
\end{align*}
%
In this genetics application,
we call $\latentadmix_{\n}$ the
\textit{admixture} of individual $\n$.


The joint log-likelihood decomposes as
\begin{align*}
\logp(\x, \latentpop, \z, \nu) &=
\sum_{\n=1}^\nindiv \sum_{\l=1}^\nloci \sum_{i = 1}^2 \sum_{\k=1}^{\infty}
        \z_{\n\l\i\k} \left(
            \logp(\x_{\n\l\i} \vert \latentpop_{\k}) + \log \pi_{\n\k}
        \right)
\nonumber\\&
    \quad +
    \sum_{\n=1}^\nindiv \sum_{k=1}^{\infty} \log \pstick(\nu_{\n\k})
    + \sum_{k=1}^{\infty} \log \pbetaprior(\latentpop_{\k}).
\end{align*}

This model is identical to STRUCTURE,
a model proposed in \citet{pritchard:2000:structure, raj:2014:faststructure},
except that we replace the Dirichlet prior in STRUCTURE
with an infinite stick-breaking process.
The result is a model similar to a hierarchical Dirichlet process for topic modeling,
\citep{teh:2006:hdp},
but without the top-level Dirichlet process.
%
\end{ex}


%%%%%%%%%%%%%%%%%%%%%%%%%%%%%%%%%%%%%%%%%%%%%%%%%%%%%%%%%%%%%%%%%%%%%%%%%%%%%%%
