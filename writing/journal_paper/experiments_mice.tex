%%%%%%%%%%%%%%%%%%%%%%%%%%%%%%%%%%%%%%
%%%%%%%%%%%%%%%%%%%%%%%%%%%%%%%%%%%%%%
% Do not edit the TeX file your work
% will be overwritten.  Edit the RnW
% file instead.
%%%%%%%%%%%%%%%%%%%%%%%%%%%%%%%%%%%%%%
%%%%%%%%%%%%%%%%%%%%%%%%%%%%%%%%%%%%%%



In this section, we study a publicly available data set of mice gene expression
\citep{shoemaker:2015:ultrasensitive}. Mice were infected with influenza virus,
and expression levels of a set of genes were assessed at 14 time points after
infection. Three measurements were taken at each time point (called biological
replicates), for a total of $\ntimepoints = 42$ measurements per gene.

The goal of the analysis is to cluster the time-course gene expression data
under the assumption that genes with similar time-course behavior may
have similar function.
Clustering gene expressions is often used for exploratory analysis
and is a first step before further downstream investigation.
It is important, therefore, to ascertain the
stability of the discovered clusters.

\subsubsection*{The model}

% \todo{This could be much clearer.
% Start by referencing the figure as an example.
% Then say that a cluster defines an average expression level over time, that
% each gene is modeled as belonging to a cluster, and that the gene's
% expression is modeled as IID noise added to its cluster's time series with
% a per-gene offset.  Then you can say that we use B-splines to represent
% the cluster time series, and list the parameters.}


The left plot of \figref{example_genes}
shows the measurements of a single gene over time.
We model each gene as belonging to a latent component,
where each component defines a smooth expression curve over time.
Then, observations are drawn by adding i.i.d. noise to the smoothed
curve along with a gene-specific offset.
Following \citet{Luan:2003:clustering} we construct the smoothers
using cubic B-splines.

Let $\x_\n\in\mathbb{R}^\ntimepoints$ be measurements of gene $\n$ at
$\ntimepoints$ time points. Let $\regmatrix$ be the $\ntimepoints \times \d$
B-spline regressor matrix, so that the $ij$-th entry of $\regmatrix$ is the
$j$-th B-spline basis vector evaluated at the $i$-th time point. The right plot
of \figref{example_genes} shows the B-spline basis. The distribution of the data
arising from component $k$ is
%
\begin{align}\eqlabel{mice_model}
\p(\x_\n | \beta_\k, \b_\n) =
\normdist{\x_\n | \regmatrix\mu_\k + \b_\n,
\tau_\k^{-1}I_{\ntimepoints \times \ntimepoints}},
\end{align}
%
where $\b_\n$ is a gene-specific additive offset and $I$ is the identity matrix.
We include the additive offset because we
are interested in clustering gene expressions based on their patterns over time,
not their absolute level.
In this model, the component variables are $\beta_\k = (\mu_\k, \tau_\k)$,
the regression coefficients and the inverse noise variance.
%
The mixture weights $\pi$ and cluster assignments $\z$ are drawn from the
stick-breaking process described in \secref{model_bnp}.

Our variational approximation factorizes similarly to \eqref{vb_mf}
except with an additional factor for the additive shift.
In our variational approximation, we also make a simplification by letting
$\q(\beta_\k \vert \eta) = \delta (\beta_k \vert \eta)$,
where $\delta(\cdot \vert \eta)$ denotes a point mass at a parameterized location.
See \appref{app_mice} for further details concerning the model and
variational approximation.


\subsubsection*{Quantity of interest: the co-clustering matrix}

In this application,
we are particularly interested in which genes cluster together,
a quantity which we express through the following posterior co-clustering
matrix.  Let $\gcoclustering(\eta)\in\mathbb{R}^{\N\times\N}$ denote the matrix
whose $(i,j)$-th entry is the posterior probability that gene $i$ belongs to the
same cluster as gene $j$, given by
%
\begin{align*}
%
[\gcoclustering(\eta)]_{ij} =
\expect{\q(\z\vert\eta)}{\ind{\z_{i} = \z_{j}}}  =
\begin{cases}
\sum_{k=1}^{\kmax}\left(\expect{\q(\z_i\vert\eta)}{\z_{ik}}
\expect{\q(\z_j\vert\eta)}{\z_{jk}}\right)
& \text{for } i \not= j\\
1 & \text{for } i = j
\end{cases}.
%
\end{align*}

Below, we will use the influence function (\corref{etafun_worst_case}) to try
and find a perturbation that produces large changes in the co-clustering matrix.
To compute the worst-case perturbation, we must choose a univariate summary of
the $\ngenes^2$-dimensional co-clustering matrix whose derivative we wish to
extremize. We use the sum of the eigenvalues of the symmetrically normalized
graph Laplacian, as given by
%
\begin{align*} \laplacianevsum(\eta) = \text{Tr}\left( I - D(\eta)^{-1/2}
\gcoclustering(\eta) D(\eta)^{-1/2} \right), \end{align*}
%
where $D(\eta)^{-1/2}$ is the diagonal matrix with entries $d_i =
\sum_{j=1}^{\ngenes}[\gcoclustering(\eta)]_{ij}$. The quantity $\laplacianevsum$
is differentiable, and has close connection with the number of distinct
components in a graph~\citep{luxburg:2007:spectralcluster}. We expect that prior
perturbations which produce large changes in $\laplacianevsum$ will also produce large changes in the full co-clustering matrix.

\newcommand{\MiceSmoothers}{
% moving this to appendix

\begin{knitrout}
\definecolor{shadecolor}{rgb}{0.969, 0.969, 0.969}\color{fgcolor}\begin{figure}[!h]

{\centering \includegraphics[width=0.980\linewidth,height=0.627\linewidth]{figure/gene_centroids-1} 

}

\caption[Inferred clusters in the mice gene expression dataset.
    Shown are the twelve most occupied clusters.
    In blue, the inferred cluster centroid.
    In grey, gene expressions averaged over replicates and
    shifted by their inferred intercepts]{Inferred clusters in the mice gene expression dataset.
    Shown are the twelve most occupied clusters.
    In blue, the inferred cluster centroid.
    In grey, gene expressions averaged over replicates and
    shifted by their inferred intercepts. }\label{fig:gene_centroids}
\end{figure}


\end{knitrout}
}



\begin{knitrout}
\definecolor{shadecolor}{rgb}{0.969, 0.969, 0.969}\color{fgcolor}\begin{figure}[!h]

{\centering \includegraphics[width=0.588\linewidth,height=0.470\linewidth]{figure/gene_initial_coclustering-1} 

}

\caption[The inferred co-clustering matrix of gene expressions at $\alpha_0 = 6.$ ]{The inferred co-clustering matrix of gene expressions at $\alpha_0 = 6.$ }\label{fig:gene_initial_coclustering}
\end{figure}


\end{knitrout}


\subsubsection*{Parametric sensitivity}

We first evaluate the sensitivity of the co-clustering matrix $\gcoclustering$
to the choice of $\alpha$ in the GEM prior.
We use $\p(\nu_\k\vert\alpha_0) = \betadist{\nu_\k \vert 1, \alpha_0}$
with $\alpha_0 = 6$ as the initial prior at
which we form the linear approximation.
\figref{gene_initial_coclustering} shows the inferred
co-clustering matrix at $\alpha_0$.


\begin{knitrout}
\definecolor{shadecolor}{rgb}{0.969, 0.969, 0.969}\color{fgcolor}\begin{figure}[!h]

{\centering \includegraphics[width=0.980\linewidth,height=0.392\linewidth]{figure/example_genes-1} 

}

\caption[(Left) An example gene and its expression measured at 14 unique time points
    with three biological replicates at each time point.
     (Right) The cubic B-spline basis with 7 degrees of freedom,
    along with three indicator functions for the last three time points,
    $\timeindx = 72, 120, 168$]{(Left) An example gene and its expression measured at 14 unique time points
    with three biological replicates at each time point.
     (Right) The cubic B-spline basis with 7 degrees of freedom,
    along with three indicator functions for the last three time points,
    $\timeindx = 72, 120, 168$.}\label{fig:example_genes}
\end{figure}


\end{knitrout}
%

We use the linear approximation to extrapolate the co-clustering matrix
under prior parameters
$\alpha = 0.1$ and $\alpha = 12$,
Using \eqref{prior_num_clusters} with $\N = 1000$,
we see that at $\alpha = 0.1$, only two clusters are expected
{\em a priori} under the GEM prior;
at $\alpha = 12$, more than fifty are expected.


Despite this wide prior range, for either $\alpha$, the change in the posterior
co-clustering matrix is minuscule (\figref{gene_alpha_coclustering}): the
largest absolute changes in the co-clustering matrix is of order $10^{-2}$.
Refitting the approximate posterior at $\alpha = 0.1$ and
$\alpha = 12$ confirms the insensitivity predicted by the
linearized variational global parameters. Beyond capturing insensitivity, the
linearized parameters were also able to capture the sign and size of the
changes in the individual entries of the coclustering matrix (these changes
albeit small).


\begin{knitrout}
\definecolor{shadecolor}{rgb}{0.969, 0.969, 0.969}\color{fgcolor}\begin{figure}[!h]

{\centering \includegraphics[width=0.980\linewidth,height=0.784\linewidth]{figure/gene_alpha_coclustering-1} 

}

\caption[Differences in the
     co-clustering matrix at $\alpha = 0.1$ (top row)
     and $\alpha = 12$ (bottom row),
     relative to the co-clustering matrix at $\alpha_0 = 6$.
     (Left) A scatter plot of differences under the linear approximation
     against differences after refitting, where
     each point represents an entry of the co-coclustering matrix.
     Note the scales of the axes:
     the largest change in an entry of the co-clustering matrix is
     $\approx 0.03$.
     (Middle) Sign changes in the co-clustering matrix observed after refitting,
     ignoring the magnitude of the change.
     (Right) Sign changes under the linearly approximated variational
     parameters.
     For visualization, changes with absolute value $< 1e-5$ are not colored]{Differences in the
     co-clustering matrix at $\alpha = 0.1$ (top row)
     and $\alpha = 12$ (bottom row),
     relative to the co-clustering matrix at $\alpha_0 = 6$.
     (Left) A scatter plot of differences under the linear approximation
     against differences after refitting, where
     each point represents an entry of the co-coclustering matrix.
     Note the scales of the axes:
     the largest change in an entry of the co-clustering matrix is
     $\approx 0.03$.
     (Middle) Sign changes in the co-clustering matrix observed after refitting,
     ignoring the magnitude of the change.
     (Right) Sign changes under the linearly approximated variational
     parameters.
     For visualization, changes with absolute value $< 1e-5$ are not colored. }\label{fig:gene_alpha_coclustering}
\end{figure}


\end{knitrout}

\subsubsection*{Functional sensitivity}


\begin{knitrout}
\definecolor{shadecolor}{rgb}{0.969, 0.969, 0.969}\color{fgcolor}\begin{figure}[!h]

{\centering \includegraphics[width=0.980\linewidth,height=0.823\linewidth]{figure/gene_fpert_coclustering-1} 

}

\caption[Effect on the co-clustering matrix after a multiplicative functional
     perturbation.
     (Top left) the perturbation $\phi$ in grey,
     and the influence function in purple.
     (Top right) the effect of this perturbation on the prior density.
     (Bottom) the effect of this perturbation on
    the coclustering matrix.
    Note the scale of the scatter plot axes compared with
    the scatter plots in \figref{gene_alpha_coclustering}]{Effect on the co-clustering matrix after a multiplicative functional
     perturbation.
     (Top left) the perturbation $\phi$ in grey,
     and the influence function in purple.
     (Top right) the effect of this perturbation on the prior density.
     (Bottom) the effect of this perturbation on
    the coclustering matrix.
    Note the scale of the scatter plot axes compared with
    the scatter plots in \figref{gene_alpha_coclustering}. }\label{fig:gene_fpert_coclustering}
\end{figure}


\end{knitrout}


\begin{knitrout}
\definecolor{shadecolor}{rgb}{0.969, 0.969, 0.969}\color{fgcolor}\begin{figure}[!h]

{\centering \includegraphics[width=0.882\linewidth,height=0.423\linewidth]{figure/alpha_pert_logphi-1} 

}

\caption[The multiplicative perturbations $\phi_\alpha(\cdot)$ that
    corresponds to decreasing (left) or increasing (right)
    the $\alpha$ parameter]{The multiplicative perturbations $\phi_\alpha(\cdot)$ that
    corresponds to decreasing (left) or increasing (right)
    the $\alpha$ parameter. }\label{fig:alpha_pert_logphi}
\end{figure}


\end{knitrout}


As in \secref{results_iris}, we now investigate whether the co-clustering matrix
is robust to deviations from the Beta prior.
In \figref{gene_fpert_coclustering}, we use the influence function for
$\laplacianevsum$ to construct a nonparametric prior perturbation which we
expect to have a large, positive effect.  The resulting prior does indeed
produce changes an order of magnitude larger than those produced by $\alpha$
perturbations shown in \figref{gene_alpha_coclustering}, and the approximation
is again able to capture the qualitative changes.
The influence function is also able to
explain why $\alpha$ perturbations were unable to produce large changes in this
case: \figref{alpha_pert_logphi} shows that changing $\alpha$ (as in
\exref{beta_inf_norm}) induces large changes in the prior only where the
influence function is small.

However, even with the (unreasonable-looking) selected functional perturbation,
the size of the differences in the co-clustering matrix remains modest. It is
unlikely that any scientific conclusions derived from the co-clustering matrix would have
changed after the functional perturbation. The co-clustering matrix appears
robust to perturbations in the stick-breaking distribution.
