As we saw in \corref{etafun_deriv_form}, the derivative of perturbations given
by \defref{prior_nl_pert} takes the form of an integral of the influence
function against the perturbation.  It is natural to use the influence function
to {\em explore} the space of priors, e.g., to find alternative priors with
large influence but small $\norminf{\phi}$.  Consider as an example the
following corollary is an example, which is the VB analogue of \citet[Result
11]{gustafson:1996:local}.

% , where the integrand is known as the ``influence function.'' In
% addition to providing an interpretable summary of the effect of different
% perturbations, the influence function motivates the consideration of
% ``worst-case'' prior perturbations which maximize the influence subject to a
% size constraint.  To justify this using the influence function in this way, we
% prove that the VB optimum is in fact Fr{\'e}chet differentiable as a function of
% the perturbation.

%%%%%%%%%%%%%%%%%%%%%%%%%%%%%%%%%%%%%%%%%%%%%%%%%%%%%%%%%%%%%%%%%%%%%%%%%
%%%%%%%%%%%%%%%%%%%%%%%%%%%%%%%%%%%%%%%%%%%%%%%%%%%%%%%%%%%%%%%%%%%%%%%%%

\begin{cor}\corlabel{etafun_worst_case}
%
The ``worst-case'' derivative in $\ball_\phi(\delta)$ is given by
%
\begin{align*}
%
\sup_{\phi \in \ball_\phi(\delta)}
    \fracat{d g(\etaopt(\t \phi))}{d \t}{0} =
        \delta \int \abs{\infl(\theta)} \mu(d\theta),
%
\end{align*}
%
which is acheived at the perturbation
$\phi^*(\theta) = \delta \, \mathrm{sign}\left(\infl(\theta)\right)$.
%
\begin{proof}
%
The result follows immediately from applying H{\"o}lder's inequality
(\citet[Theorem 5.1.2]{dudley:2018:real} and subsequent disscussion)
to \eqref{vb_eta_infl_sens}.
%
\end{proof}
%
\end{cor}

%%%%%%%%%%%%%%%%%%%%%%%%%%%%%%%%%%%%%%%%%%%%%%%%%%%%%%%%%%%%%%%%%%%%%%%%%

Even if one is not interested formally in the worst-case, one can still
use the influence function as a low-dimensaional summary of
the infinite dimensional space of prior densities, as illustrated by
the following example.

However, \corref{etafun_deriv_form} does not precisely justify using
the influence function as in \corref{etafun_worst_case}.


In turn, \corref{etafun_worst_case} motivates the question of whether
\eqref{vb_eta_infl_sens} provides a uniformly good linear approximation to
$\etaopt(\t)$ in a neighborhood of $0$: that is, whether $\phi \mapsto
\etaopt(\phi)$ is Fr{\'e}chet differentiable. By \citep[Theorem
5.2.1]{dudley:2018:real}, $L_\infty$ is a Banach space.

%%%%%%%%%%%%%%%%%%%%%%%%%%%%%%%%%%%%%%%%%%%%%%%%%%%%%%%%%%%%%%%%%%%%%%%%%%%
%%%%%%%%%%%%%%%%%%%%%%%%%%%%%%%%%%%%%%%%%%%%%%%%%%%%%%%%%%%%%%%%%%%%%%%%%%%
\begin{defn}\deflabel{diffable_classes}
    (\citep[Definition 4.5]{zeidler:2013:functional})
%
Let $B_1$ and $B_2$ denote Banach spaces, and let $\ball_1 \subseteq B_1$ define
an open neighborhood of $\phi_0 \in B_1$.  Fix a function $f: \ball_1
\mapsto B_2$.
%
A function $f$ is {\em Fr{\'echet} differentiable} (also known as boundedly
differentiable) at $\phi_0$ if
%
\begin{align*}
%
\lim_{t \rightarrow 0}
    \sup_{\phi: \norm{\phi - \phi_0} = 1}
    \frac{f(\phi) - f(\phi_0) -
          f^{\mathrm{lin}}(t (\phi - \phi_0))
         }{t} \rightarrow 0.
%
\end{align*}
%
\end{defn}
%%%%%%%%%%%%%%%%%%%%%%%%%%%%%%%%%%%%%%%%%%%%%%%%%%%%%%%%%%%%%%%%%%%%%%%%%%%

%%%%%%%%%%%%%%%%%%%%%%%%%%%%%%%%%%%%%%%%%%%%%%%%%%%%%%%%%%%%%%%%%%%%%%%%%%%%
%%%%%%%%%%%%%%%%%%%%%%%%%%%%%%%%%%%%%%%%%%%%%%%%%%%%%%%%%%%%%%%%%%%%%%%%%%%%
\begin{thm}\thmlabel{eta_phi_deriv}
%
Let \assuref{kl_opt_ok, exchange_order_q} hold. Then the map $\phi \mapsto
\etaopt(\phi)$ is well-defined and continuously Fr{\'e}chet differentiable in a
neighborhood of $\phiz$ as a map from $\lp{\mu,\infty}$ to $\mathbb{R}^\etadim$,
with the derivative given in \corref{etafun_deriv_form}.

(For a proof, see \appref{proofs} \proofref{eta_phi_deriv}.)

\end{thm}
%%%%%%%%%%%%%%%%%%%%%%%%%%%%%%%%%%%%%%%%%%%%%%%%%%%%%%%%%%%%%%%%%%%%%%%%%%%%
