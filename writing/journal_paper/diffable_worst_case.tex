As we saw in \corref{etafun_deriv_form}, the derivative of perturbations given
by \defref{prior_nl_pert} takes the form of an integral of the influence
function against the perturbation.  It is natural to use the influence function
to {\em explore} the space of priors, e.g., to find alternative priors with
large influence but small $\norminf{\phi}$.  Consider as an example the
following corollary, which is the VB analogue of \citet[Result
11]{gustafson:1996:local}.

%%%%%%%%%%%%%%%%%%%%%%%%%%%%%%%%%%%%%%%%%%%%%%%%%%%%%%%%%%%%%%%%%%%%%%%%%
%%%%%%%%%%%%%%%%%%%%%%%%%%%%%%%%%%%%%%%%%%%%%%%%%%%%%%%%%%%%%%%%%%%%%%%%%

\begin{cor}\corlabel{etafun_worst_case}
%
The ``worst-case'' derivative in $\ball_\phi(\delta)$ is given by
%
\begin{align*}
%
\sup_{\phi \in \ball_\phi(\delta)}
    \fracat{d g(\etaopt(\t \phi))}{d \t}{0} =
        \delta \int \abs{\infl(\theta)} \mu(d\theta),
%
\end{align*}
%
which is acheived at the perturbation
$\phi^*(\theta) = \delta \, \mathrm{sign}\left(\infl(\theta)\right)$.
%
\begin{proof}
%
The result follows immediately from applying H{\"o}lder's inequality
(\citet[Theorem 5.1.2]{dudley:2018:real} and subsequent disscussion)
to \eqref{vb_eta_infl_sens}.
%
\end{proof}
%
\end{cor}

%%%%%%%%%%%%%%%%%%%%%%%%%%%%%%%%%%%%%%%%%%%%%%%%%%%%%%%%%%%%%%%%%%%%%%%%%

Even if one is not interested formally in the worst-case, one can still use the
influence function to informally choose influential prior perturbations, perhaps
after dimension reduction, as in \exref{infl_univariate}.  Again, see
\secref{results} below for motivating examples.

However, \corref{etafun_deriv_form} does not precisely justify using the
influence function for exploration in this way.  \Corref{etafun_deriv_form}
states only that, for a {\em particular} direction $\phi$, $\t \mapsto
\etaopt(\t \phi)$ is continuously differentiable.  Observing that $\t \phi \in
\ball_\phi(\t \norminf{\phi})$, this guarantees only that, for a fixed $\phi$,
one can make $\t$ sufficiently small so that the error $\abs{\etaopt(\t \phi) -
\etalin(\t \phi)}$ goes to zero faster than $\t$. However, this is not to say
that, for a fixed $\delta$ (no matter how small), the worst-case error
$\sup_{\phi \in \ball_\phi(\delta)} \abs{\etaopt(\phi) - \etalin(\phi)}$ is
bounded, much less that it goes to zero faster than $\delta$.

To justify using linear approximations to exploring the unit ball
$\ball_\phi(\delta)$, we thus require a stronger result than
\corref{etafun_deriv_form}.  Observe that $\phi$ is a member of the the Banach
space $L_\infty$ \citep[Theorem 5.2.1]{dudley:2018:real}.  We require that the
map $\phi \mapsto \etaopt(\phi)$, which maps $L_\infty$ to
$\mathbb{R}^\etadim$, admits a {\em uniformly good linear approximation}, i.e.,
to be Fr{\'e}chet differentiable, as described in \defref{diffable_classes}.

%%%%%%%%%%%%%%%%%%%%%%%%%%%%%%%%%%%%%%%%%%%%%%%%%%%%%%%%%%%%%%%%%%%%%%%%%%%
%%%%%%%%%%%%%%%%%%%%%%%%%%%%%%%%%%%%%%%%%%%%%%%%%%%%%%%%%%%%%%%%%%%%%%%%%%%
\begin{defn}\deflabel{diffable_classes}
    (Fr{\'e}chet differentiability,
    \citep[Definition 4.5]{zeidler:2013:functional})
%
Let $B_1$ and $B_2$ denote Banach spaces, and let $\ball_1 \subseteq B_1$ define
an open neighborhood of $\phi_0 \in B_1$.
%
A function $f: \ball_1 \mapsto B_2$ is {\em Fr{\'echet} differentiable} (also
known as boundedly differentiable) at $\phi_0$ if there exists a  bounded linear
operator, $f^{\mathrm{lin}}: B_1 \mapsto B_2$, such that
%
\begin{align*}
%
\lim_{t \rightarrow 0}
    \sup_{\phi: \norm{\phi - \phi_0} = 1}
    \frac{f(\phi) - f(\phi_0) -
          f^{\mathrm{lin}}(t (\phi - \phi_0))
         }{t} \rightarrow 0.
%
\end{align*}
%
\end{defn}
%%%%%%%%%%%%%%%%%%%%%%%%%%%%%%%%%%%%%%%%%%%%%%%%%%%%%%%%%%%%%%%%%%%%%%%%%%%

By \citep[Proposition 4.8]{zeidler:2013:functional}, if a function is
Fr{\'e}chet differentiable, then the linear operator $f^{\mathrm{lin}}$ is given
precisely by the directional derivative $d f(t (\phi - \phi_0)) / d t$. Thus, if
$\phi \mapsto \etaopt(\phi)$ is Fr{\'e}chet differentiable, its derivative is
given by \corref{etafun_deriv_form}.  Fr{\'e}chet differentiability guarantees
that the error of the linear approximation given by \corref{etafun_deriv_form}
does not blow up in the ball $\ball_\phi(\delta)$.

We emphasize that Fr{\'e}chet differentiability is neither sufficient nor
necessary for a derivative to be useful.  For example, it is possible in
principle for a function to be Fr{\'e}chet differentiable but still have a very
large finite second derivative, and so fail to extrapolate meaningfully to any
alternatives one cares about.  Conversely, if a function fails to be Fr{\'e}chet
differentiable, the derivative may still perform well in particular directions,
including that chosen by \corref{etafun_worst_case}.  Nevertheless, Fr{\'e}chet
differentiability is a strong local result, and provides some assurance that one
can use results such as \corref{etafun_worst_case} without uncovering
pathological behavior.

As we now state in \thmref{eta_phi_deriv}, the perturbation given in
\defref{prior_nl_pert} induces a Fr{\'e}chet differentiable map.

%%%%%%%%%%%%%%%%%%%%%%%%%%%%%%%%%%%%%%%%%%%%%%%%%%%%%%%%%%%%%%%%%%%%%%%%%%%%
%%%%%%%%%%%%%%%%%%%%%%%%%%%%%%%%%%%%%%%%%%%%%%%%%%%%%%%%%%%%%%%%%%%%%%%%%%%%
\begin{thm}\thmlabel{eta_phi_deriv}
%
Let \assuref{kl_opt_ok, exchange_order_q} hold. Then the map $\phi \mapsto
\etaopt(\phi)$ is well-defined and continuously Fr{\'e}chet differentiable in a
neighborhood of $\phiz$ as a map from $\lp{\mu,\infty}$ to $\mathbb{R}^\etadim$,
with the derivative given in \corref{etafun_deriv_form}.

(For a proof, see \appref{proofs} \proofref{eta_phi_deriv}.)

\end{thm}
%%%%%%%%%%%%%%%%%%%%%%%%%%%%%%%%%%%%%%%%%%%%%%%%%%%%%%%%%%%%%%%%%%%%%%%%%%%%

However, the next section will show that our concern is not idle: for the other
classes of non-multiplicative perturbations considered by
\citet{gustafson:1996:local}, no such result will be possible.
