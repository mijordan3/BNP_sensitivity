We observe that the present work is essentially a
VB extension of the local Bayesian robustness literature, which was
traditionally based on MCMC
\citep{gustafson:1996:local, basu:1996:local}.
Work specific to robustness in Dirichlet process mixture models can be found in
\citet{Basu:2000:robustnessBNP, barajas:2009:densitysens}; and \citet{saha:2019:geometricsens},
with the latter two focusing on density estimation problems.

To the best of our knowledge, the present work is the first to apply the
correspondence of MCMC covariances and VB sensitivity derived in
\citep{giordano:2018:covariances} to study the sensitivity of VB approximations to BNP.
In a sense, VB is more naturally
amenable to sensitivity analysis than MCMC, since the derivative of VB
approximations is typically available in closed form, whereas the derivative of
Bayesian posteriors must be approximated using potentially noisy posterior
covariances and/ or posterior conditional densities that are not readily
available from MCMC draws (e.g., \citet{gustafson:1996:marginal}).

Unlike previous work on robustness, we pay
special attention to the ability of using local sensitivity measures
to accurately and usefully \textit{extrapolate}
VB posterior inferences to different priors, rather than simply considering
large derivatives to be indicative of non-robustness \textit{per se}.

One posterior quantity of interest in our experiments below will be
the expected number of clusters under the posterior distribution,
and we will evaluate the sensitivity of the expected number of clusters
to the DP prior.
That the number of clusters is unknown in many applications
is a primary motivator for the use of BNP models, which does not
require an \textit{a priori} specification of the number of clusters.
However, as \citet{miller:2013:neurips, miller:14:inconsistency} prove,
the posterior number of clusters in a BNP model is,
in general, an inconsistent estimate for the true number of clusters if the
true data generating process is a finite mixture model.
This inconsistency is not unique to a BNP approach either, as
\citet{cai:2020:finite, cai:2020:power} show
a similar inconsistency for a finite mixture model with a prior on the number
of clusters.

As \citet{miller:14:inconsistency} observes empirically,
this inconsistency appears to be due to the fact
that in a BNP model, there are typically a few small, rare clusters,
corresponding to the stick-breaking proportions late in the process.
In our results, we see this phenomenon particularly pronounced in
the population genetics model of \secref{results_structure}
(see also \appref{app_structure}),
where small, rare clusters correspond to outlying genetic loci.
We will see that the expected number of clusters appear non-robust to prior
perturbations, and it is the small, rare clusters that drive
the change.
\todo{I removed the figure that shows this. Need to add it back in the appendix. }

In a way, our sensitivity analysis on the expected number of clusters can be viewed as
orthogonal to the proven inconsistency results for BNP models: inconsistency says that the
posterior number of clusters may be unreliable even as the number of data points
tend towards infinity; we show here that with a fixed data set, the number of clusters
may be unreliable due to the subjective nature of the prior specification.

In our experiments below, we are able to construct robust posterior quantities
by counting only the clusters with significant probability mass,
an approach used in practice in other data applications such as \citet{fox:2007:dptracking}.
We take care to construct posterior quantities that are of scientific interest,
and in many cases, such quantities involve only the
largest clusters and can be shown to be robust.
