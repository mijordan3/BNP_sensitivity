
A standard consequence of the dominated convergence theorem is the ability to
exchange integration and differentiation.  Since we will use this result
frequently, we state it here in our own notation as \thmref{dct}.

%%%%%%%%%%%%%%%%%%%%%%%%%%%%%%%%%%%%%%%%%%%%%%%%%%%%%%%%%%
%%%%%%%%%%%%%%%%%%%%%%%%%%%%%%%%%%%%%%%%%%%%%%%%%%%%%%%%%%
\begin{thm}\thmlabel{dct}
\citep[Theorem 16.8]{billingsley:1986:probability}
%
Let $\mu$ be sigma-finite measure on $\thetadom$, and let $S_\t \subseteq
\mathbb{R}$.  Let $f:\thetadom \times S_\t \mapsto \mathbb{R}$.

If there exists a function $M(\theta)$ with $\int M(\theta) \mu(d\theta) <
\infty$ such that $\abs{f(\theta, \t)} \le M(\theta)$, $\mu$-almost surely,
for all $\t \in S_\t$, then the map $\t \mapsto \int f(\theta, \t)
\mu(d\theta)$ is continuous.

Further, suppose that the derivative $\fracat{\partial f(\theta, \t)}{\partial
\t}{\t}$ exist $\mu$-almost surely for $\t \in S_\t$.  If there exists
an $M'(\theta)$ such that $\int M'(\theta) \mu(d\theta) < \infty$ and
$\abs{\fracat{\partial f(\theta, \t)}{\partial \t}{\t}} \le M'(\theta)$,
$\mu$-almost surely and for all $\t \in S_\t$, then
%
\begin{align*}
%
\fracat{\partial \int f(\theta, \t) \mu(d\theta)}{\partial \t}{\t} =
     \int \fracat{\partial f(\theta, \t)}{\partial \t}{\t} \mu(d\theta).
%
\end{align*}
%
\end{thm}
%%%%%%%%%%%%%%%%%%%%%%%%%%%%%%%%%%%%%%%%%%%%%%%%%%%%%%%%%%


%%%%%%%%%%%%%%%%%%%%%%%%%%%%%%%%%%%%%%%%%%%%%%%%%%%%%%%%%%%%%%%%%%%%%%%%%%%%
%%%%%%%%%%%%%%%%%%%%%%%%%%%%%%%%%%%%%%%%%%%%%%%%%%%%%%%%%%%%%%%%%%%%%%%%%%%%
%
\proofof{\lemref{exchange_order}}\prooflabel{exchange_order}

Let $\eta_d$ denote the $d-$th entry of the vector $\eta$.  Then
%
\begin{align*}
%
\abs{\partial f(\theta, \eta, \t) / \partial \eta_d} \le{}&
  \norm{\partial f(\theta, \eta, \t) / \partial \eta}_2
\textrm{,}\\
\abs{\partial^2 f(\theta, \eta, \t) / \partial \eta_d \partial \t} \le{}&
    \norm{\partial f(\theta, \eta, \t) / \partial \eta \partial \t}_2
\textrm{, and}\\
\abs{\partial^2 f(\theta, \eta, \t) /
       \partial \eta_{d_1} \partial \eta_{d_2}} \le{}&
     \norm{\partial f(\theta, \eta, \t) / \partial \eta \partial\eta^T}_2.
%
\end{align*}
%
The conclusion follows by repeatedly applying \thmref{dct} to the components
of the derivatives.
%
%%%%%%%%%%%%%%%%%%%%%%%%%%%%%%%%%%%%%%%%%%%%%%%%%%%%%%%%%%%%%%%%%%%%%%%%%%%%



%%%%%%%%%%%%%%%%%%%%%%%%%%%%%%%%%%%%%%%%%%%%%%%%%%%%%%%%%%%%%%%%%%%%%%%%%%%%
%%%%%%%%%%%%%%%%%%%%%%%%%%%%%%%%%%%%%%%%%%%%%%%%%%%%%%%%%%%%%%%%%%%%%%%%%%%%

\begin{lem}\lemlabel{logq_continuous}

Under \assuref{exchange_order}, the map $\eta, \t \mapsto \expect{\q(\theta
\vert \eta)}{\ptil(\theta \vert \t)}$ has continuous partial derivatives
$\partial / \partial \eta$, $\partial^2 / \partial \eta^2$, and $\partial^2 /
\partial \eta \partial \t$ at all $\eta, \t \in \ball_\eta \times \ball_\t$.
Furthermore,

\begin{align}
\fracat{\partial \expect{\q(\theta \vert \eta)}
              {\ptil(\theta \vert \t)}}{\partial \eta}{\eta}
={}&
\expect{\q(\theta \vert \eta)}
       {\lqgradbar{\theta \vert \eta}
       \ptil(\theta \vert \t)}
       \eqlabel{q_sens_is_cov}\\
%
\fracat{\partial^2 \expect{\q(\theta \vert \eta)}
      {\ptil(\theta \vert \t)}}{\partial \eta \partial \t}{\eta, \t}
={}&
\expect{\q(\theta \vert \eta)}
       {\lqgradbar{\theta \vert \eta}
       \fracat{\partial \ptil(\theta \vert \t)}{\partial \t}{\t}}
\eqlabel{q_sens_psi_grad_is_cov}.
%
\end{align}
%
\begin{proof}
%
We can write
%
\begin{align*}
%
R(a, b) :={} \frac{a}{b} \quad \Rightarrow
\expect{\q(\theta \vert \eta)}{\psi(\theta, \t)} ={}
R\left(\int \qtil(\theta \vert \eta) \psi(\theta, \t) \mu(d\theta),
  \int \qtil(\theta \vert \eta) \mu(d\theta)\right).
%
\end{align*}
%
If necessary, we can shrink $\ball_\eta$ so that the denominator $\int
\qtil(\theta \vert \eta) \mu(d\theta)$ is bounded below by a positive constant
for all $\eta \in \ball_\eta$.  With the denominator strongly positive, $R(a,b)$
is is a continuously differentiable function to all orders for all $\t, \eta \in
\ball_\t \times \ball_\eta$.  The desired results follow from
\assuref{exchange_order} by the chain rule.

\end{proof}
%
\end{lem}
%%%%%%%%%%%%%%%%%%%%%%%%%%%%%%%%%%%%%%%%%%%%%%%%%%%%%%%%%%%%%%%%%%%%%%%%%%%%
