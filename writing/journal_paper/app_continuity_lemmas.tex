
A standard consequence of the dominated convergence theorem is the ability to
exchange integration and differentiation.  Since we will use this result
frequently, we state it here in our own notation as \thmref{dct}.

%%%%%%%%%%%%%%%%%%%%%%%%%%%%%%%%%%%%%%%%%%%%%%%%%%%%%%%%%%
%%%%%%%%%%%%%%%%%%%%%%%%%%%%%%%%%%%%%%%%%%%%%%%%%%%%%%%%%%
\begin{thm}\thmlabel{dct}
\citep[Theorem 16.8]{billingsley:1986:probability}
%
Let $\mu$ be sigma-finite measure on $\thetadom$, and let $S_\t \subseteq
\mathbb{R}$.  Let $f:\thetadom \times S_\t \mapsto \mathbb{R}$.

If there exists a function $M(\theta)$ with $\int M(\theta) \mu(d\theta) <
\infty$ such that $\abs{f(\theta, \t)} \le M(\theta)$, $\mu$-almost surely,
for all $\t \in S_\t$, then the map $\t \mapsto \int f(\theta, \t)
\mu(d\theta)$ is continuous.

Further, suppose that the derivative $\fracat{\partial f(\theta, \t)}{\partial
\t}{\t}$ exist $\mu$-almost surely for $\t \in S_\t$.  If there exists
an $M'(\theta)$ such that $\int M'(\theta) \mu(d\theta) < \infty$ and
$\abs{\fracat{\partial f(\theta, \t)}{\partial \t}{\t}} \le M'(\theta)$,
$\mu$-almost surely and for all $\t \in S_\t$, then
%
\begin{align*}
%
\fracat{\partial \int f(\theta, \t) \mu(d\theta)}{\partial \t}{\t} =
     \int \fracat{\partial f(\theta, \t)}{\partial \t}{\t} \mu(d\theta).
%
\end{align*}
%
\end{thm}
%%%%%%%%%%%%%%%%%%%%%%%%%%%%%%%%%%%%%%%%%%%%%%%%%%%%%%%%%%


%%%%%%%%%%%%%%%%%%%%%%%%%%%%%%%%%%%%%%%%%%%%%%%%%%%%%%%%%%%%%%%%%%%%%%%%%%%%
%%%%%%%%%%%%%%%%%%%%%%%%%%%%%%%%%%%%%%%%%%%%%%%%%%%%%%%%%%%%%%%%%%%%%%%%%%%%
%
\proofof{\lemref{exchange_order}}\prooflabel{exchange_order}

Let $\eta_d$ denote the $d-$th entry of the vector $\eta$.  Then
$\abs{\partial f(\theta, \eta, \t) / \partial \eta_d} \le
  \norm{\partial f(\theta, \eta, \t) / \partial \eta}_2$
,
$\abs{\partial^2 f(\theta, \eta, \t) / \partial \eta_d \partial \t} \le
    \norm{\partial f(\theta, \eta, \t) / \partial \eta \partial \t}_2$,
and
$\abs{\partial^2 f(\theta, \eta, \t) /
       \partial \eta_{d_1} \partial \eta_{d_2}} \le
     \norm{\partial f(\theta, \eta, \t) / \partial \eta \partial\eta^T}_2$.
%
The conclusion follows by repeatedly applying \thmref{dct} to the components
of the derivatives.
%
%%%%%%%%%%%%%%%%%%%%%%%%%%%%%%%%%%%%%%%%%%%%%%%%%%%%%%%%%%%%%%%%%%%%%%%%%%%%



%%%%%%%%%%%%%%%%%%%%%%%%%%%%%%%%%%%%%%%%%%%%%%%%%%%%%%%%%%%%%%%%%%%%%%%%%%%%
%%%%%%%%%%%%%%%%%%%%%%%%%%%%%%%%%%%%%%%%%%%%%%%%%%%%%%%%%%%%%%%%%%%%%%%%%%%%
\begin{defn}
%
Throughout the appendix, use the following notation.
%
\begin{align*}
%
\lqgrad{\theta \vert \eta} :={}&
    \fracat{\partial \log \qtil(\theta \vert \eta)}{\partial \eta}{\eta} \\
%
\lqhess{\theta \vert \eta} :={}&
    \fracat{\partial^2 \log \qtil(\theta \vert \eta)}
           {\partial \eta \partial \eta^T}{\eta} \\
%
\psi(\theta, \t) :={}&
    \log \ptil(\theta \vert \t) - \log \ptil(\theta \vert \t=0)\\
%
\psigrad{\theta, \t} :={}&
    \fracat{\partial \psi(\theta, \t)}{\partial \t}{\t} \\
%
\lqgradbar{\theta \vert \eta} :={}& \lqgrad{\theta \vert \eta}
  - \expect{\q(\theta \vert \eta)}{\lqgrad{\theta \vert \eta}} \\
%
\lqhessbar{\theta \vert \eta} :={}& \lqhess{\theta \vert \eta}
 - \expect{\q(\theta \vert \eta)}{\lqhess{\theta \vert \eta}}.
%
\end{align*}
%
\end{defn}
%%%%%%%%%%%%%%%%%%%%%%%%%%%%%%%%%%%%%%%%%%%%%%%%%%%%%%%%%%%%%%%%%%%%%%%%%%%%

%
% %%%%%%%%%%%%%%%%%%%%%%%%%%%%%%%%%%%%%%%%%%%%%%%%%%%%%%%%%%%%%%%%%%%%%%%%%%%%
% %%%%%%%%%%%%%%%%%%%%%%%%%%%%%%%%%%%%%%%%%%%%%%%%%%%%%%%%%%%%%%%%%%%%%%%%%%%%
% \begin{assu}\assulabel{dist_fun_nice}
% %
% \todo{What from here needs to be kept?}
%
%
% Assume that the map $\eta \mapsto \log \qtil(\theta \vert \eta)$ is twice
% continuously differentiable. Let $\psi(\theta, \t)$ be a scalar-valued
% $\mu$-measurable function of $\theta$ and $\t$.  Assume that the map $\t \mapsto
% \psi(\theta, \t)$ is continuously differentiable.
% %
% For a given $\t_0$ and $\eta_0$, assume there exists some neighborhood of
% $\t_0$, $\ball_\t$, some neighborhood of $\eta_0$, $\ball_\eta$, and a
% $\mu$-integrable $M_\psi(\theta)$ with $\int M_\psi(\theta) \mu(d\theta) <
% \infty$ such that the following bounds hold for all $\eta, \t \in \ball_\eta
% \times \ball_\t$:
% %
% \begin{enumerate}
% %
% \item \itemlabel{fundom}
% $\qtil(\theta \vert \eta) \psi(\theta, \t) \le M_\psi(\theta)$.
% %
% \item \itemlabel{funqgraddom}
% $\qtil(\theta \vert \eta) \norm{\lqgrad{\theta \vert \eta}}_2 \psi(\theta, \t) \le
% M_\psi(\theta)$.
% %
% \item \itemlabel{funqhessdom}
% $\qtil(\theta \vert \eta) \norm{\lqhess{\theta \vert \eta}}_2 \psi(\theta, \t) \le
% M_\psi(\theta)$.
% %
% \item \itemlabel{fungradqgraddom}
% $\qtil(\theta \vert \eta) \norm{\lqgrad{\theta \vert \eta}}_2 \psigrad{\theta, \t}
% \le M_\psi(\theta)$.
% %
% \item \itemlabel{funqgradsqdom}
% $\qtil(\theta \vert \eta) \norm{\lqgrad{\theta \vert \eta}}^2_2 \psi(\theta, \t) \le
% M_\psi(\theta)$.
% %
% \end{enumerate}
% %
% \end{assu}
% %%%%%%%%%%%%%%%%%%%%%%%%%%%%%%%%%%%%%%%%%%%%%%%%%%%%%%%%%%%%%%%%%%%%%%%%%%%%


%%%%%%%%%%%%%%%%%%%%%%%%%%%%%%%%%%%%%%%%%%%%%%%%%%%%%%%%%%%%%%%%%%%%%%%%%%%%%
%%%%%%%%%%%%%%%%%%%%%%%%%%%%%%%%%%%%%%%%%%%%%%%%%%%%%%%%%%%%%%%%%%%%%%%%%%%%%

\begin{lem}\lemlabel{logq_derivs}
%
Under \assuref{exchange_order}, the following equalties hold:
%
\begin{align}
%
\MoveEqLeft
\fracat{\partial \expect{\q(\theta \vert \eta)}
              {\psi(\theta, \t)}}{\partial \eta}{\eta}
\nonumber\\ \quad={}&
\expect{\q(\theta \vert \eta)}
       {\lqgradbar{\theta \vert \eta} \left(
        \psi(\theta,\t) - \expect{\q(\theta \vert \eta)}{\psi(\theta, \t)}
       \right)
       }\eqlabel{q_sens_is_cov}\\\nonumber\\
%
\MoveEqLeft
\fracat{\partial^2 \expect{\q(\theta \vert \eta)}
      {\psi(\theta, \t)}}{\partial \eta \partial \t}{\eta, \t}
\nonumber\\ \quad={}&
  \expect{\q(\theta \vert \eta)}
         {\lqgradbar{\theta \vert \eta} \left(
          \psigrad{\theta,\t} - \expect{\q(\theta \vert \eta)}{\psigrad{\theta, \t})}
         \right)
         } \eqlabel{q_sens_psi_grad_is_cov} \\\nonumber\\
 %
 \MoveEqLeft
 \fracat{\partial^2 \expect{\q(\theta \vert \eta)}
       {\psi(\theta, \t)}}{\partial \eta \partial \eta^T}{\eta}
\nonumber\\={}&
 \expect{\q(\theta \vert \eta)}
        {\lqgradbar{\theta \vert \eta} \lqgradbar{\theta \vert \eta}^T
        \left(
         \psi(\theta,\t) - \expect{\q(\theta \vert \eta)}{\psi(\theta, \t)}
        \right)
        } +
\nonumber\\ \quad&
 \expect{\q(\theta \vert \eta)}{
        \lqhessbar{\theta \vert \eta}
        \left(
         \psi(\theta,\t) - \expect{\q(\theta \vert \eta)}{\psi(\theta, \t)}
        \right)
        }. \eqlabel{q_score_sens_is_cov}
%
\end{align}
%
\begin{proof}
%
The proof follows by repeatedly using \thmref{dct} to interchange the order of
integration and differentiation as in  \citep[Theorem
1]{giordano:2018:covariances}, from which \eqref{q_sens_is_cov} follows
directly.  \Eqref{q_sens_psi_grad_is_cov} follows by applying
\assuref{exchange_order} to \eqref{q_sens_is_cov}.

To compute \eqref{q_score_sens_is_cov}, we differentiate \eqref{q_sens_is_cov}
with respect to $\eta$, which involves terms of the form $\expect{\q(\theta
\vert \eta)}{\lqgrad{\theta \vert \eta}}$ and $\expect{\q(\theta \vert
\eta)}{\lqgrad{\theta \vert \eta} \psi(\theta, \t)}$.  For both of these terms,
we can exchange the order of the expectation and differentiation by
\assuref{exchange_order}.  \Eqref{q_score_sens_is_cov} then follows by
differentiating and collecting terms.
%
\end{proof}
%
\end{lem}
%%%%%%%%%%%%%%%%%%%%%%%%%%%%%%%%%%%%%%%%%%%%%%%%%%%%%%%%%%%%%%%%%%%%%%%%%%%%%


%%%%%%%%%%%%%%%%%%%%%%%%%%%%%%%%%%%%%%%%%%%%%%%%%%%%%%%%%%%%%%%%%%%%%%%%%%%%%
%%%%%%%%%%%%%%%%%%%%%%%%%%%%%%%%%%%%%%%%%%%%%%%%%%%%%%%%%%%%%%%%%%%%%%%%%%%%%



%%%%%%%%%%%%%%%%%%%%%%%%%%%%%%%%%%%%%%%%%%%%%%%%%%%%%%%%%%%%%%%%%%%%%%%%%%%%
%%%%%%%%%%%%%%%%%%%%%%%%%%%%%%%%%%%%%%%%%%%%%%%%%%%%%%%%%%%%%%%%%%%%%%%%%%%%

\begin{lem}\lemlabel{logq_continuous}\seeproof{logq_continuous}

Under \assuref{exchange_order}, each of the following functions
are continuous on $\ball_\eta \times \ball_\t$.
%
\begin{align*}
%
\eta, \t \mapsto{}& \fracat{\partial
\expect{\q(\theta \vert \eta)} {\psi(\theta, \t)}}{\partial \eta}{\eta, \t}
%
\mathtxt{,}\\
%
\eta, \t \mapsto{}& \fracat{\partial^2
\expect{\q(\theta \vert \eta)} {\psi(\theta, \t)}}{\partial \eta \partial
\t}{\eta, \t}
%
\mathtxt{, and}\\
%
\eta, \t \mapsto{}&  \fracat{\partial^2
\expect{\q(\theta \vert \eta)} {\psi(\theta, \t)}}{\partial \eta \partial
\eta^T}{\eta}
%
\end{align*}
%
\end{lem}
%%%%%%%%%%%%%%%%%%%%%%%%%%%%%%%%%%%%%%%%%%%%%%%%%%%%%%%%%%%%%%%%%%%%%%%%%%%%

TODO: you need to assume the dominated version, \assuref{exchange_order}
is not enough.

TODO: Prove separately the continuiuty and differentiability of $\eta, \t
\mapsto \int \qtil(\theta \vert \eta) \psi(\theta, \t) \mu(d\theta)$ and of
$\eta \mapsto \int \qtil(\theta \vert \eta) \mu(d\theta)$.


%
\begin{proof}[Proof of \lemref{logq_continuous}]\prooflabel{logq_continuous}
%
Since
%
\begin{align*}
%
\expect{\q(\theta \vert \eta)}{\psi(\theta, \t)} ={}&
\frac{\int \qtil(\theta \vert \eta) \psi(\theta, \t) \mu(d\theta)}
     {\int \qtil(\theta \vert \eta) \mu(d\theta)},
%
\end{align*}
%
and $\int \qtil(\theta \vert \eta) \mu(d\theta) > 0$ for all $\eta$,
by definition, it suffices to prove that
%
\begin{align*}
%
\eta, \t \mapsto{}
    \int \qtil(\theta \vert \eta) \psi(\theta, \t) \mu(d\theta) \mathand
\eta \mapsto{}
    \int \qtil(\theta \vert \eta) \mu(d\theta)
%
\end{align*}
%
are continuously differentiable to the required degree.

Consider first the continuity of $\partial \int \qtil(\theta \vert \eta)
\psi(\theta, \t) \mu(d\theta) / \partial \eta$.  By \assuref{exchange_order},
and \lemref{exchange_order},
%
\begin{align*}
%
\fracat{\partial \int \qtil(\theta \vert \eta) \psi(\theta, \t) \mu(d\theta) }
       {\partial \eta}{\eta, \t} ={}&
\int \fracat{\partial \qtil(\theta \vert \eta) }
      {\partial \eta}{\eta}  \psi(\theta, \t) \mu(d\theta).
%
\end{align*}
%
By DCT ASSU, we can take the limits...
%
\end{proof}
%
%%%%%%%%%%%%%%%%%%%%%%%%%%%%%%%%%%%%%%%%%%%%%%%%%%%%%%%%%%%%%%%%%%%%%%%%%%%%%




The next lemma is a different kind of continuity lemma.

%%%%%%%%%%%%%%%%%%%%%%%%%%%%%%%%%%%%%%%%%%%%%%%%%%%%%%%%%%%%%%%%%%%%%%%%%%%%%
%%%%%%%%%%%%%%%%%%%%%%%%%%%%%%%%%%%%%%%%%%%%%%%%%%%%%%%%%%%%%%%%%%%%%%%%%%%%%

\begin{lem}\lemlabel{continuity_partition}
%
Let $\epsilon'_n = n^{-1}$.  Since $\pbase \ll \mu \ll \lambda$ (where $\lambda$
is the Lebesgue measure), by applying \citet[Proposition
15.5]{nielsen:1997:measure}, for each $n$ there exists a $\delta'_n$ such that,
for any measureable set $A$ with $\mu(A) < \delta'_n$, $\pbase(A) <
\epsilon'_n$.  Again applying \citet[Proposition 15.5]{nielsen:1997:measure},
there similarly exists a $\delta_n$ such that for any measureable set $A$ with
$\lambda(A) < \delta_n$, $\mu(A) < \delta'_n \Rightarrow \pbase(A) <
\epsilon'_n$.

For each $n$, partition $\thetadom$ into a countable number of sets $A_{m}$ such
that $\sum_{m} \lambda(A_{m}) = 1$ and $\lambda(A_{m}) < \delta_n$. (This is
possible by dividing $\thetadom$ into sufficiently small rectangles, for
example.)  Then $\pbase(A_{m}) < \epsilon'_n$ for all $m$.  Since $\pbase$ is a
probability measure, $\sum_m \pbase(A_{m}) = 1$, so there must exist at least $1 /
\epsilon'_n$ indices $m'$ such that $\pbase(A_{m'}) > 0$. Take any such $m'$ and
let $\epsilon_n = \pbase(A_{m'})$ and $S_n = A_{m'}$.

%
\end{lem}
%%%%%%%%%%%%%%%%%%%%%%%%%%%%%%%%%%%%%%%%%%%%%%%%%%%%%%%%%%%%%%%%%%%%%%%%%%%%%
