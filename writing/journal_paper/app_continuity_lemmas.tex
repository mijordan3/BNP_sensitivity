
A standard consequence of the dominated convergence theorem is the ability to
exchange integration and differentiation.  Since we will use this result
frequently, we state it here in our own notation as \thmref{dct}.

%%%%%%%%%%%%%%%%%%%%%%%%%%%%%%%%%%%%%%%%%%%%%%%%%%%%%%%%%%
%%%%%%%%%%%%%%%%%%%%%%%%%%%%%%%%%%%%%%%%%%%%%%%%%%%%%%%%%%
\begin{thm}\thmlabel{dct}
\citep[Theorem 16.8]{billingsley:1986:probability}
%
Let $\mu$ be sigma-finite measure on $\thetadom$, and let $S_\t \subseteq
\mathbb{R}$.  Let $f:\thetadom \times S_\t \mapsto \mathbb{R}$.

If there exists a function $M(\theta)$ with $\int M(\theta) \mu(d\theta) <
\infty$ such that $\abs{f(\theta, \t)} \le M(\theta)$, $\mu$-almost surely,
for all $\t \in S_\t$, then the map $\t \mapsto \int f(\theta, \t)
\mu(d\theta)$ is continuous.

Further, suppose that the derivative $\fracat{\partial f(\theta, \t)}{\partial
\t}{\t}$ exist $\mu$-almost surely for $\t \in S_\t$.  If there exists
an $M'(\theta)$ such that $\int M'(\theta) \mu(d\theta) < \infty$ and
$\abs{\fracat{\partial f(\theta, \t)}{\partial \t}{\t}} \le M'(\theta)$,
$\mu$-almost surely and for all $\t \in S_\t$, then
%
\begin{align*}
%
\fracat{\partial \int f(\theta, \t) \mu(d\theta)}{\partial \t}{\t} =
     \int \fracat{\partial f(\theta, \t)}{\partial \t}{\t} \mu(d\theta).
%
\end{align*}
%
\end{thm}
%%%%%%%%%%%%%%%%%%%%%%%%%%%%%%%%%%%%%%%%%%%%%%%%%%%%%%%%%%


%%%%%%%%%%%%%%%%%%%%%%%%%%%%%%%%%%%%%%%%%%%%%%%%%%%%%%%%%%%%%%%%%%%%%%%%%%%%%
%%%%%%%%%%%%%%%%%%%%%%%%%%%%%%%%%%%%%%%%%%%%%%%%%%%%%%%%%%%%%%%%%%%%%%%%%%%%%

\begin{lem}\lemlabel{logq_derivs}
%
Let \assuref{dist_fun_nice} hold for some $\psi(\theta, \t)$ as well as
for $\psi(\theta, \t) = 1$.  Define
%
\begin{align*}
%
\lqgradbar{\theta \vert \eta} :={}& \lqgrad{\theta \vert \eta}
  - \expect{\q(\theta \vert \eta)}{\lqgrad{\theta \vert \eta}} \\
\lqhessbar{\theta \vert \eta} :={}& \lqhess{\theta \vert \eta}
 - \expect{\q(\theta \vert \eta)}{\lqhess{\theta \vert \eta}}.
%
\end{align*}
%
Then the following equalties hold:
%
\begin{align}
%
\MoveEqLeft
\fracat{\partial \expect{\q(\theta \vert \eta)}
              {\psi(\theta, \t)}}{\partial \eta}{\eta} ={}
\expect{\q(\theta \vert \eta)}
       {\lqgradbar{\theta \vert \eta} \left(
        \psi(\theta,\t) - \expect{\q(\theta \vert \eta)}{\psi(\theta, \t)}
       \right)
       }\eqlabel{q_sens_is_cov}\\\nonumber\\
%
\MoveEqLeft
\fracat{\partial^2 \expect{\q(\theta \vert \eta)}
      {\psi(\theta, \t)}}{\partial \eta \partial \t}{\eta, \t} ={}\nonumber\\&
  \expect{\q(\theta \vert \eta)}
         {\lqgradbar{\theta \vert \eta} \left(
          \psigrad{\theta,\t} - \expect{\q(\theta \vert \eta)}{\psigrad{\theta, \t})}
         \right)
         } \eqlabel{q_sens_psi_grad_is_cov} \\\nonumber\\
 %
 \MoveEqLeft
 \fracat{\partial^2 \expect{\q(\theta \vert \eta)}
       {\psi(\theta, \t)}}{\partial \eta \partial \eta^T}{\eta} ={}
 \nonumber\\&
 \expect{\q(\theta \vert \eta)}
        {\lqgradbar{\theta \vert \eta} \lqgradbar{\theta \vert \eta}^T
        \left(
         \psi(\theta,\t) - \expect{\q(\theta \vert \eta)}{\psi(\theta, \t)}
        \right)
        } +
 \nonumber\\ &
 \expect{\q(\theta \vert \eta)}{
        \lqhessbar{\theta \vert \eta}
        \left(
         \psi(\theta,\t) - \expect{\q(\theta \vert \eta)}{\psi(\theta, \t)}
        \right)
        }. \eqlabel{q_score_sens_is_cov}
%
\end{align}
%
\begin{proof}
%
The proof follows by repeatedly using \thmref{dct} to interchange the order of
integration and differentiation as in \citep[Theorem
1]{giordano:2018:covariances}.  For example,
%
\begin{align*}
%
\MoveEqLeft
\fracat{\partial \int \q(\theta \vert \nu) \psi(\theta \vert \t) \mu(d\theta)}
       {\partial \eta}{\eta}
={}\\&
\int \fracat{\partial \q(\theta \vert \nu) \psi(\theta \vert \t) }
          {\partial \eta}{\eta} \mu(d\theta)={}
&\mathtxt{(\assuitemref{dist_fun_nice}{fundom} and \thmref{dct})}
\\&
\int \lqgrad{\theta, \eta} \psi(\theta \vert \t) \q(\theta, \eta) \mu(d\theta) ={}
\\&
\expect{\q(\theta, \eta)}{\lqgrad{\theta, \eta} \psi(\theta \vert \t) }.
%
\end{align*}
%
Applying analogous reasoning to the denominator of
%
\begin{align*}
%
\expect{\q(\theta \vert \eta)}{\psi(\theta, \t)} =
\frac{\int \psi(\theta, \t) \q(\theta \vert \eta) \mu(d\theta)}
     {\int \q(\theta \vert \eta) \mu(d\theta)}
%
\end{align*}
%
and applying the chain rule gives \eqref{q_sens_is_cov}.

For \eqref{q_sens_psi_grad_is_cov}, by anaologously applying
\assuitemref{dist_fun_nice}{fundom} and the DCT gives
%
\begin{align*}
%
\fracat{\partial \expect{\q(\theta \vert \eta)}{\psi(\theta \vert \eta)}}
       {\partial \t}{\t} ={}&
\expect{\q(\theta \vert \eta)}{\psigrad{\theta \vert \eta}}.
%
\end{align*}
%
Applying \assuitemref{dist_fun_nice}{funqgraddom} and \thmref{dct} gives
%
\begin{align*}
%
\fracat{\partial \expect{\q(\theta \vert \eta)}
                        {\lqgrad{\theta \vert \eta}\psi(\theta \vert \eta)}}
       {\partial \t}{\t} ={}&
\expect{\q(\theta \vert \eta)}
       {\lqgrad{\theta \vert \eta} \psigrad{\theta \vert \eta}},
%
\end{align*}
%
where we have used the fact that the absolute value of any component of the
vector $\lqgrad{\theta \vert \eta}\psi(\theta \vert \eta)$ is bounded above by a
constant times $\norm{\lqgrad{\theta \vert \eta}\psi(\theta \vert \eta)}_2$.
From the preceding two displays, \eqref{q_sens_psi_grad_is_cov} follows.

Finally, for \eqref{q_score_sens_is_cov}, we need to differentiate
\eqref{q_sens_is_cov}.  In addition to quantities already considered
above, \eqref{q_sens_is_cov} involves terms of the following form,
to which we can apply \thmref{dct} using the corresponding assumptions:
%
\begin{align*}
%
\expect{\q{\theta \vert \eta}}
    {\lqgrad{\theta \vert \eta}} &&
    \textrm{\assuitemref{dist_fun_nice}{funqgraddom}}\\
\expect{\q{\theta \vert \eta}}
    {\lqgrad{\theta \vert \eta} \psi(\theta, \t)}. &&
    \textrm{\assuitemref{dist_fun_nice}{funqgraddom}}
%
\end{align*}
%
\Eqref{q_score_sens_is_cov} then follows by differentiating as above and
collecting terms.
%
\end{proof}
%
\end{lem}
%%%%%%%%%%%%%%%%%%%%%%%%%%%%%%%%%%%%%%%%%%%%%%%%%%%%%%%%%%%%%%%%%%%%%%%%%%%%%


%%%%%%%%%%%%%%%%%%%%%%%%%%%%%%%%%%%%%%%%%%%%%%%%%%%%%%%%%%%%%%%%%%%%%%%%%%%%%
%%%%%%%%%%%%%%%%%%%%%%%%%%%%%%%%%%%%%%%%%%%%%%%%%%%%%%%%%%%%%%%%%%%%%%%%%%%%%



%%%%%%%%%%%%%%%%%%%%%%%%%%%%%%%%%%%%%%%%%%%%%%%%%%%%%%%%%%%%%%%%%%%%%%%%%%%%
%%%%%%%%%%%%%%%%%%%%%%%%%%%%%%%%%%%%%%%%%%%%%%%%%%%%%%%%%%%%%%%%%%%%%%%%%%%%

\begin{lem}\lemlabel{logq_continuous}\seeproof{logq_continuous}
%
Let \assuref{dist_fun_nice} hold for some $\psi$ as well as with $\psi(\theta,
\t) = 1$.  Then
%
\begin{align*}
%
\eta, \t \mapsto{}& \fracat{\partial
\expect{\q(\theta \vert \eta)} {\psi(\theta, \t)}}{\partial \eta}{\eta, \t}
%
\mathtxt{,}\\
%
\eta, \t \mapsto{}& \fracat{\partial^2
\expect{\q(\theta \vert \eta)} {\psi(\theta, \t)}}{\partial \eta \partial
\t}{\eta, \t}
%
\mathtxt{, and}\\
%
\eta, \t \mapsto{}&  \fracat{\partial^2
\expect{\q(\theta \vert \eta)} {\psi(\theta, \t)}}{\partial \eta \partial
\eta^T}{\eta}
%
\end{align*}
%
are continuous on $\ball_\eta \times \ball_\t$.
%
\end{lem}
%%%%%%%%%%%%%%%%%%%%%%%%%%%%%%%%%%%%%%%%%%%%%%%%%%%%%%%%%%%%%%%%%%%%%%%%%%%%

%
\begin{proof}[Proof of \lemref{logq_continuous}]\prooflabel{logq_continuous}
%
By \lemref{logq_derivs}, the mixed partial $ \eta, \t \mapsto \fracat{\partial^2
\expect{\q(\theta \vert \eta)} {\psi(\theta, \t)}}{\partial \eta \partial
\t}{\eta, \t}$ is a continuous combination of terms of the form
%
\begin{align*}
%
\expect{\q(\theta \vert \eta)}
       {\lqgrad{\theta \vert \eta} \psigrad{\theta,\t}}
       && \mathtxt{\assuitemref{dist_fun_nice}{fungradqgraddom}} \\
\expect{\q(\theta \vert \eta)}
      {\lqgrad{\theta \vert \eta}}
      && \mathtxt{\assuitemref{dist_fun_nice}{funqgraddom}} \\
\expect{\q(\theta \vert \eta)}
    {\psigrad{\theta,\t}}.
    && \mathtxt{\assuitemref{dist_fun_nice}{funqgraddom}}
%
\end{align*}
%
By the corresponding assumptions, \thmref{dct} applies to each of these terms,
and by \assuref{dist_fun_nice}, each of the expressions in the preceding display
are continuous.  For example,
%
\begin{align*}
%
\MoveEqLeft
\norm{\expect{\q(\theta \vert \eta)}
       {\lqgrad{\theta \vert \eta} \psigrad{\theta,\t}} -
   \expect{\q(\theta \vert \eta')}
          {\lqgrad{\theta \vert \eta'} \psigrad{\theta,\t'}}
      }_2 =\\&
%
\norm{\int \left(
\q(\theta \vert \eta) \lqgrad{\theta \vert \eta} \psigrad{\theta,\t} -
\q(\theta \vert \eta') \lqgrad{\theta \vert \eta'} \psigrad{\theta,\t'}
\right)\mu(d\theta)
}_2  \le\\&
%
\int \norm{
\q(\theta \vert \eta) \lqgrad{\theta \vert \eta} \psigrad{\theta,\t} -
\q(\theta \vert \eta') \lqgrad{\theta \vert \eta'} \psigrad{\theta,\t'}
}_2 \mu(d\theta) \le\\&
%
\int \norm{
\left(\q(\theta \vert \eta) - \q(\theta \vert \eta')\right)
    \lqgrad{\theta \vert \eta} \psigrad{\theta, \t}
}_2 \mu(d\theta) + \\&\quad
%
\int \norm{
\q(\theta \vert \eta')
    \left( \lqgrad{\theta \vert \eta} - \lqgrad{\theta \vert \eta'} \right)
    \psigrad{\theta, \t}
}_2 \mu(d\theta) + \\&\quad
%
\int \norm{
\q(\theta \vert \eta')\lqgrad{\theta \vert \eta'}
    \left( \psigrad{\theta, \t} - \psigrad{\theta, \t'} \right)
}_2 \mu(d\theta).
%
\end{align*}
%
By \assuitemref{dist_fun_nice}{fungradqgraddom} we can apply \thmref{dct} to
each term in the final line of the preceding display, giving
%
\begin{align*}
%
\MoveEqLeft
\lim_{\eta' \rightarrow \eta} \lim_{\t' \rightarrow \t}
\norm{\expect{\q(\theta \vert \eta)}
       {\lqgrad{\theta \vert \eta} \psigrad{\theta,\t}} -
   \expect{\q(\theta \vert \eta')}
          {\lqgrad{\theta \vert \eta'} \psigrad{\theta,\t'}}
      }_2 \le\\&
%
\int \lim_{\eta' \rightarrow \eta} \lim_{\t' \rightarrow \t} \norm{
\left(\q(\theta \vert \eta) - \q(\theta \vert \eta')\right)
    \lqgrad{\theta \vert \eta} \psigrad{\theta, \t}
}_2 \mu(d\theta) + \\&\quad
%
\int \lim_{\eta' \rightarrow \eta} \lim_{\t' \rightarrow \t} \norm{
\q(\theta \vert \eta')
    \left( \lqgrad{\theta \vert \eta} - \lqgrad{\theta \vert \eta'} \right)
    \psigrad{\theta, \t}
}_2 \mu(d\theta) + \\&\quad
%
\int \lim_{\eta' \rightarrow \eta} \lim_{\t' \rightarrow \t} \norm{
\q(\theta \vert \eta')\lqgrad{\theta \vert \eta'}
    \left( \psigrad{\theta, \t} - \psigrad{\theta, \t'} \right)
}_2 \mu(d\theta) = 0,
%
\end{align*}
%
the final equality following from the continuity assumptions of
\assuref{dist_fun_nice}.

Similarly, $\fracat{\partial^2
\expect{\q(\theta \vert \eta)} {\psi(\theta, \t)}}{\partial \eta \partial
\eta^T}{\eta}$ involves terms of the form
%
\begin{align*}
%
\expect{\q(\theta \vert \eta)}
       {\lqgrad{\theta \vert \eta} \lqgrad{\theta \vert \eta}^T
        \psi(\theta,\t)}
       && \mathtxt{\assuitemref{dist_fun_nice}{funqgradsqdom}} \\
       %
\expect{\q(\theta \vert \eta)}
      {\lqgrad{\theta \vert \eta} \lqgrad{\theta \vert \eta}^T}
      && \mathtxt{\assuitemref{dist_fun_nice}{funqgradsqdom}} \\
%
\expect{\q(\theta \vert \eta)}
       {\lqhess{\theta \vert \eta}
        \psi(\theta,\t)}
       && \mathtxt{\assuitemref{dist_fun_nice}{funqhessdom}} \\
%
\expect{\q(\theta \vert \eta)}
       {\lqhess{\theta \vert \eta}},
       && \mathtxt{\assuitemref{dist_fun_nice}{funqhessdom}}
%
\end{align*}
%
to which we can apply \thmref{dct} by the corresponding assumption.  Reasoning
analogously to the other term, the conclusion follows.
%
\end{proof}
%
%%%%%%%%%%%%%%%%%%%%%%%%%%%%%%%%%%%%%%%%%%%%%%%%%%%%%%%%%%%%%%%%%%%%%%%%%%%%%


Next, we prove a result for directional derivatives in $\lp{p}$ spaces.


%%%%%%%%%%%%%%%%%%%%%%%%%%%%%%%%%%%%%%%%%%%%%%%%%%%%%%%%%%%%%%%%%%%%%%%%%%%%%
%%%%%%%%%%%%%%%%%%%%%%%%%%%%%%%%%%%%%%%%%%%%%%%%%%%%%%%%%%%%%%%%%%%%%%%%%%%%%
\begin{lem}\lemlabel{lp_integral_bound}
%
Under \assuref{lp_regular}, if $\norm{\gamma}_{\pbase, p} < \infty$
and $\inf_\theta \gamma(\theta) > -\infty$, then there
exists an $M(\theta)$ with $\int M(\theta) \mu(d\theta) < \infty$
such that, for all $0 \le \t \le (2 \abs{\inf_\theta \gamma(\theta)})^{-1}$,
%
\begin{align}
%
\abs{\psi(\theta)}\abs{\log\left(1 + \t \gamma(\theta)\right)}\q(\theta)
    <{}& M(\theta) \mathand \eqlabel{lp_integral_bound}\\
%
\abs{\psi(\theta)}\abs{\frac{\t \gamma(\theta)}{1 + \t \gamma(\theta)}}\q(\theta)
    <{}& M(\theta). \eqlabel{lp_integral_deriv_bound}
%
\end{align}
%
\begin{proof}

Let $\t_{max} := (2 \abs{\inf_\theta \gamma(\theta)})^{-1}$.  Then
%
\begin{align*}
%
1 + \t \gamma(\theta) \ge{}& 1 - \t \abs{\inf_\theta \gamma(\theta)}
\\={}
    1 - \frac{\t}{2 \t_{max}}.
%
\end{align*}
%
So $\t \le \t_{max} \Rightarrow 1 + \t \gamma(\theta) \ge 1/2$ for all $\theta$.
Expanding the integral,
%
\begin{align*}
%
\MoveEqLeft
\int \abs{\psi(\theta)}
    \abs{\log\left(1 + \t \gamma(\theta)\right)}\q(\theta)
    \mu(d\theta) ={}
\\&
\int \abs{\psi(\theta)}
    \abs{\log\left(1 + \t \gamma(\theta)\right)}
    \ind{\gamma(\theta) \le 0}
    \q(\theta)\mu(d\theta)  +
\\&
\int \abs{\psi(\theta)}
    \abs{\log\left(1 + \t \gamma(\theta)\right)}
    \ind{\gamma(\theta) > 0}
    \q(\theta)\mu(d\theta).
%
\end{align*}
%
For all $0 \le \t \le \t_{max}$, $0 \le \abs{\log\left(1 + \t
\gamma(\theta)\right)} \ind{\gamma(\theta) \le 0} \le \log(1/2)$,
so the first term in the preceding display is dominated by the integrable
function $\log(1/2) \abs{\psi(\theta)}\q(\theta)$.

For the second term, by the fact that $x \ge 0 \Rightarrow \log (1 + x) \le x$
and Holder's inequality with respect to $\pbase(\theta)$,
%
\begin{align*}
%
\MoveEqLeft
\int \abs{\psi(\theta)}
    \abs{\log\left(1 + \t \gamma(\theta)\right)}
    \ind{\gamma(\theta) > 0}
    \q(\theta)\mu(d\theta)
\\\le{}&
\t
\int \abs{\psi(\theta)}
    \abs{\gamma(\theta)}
    \ind{\gamma(\theta) > 0}
    \q(\theta)\mu(d\theta)
\\={}&
\t
\int \abs{\psi(\theta)}
    \abs{\gamma(\theta)}
    \ind{\gamma(\theta) > 0}
    \frac{\q(\theta)}{\pbase(\theta)} \pbase(\theta) \mu(d\theta)
\\\le{}&
\t
\int \abs{\psi(\theta)}
    \abs{\gamma(\theta)}
    \frac{\q(\theta)}{\pbase(\theta)} \pbase(\theta) \mu(d\theta)
\\\le{}&
\t_{max}
\left(
\int
    \abs{\frac{\psi(\theta)\q(\theta)}{\pbase(\theta)}}^q
    \pbase(\theta) \mu(d\theta)
\right)^{1/q}
\left(
\int\abs{\gamma(\theta)}^p \pbase(\theta) \mu(d\theta)
\right)^{1/p}
\\\le{}&
\t_{max}
\left(
\int
    \abs{\frac{\psi(\theta)\q(\theta)}{\pbase(\theta)}}^q
    \pbase(\theta) \mu(d\theta)
\right)^{1/q}
\norm{\gamma}_{\pbase,p}.
%
\end{align*}
%
By assumption, the final equation in the preceding display is finite, and the
second line does not depend on $\t$, so we have proven the existence of an
integrable bounding function.  \Eqref{lp_integral_bound} follows.

For \eqref{lp_integral_deriv_bound} observe that $\t \le \t_{max}$ implies
%
\begin{align*}
%
\abs{\psi(\theta)}
    \abs{\frac{\t \gamma(\theta)}{1 + \t \gamma(\theta)}} \q(\theta)
\le{}&
2 \t \abs{\psi(\theta)} \abs{\gamma(\theta)} \q(\theta).
%
\end{align*}
%
Thus \eqref{lp_integral_deriv_bound} was proven along the way to
proving \eqref{lp_integral_bound}.
%
\end{proof}
%
\end{lem}
%%%%%%%%%%%%%%%%%%%%%%%%%%%%%%%%%%%%%%%%%%%%%%%%%%%%%%%%%%%%%%%%%%%%%%%%%%%%%




The next lemma is a different kind of continuity lemma.

%%%%%%%%%%%%%%%%%%%%%%%%%%%%%%%%%%%%%%%%%%%%%%%%%%%%%%%%%%%%%%%%%%%%%%%%%%%%%
%%%%%%%%%%%%%%%%%%%%%%%%%%%%%%%%%%%%%%%%%%%%%%%%%%%%%%%%%%%%%%%%%%%%%%%%%%%%%

\begin{lem}\lemlabel{continuity_partition}
%
Let $\epsilon'_n = n^{-1}$.  Since $\pbase \ll \mu \ll \lambda$ (where $\lambda$
is the Lebesgue measure), by applying \citet[Proposition
15.5]{nielsen:1997:measure}, for each $n$ there exists a $\delta'_n$ such that,
for any measureable set $A$ with $\mu(A) < \delta'_n$, $\pbase(A) <
\epsilon'_n$.  Again applying \citet[Proposition 15.5]{nielsen:1997:measure},
there similarly exists a $\delta_n$ such that for any measureable set $A$ with
$\lambda(A) < \delta_n$, $\mu(A) < \delta'_n \Rightarrow \pbase(A) <
\epsilon'_n$.

For each $n$, partition $\thetadom$ into a countable number of sets $A_{m}$ such
that $\sum_{m} \lambda(A_{m}) = 1$ and $\lambda(A_{m}) < \delta_n$. (This is
possible by dividing $\thetadom$ into sufficiently small rectangles, for
example.)  Then $\pbase(A_{m}) < \epsilon'_n$ for all $m$.  Since $\pbase$ is a
probability measure, $\sum_m \pbase(A_{m}) = 1$, so there must exist at least $1 /
\epsilon'_n$ indices $m'$ such that $\pbase(A_{m'}) > 0$. Take any such $m'$ and
let $\epsilon_n = \pbase(A_{m'})$ and $S_n = A_{m'}$.

%
\end{lem}
%%%%%%%%%%%%%%%%%%%%%%%%%%%%%%%%%%%%%%%%%%%%%%%%%%%%%%%%%%%%%%%%%%%%%%%%%%%%%
