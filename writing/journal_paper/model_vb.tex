Let $\zeta$ denote the full vector of
latent variables.
In the GMM and topic model (\exref{iris_bnp_process, structure_bnp_process}),
$\zeta := (\beta, \z, \nu)$;
in the regression example (\exref{mice_bnp_process}),
$\zeta$ includes the additive shifts, $\zeta := (\beta, \z, \nu, \b)$.
In each model,
the exact posterior distribution $\p(\zeta \vert \x)$
is intractable.
Variational Bayes (VB) is an approach that seeks an approximate posterior
through solving a numerical optimization problem
\citep{jordan:1999:vi, wainwright:2008:graphical_models, blei:2017:vi_review}.

VB specifies a
family of approximating distributions $\q(\zeta \vert \eta)$
parameterized by a finite-dimensional vector $\eta \in \etadom \subseteq
\mathbb{R}^{\etadim}$
and solves for $\q(\zeta\vert\etaopt)$
that is closest to the posterior $\p(\zeta \vert \x)$ in
Kullback-Leibler (KL) divergence:
%
\begin{align}\eqlabel{vb_optimization}
%
\etaopt :={}&
    \argmin_{\eta \in \etadom}
        \KL{\q(\zeta \vert \eta) || \p(\zeta \vert \x)} \mathwhere \\
\KL{\q(\zeta \vert \eta) || \p(\zeta \vert \x)}
={}&    \expect{\q(\zeta \vert \eta)}{
        \log \q(\zeta \vert \eta) - \logp(\x, \zeta)} + \logp(\x). \nonumber
%
\end{align}
%
As we discuss
below, we will choose $\q(\zeta \vert \eta)$ so that we can easily evaluate (or approximate)
the above expectation with respect to $\q(\zeta \vert \eta)$
as a closed-form function of $\eta$.
Notice that the intractable $\logp(\x)$ term does not depend on $\eta$, and
so can be neglected in the optimization.

In practice, forming an approximating posterior for BNP
can be challenging since the latent variables $\nu$ and $\beta$ are
(countably) infinite dimensional.
We would like to keep dimension of the variational parameter $\eta$ finite
in order for the optimization in \eqref{vb_optimization} to be tractable.
In the present paper, we will follow \citet{blei:2006:vi_for_dp} and
use a truncated stick-breaking representation in
the variational distribution.
We choose a truncation parameter $\kmax$ large but finite,
and we set $\q(\nu_\k = 1 | \eta) = 1$ for all $\k > \kmax$.
This implies that under $\q$, $\pi_\k = 0$ with probability one
for all $\k > \kmax$
(\eqref{stick_breaking}).
Correspondingly, we also set $\q(\z_{\n\k} = 0 | \eta) = 1$ for $\k > \kmax$.

For the generic BNP mixture model in \eqref{bnp_model},
we propose a mean-field variational approximating family of the following form:
%
\begin{align}\eqlabel{vb_mf}
%
\q(\zeta \vert \eta) =
    \left( \prod_{\k=1}^{\kmax - 1} \q(\nuk \vert \eta) \right)
    \left( \prod_{\k=1}^{\kmax} \q(\beta_\k \vert \eta) \right)
    \left( \prod_{\n=1}^{\N} \q(\z_{\n} \vert \eta) \right).
%
\end{align}
Because $\pi_\k = 0$ for all $\k > \kmax$,
we can ignore the latent variables $\beta_\k$ for $\k > \kmax$ in
defining our variational approximation.

Notice that only our variational approximation is truncated---the model
(\eqref{bnp_model}) itself is not finite.
We set $\kmax$ large enough in our variational approximation to ensure that a large
proportion of the components are unoccupied with high probability under $\q$,
in which case the truncation approximates the fully nonparametric model with $\kmax = \infty$.

The variational approximation for the topic model (\exref{structure_bnp_process})
is similarly mean-field: the distribution on stick-breaking proportions $\nu$
factorizes over
both individuals $\n$ and components $\k$, while the
assignments $\z$
factorize over individuals $\n$, loci $\l$, and chromosomes $\i$.
For the regression model (\exref{mice_bnp_process}), all terms in the
variational approximation fully-factorize
except for the cluster assignments $\z$ and additive shifts $\b$.
While we assume $(\z, \b)$ to be independent from all other latent variables
under $\q$, we will allow conditional dependence between $\z$ and $\b$ (\appref{appendix}).

\subsubsection{Conditional conjugacy}\seclabel{vb_conjugacy}

For $\z$ and $\beta$ in all models we consider, we will
take advantage of conditional conjugacy to choose distributions
$\q(\z_\n\vert\eta)$ and $\q(\beta_k\vert\eta)$, unless otherwise stated.
This means that we will take $\q(\z_{\n}
\vert \eta)$ to be multinomial, matching $\p(\z_{\n}
\vert \x, \beta, \nu)$, and we will take $\q(\beta_\k \vert \eta)$
to match the distribution of $\p(\beta_{\k} \vert \x, \z, \nu)$.


%%%%%%%%%%%%%%%%%%%%%%%%%%%%%%%%%%%%%%%%%%%%%%%%%%%%%%%%%%%%%%%%%%%%%%%%%%%%%%%
%%%%%%%%%%%%%%%%%%%%%%%%%%%%%%%%%%%%%%%%%%%%%%%%%%%%%%%%%%%%%%%%%%%%%%%%%%%%%%%

\begin{ex}[VB approximation for $\beta_\k$ in a GMM]\exlabel{iris_var_distr}
%
To evaluate the expectation in \eqref{vb_optimization}, we need to compute
the expected joint log-likelihood
\begin{align*}
  \expect{\q(\beta_\k \vert \eta)}{\log \p(\x_\n, \beta_\k)}. % , \quad
  % \expect{\q(\beta_\k \vert \eta)}{\log \q(\beta_\k \vert \eta)}.
\end{align*}

In \exref{iris_bnp_process}, $\beta_\k = (\mu_\k, \Lambda_\k)$,
and the likelihoods are Gaussian,
$\p(\x_\n \vert \beta_\k) = \normdist{\x_n \vert \mu_\k, \Lambda_\k^{-1}}$.
The prior $\pbetaprior(\beta_\k)$ a normal-Wishart.
Using the log densities displayed in \exref{iris_bnp_process},
observe that $\beta_\k$ enters the expected joint log-likelihood only through the
expected moments
%
\begin{align*}
\expect{\q(\beta_\k \vert \eta)}{\Lambda_\k},  \quad
\expect{\q(\beta_\k \vert \eta)}{\log|\Lambda_\k|},  \quad
\expect{\q(\beta_\k \vert \eta)}{\Lambda_\k\mu_\k}, \quad
\expect{\q(\beta_\k \vert \eta)}{\mu_\k\Lambda_\k\mu_\k}.
\end{align*}

The conditionally conjugate variational distribution on $\beta_\k$ is
normal-Wishart, which we denote as
$\q(\beta_\k \vert \eta) = \normalwishart{\beta_k \vert \eta}$.
With this choice of $\q(\beta_\k \vert \eta)$,
all the preceding expected moments
can be provided as closed-form functions of $\eta$.
%
% With this choice of $\q$, the expectations
% \begin{align*}
% \expect{\q(\beta_\k \vert \eta)}{\log \p(\x_\n \vert \beta_\k)} \quad
% \expect{\q(\beta_\k \vert \eta)}{\log \pbetaprior(\beta_\k)} \quad
% \expect{\q(\beta_\k \vert \eta)}{\log \q(\beta_\k \vert \eta)}
% \end{align*}
% all have closed form expressions as functions of $\eta$.
%
\end{ex}

%%%%%%%%%%%%%%%%%%%%%%%%%%%%%%%%%%%%%%%%%%%%%%%%%%%%%%%%%%%%%%%%%%%%%%%%%%%%%%%

Under the mean-field factorization (\eqref{vb_mf}),
the vector $\eta$ will partition into parameters
governing $\nu$, $\beta$, and $\z$. Let the parameters governing a
particular latent variable or latent vector be denoted with a subscript: for example,
$\q(\beta \vert \eta) = \q(\beta \vert \etabeta)$,
$\q(\z_\n \vert \eta) = \q(\z \vert \eta_{\z_{\n}})$, and so on.
With conditionally conjugate distributions and our mean-field assumption,
the parameters $\eta_{\z_{\n}}$ can be optimally set as a function of
parameters $\etabeta$ and $\etanu$.
The next example details this point.

%%%%%%%%%%%%%%%%%%%%%%%%%%%%%%%%%%%%%%%%%%%%%%%%%%%%%%%%%%%%%%%%%%%%%%%%%%%%%%%
%%%%%%%%%%%%%%%%%%%%%%%%%%%%%%%%%%%%%%%%%%%%%%%%%%%%%%%%%%%%%%%%%%%%%%%%%%%%%%%

\begin{ex}[VB approximation for $\z_\n$ in a GMM]\exlabel{qz_form}
%
The conditionally conjugate variational distribution for $\z_\n$
is multinomial.
Our variational approximation is truncated at $\kmax$ so
$\z_{\n\k} = 0$ for all $\k > \kmax$;
the multinomial distribution under $\q$ has $\kmax$ discrete categories.

We parameterize the the multinomial distribution
using its natural parameterization
in exponential family form. That is,
we let $\eta_{\z_\n} = (\rho_{\n1}, \rho_{\n2}, ..., \rho_{\n(\kmax-1)})$
be an unconstrained vector in $\mathbb{R}^{\kmax-1}$;
in this parameterization, the multinomial expectations are
%
\begin{align*}
  p_{\n\k} := \expect{\q(\z_\n \vert \etaz)}{\z_{\n\k}} =
  \frac{\exp(\rho_{\n\k})}{1 + \sum_{\k'=1}^{\kmax-1}\exp(\rho_{\n\k})}
\end{align*}
%
We use the exponential family parameterization because
we will require the optimal variational parameters $\etaopt$
to be interior to $\etadom$ in our sensitivity analysis
(\secref{local_sensitivity}).
In the mean parameterization,
$\sum_{\k=1}^\kmax p_{\n\k} = 1$, so the
optimal mean parameters $\hat p_{\n}$ cannot be
interior to $\Delta^{\kmax - 1}$.
On the other hand, $\eta_{\z_\n}$ as defined
is unconstrained in $\mathbb{R}^{\kmax - 1}$.

Moreover, with the distributions $\q(\beta\vert\etabeta)$ and $\q(\nu\vert\etanu)$ fixed,
the parameter vector $\eta_{\z_\n}$ that minimizes \eqref{vb_optimization}
has a closed form.
Fixing $\q(\beta\vert\etabeta)$ and $\q(\nu\vert\etanu)$,
the optimal $\etaopt_{\z_\n}$ must satisfy
%
\begin{align*}
& \q(\z_\n | \etaopt_{\z_\n}) \propto \exp\left(\tilde \rho_{\n\k}\right)\\
& \mathwhere \tilde \rho_{\n\k} := \expect{\q(\beta, \nu \vert \eta)}
       {\logp(\x_n \vert \beta_\k) + \log \pi_\k}.
\end{align*}
%
See \citet{bishop:2006:PRML} and \citet{blei:2017:vi_review} for details.
To satisfy this optimality condition,
we set the optimal $\etaopt_{\z_\n}$ to be
%
\begin{align*}
%
\etaopt_{\z_\n} = \left(\log\frac{\tilde\rho_{\n1}}{\tilde\rho_{\n\kmax}},
\log\frac{\tilde\rho_{\n2}}{\tilde\rho_{\n\kmax}}, \ldots,
\log\frac{\tilde\rho_{\n(\kmax-1)}}{\tilde\rho_{\n\kmax}}\right).
%
\end{align*}
%
Thus, as long as the expectation $\tilde\rho_{\n\k}$ has a closed-form as a function of
$(\etabeta, \etanu)$, the optimal $\etaopt_{\z_\n}$ can be also be set in closed-form as
a function of $(\etabeta, \etanu)$.
%
\end{ex}

For fixed $(\etabeta, \etanu)$, the option of setting $\etaz$ at its optimum
extends beyond the GMM example and will play
a key role in computing our local sensitivity
measures in practice (\secref{computing_sensitivity}).
In greater generality, each of our example models
has latent variables that factorize in a global/local structure.
In the GMM example discussed above, we call the variables $(\beta, \nu)$ ``global"
because they are shared across all data points; the $\z$ is ``local"
because each $\z_\n$ is unique to a single data point.
In the regression model (\exref{mice_bnp_process}),
the global variables are again $(\beta, \nu)$,
but the local variables comprise of both the cluster assignments $\z$ and additive shifts $\b$.
In these two models, notice that the dimension of global variables scale with $\kmax$, while
the dimension of local variables scale with the number of observations $\N$.

In the topic model (\exref{structure_bnp_process}),
we still call $(\beta, \nu)$ the global latent variables, even though they scale
with the number of individuals $\N$;
they do not, however, scale with both the number of individuals and the number of loci
like $\z$ does. In the topic model, we call $\z$ the local latent variables.

Let $\gamma = (\beta,\nu)$ be the global latent variables
and let $\etaglob = (\etabeta, \etanu)$ be their variational parameters.
Let $\etalocal$ be the local variational parameters. In \exref{iris_bnp_process} and
\exref{structure_bnp_process}, $\etalocal = \etaz$, while
in \exref{mice_bnp_process}, $\etalocal = (\etaz, \eta_\b)$.

In each model we consider, for $\etaglob$ fixed, the optimal $\etalocal$ that minimizes
the $\mathrm{KL}$ can be set in closed form as a function of $\etaglob$.
The multinomial parameters for $\eta_{\z_\n}$ in the regression and topic models
can be set in the same way as described in \exref{qz_form}.
For more details concerning the optimal shift parameters $\eta_\b$
in the regression model, see \appref{put_in_appendix}.



%%%%%%%%%%%%%%%%%%%%%%%%%%%%%%%%%%%%%%%%%%%%%%%%%%%%%%%%%%%%%%%%%%%%%%%%%%%%%%%
%%%%%%%%%%%%%%%%%%%%%%%%%%%%%%%%%%%%%%%%%%%%%%%%%%%%%%%%%%%%%%%%%%%%%%%%%%%%%%%

\subsubsection{Evaluating stick expectations}

To evaluate the $\mathrm{KL}$ in \eqref{vb_optimization}, we also need
the expectations over stick-breaking proportions,
\begin{align}\eqlabel{stick_expectations}
%
\expect{\q(\nuk \vert \eta)}{\log \nuk}
\textrm{,}\quad
\expect{\q(\nuk \vert \eta)}{\log (1 - \nuk)}
\textrm{,}\quad\textrm{and}\quad
\expect{\q(\nuk \vert \eta)}{\log \pstick(\nuk)}.
%
\end{align}
(The discussion in this subsection applies to the topic model as well,
with stick-breaking proportions indexed by $\n\k$).
The first two expectations appear in the $\mathrm{KL}$
when decomposing the mixture weights
$\expect{}{\log \pi}$ into its component stick-breaking proportions (\eqref{stick_breaking}).

If the prior $\pstick(\nuk)$ were Beta-distributed like in the GEM construction,
then the conditionally conjugate distribution for $\q(\nuk \vert \eta)$ would also be Beta.
In this case, all the displayed expectations in \eqref{stick_expectations}
can be computed analytically as a function of
the Beta parameters in the variational approximation.

However, we will be considering stick-breaking distributions $\pstick(\nuk)$ that
are outside the family of Beta distributions.
To accommodate a generic prior $\pstick(\nuk)$,
we approximate the expectations in \eqref{stick_expectations} numerically.
Each expectation is a univariate integral.
A particularly easy approximation method is
Gauss-Hermite (GH) quadrature, which we now describe.

To take advantage of GH quadrature, we first logit transform the stick-breaking
proportion $\nuk$ so that the transformed variable
\begin{align*}
  \lnuk := \log\left(\frac{\nuk}{1 - \nuk}\right)
\end{align*}
is not constrained to be between $(0, 1)$ and can take values in all of $\mathbb{R}$.
Let $\s$ be the sigmoid function,
which provides the inverse transformation,
\begin{align*}
  \nuk = \s(\tilde\nuk) := \frac{\exp(\lnu_\k)}{1 + \exp(\lnu_\k)}.
\end{align*}

It will be useful later to have at hand the following
derivatives of the transforms between $\nu$ and $\lnu$,
%
\begin{align}\eqlabel{lnuk_derivatives}
%
\fracat{d \lnu_\k}{ d\nuk}{\nuk} ={}
%     \frac{1-\nuk}{\nuk}
%     \left(\frac{1}{1 - \nuk} + \frac{\nuk}{(1 - \nuk)^2} \right)
% \\={}& \frac{1}{\nuk} + \frac{1}{1 - \nuk}
% \\={}&
    \frac{1}{\nuk (1 - \nuk)} \mathand
%
\fracat{d \nuk}{ d\lnuk}{\lnuk} ={}
    \frac{\exp(\lnuk)}{(1 + \exp(\lnuk))^2}.
%
\end{align}


We let $\q(\lnu_\k \vert \eta)$ be normally distributed with
location parameter $\lnumean_\k$ and scale parameter $\lnusd_\k$.
This then induces a logit-normal
distribution on our original variable of interest, $\nuk$.
In other words,
\begin{align}\eqlabel{lnuk_vb_approximation}
%
\q(\lnu_\k \vert \eta) ={}& \normdist{\lnu_\k \vert \lnumean_\k, \lnusd_\k}
\Rightarrow \\
\q(\nuk \vert \eta) ={}&
    \normdist{\log \left( \frac{\nuk}{1 - \nuk} \right)
        \vert \lnumean_\k, \lnusd_\k}
    \left|\fracat{d \lnu_\k}{ d\nuk}{\nuk}\right|
\nonumber\\={}&
\normdist{\log \left( \frac{\nuk}{1 - \nuk} \right)
        \vert \lnumean_\k, \lnusd_\k}
    \left|\frac{1}{\nuk (1 - \nuk)}\right|.
\nonumber
%
\end{align}


With this choice of a logit-normal $\q(\nuk \vert \eta)$,
we can approximate expectations of smooth functions
$f(\nuk)$ (such as $f(\nuk) = \pstick(\nuk)$) using GH quadrature with $\ngh$ knots,
located at $\xi_g$, weighted by $\omega_g$:
%
\begin{align}\eqlabel{gh_integral}
%
\expect{\q(\nuk \vert \eta)}{f(\nuk)} ={}&
\expect{\q(\lnu_\k \vert \eta)}
       {f\circ \s\left(\lnu_\k\right)}
\nonumber\\\approx{}&
    \sum_{g=1}^{\ngh} \omega_g f\circ \s \left(\lnusd_\k \xi_{g} + \lnumean_\k\right)
 \nonumber\\=:{}&
\expecthat{\q(\nuk \vert \eta)}{f(\nuk)}.
%
\end{align}
%
Conveniently, $\expecthat{\q(\nuk \vert \eta)}{f(\nuk)}$ is a deterministic, differentiable
function of $\lnumean_\k$ and $\lnusd_\k$, and so also of $\eta$.  (This
technique is similar to the ``reparameterization trick,'' only using
GH points rather than standard normal draws).

%%%%%%%%%%%%%%%%%%%%%%%%%%%%%%%%%%%%%%%%%%%%%%%%%%%%%%%%%%%%%%%%%%%%%%%%%%%%%%%
%%%%%%%%%%%%%%%%%%%%%%%%%%%%%%%%%%%%%%%%%%%%%%%%%%%%%%%%%%%%%%%%%%%%%%%%%%%%%%%

\subsection{Posterior quantities}

In a VB approach, all posterior quantities of interest can be expressed as
functions of the variational parameter $\eta$. We will use $\g(\eta)$ to denote
such quantities. Often, $\g(\eta)$ takes the form of an expectation over $\q$,
\begin{align*}
  g(\eta) = \expect{q(\zeta\vert\eta)}{f(\zeta)}
\end{align*}
for some function $f(\eta)$.

In the next few examples, we define some posterior quantities that we will
consider in \secref{results}. We will evaluate the sensitivity
of these quantities to the prior specification $\pstick$.

\begin{ex}[The in-sample number of clusters]\exlabel{insample_nclusters}

One might ask, \textit{how many clusters are present in the data set}?
For example, in the iris data set, answering this question has the interpretation
of counting the number of iris species present.
To estimate the number of clusters in the context of a BNP model,
define the random variable
\begin{align*}
  \nclusters_\tau := \sum_{k=1}^\kmax \ind{ \left(\sum_{n=1}^{N}
  \z_{\n\k}\right) > \tau},
\end{align*}
where $\ind{\cdot}$ is the indicator function.
$\nclusters_\tau$ counts the number of clusters with at least $\tau$
observations in a set of assignments $\z$.
The expected number of clusters under the variational posterior is
\begin{align*}
  \gclusters(\eta) := \expect{\q(\z\vert\eta)}{\nclusters_\tau}.
\end{align*}

When $\tau = 0$, $\gclusterszero$ can be written as a function with respect to
the assignment probabilities
\begin{align*}
  \gclusterszero(\eta) = \sum_{k=1}^\kmax \left(1 -  \prod_{n=1}^N
  \left(1 - \expect{\q(\z_{nk}\vert\eta)}{\z_{nk}}\right)\right).
\end{align*}
%
\end{ex}

\begin{ex}[The predictive number of clusters]\exlabel{predictive_nclusters}

In the Bayesian approach, we can formulate the posterior predictive question,
\textit{how many clusters would be present if a new data set were collected}?
In the iris example, this can interpreted as predicting the number of species
one might see if a fresh sample of iris flowers were collected.
Under the BNP model, the expected number of predictive clusters is defined as
\begin{align*}
  \gclusterspred(\eta) &:= \expect{\q(\pi\vert\eta)}{\expect{\p(\z\vert\pi)}{\nclusters_\tau}}.
\end{align*}
Notice that the inner expectation conditions on $\pi$ and the randomness is
over $\z$ sampled from the generative model $\z\sim\p(\z\vert\pi)$.
We can write out the inner expectation:
%
\begin{align*}
  \gclusterspred(\eta) &= \expect{\q(\pi\vert\eta)}{\sum_{k=1}^\kmax\left(1 -
  \sum_{i=0}^{\lfloor\tau\rfloor} {\N \choose i} (1 - \pi_k)^\N \right)},
\end{align*}
where we use the convention that ${\N\choose 0} = 1$.
%
\end{ex}

\begin{ex}[Co-clustering]\exlabel{posterior_coclustering}

Finally, in a clustering problem, we are often interested in understanding
which observations group with each other.
One way to visualize the clusters is to construct the co-clustering matrix,
\begin{align*}
\gcoclustering(\eta) := \expect{\q(\z\vert\eta)}{\z\z^{T}},
\end{align*}
where we view $\z$ as a $\N\times \kmax$ matrix of cluster assignments.
Unlike the quantities in \exref{insample_nclusters, predictive_nclusters},
$\gcoclustering$ is a matrix quantity, not a scalar quantity.

\end{ex}

For some posterior quantities, the expectation over $\q$ will not be a simple
closed-form function of $\eta$.
For example, computing $\gclusters$ with a threshold $\tau > 0$
requires forming all ${\N\choose\tau}$ combination of $\tau$-length products
$\expect{}{\z_{\n_1\k}}\times \ldots \times \expect{}{\z_{\n_\tau\k}}$
for each $\k$.
In such cases, we resorted to Monte-Carlo approximations of the expectation.
Specifically, we used the ``reparameterization trick"
to sample from the variational distribution.
In the case of $\gclusters$,
we constructed an $\eta$-dependent transformation
$f(\cdot, \eta)$ that satisfies
\begin{align*}
  u \iid\text{Uniform}(0, 1)^{\N\kmax} \implies
  f(u, \eta) \stackrel{d}{=} \z \sim \q(\cdot | \eta).
\end{align*}
To form a Monte Carlo estimate of $\gclusters$,
we sampled $u_1, \dots, u_m$ uniformly,
and then averaged the expression inside the expectation evaluated at points
$f(u_1, \eta), \ldots, f(u_m, \eta)$.
The uniform draws $u_1, \dots, u_m$ can be fixed beforehand.
This is important for two reasons.
First, we will be evaluating the same $\g$ at different parameter vectors $\eta$;
conditional on the fixed $m$ uniform draws, $\g$ will be a deterministic function
of $\eta$, and we can compare how $\g$ changes without stochasticity.
Secondly, in the construction of our ``influence function" (\secref{functional_perturbations}),
it will be useful to evaluate the
gradient of $\g$ with respect to $\eta$; conditional on the random draws,
$\g$ (or more precisely, our Monte Carlo approximation of $\g$), will be
differentiable with respect to $\eta$.

% \subsection*{subsection name}
%
% we choose $\q(\nuk \vert \eta)$ to be logit-normally distributed,
% and express the expectations in \eqref{stick_expectations} as Gaussian integrals.
% Define
% \begin{align*}
%   \tilde \nuk := \log\left(\frac{\nuk}{1 - \nuk}\right),
% \end{align*}
% which will be normally distributed under our choice of a
% logit-normal $\q(\nuk \vert \eta)$.
%
% Let $\lnumean_\k$ and $\lnusd_\k$ be entries of $\eta$ corresponding to
% the logit-normal parameters of $\nuk$.
%
%
% In order to optimize the variational objective \eqref{vb_optimization} we see
% from \eqref{stick_log_post} that we need to evaluate or approximate expectations
% of the form
% %
% \begin{align*}
% %
% \expect{\q(\nuk \vert \eta)}{\log \nuk}
% \textrm{,}\quad
% \expect{\q(\nuk \vert \eta)}{\log (1 - \nuk)}
% \textrm{,}\quad\textrm{and}\quad
% \expect{\q(\nuk \vert \eta)}{\log \pstick(\nuk)}.
% %
% \end{align*}
%
%
%
%
% First, define a version of $\nuk$ that is not constrained to $(0,1)$:
% %
% \begin{align}\eqlabel{lnuk_transform}
% %
% \lnuk :={} \log \left( \frac{\nuk}{1 - \nuk} \right)
% \quad\Leftrightarrow\quad
% \nuk :={} \frac{\exp(\lnuk)}{1 + \exp(\lnuk)}.
% %
% \end{align}
% %
% It will be useful later to have at hand the transform between densities
% expressed in the space of $\nu$ and $\lnu$, which is given by
% %
% \begin{align}\eqlabel{lnuk_derivatives}
% %
% \fracat{d \lnu_\k}{ d\nuk}{\nuk} ={}
% %     \frac{1-\nuk}{\nuk}
% %     \left(\frac{1}{1 - \nuk} + \frac{\nuk}{(1 - \nuk)^2} \right)
% % \\={}& \frac{1}{\nuk} + \frac{1}{1 - \nuk}
% % \\={}&
%     \frac{1}{\nuk (1 - \nuk)} \mathand
% %
% \fracat{d \nuk}{ d\lnuk}{\lnuk} ={}
%     \frac{\exp(\lnuk)}{(1 + \exp(\lnuk))^2}.
% %
% \end{align}
% %
% We wish to let $\lnu_\k$ be distributed normally under the variational
% distribution.  Let $\lnumean_\k$ and $\lnusd_\k$ be entries of the parameter
% vector $\eta$, and write
% %
% \begin{align}\eqlabel{lnuk_vb_approximation}
% %
% \q(\lnu_\k \vert \eta) ={}& \normdist{\lnu_\k \vert \lnumean_\k, \lnusd_\k}
% \Rightarrow \\
% \q(\nuk \vert \eta) ={}&
%     \normdist{\log \left( \frac{\nuk}{1 - \nuk} \right)
%         \vert \lnumean_\k, \lnusd_\k}
%     \left|\fracat{d \lnu_\k}{ d\nuk}{\nuk}\right|
% \nonumber\\={}&
% \normdist{\log \left( \frac{\nuk}{1 - \nuk} \right)
%         \vert \lnumean_\k, \lnusd_\k}
%     \left|\frac{1}{\nuk (1 - \nuk)}\right|.
% \nonumber
% %
% \end{align}
% %
% Given this, we can approximate expectations of smooth functions
% $f(\nuk)$ using GH quadrature with $\ngh$ knots,
% located at $\xi_g$, weighted by $\omega_g$:
% %
% \begin{align}\eqlabel{gh_integral}
% %
% \expect{\q(\nuk \vert \eta)}{f(\nuk)} ={}&
% \expect{\q(\lnu_\k \vert \eta)}
%        {f\left(\frac{\exp(\lnu_\k)}{1 + \exp(\lnu_\k)}\right)}
% \nonumber\\\approx{}&
%     \sum_{g=1}^{\ngh} \omega_g f\left(\lnusd_\k \xi_{g} + \lnumean_\k\right)
%  \nonumber\\=:{}&
% \expecthat{\q(\nuk \vert \eta)}{f(\nuk)}.
% %
% \end{align}
% %
% Conveniently, $\expecthat{\q(\nuk \vert \eta)}{f(\nuk)}$ is a differentiable
% function of $\lnumean_\k$ and $\lnusd_\k$, and so also of $\eta$.  (This
% technique is similar to the ``reparameterization trick,'' only using
% GH points rather than standard normal draws.)
