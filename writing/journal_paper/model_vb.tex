To assess the sensitivity of a procedure in practice, we need to consider the approximate
Bayesian inference algorithm used as well. Here we focus on a variational Bayes approximation
due to \citet{blei:2006:vi_for_dp}.

Variational Bayes (VB) posits a class of tractable distributions over the model parameters and chooses the 
element of this class that minimizes the reverse Kullback-Leibler (KL) divergence to the exact posterior. The KL divergence
for a full Bayesian nonparametric posterior is not well-defined due the infinitude of parameters. One approach
to extend VB to Dirichlet process stick-breaking models 
assumes $\nu_\kmax = 1$ for all distributions in the variational class and some truncation level $\kmax$.
Let $\zeta$ collect the first $\kmax - 1$ elements of $\nu$, the first $\kmax$ elements of $\beta$,
and the first $\kmax$ elements of $\z_\n$ across $n$.
In what follows, then, we effectively consider approximations to the posterior marginal $\p(\zeta \vert \x)$.
By setting $\kmax$ sufficiently large, one can make this truncation as expressive as desired.

Mean-field VB is a particularly popular VB variant where the tractable approximating distributions $\q$ factorize
over the parameters. In our case, then, we allow approximations of the form
%
\begin{align}\eqlabel{vb_mf}
%
\q(\zeta \vert \eta) =
    \left( \prod_{\k=1}^{\kmax - 1} \q(\nuk \vert \eta) \right)
    \left( \prod_{\k=1}^{\kmax} \q(\beta_\k \vert \eta) \right)
    \left( \prod_{\n=1}^{\N} \q(\z_{\n} \vert \eta) \right),
%
\end{align}
%
where $\eta \in \etadom \subseteq
\mathbb{R}^{\etadim}$ represents \emph{variational parameters} that determine the factors of
the $\q$ distribution. When the observation likelihood $\p(\x_n \vert \beta_\k)$ is conditionally
conjugate with the component-parameter prior $\pbetaprior(\beta_\k)$, no further assumptions
are needed on the form of $\q(\beta_\k \vert \eta)$; one can show that it will
take the form of the conjugate exponential family after the KL optimization ***cite.
Similarly, when $\pstick$ is a beta distribution,
no further assumptions are needed on $\q(\nuk \vert \eta)$; it will take a beta form ***cite.
However, since we will consider non-beta forms of $\pstick$, we must specify a more generic
approximation -- one that will work even when conditional conjugacy does not hold. To that end, we
follow ***cite to first transform the $\nuk$ to a form that takes any real values and then use a Gaussian approximation.
Define the logit-transformed stick-breaking
proportions $\lnuk$:
%
\begin{align*}
  \lnu_\k := \log(\nu_\k) - \log(1 - \nu_\k)
  \quad \Leftrightarrow \quad
  \nuk = \frac{\exp(\lnu_\k)}{1 + \exp(\lnu_\k)}.
\end{align*}
%
We take $\q(\lnuk \vert \eta)$ to be a normal distribution, which induces a
logit-normal distribution on $\nuk$. We approximate all resulting integrals
over $\q(\lnuk \vert \eta)$, as in the KL objective for VB or in our later sensitivity calculations, with GH quadrature;
see \appref{gh_quadrature} for details.

GH quadrature gives us an approximation, which we will call $\KL{\eta}$, to the full KL, $\KL{\q(\zeta \vert \eta) || \p(\zeta \vert \x)}$.
We minimize that approximation
to perform approximate posterior inference:
%
\begin{align}
%
\eqlabel{kl_def}
\KL{\q(\zeta \vert \eta) || \p(\zeta \vert \x)}
={}    \expect{\q(\zeta \vert \eta)}{
        \log \q(\zeta \vert \eta) - \log\p(\x, \zeta)} + \log\p(\x) \\
%
\eqlabel{vb_optimization}
\etaopt :={} \argmin_{\eta \in \etadom} \KL{\eta} \mathwhere
\KL{\eta} \approx{} \KL{\q(\zeta \vert \eta) || \p(\zeta \vert \x)}.
%
\end{align}
%
Our final approximation to the marginal posterior $\p(\zeta \vert \x)$
is $\q(\zeta \vert \etaopt)$.

%%
\noindent \textbf{Posterior quantities of interest.}
To approximate any functional of the exact posterior, we apply the equivalent functional to
$\q(\zeta \vert \etaopt)$. For instance, the approximation to the posterior expected number of clusters
among the
$N$ observed data points is
%
\begin{align} \eqlabel{num_clust_vb}
%
\expect{\q(\zeta \vert\etaopt)}{\nclusters(\z)} =
\expect{\q(\z\vert\etaopt)}{\nclusters(\z)} =
\sumkm \left(1 -  \prod_{\n=1}^\N
    (1 - \expect{\q(\z_\n \vert \etaoptz)}{\z_{\n\k}})\right).
%
\end{align}
%

We will see examples in \secref{results} where our quantity of interest is
(a) the expected posterior number of clusters in the observed data, (b) the expected posterior
number of clusters in a new set of (as yet unobserved) data, (c) some
aspect of a co-clustering matrix, or (d) the topic assignments of certain data points.
In all of these cases, as in \eqref{num_clust_vb}, we can express our (approximate) posterior quantity of interest as some function $g$ of the optimized variational parameters $\etaopt$: $g(\etaopt)$.

Once we have an (approximate) posterior quantity of interest, we can ask how this quantity would change -- and
whether our substantive scientific conclusions would change -- if we had made reasonably different prior choices.