There are two practical problems with forming the posterior
$p(\theta, \z \vert \x)$ based on the joint distribution given in
\eqref{bnp_model}.  First, there are an infinite number of parameters,
and second, the posterior is intractable.  In this section, we describe
how we circumvent these difficulties using a truncated variational Bayes
approximation (CITE).

In practice, forming a posterior based on \eqref{bnp_model} with $\kmax =
\infty$ can be challenging.  In the present paper, we will follow CITE BLEI and
use a ``truncated model'' where $\kmax$ is large but finite. We ensure that a
large proportion of the clusters are unoccupied with high posterior probability,
in which case the truncation approxes the fully nonparametric case with $\kmax =
\infty$ (CITE HUGGINS).  Under this truncation, the final cluster (indexed by
$\kmax$) can be thought of as capturing the contribution from the tail of
clusters that are not explicitly represented in the truncated model.

Under the truncation, our model differs formally from standard finite mixture
models principally in our usage of the stick breaking prior rather than, say,
the Dirichlet prior.  Using the stick breaking prior is appealing due to
its connection to the fully nonparametric model.  Additionally, the stick
breaking prior deals more gracefully than the Dirichlet prior
with a large $\kmax$.  {\color{red}TODO: is this true?  Doesn't $\alpha$
have to do something as $\kmax \rightarrow \infty$ to approximate a DP?}

Second, even with finite $\kmax$, the posterior for \eqref{bnp_model} is
intractable, so we again follow CITE BLEI and emply a mean field variational
approximation.  Let $\zeta := (\theta, \z, \nu)$ denote the full vector of
posterior parameters.  We form our variational approximation by specifying a
family of approximating distributions of the form $\q(\zeta \vert \eta)$,
parameterized by a finite-dimensional $\eta \in \etadom \subseteq
\mathbb{R}^{\etadim}$, such that $\q(\zeta \vert \eta)$ is absolutely continuous
with respect to the prior $p(\zeta)$ for all $\eta \in \etadom$.  As we discuss
below, we will choose $\q(\zeta \vert \eta)$ so that we can easily compute or
approximation expectations with respect to $\q(\zeta \vert \eta)$, and so that
$\q(\zeta \vert \eta)$ has tractable entropy as a function of $\eta$.

Given our family of variational approximations, we wish to find the member of
the family that is closest to the posterior $p(\zeta \vert \x)$ in
Kullback-Leibler (KL) divergence:
%
\begin{align}\eqlabel{vb_optimization}
%
\etaopt :={}&
    \argmin_{\eta \in \etadom}
        \KL{\q(\zeta \vert \eta) || p(\zeta \vert \x)} \mathwhere \\
\KL{\q(\zeta \vert \eta) || p(\theta \vert \x)}
={}&    \expect{\q(\zeta \vert \eta)}{
        \log \q(\zeta \vert \eta) - \logp(\x \vert \zeta) -
        \logp(\zeta)} + \logp(\x). \nonumber
%
\end{align}
%
Due to properties of $\q(\zeta \vert \eta)$ or as given in \eqref{bnp_model},
all the terms in $\KL{\q(\zeta \vert \eta) || p(\theta \vert \x)}$ are tractable
except for $\logp(\x)$.  However, $\logp(\x)$ does not depend on $\eta$, and
so can be negelcted in the optimziation.

We will use mean-field variational approximating families of the following form:
%
\begin{align*}
%
\q(\zeta \vert \eta) =
    \left( \prod_{\k=1}^{\kmax - 1} \q(\nu_\k \vert \eta) \right)
    \left( \prod_{\k=1}^{\kmax} \q(\theta_\k \vert \eta) \right)
    \left( \prod_{\n=1}^{\N} \q(\z_{\n} \vert \eta) \right).
%
\end{align*}
%
For $\z$ and $\theta$, in the present work, we will be always able to take
advantage of conditional conjugatcy.  Specifically, we will take $\q(\z_{\n}
\vert \eta)$ to be multinomial with a single observation, matching $p(\z_{\n}
\vert \x, \theta, \nu)$, and we will take $\q(\theta_\k \vert \eta)$, matching
the distribution of $p(\theta_{\k} \vert \x, \z, \nu)$.

For the stick-breaking distributions $\q(\nu_\k \vert \eta)$ we will need to do
something more complicated, since we wish to accomodate generic stick breaking
distributions.  From \eqref{bnp_model} we see that, up to a constant not
depending on $\nu_\k$,
%
\begin{align}\eqlabel{stick_log_post}
%
\log \pi_\k ={}&
    \log \nu_\k + \sum_{\k' < \k} \log (1 - \nu_\k) \nonumber \\
\logp(\nu_{\k} \vert \x, \theta, \z) ={}&
    \left(\sum_{\n=1}^\N \z_{\n\k'}\right) \log \nu_\k +
    \left( \sum_{\k' > \k} \sum_{\n=1}^\N \z_{\n\k'} \right) \log (1 - \nu_\k) +
    \log \pstick(\nu_\k).
%
\end{align}
%
When $\pstick(\cdot) = \mathrm{Beta}(\cdot | 1, \alpha)$, then, up to a constant
not depending on $\nu_\k$, $\log \pstick(\nu_\k) = (\alpha - 1) \log (1 -
\nu_\k)$, so $\logp(\nu_{\k} \vert \x, \theta, \z)$ is proportional to the
sufficient statistics $\log \nu_\k$ and $\log(1 - \nu_\k)$ and so in the
Beta family.  However, for a generic $\pstick(\cdot)$, the posterior
$p(\nu_{\k} \vert \x, \theta, \z)$ does not have a standard form.

In order to optimize the variational objective \eqref{vb_optimization} we
see from \eqref{stick_log_post} that we need to evaluate expectations of
the form
%
\begin{align*}
%
\expect{\q(\nu_\k \vert \eta)}{\log \nu_\k}
\textrm{,}\quad
\expect{\q(\nu_\k \vert \eta)}{\log (1 - \nu_\k)}
\textrm{,}\quad\textrm{and}\quad
\expect{\q(\nu_\k \vert \eta)}{\log \pstick(\nu_\k)}.
%
\end{align*}
%
