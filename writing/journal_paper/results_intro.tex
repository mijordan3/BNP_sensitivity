We evaluate the prior sensitivity in BNP models applied to three distinct data analysis examples.
We first fit a Gaussian mixture model (\exref{iris_bnp_process}) to the canonical iris data set.
Secondly, we cluster time-course gene expression data using a regression model
and study the resulting co-clustering matrix.
Finally, we fit a topic model
on a data set of sampled genotypes in an endangered bird species.
From the inferred population structure,
we reconstruct ancestral migration patterns.

In each data example, we first fit the variational approximation to a model
with a $\gem$ prior at some chosen parameter $\alpha = \alpha_0$.
We then evaluate sensitivity to the chosen $\alpha$ parameter
by refitting the variational approximation for each $\alpha$
in a set $\{\alpha_1, ..., \alpha_m\}$ of plausible values.
We also examine the effects of changing the functional form the Beta prior itself,
using the influence function to guide our choice of prior perturbation.
To speed up the refitting process, we used the intitial
variational optimum at $\alpha = \alpha_0$ as a warm-start
for subsequent refits after a prior pertrubation.
For each prior perturbation,
we validate the performace of the linear approximation against
re-fitting the model.
