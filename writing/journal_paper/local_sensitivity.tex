We would like to understand how our quantity of interest $g(\etaopt)$ changes
when the concentration parameter or, more generally, the stick-breaking density
$\pstick$ changes. To efficiently compute these changes, we use a first-order
Taylor series approximation in the optimal VB parameters. In this section, we
first give the Taylor series and then show how to compute its terms.

%%
\noindent \textbf{Sensitivity to the concentration parameter.} First, we show
how to approximate the sensitivity of $g(\etaopt)$ to the choice of
concentration parameter $\alpha$. Let $\etaopt(\alpha)$ represent the value of
$\etaopt$ for a particular choice of $\alpha$. For our approximation, we choose
some initial value $\alpha_0$ of the concentration parameter and solve the
optimization problem to compute $\etaopt(\alpha_0)$. We then approximate
$\etaopt(\alpha)$ with the linear approximation $\etalin(\alpha)$, and in turn
approximate $g(\etaopt(\alpha))$ with $g(\etalin(\alpha))$:
%
\begin{align*}
%
\etalin(\alpha) :=
    \etaopt(\alpha_0) +
    \fracat{d\etaopt(\alpha)}{d\alpha}{\alpha_0} (\alpha - \alpha_0)
\mathand
g(\etaopt(\alpha)) \approx g(\etalin(\alpha)).
%
\end{align*}
%
If $\alpha \mapsto \etaopt(\alpha)$ is continuously differentiable, and $g$ is
sufficiently smooth, then we expect $g(\etaopt(\alpha)) \approx
g(\etalin(\alpha))$ when $\abs{\alpha - \alpha_0}$ is small.

We show (\thmref{etat_deriv} in \appref{diffable_concentration}) that the map
$\alpha \mapsto \etaopt(\alpha)$ is continuously differentiable for the VB
approximation given in \secref{model_vb} under the following set of mild
conditions on the VB objective and optimum.


%%%%%%%%%%%%%%%%%%%%%%%%%%%%%%%%%%%%%%%%%%%%%%%%%%%%%%%%%%%%%%%%%%%%%%%%%%%%
%%%%%%%%%%%%%%%%%%%%%%%%%%%%%%%%%%%%%%%%%%%%%%%%%%%%%%%%%%%%%%%%%%%%%%%%%%%%
\begin{assu}\assulabel{kl_opt_ok}
%
Assume that:
%
%%%%%%%%%%%%%%%%%%%%%%%%%%%%%%%%%%%%%%%%%%%%%%%%%%%%%%%%%%%%%%%%%%%%%%
\begin{enumerate*}[label=(\arabic*)]
%
    \item \itemlabel{kl_diffable} The map $\eta \mapsto \KL{\eta}$ is twice
    continuously differentiable at $\etaopt$;

    \item\itemlabel{kl_hess} The Hessian matrix $\fracat{\partial^2 \KL{\eta}}
    {\partial \eta \partial \eta^T} {\etaopt}$ is non-singular; and

    \item \itemlabel{kl_opt_interior} There exists an open ball $\ball_\eta
    \subset \mathbb{R}^\etadim$ such that $\etaopt \in \ball_\eta \subset
    \etadom$.
%
\end{enumerate*}
%
\end{assu}
%%%%%%%%%%%%%%%%%%%%%%%%%%%%%%%%%%%%%%%%%%%%%%%%%%%%%%%%%%%%%%%%%%%%%%%%%%%%


%%
\noindent \textbf{Sensitivity to the stick-breaking density.} Next, we show how
to approximate the sensitivity of $g(\etaopt)$ to the choice of concentration
stick distribution $\pstick$. Technically, perturbations of $\alpha$ are
perturbations of $\pstick$. But here we consider more general perturbations of
the form of $\pstick$, potentially outside the beta class. To define our
perturbations, let $\ptil$ represent a potentially unnormalized (but
normalizable) density with respect to Lebesgue measure; the same notation
without the tilde will give the normalized density. Now start from an initial
setting of $\pstick$ at $\pbase$; we will typically start from Dirichlet-process
stick-breaking, i.e.\ $\pbase = \betadist{1,\alpha_0}$. Then take any
Lebesgue-measurable function $\phi(\cdot)$ on $[0,1]$. We consider a range of
alternative (potentially unnormalized) stick-breaking forms $\ptil(\cdot \vert
\t)$ defined by
%
\begin{equation} \eqlabel{mult_perturbation}
	\log \ptil(\cdot \vert \t) = \log \pbase(\cdot) + \t \phi(\cdot)
\end{equation}
%
on $[0,1]$. Note that the perturbation applies equally to every stick break
$\nuk$. This style of multiplicative functional perturbation was proposed by
\citet{gustafson:1996:local}; we deviate from \citet{gustafson:1996:local} by
allowing $\phi$ to take on negative values and by considering VB (rather than
MCMC) approximations.

If we now let $\etaopt(t)$ represent the value of $\etaopt$ for a particular
choice of $\ptil(\cdot \vert \t)$, we can form an analogous approximation to the
$\alpha$ case above:
%
\begin{align} \eqlabel{taylor_series_t}
%
% \etaopt(\alpha) \approx
\etalin(\t) :=
    \etaopt(0) +
    \fracat{d\etaopt(t)}{dt}{\t=0} (\t - 0)
\mathand
g(\etaopt(\t)) \approx g(\etalin(\t)).
%
\end{align}
%
As in the $\alpha$ case, we will show that the map $t \mapsto \etaopt(t)$ is
continuously differentiable \todo{TODO: back this up, possibly with concise
additional assumptions.  \par Boundedness of $\phi$ is what you need.
Should that go here, or later?}. So, for sufficiently smooth $g$, we expect this
approximation to be good for small $t$.

We next provide some additional justification for why this particular functional
perturbation is useful. Note first that if we consider any other distribution
$\palt$ for $\pstick$, we can continuously warp $\pbase$ to $\palt$ by setting
$\phi(\cdot) = \log \left( \palt(\cdot) / \pbase(\cdot) \right)$ so long as
$\palt \ll \pbase$, i.e.\ $\palt$ is absolutely continuous with respect to
$\pbase$. Moreover, we will see in \secref{influence_function} that we can
compute an \emph{influence function} to provide an interpretable summary of the
effect of arbitrary changes $\phi$ and also find a worst-case $\phi$ in some
appropriate ball. Second, while the perturbations above are essentially
multiplicative relative to $\pbase$, one might wonder whether additive
perturbations (subject to renormalization) or other perturbations would work
instead; indeed \citet{gustafson:1996:local} proposes a number of alternative
perturbations. We show in \secref{influence_function} that, among a class of
potential functional perturbations, the one we consider here is the only one
that admits a uniformly good approximation over an appropriate ball of $\phi$
choices -- and thus can be used to reason about worst-case $\phi$.
\todo{I'd like to tighten up the language about ``uniformly good approximation''}

%%
\noindent \textbf{Computing the terms in the Taylor series.} First observe that
we can write the $\alpha$ perturbation using the same notation as the stick-form
perturbation by setting $t = \alpha - \alpha_0$. Then, as before, let
$\ptil(\cdot \vert \t)$ be the corresponding (potentially unnormalized)
distribution on a stick-breaking parameter, and let $\etaopt(t)$ be the value of
$\etaopt$ for $\ptil(\cdot \vert \t)$. It follows that the concentration
parameter perturbation is also given by \eqref{taylor_series_t}. Whether we
perturb the concentration parameter or stick-breaking form, $\etaopt(0)$ is the
VB optimum for the original, unperturbed problem, and $1-0$ is trivial to
compute. It remains to provide a formula for the derivative
$\fracat{d\etaopt(t)}{dt}{\t=0}$.
%
\todo{Something may be confusing to the reader here.  We have observed that the
$\alpha$ perturbation could have been represented with a particular choice of
$\phi$.  Now we are suggesting replacing $\alpha$ with $\t$. We should be clear
that we are not suggesting using the functional parameterization for all
hyperparameter perturbations.  The two happen to be equivalent in this case
because the beta prior is log-linear in $\alpha$, but in general, the two
parameterizations are not equivalent.}

Our next result gives a computable formula for the derivative. In what follows,
we write $\qtil$ for a potentially unnormalized (but normalizable) approximation
distribution; as with $\ptil$ and $\p$, we again assume $\q$ is the normalized
form of $\qtil$.
%
\begin{thm}\thmlabel{bnp_deriv}
%
\todo{I think this should all work given the proofs in the appendix for
$\alpha$. Can we get away with adding some very minimal assumptions here to make
sure it applies for the functional perturbation?
\par\vspace{1em}
Ryan: I see a few options.
\par\vspace{1em}
1) Say that this only applies to the $\phi$ and $\alpha$ peturbations.
That seems unnecessarily limiting, because it's actually quite general.
\par\vspace{1em}
2) Say that it applies whenever you can apply the implicit function theorem.
That's basically what the original assumptions provide.  That puts us in the
spot of either saying them here or referring to things in the appendix.
\par\vspace{1em}
3) It would be great if we could say something like {\em when}
$\t \mapsto \etaopt(\t)$ is continuously differentiable, its derivative has
this form.  I doubt that's true, but I could be wrong.  AFAIK the implication
of the implicit function theorem only goes one way.
\par\vspace{1em}
4) Take option (1), but say in the text that it applies generally, and refer
to a proof in the appendix.
}
%
The map $\t \mapsto
\etaopt(\t)$ is continuously differentiable at $\t=0$ with derivative
%
\begin{align}
%
\eqlabel{bnp_vb_eta_sens}
\fracat{d \etaopt(\t)}{d \t}{\t=0} ={}&
    - \hessopt^{-1} \crosshessian, \mathwhere \\
%
\hessopt :={}& \fracat{\partial^2 \KL{\eta}}
                      {\partial \eta \partial \eta^T}
                      {\eta = \etaopt},
\quad \lqgradbar{\zeta \vert \eta} :={}
    \lqgrad{\zeta \vert \eta} -
    \expect{\q(\zeta \vert \eta)}{\lqgrad{\zeta \vert \eta}}, \\
%
\eqlabel{bnp_vb_crosshessian}
\crosshessian :={}&
%    \fracat{\partial
%            \expect{\q(\zeta \vert \eta)}
%                   {\fracat{\partial \log \prod_{k=1}^{\kmax -1}\ptil(\nu_k \vert \t)}
%                           {\partial \t}{\t=0} }
%            }
%        {\partial \eta}{\eta = \etaopt} \\
%={}&
    \expect{\q(\zeta \vert \etaopt)}{
          \lqgradbar{\zeta \vert \etaopt}
          \sum_{k=1}^{\kmax-1}
          \fracat{\partial \log \ptil(\nu_k \vert \t)}
                 {\partial \t}{\t=0}}.
%
\end{align}
%
\todo{We had the other form of the cross Hessian for a reason, make sure
it's still covered later.}
%
\end{thm}
%
\begin{proof}
%
Under differentiability and regularity conditions on a (generic, not just BNP)
model and VB approximation, we show in \thmref{etat_deriv} that the maps $t
\mapsto \etaopt(t)$ are differentiable and the formula for their computation
holds. \appref{diffable_concentration} shows that the differentiability and
regularity conditions hold when we vary the concentration parameter $\alpha$ in
our stick-breaking model.
\todo{Fix this proof once we decide about the assumptions.}
%
\end{proof}
%

\eqref{vb_eta_sens} requires computation of two terms: $\hessopt^{-1}$ and
$\crosshessian$.  Typically, $\crosshessian$, which is a derivative of a
variational expectation, is straightforward to evaluate: the requisite
expectation is evaluated either in closed form or approximated numerically;
then, in either case, an application of automatic differentiation provides the
gradient. Forming and inverting or factorizing $\hessopt$ can present a
challenge due to its high dimensionality. $\hessopt$ has dimensions $\etadim
\times \etadim$, where $\etadim$ is the dimension of $\eta$. There are
variational parameters in $\eta$ corresponding to each parameter $\z_\n$. Also,
the truncation level $\kmax$ typically needs to grow with $N$ to keep pace with
the growing number of clusters -- though this growth is sublinear. It follows
from these observations that $\etadim$ can grow linearly with $N$ and thus
become very large.  However, in many cases -- including our BNP problem here --
we can take advantage of model sparsity to efficiently compute
\eqref{vb_eta_sens}. We show how to do so in \secref{computing_sensitivity}. And
our experiments confirm that we can compute $\fracat{d \etaopt(\t)}{d \t}{\t=0}$
much more efficiently than re-optimizing the VB objective directly
(\secref{compute_time}). Moreover, we will see that most of the cost of using
the Taylor series approximation can be shared across multiple values of $t$, so
the savings increases dramatically when we are interested in a range of $t$
values (\secref{compute_time}).

%%
\noindent \textbf{More on accuracy and speed trade-offs in the Taylor series.}
One might save even more compute time, though likely at the cost of some
accuracy, by extending the approximation to $g$ as well. If the cost of
computing $g(\etalin(t))$ is prohibitive and if $g(\eta)$ is itself continuously
differentiable, we may employ an additional first-order expansion of $g$ to
further approximate $g(\etalin(t))$. In all of our experiments, computing
$g(\etalin(t))$ is not prohibitive, so we do not employ this additional
approximation.
\todo{The other big reason to linearize $g$ is for the influence function.}

Finally, we note that even when the approximation $\etalin(t)$ is an imperfect
substitute for exact re-optimization, the derivative can be a useful guide for
what sorts of prior perturbations might be problematic and thereby inform
further exploration based on re-optimizing; see \secref{results_structure} for
an example and discussion.
