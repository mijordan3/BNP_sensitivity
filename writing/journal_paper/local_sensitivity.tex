We now turn to the optimization problem \eqref{vb_optimization} where the priors
$\pstick(\nuk)$ which enter the term $\logp(\zeta)$ depend on a real-valued
parameter, $\t \in \tdom \subseteq \mathbb{R}$, writing $\pstick(\nuk \vert
\t)$.  Starting now, we will state general results in terms of a generic
parameter $\theta$, specializing to the BNP problem in examples and in the
experiments below.

%

%%%%%%%%%%%%%%%%%%%%%%%%%%%%%%%%%%%%%%%%%%%%%%%%%%%%%%%%%%%%%%%%%%%%%%%%%%%
%%%%%%%%%%%%%%%%%%%%%%%%%%%%%%%%%%%%%%%%%%%%%%%%%%%%%%%%%%%%%%%%%%%%%%%%%%%%
\begin{defn}\deflabel{prior_t}
%
Let $\mu$ denote a sigma finite measure, and let $\p(\theta \vert \t)$ denote a
class of probability densities relative $\mu$.  Let $\p(\theta \vert \t)$ be
defined for $\t$ in an open set $\ball_\t \subseteq \mathbb{R}$ containing $0$.
Assume that the variational densities $\q(\theta \vert \eta)$ are also defined
relative to $\mu$.
\todo{need a note about how the truncated approximation uses point masses}

Let $\qtil$ and $\ptil$ refer to potentially
unnormalized (but normalizable) versions of the respectively corresponding $\q$
and $\p$, so that,
%
\begin{align*}
%
\q(\theta \vert \eta) :={}
    \frac{\qtil(\theta \vert \eta)}
    {\int \qtil(\theta' \vert \eta) \mu(d\theta')} \mathand
\p(\theta \vert \t) :={}
    \frac{\ptil(\theta \vert \t)}
    {\int \ptil(\theta' \vert \t) \mu(d\theta')}.
%
\end{align*}
%
\end{defn}
%%%%%%%%%%%%%%%%%%%%%%%%%%%%%%%%%%%%%%%%%%%%%%%%%%%%%%%%%%%%%%%%%%%%%%%%%%%%

The prior depends on $\t$; in turn, the posterior $\p(\theta \vert \x, \t)$
depends on $\t$; in turn, the KL divergance depends on $\t$; in turn, the
optimal variational parameters depend on $\t$.  Define the shorthand notation
%
\begin{align}\eqlabel{kl_shorthand}
%
\KL{\eta, \t} := \KL{\q(\theta \vert \eta) || \p(\x \vert \zeta, \t)}
\mathand
\etaopt(\t) := \argmin_{\zeta \in \etadom} \KL{\eta, \t},
%
\end{align}
%
where we write $\etaopt(\t)$ to emphasize the dependence of the optimum on $\t$.
In \defref{prior_t}, we take $\t = 0$ at the ``original'' problem,
\eqref{vb_optimization}, without loss of generality.  We will thus continue to
use $\etaopt$ with no argument to refer to $\etaopt(0)$.


%%%%%%%%%%%%%%%%%%%%%%%%%%%%%%%%%%%%%%%%%%%%%%%%%%%%%%%%%%%%%%%%%%%%%%%%%%%%%%%%
%%%%%%%%%%%%%%%%%%%%%%%%%%%%%%%%%%%%%%%%%%%%%%%%%%%%%%%%%%%%%%%%%%%%%%%%%%%%%%%%
\begin{ex}\exlabel{alpha_perturbation}
%
When drawing from the classical $\mathrm{GEM}(\alpha)$ distribution, we take
$\mu$ to be the Lebesgue measure on $[0,1]$ and model
%
\begin{align*}
%
\pstick(\nuk \vert \alpha) ={}&
    \betadist{\nuk \vert 1, \alpha} \Rightarrow\\
\log \pstick(\nuk \vert \alpha) ={}&
    (\alpha - 1) \log(1 - \nuk) + \const. &
    \constdesc{\nuk}
%
\end{align*}
%
Fix some ``original'' $\alpha_0$.  In this case, we represent deviations from
the choice $\alpha_0$ by identifying $\t$ with $\alpha - \alpha_0$:
%
\begin{align*}
%
\log \pstick(\nuk \vert \t) ={}&
(\t + \alpha_0 - 1) \log(1 - \nuk) + \const.
\end{align*}
%
The prior on the full $\nu$ vector is given by
%
\begin{align*}
%
\log \pstick(\nu \vert \t) ={}&
    \sumkm (t + \alpha_0 - 1) \log(1 - \nuk)+ \const.
%
\end{align*}
%
\end{ex}
%%%%%%%%%%%%%%%%%%%%%%%%%%%%%%%%%%%%%%%%%%%%%%%%%%%%%%%%%%%%%%%%%%%%%%%%%%%%%%%%

Assuming for the moment that $\t \mapsto \etaopt(\t)$ is continuously
differentiable (we will consider its differentiability in detail below), and
that we have already computed the solution $\etaopt$ to the ``original'' problem
\eqref{vb_optimization} with $\t=0$, then we can form a Taylor series
approximation to $\etaopt(\t)$.  Specifically, we define
%
\begin{align}\eqlabel{etalin_def}
%
\etalin(\t) := \etaopt + \fracat{d \etaopt(\t)}{d \t}{\t=0} \t .
%
\end{align}
%
Evaluating $\etaopt(\t)$ requires solving a new optimization problem, but, given
$d\etaopt(\t) / d\t | 0$, evaluating $\etalin(\t)$ involves only
multiplication and addition.  When $|\t|$ is small, by continuous
differentiability of $\etaopt(\t)$, we might hope that $\etaopt(\t) \approx
\etalin(\t)$, and so we can use $\etalin(\t)$ to quickly approximate a
time-consuming optimization problem.

Futhermore, or functions of interest $\g(\eta)$ which are themselves
differentiable, we can use the chain rule to compute
%
\begin{align}\eqlabel{vb_g_sens}
%
\fracat{d g(\etaopt(\t))}{d\t}{\t=0}&
    \fracat{\partial g(\eta)}{\partial \eta^T}{\etaopt}
    \fracat{d \etaopt(\t)}{d \t}{\t=0} \\
\glin(\t) :={}& \g(\etaopt) + \fracat{d g(\etaopt(\t))}{d\t}{\t=0} \t.
%
\end{align}
%
For non-differentiable functions of $\eta$, we can still form the approximation
%
\begin{align*}
%
\gapprox(\t) :={}& \g(\etalin(\t)).
%
\end{align*}
%
The advantage of $\glin(\t)$ relative to $\gapprox(\t)$ is that for the former
we can compute influence functions and worst-case perturbations, as we discuss
below.  Converse, $\gapprox(\t)$ may be expected to provide a better
approximation in some cases since it retains non-linearities in the map $\eta
\mapsto \g(\eta)$, linearizing only the computationally intensive map $\t
\mapsto \etaopt(\t)$.

When, then, is $\etaopt(\t)$ continuously differentiable?  We will now state
some sufficient conditions under which we can apply the implicit function
theorem (e.g., \citet{krantz:2012:implicit}) to prove the continuous
differentiability of $\etaopt(\t)$.

The first key assumption, \assuref{dist_fun_nice}, states sufficient conditions
for which we can apply the dominated convergence theorem to variational
expectations of some generic function $\psi(\theta, \t)$, allowing us to
translate continuity of the variational and model densitites into continuity of
the variational objective.  The details can be found in \lemref{logq_derivs,
logq_continuous} of \appref{cont_lemmas}.
%
In case \assuref{dist_fun_nice} seems forbidding, observe that, in lieu of
\assuref{dist_fun_nice} we might equivalently have said that we can exchange
limits and variational expectations whenever needed.  \Assuref{dist_fun_nice} is
simply a precise catalogue of what is needed.

%%%%%%%%%%%%%%%%%%%%%%%%%%%%%%%%%%%%%%%%%%%%%%%%%%%%%%%%%%%%%%%%%%%%%%%%%%%%
%%%%%%%%%%%%%%%%%%%%%%%%%%%%%%%%%%%%%%%%%%%%%%%%%%%%%%%%%%%%%%%%%%%%%%%%%%%%
\begin{assu}\assulabel{dist_fun_nice}
%
% Let $\qtil(\theta \vert \eta)$ be a (integrable, but possibly unnormalized)
% density over the random variable $\theta$ parameterized by $\eta$ defined
% relative to a dominating measure $\mu$.
%
Assume that the map $\eta \mapsto \log \qtil(\theta \vert \eta)$ is twice
continuously differentiable. Let $\psi(\theta, \t)$ be a scalar-valued
$\mu$-measurable function of $\theta$ and $\t$.  Assume that the map $\t \mapsto
\psi(\theta, \t)$ is continuously differentiable.

Define the following shorthand notation:
%
\begin{align*}
%
\lqgrad{\theta \vert \eta} :={}&
    \fracat{\partial \log \qtil(\theta \vert \eta)}{\partial \eta}{\eta} \\
%
\lqhess{\theta \vert \eta} :={}&
    \fracat{\partial^2 \log \qtil(\theta \vert \eta)}
           {\partial \eta \partial \eta^T}{\eta} \\
%
\psigrad{\theta, \t} :={}& \fracat{\partial \psi(\theta, \t)}{\partial \t}{\t}.
%
\end{align*}
%
For a given $\t_0$ and $\eta_0$, assume there exists some neighborhood of
$\t_0$, $\ball_\t$, some neighborhood of $\eta_0$, $\ball_\eta$, and a
$\mu$-integrable $M_\psi(\theta)$ with $\int M_\psi(\theta) \mu(d\theta) <
\infty$ such that the following bounds hold for all $\eta, \t \in \ball_\eta
\times \ball_\t$:
%
\begin{enumerate}
%
\item \itemlabel{fundom}
$\qtil(\theta \vert \eta) \psi(\theta, \t) \le M_\psi(\theta)$.
%
\item \itemlabel{funqgraddom}
$\qtil(\theta \vert \eta) \norm{\lqgrad{\theta \vert \eta}}_2 \psi(\theta, \t) \le
M_\psi(\theta)$.
%
\item \itemlabel{funqhessdom}
$\qtil(\theta \vert \eta) \norm{\lqhess{\theta \vert \eta}}_2 \psi(\theta, \t) \le
M_\psi(\theta)$.
%
\item \itemlabel{fungradqgraddom}
$\qtil(\theta \vert \eta) \norm{\lqgrad{\theta \vert \eta}}_2 \psigrad{\theta, \t}
\le M_\psi(\theta)$.
%
\item \itemlabel{funqgradsqdom}
$\qtil(\theta \vert \eta) \norm{\lqgrad{\theta \vert \eta}}^2_2 \psi(\theta, \t) \le
M_\psi(\theta)$.
%
\end{enumerate}
%
\end{assu}
%%%%%%%%%%%%%%%%%%%%%%%%%%%%%%%%%%%%%%%%%%%%%%%%%%%%%%%%%%%%%%%%%%%%%%%%%%%%

%%%%%%%%%%%%%%%%%%%%%%%%%%%%%%%%%%%%%%%%%%%%%%%%%%%%%%%%%%%%%%%%%%%%%%%%%%%%
%%%%%%%%%%%%%%%%%%%%%%%%%%%%%%%%%%%%%%%%%%%%%%%%%%%%%%%%%%%%%%%%%%%%%%%%%%%%
\begin{lem}
%
Under \assuref{dist_fun_nice}, we can exchange the order of integration
and differentiation in
%
\begin{align*}
%
\fracat{\partial\expect{\q(\theta \vert \eta)}{\psi(\theta, \t)}}
       {\partial\eta}{\etaopt, \t=0}
\textrm{, }
\fracat{\partial^2\expect{\q(\theta \vert \eta)}{\psi(\theta, \t)}}
       {\partial\eta \partial\eta}{\etaopt, \t=0}
\textrm{, and }
\fracat{\partial^2\expect{\q(\theta \vert \eta)}{\psi(\theta, \t)}}
       {\partial\eta \partial \t}{\etaopt, \t=0}.
%
\end{align*}
%
(See also \lemref{logq_derivs, logq_continuous} of \appref{cont_lemmas}
for a more detailed statement.)
%
\end{lem}
%%%%%%%%%%%%%%%%%%%%%%%%%%%%%%%%%%%%%%%%%%%%%%%%%%%%%%%%%%%%%%%%%%%%%%%%%%%%

Next, in \assuref{kl_opt_ok}, we first require some regularity conditions on
the KL divergence and its optimum.

%%%%%%%%%%%%%%%%%%%%%%%%%%%%%%%%%%%%%%%%%%%%%%%%%%%%%%%%%%%%%%%%%%%%%%%%%%%%
%%%%%%%%%%%%%%%%%%%%%%%%%%%%%%%%%%%%%%%%%%%%%%%%%%%%%%%%%%%%%%%%%%%%%%%%%%%%
\begin{assu}\assulabel{kl_opt_ok}
%
Let the following conditions on the variational approximation hold.
%
%%%%%%%%%%%%%%%%%%%%%%%%%%%%%%%%%%%%%%%%%%%%%%%%%%%%%%%%%%%%%%%%%%%%%%
\begin{enumerate}
%
    \item \itemlabel{kl_diffable} The map $\eta \mapsto \KL{\eta, 0}$ is twice
    continuously differentiable at $\etaopt$.

    \item \itemlabel{kl_opt_interior} The optimal $\etaopt$ is interior
    to $\etadom$.

    \item\itemlabel{kl_hess} The Hessian matrix $\fracat{\partial^2 \KL{\eta,
    0}} {\partial \eta \partial \eta^T} {\etaopt}$ is positive definite.

    \item\itemlabel{kl_dct} The unnormalized variational densities $\qtil(\theta
    \vert \eta)$ satisfy \assuref{dist_fun_nice} with $\psi(\theta, \t) \equiv
    1$ (no $\theta$ dependence).
%
\end{enumerate}
%
\end{assu}
%%%%%%%%%%%%%%%%%%%%%%%%%%%%%%%%%%%%%%%%%%%%%%%%%%%%%%%%%%%%%%%%%%%%%%%%%%%%

Observe that \assuref{dist_fun_nice}, applied with $\logp(\x \vert \theta)$, is
one way to prove \assuitemref{kl_opt_ok}{kl_diffable}.  Since our primary focus
is on the prior $\logp(\theta \vert \t)$, we prefer to simply state
\assuitemref{kl_opt_ok}{kl_diffable} directly and reserve our detailed attention
for the prior $\logp(\theta \vert \t)$.

Finally, in \assuref{q_stick_regular} we draw the needed connection between the
class of prior perturbations and the variational approximation.

%%%%%%%%%%%%%%%%%%%%%%%%%%%%%%%%%%%%%%%%%%%%%%%%%%%%%%%%%%%%%%%%%%%%%%%%%%%%
%%%%%%%%%%%%%%%%%%%%%%%%%%%%%%%%%%%%%%%%%%%%%%%%%%%%%%%%%%%%%%%%%%%%%%%%%%%%
\begin{assu}\assulabel{q_stick_regular}
%
Under \defref{prior_t}, assume that the variational densities $\q(\theta \vert
\eta)$ satisfy \assuref{dist_fun_nice} with both $\psi(\theta, \t) \equiv 1$ (no
$\theta$ dependence) and with and $\psi(\theta, \t) = \log \p(\theta \vert \t) -
\log \p(\theta \vert \t=0)$.
%
\end{assu}
%%%%%%%%%%%%%%%%%%%%%%%%%%%%%%%%%%%%%%%%%%%%%%%%%%%%%%%%%%%%%%%%%%%%%%%%%%%%


%%%%%%%%%%%%%%%%%%%%%%%%%%%%%%%%%%%%%%%%%%%%%%%%%%%%%%%%%%%%%%%%%%%%%%%%%%%%
%%%%%%%%%%%%%%%%%%%%%%%%%%%%%%%%%%%%%%%%%%%%%%%%%%%%%%%%%%%%%%%%%%%%%%%%%%%%
\begin{thm}\thmlabel{etat_deriv}
%
Under the conditions of \defref{prior_t}, let \assuref{kl_opt_ok} hold at
$\eta_0 = \etaopt$.  Further, let \assuref{dist_fun_nice} hold with
%
\begin{align*}
%
\psi(\theta, \t) = \log \p(\theta \vert \t) - \log \p(\theta \vert \t=0)
%
\end{align*}
%
at $\t_0 = 0$.
%
Define\footnote{Note that if $\qtil(\theta \vert \eta)$ is already normalized
($\qtil = \q$), then $\expect{\q(\theta \vert \eta)}{\lqgrad{\theta \vert \eta}} =
0$ for all $\eta$ and $\lqgradbar{\theta \vert \etaopt} = \lqgrad{\theta \vert
\etaopt}$. Since it can sometimes be more convenient to work with unnormalized
distributions, we keep the notation general.}
%
\begin{align*}
%
\lqgradbar{\theta \vert \etaopt} :={}&
    \lqgrad{\theta \vert \etaopt} -
    \expect{\q(\theta \vert \etaopt)}{\lqgrad{\theta \vert \etaopt}}, \\
\crosshessian :={}&   \expect{\q(\theta \vert \etaopt)}{
      \lqgradbar{\theta \vert \etaopt}
      \fracat{\partial \log \p(\theta \vert \t)}{\partial \t}{\t=0}} \mathand \\
\hessopt :={}& \fracat{\partial^2 \KL{\eta, 0}}
                      {\partial \eta \partial \eta^T}
                      {\etaopt}.
%
\end{align*}
%
Then the map $\t \mapsto \etaopt(\t)$ is continuously differentiable at $\t=0$
with derivative
%
\begin{align}\eqlabel{vb_eta_sens}
%
\fracat{d \etaopt(\t)}{d \t}{0} ={}&
    - \hessopt^{-1}
    \crosshessian
    % \expect{\q(\theta \vert \etaopt)}{
    %     \lqgradbar{\theta \vert \etaopt}
    %     \fracat{\partial \log \p(\theta \vert \t)}{\partial \t}{\t=0}
    % }.
    % \fracat{\partial^2 \KL{\eta, \t}}
    %        {\partial \eta \partial \t}
    %        {\etaopt, 0}.
%
\end{align}
%
(For a proof, see \appref{proofs} \proofref{etat_deriv}.)
%
\end{thm}
%%%%%%%%%%%%%%%%%%%%%%%%%%%%%%%%%%%%%%%%%%%%%%%%%%%%%%%%%%%%%%%%%%%%%%%%%%%%

We conclude this section by showing that \thmref{etat_deriv} applies to the
setting of BNP stick-breaking.

%%%%%%%%%%%%%%%%%%%%%%%%%%%%%%%%%%%%%%%%%%%%%%%%%%%%%%%%%%%%%%%%%%%%%%%%%%%%
%%%%%%%%%%%%%%%%%%%%%%%%%%%%%%%%%%%%%%%%%%%%%%%%%%%%%%%%%%%%%%%%%%%%%%%%%%%%
\begin{lem}\lemlabel{normal_q_is_regular}
%
In the setting of \assuref{dist_fun_nice}, let $\theta \in \mathbb{R}$, let
$\mu$ be the Lebesgue measure, and let $\q(\theta \vert \eta) = \normdist{\theta
\vert \eta}$ be a normal distribution parameterized by its natural exponential
family parameters.

Let $\sigma(\eta)$ denote the standard deviation of the normal distribution.
Fix $\eta_0$ such that $\sigma(\eta_0) > 0$, and let $\ball_\eta$ be an open set
containing $\eta_0$ such a that $\sigma_{max} := \sup_{\eta \in \ball_\eta}
\sigma(\eta) < \infty$.

If there exists a neighborhood $\ball_\t$ of $\t_0$ such that $\abs{\psi(\theta,
\t)}$ and $\abs{\psigrad{\theta, \t}}$ are uniformly bounded by some multiple of
$\exp(-\abs{\theta})$ for all $\t \in \ball_\t$, then $\q$ and $\psi$ satisfy
\assuref{dist_fun_nice}.

\begin{proof}
%
By properties of the exponential family,
%
\begin{align*}
%
\lqgrad{\theta, \eta} ={} (\theta, \theta^2)^T \mathand&
\lqhess{\theta, \eta} ={} 0_{2\times2} \Rightarrow\\
%
\norm{\lqgrad{\theta, \eta}}_2^2 ={} \theta^2 + \theta^4 \mathand&
\norm{\lqhess{\theta, \eta}}_2 ={} 0.
%
\end{align*}
%
Let $\ballclosed_\eta$ denote the closure of $\ball_\eta$, and let
%
\begin{align*}
%
\eta^* := \argmax_{\eta \in \ballclosed_\eta}
    \expect{\q(\theta \vert \eta)}{\exp(\abs{\theta})}.
%
\end{align*}
%
By standard properties of the normal and the boundedness of $\sigma(\eta)$, the
right hand side of the preceding display is finite.
%
Then
%
\begin{align*}
\int \q(\theta \vert \eta) \psi(\theta, \t) \mu(d \theta) \le{}&
    \left( \sup_{\theta} \sup_{\t \in \ball_\t}
        \abs{\psi(\theta, \t)} \exp(-\abs{\theta}) \right)
    \int \q(\theta \vert \eta) \exp(\theta) \mu(d \theta)
%
\\\le{}&
    \const
    \expect{\q(\theta \vert \eta^*)}{\exp(\abs{\theta})}.
    \quad\constdesc{\eta, \t}
%
\end{align*}
%
Therefore, for \assuitemref{dist_fun_nice}{fundom}, we can take $M(\theta)
\propto \q(\theta \vert \eta^*) \exp(\abs{\theta})$. The other terms follow
similarly, since each multiplier of $\q(\theta \vert \eta)$ is dominated by
$\exp(-\abs{\theta})$.  The final $M(\theta)$ simply takes the largest
of the five constants.
%
\end{proof}
%
\end{lem}
%%%%%%%%%%%%%%%%%%%%%%%%%%%%%%%%%%%%%%%%%%%%%%%%%%%%%%%%%%%%%%%%%%%%%%%%%%%%


%%%%%%%%%%%%%%%%%%%%%%%%%%%%%%%%%%%%%%%%%%%%%%%%%%%%%%%%%%%%%%%%%%%%%%%%%%%%
%%%%%%%%%%%%%%%%%%%%%%%%%%%%%%%%%%%%%%%%%%%%%%%%%%%%%%%%%%%%%%%%%%%%%%%%%%%%
\begin{ex}\exlabel{gem_pert_ok}
%
To analyze \exref{alpha_perturbation}, we take $\theta$
in \lemref{normal_q_is_regular}
be the unconstrained stick-breaking proportion $\lnuk$, which
recall from \secref{stick_expectations}
was normally distributed under $\q$.
Let $\mu$ be the Lebesgue measure on $\mathbb{R}$.

In \exref{alpha_perturbation},
the parameterization of the stick-breaking distribution was given by
\begin{align*}
  \log \pstick(\nuk \vert \t) - \log \pstick(\nuk \vert \t=0) ={}&
  \t \log(1 - \nuk).
\end{align*}
%
Expressing the perturbation in terms of
$\lnuk$,
%
\begin{align*}
%
\log \pstick(\lnuk \vert \t) - \log \pstick(\lnuk \vert \t=0) ={}&
\t \log\left(1 - \frac{\exp(\lnuk)}{1 + \exp(\lnuk)}\right)
\\={}&
-\t \log\left(1 + \exp(\lnuk)\right).
%
\end{align*}
%
So, by \lemref{normal_q_is_regular}, \assuref{q_stick_regular} is satisfied with
the VB approximation given in \secref{stick_expectations} and the parametric
perturbation given in \exref{alpha_perturbation}.
%
\end{ex}
%%%%%%%%%%%%%%%%%%%%%%%%%%%%%%%%%%%%%%%%%%%%%%%%%%%%%%%%%%%%%%%%%%%%%%%%%%%%



%%%%%%%%%%%%%%%%%%%%%%%%%%%%%%%%%%%%%%%%%%%%%%%%%%%%%%%%%%%%%%%%%%%%%%%%%%%%
%%%%%%%%%%%%%%%%%%%%%%%%%%%%%%%%%%%%%%%%%%%%%%%%%%%%%%%%%%%%%%%%%%%%%%%%%%%%

\begin{cor}\corlabel{gem_approximation_ok}
%
For the variational approximation of \secref{model_vb} and perturbation
given in \exref{alpha_perturbation}, $\alpha \mapsto \etaopt(\alpha)$
is continuously differentiable.
%
\begin{proof}
%
We have already shown in \exref{gem_pert_ok} that \assuref{q_stick_regular} is
satisfied.  Given that the variational approximations to $\p(\z \vert \x, \beta,
\nu)$ and $\p(\beta \vert \x, \nu, \z)$ are conjugate exponential family
approximations, $\eta \mapsto \KL{\eta, 0}$ is continuous.  It remains only to
numerically find $\etaopt$ and verify \assuitemref{kl_opt_ok}{kl_hess}, i.e.
that the Hessian is positive definite at the optimum.
%
\end{proof}
%
\end{cor}

%%%%%%%%%%%%%%%%%%%%%%%%%%%%%%%%%%%%%%%%%%%%%%%%%%%%%%%%%%%%%%%%%%%%%%%%%%%%

We conclude this section with a brief remark about computing the expectation
$\crosshessian$ in our BNP sensitivity analysis.
We are interested in sensitivity to the stick-breaking distribution,
so only the prior terms on stick-breaking proportions
$\nu = (\nu_1, ..., \nu_{\kmax - 1})$ depends on $t$.
Because the elements of $\nu$ fully factorize
under both the prior and the variational distributions,
$\crosshessian$ decomposes as
\begin{align}
  \crosshessian &=
  \sum_{\k=1}^{\kmax - 1}
          \expect{\q(\nuk \vert \eta)}
                 {
                 \lqgrad{\nuk \vert \etaopt}
                 \fracat{\log \pstick(\nuk \vert \t)}{\partial \t}{\t = 0}
                 } \notag\\
  &= \sum_{\k=1}^{\kmax - 1}
         \evalat{\nabla_\eta \expect{\q(\nuk \vert \eta)}
                {
                \fracat{\log \pstick(\nuk \vert \t)}{\partial \t}{\t = 0}
                }}{\eta = \etaopt(0)},
\eqlabel{sens_mixed_partial}
\end{align}
where we assumed that $\q(\theta \vert \eta)$ is normalized, so
$\lqgradbar{\theta \vert \etaopt} = \lqgrad{\theta \vert \etaopt}$,
and that the assumptions of \thmref{etat_deriv} hold, so we
can freely exchange derivatives with expectations.

We approximate the expectation using GH quadrature (\eqref{gh_integral}),
with
$f(\nu_k) = \fracat{\log \pstick(\nuk \vert \t)}{\partial \t}{\t = 0}$.
In all the functional forms for
$\t \mapsto \pstick(\nuk \vert \t)$ considered below,
$f(\nu_k)$ can be provided in either closed-form or computed with automatic differentiation.
The resulting GH approximation is a deterministic function of $\eta$,
and thus the gradient in \eqref{sens_mixed_partial} can be computed
with another application of automatic differentiation.
Note that $\crosshessian$ is sparse in \eqref{sens_mixed_partial}:
it is zero for all entries of
$\eta$ other than those that parameterize the sticks.
