Suppose we have a dataset $\x$, a real-valued parameter $\theta \in \thetadom
\subseteq \mathbb{R}^{\thetadim}$, and a log likelihood $\logp(\x | \theta)$.
Let us take a family of log prior densities, $\ell(\theta \vert \t)$,
parameterized by some scalar $\t \in \tdom \subseteq \mathbb{R}$, defined
relative to some dominating measure, $\lambda$.  We can write the posterior
density (relative to $\lambda$) and expectations as
%
\begin{align*}
%
p(\theta \vert \x, \t) :={}&
    \frac{p(\x \vert \theta) p(\theta \vert \t)}
         {\int_{\thetadom} p(\x \vert \tilde{\theta}) p(\tilde{\theta} \vert \t)
               \lambda(d \tilde{\theta})}
= \frac{p(\x \vert \theta) p(\theta \vert \t)}{p(\x \vert \t)} \\
\expect{p(\theta \vert \x, \t)}{g(\theta)} :={}&
    \int_{\thetadom} p(\theta \vert \x, \t) g(\theta) \lambda{d\theta}.
%
\end{align*}
%

Suppose also we have a family of variational distributions
expressed as densities with respect to the same dominating measure, $q(\theta
\vert \eta)$, parameterized by a variational parameter $\eta \in \etadom
\subseteq \mathbb{R}^{\etadim}$.

We will form the variational approximation by solving
%
\begin{align*}
%
\etaopt(\t) :={}&
    \argmin_{\eta \in \etadom} \KL{\q(\theta \vert \eta) || p(\x \vert \theta, \t)} \\
={}& \argmin_{\eta \in \etadom}
        \expect{\q(\theta \vert \eta)}{
        \log \q(\theta \vert \eta) - \logp(\x \vert \theta) -
        \logp(\theta \vert \t) + \logp(\x \vert \t)}.
%
\end{align*}
%
That is, we wish to find the variational distribution $\q(\theta \vert
\etaopt(\t))$ that is closest in KL-divergence to $p(\theta \vert \x, \t)$.  We
have emphasized in the notation $\etaopt(\t)$ that this optimum depends on the
prior hyperparameter $\t$.  Define the shorthand notation
%
\begin{align*}
%
\KL{\eta, \t} := \KL{\q(\theta \vert \eta) || p(\x \vert \theta, \t)}.
%
\end{align*}
%
Assuming that, for all $\t \in \tdom$, $\etaopt(\t)$ is interior to $\etadom$,
and that the first derivative $\partial \KL{\eta, \t}/ \partial \eta$ is
continuous, then $\etaopt(\t)$ satisfies the first-order condition
%
\begin{align}\eqlabel{vb_first_order_condition}
%
\fracat{\partial \KL{\eta, \t}}{ \partial \eta}{\etaopt(\t), \t} = 0.
%
\end{align}
%
If the second derivative
$\partial^2 \KL{\eta, \t}/ \partial \eta \partial \eta^T$ is
positive definite in a neighborhood of $\etaopt{\t}$ and $\t$,
then $\etaopt{\t}$ is a differentiable function of $\t$ by the
implicit function thoerem.  We can then differentiate
\eqref{vb_first_order_condition} to get
%
\begin{align*}
%
\fracat{d \etaopt(\t)}{d \t}{\t_0} ={}&
    - \left( \fracat{\partial^2 \KL{\eta, \t}}
                    {\partial \eta \partial \eta^T}
                    {\etaopt(\t), \t_0} \right)^{-1}
    \fracat{\partial^2 \KL{\eta, \t}}
           {\partial \eta \partial \t}
           {\etaopt(\t), \t_0}.
%
\end{align*}
%


%
