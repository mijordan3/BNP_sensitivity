\documentclass{article}

\usepackage{etoolbox}

% Set this toggle true to compile a version for the arxiv, and
% false to compile for the currently configured conference.
% NOTE: you also must set the single_column flags in the Rnw files to
% make side-by-side figures for the arxiv version.
\newtoggle{for_arxiv}
\toggletrue{for_arxiv}
%\togglefalse{for_arxiv}

\nottoggle{for_arxiv}{%
    \usepackage[accepted]{aistats2019} % hooray!
    % \usepackage{aistats2019}
}

\usepackage{microtype}
\usepackage{graphicx}
\usepackage{subfigure}
\usepackage{booktabs} % for professional tables
\usepackage{siunitx}
\usepackage{hyperref}
\usepackage{xargs}[2008/03/08]

\iftoggle{for_arxiv} {
    \usepackage[authoryear]{natbib}
} {
    \usepackage[round]{natbib}
    \renewcommand{\bibname}{References}
    \renewcommand{\bibsection}{\subsubsection*{\bibname}}
}


% Attempt to make hyperref and algorithmic work together better:
\newcommand{\theHalgorithm}{\arabic{algorithm}}


\usepackage{prettyref}
\usepackage{refstyle}

\usepackage{amsmath}
\usepackage{amssymb}
\usepackage{amsfonts}
\usepackage{amsthm}
\usepackage{mathrsfs}
\usepackage{mathtools}
\usepackage{colonequals}
\usepackage{algpseudocode, algorithm} %typical alg typesetting packages

\usepackage{listings}
\usepackage{pdfpages}

% Define things needed to incorporate the LyX appendix.
% We can define things redundantly if necessary, but please don't modify
% this file -- if there's a name clash for a function you want to define,
% please just pick a new name.
\input{lyx_commands.tex}

% This picks up the knitr boilerplate, allowing us to \input partial knitr
% documents.
\input{knitr_header.tex}

% Paper-specific math macros (not in the appendix).
\global\long\def\wcv{W_k}
\global\long\def\wboot{W^*_B}
\global\long\def\thetapw{\thetahat(w)}
\global\long\def\wdiff{\Delta w}


\newcommand{\eq}[1]{Eq.~\ref{eq:#1}}
\newcommand{\fig}[1]{Fig.~\ref{fig:#1}}
\newcommand{\rthm}[1]{Theorem~\ref{thm:#1}}
\newcommand{\sect}[1]{Section~\ref{sec:#1}}
\newcommand{\subsect}[1]{Section~\ref{subsec:#1}}
\newcommand{\assum}[1]{Assumption~\ref{assum:#1}}
\newcommand{\lemma}[1]{Lemma~\ref{lm:#1}}
\newcommand{\corollary}[1]{Corollary~\ref{cor:#1}}
\newcommand{\app}[1]{Appendix~\ref{app:#1}}
\newcommand{\appsect}[1]{Appendix~\ref{sec:#1}}

\newcommand{\coreassum}{Assumptions \ref{assu:paper_smoothness}--\ref{assu:paper_lipschitz} }
\newcommand{\paperallcoreassum}{Assumptions \ref{assu:paper_smoothness}--\ref{assu:paper_weight_bounded} }

\iftoggle{for_arxiv}{%
    \title{Evaluating Sensitivity to the Stick Breaking Prior in Bayesian Nonparametrics}
    \author{
      Ryan Giordano\\ \texttt{rgiordano@berkeley.edu}
      \and
      Runjing Liu\\ \texttt{runjing\_liu@berkeley.edu}
      \and
      Michael I.~Jordan\\ \texttt{jordan@cs.berkeley.edu}
      \and
      Tamara Broderick\\ \texttt{tbroderick@csail.mit.edu}
    }
}


\begin{document}

\iftoggle{for_arxiv} {
    \maketitle
}

\nottoggle{for_arxiv} {
\twocolumn[
    \aistatstitle{Evaluating Sensitivity to the Stick Breaking Prior in Bayesian Nonparametrics}

    \aistatsauthor{
        Ryan Giordano \And
        Runjing Liu \And
        Michael I.~Jordan \And
        Tamara Broderick
    }
    \aistatsaddress{
        UC Berkeley \And
        UC Berkeley \And
        UC Berkeley \And
        MIT
    }
]
}

\begin{abstract}
%
This is an abstract.
%
\end{abstract}

\section{Introduction}\label{sec:introduction}

A central question in many probabilistic clustering problems is how many
distinct clusters are present in a particular dataset. Bayesian nonparametrics
(BNP) addresses this question by placing a generative process on cluster
assignment, making the number of distinct clusters present amenable to Bayesian
inference.  However, like all Bayesian approaches, BNP requires the
specification of a prior, and this prior may favor a greater or fewer number of
distinct clusters. In practice, it is important to quantitatively establish that
the prior is not too informative, particularly when---as is often the case in
BNP---the particular form of the prior is chosen for mathematical convenience
rather than because of a considered subjective belief.

We derive local sensitivity measures for a truncated variational Bayes (VB)
approximation based on the Kullback-Leibler (KL) divergence. Local sensitivity
measures approximate the nonlinear dependence of a VB optimum on prior
parameters using a local Taylor series approximation
\citep{gustafson:1996:localposterior, giordano:2017:covariances}. Using a
stick-breaking representation of a Dirichlet process, we consider perturbations
both to the scalar concentration parameter and to the functional form of the
stick-breaking distribution. As far as the authors are aware, ours is the first
analysis of the local sensitivity of BNP posteriors when using a VB
approximation.

Unlike previous work on local Bayesian sensitivity for BNP
\citep{Basu:2000:BNP_robustness}, we pay special attention to the ability of our
sensitivity measures to \emph{extrapolate} to different priors, rather than
treating the sensitivity as a measure of robustness \textit{per se}.
Extrapolation motivates the use of multiplicative perturbations to the
functional form of the prior, as the KL divergence is then linear in the
perturbation. Additionally, we linearly approximate only the computationally
intensive part of inference---the optimization of the global parameters---and
retain the non-linearity of easily computed quantities.

We apply our methods to estimate sensitivity to the BNP prior specification of
the expected number of distinct clusters present the Iris dataset
\citep{iris_data_anderson, iris_data_fisher}.  We evaluate the accuracy of our
approximations by comparing to the much more expensive process of re-fitting the
model.



\section{Stick-breaking Dirichlet processes}


\section{Local sensitivity}


\section{Results}

\subsection{Gaussian mixture modeling on Iris data}

\subsection{Regression mixture modeling}
%%%%%%%%%%%%%%%%%%%%%%%%%%%%%%%%%%%%%%
%%%%%%%%%%%%%%%%%%%%%%%%%%%%%%%%%%%%%%
% Do not edit the TeX file your work
% will be overwritten.  Edit the RnW
% file instead.
%%%%%%%%%%%%%%%%%%%%%%%%%%%%%%%%%%%%%%
%%%%%%%%%%%%%%%%%%%%%%%%%%%%%%%%%%%%%%



Here are results, and a figure. See \figref{example_genes}.
%

\begin{knitrout}
\definecolor{shadecolor}{rgb}{0.969, 0.969, 0.969}\color{fgcolor}\begin{figure}[!h]

{\centering \includegraphics[width=0.980\linewidth,height=0.392\linewidth]{figure/example_genes-1} 

}

\caption[(Left) An example gene and its expression over time.
     (Right) The cubic B-spline with 7 degrees of freedom]{(Left) An example gene and its expression over time.
     (Right) The cubic B-spline with 7 degrees of freedom. }\label{fig:example_genes}
\end{figure}


\end{knitrout}
%






\begin{knitrout}
\definecolor{shadecolor}{rgb}{0.969, 0.969, 0.969}\color{fgcolor}\begin{figure}[!h]

{\centering \includegraphics[width=0.980\linewidth,height=0.627\linewidth]{figure/gene_centroids-1} 

}

\caption[In blue, inferred centroids from the twelve most occupied clusters.
    In grey, gene expressions averaged over replicates and
    shifted by their inferred intercepts]{In blue, inferred centroids from the twelve most occupied clusters.
    In grey, gene expressions averaged over replicates and
    shifted by their inferred intercepts. }\label{fig:gene_centroids}
\end{figure}


\end{knitrout}







\begin{knitrout}
\definecolor{shadecolor}{rgb}{0.969, 0.969, 0.969}\color{fgcolor}\begin{figure}[!h]

{\centering \includegraphics[width=0.588\linewidth,height=0.400\linewidth]{figure/gene_initial_coclustering-1} 

}

\caption[The inferred co-clustering of gene expressions at $\alpha = 3.$ ]{The inferred co-clustering of gene expressions at $\alpha = 3.$ }\label{fig:gene_initial_coclustering}
\end{figure}


\end{knitrout}










\begin{knitrout}
\definecolor{shadecolor}{rgb}{0.969, 0.969, 0.969}\color{fgcolor}\begin{figure}[!h]

{\centering \includegraphics[width=0.980\linewidth,height=0.784\linewidth]{figure/gene_alpha_coclustering-1} 

}

\caption[Changes in the co-clustering matrix at alpha = 1 (top row)
     and alpha = 11 (bottom row),
     relative to the co-clustering matrix at alpha = 3.
     The left column plots differences predicted by the 
     linear approximation against differences from a model refit]{Changes in the co-clustering matrix at alpha = 1 (top row)
     and alpha = 11 (bottom row),
     relative to the co-clustering matrix at alpha = 3.
     The left column plots differences predicted by the 
     linear approximation against differences from a model refit. 
     Each point represents an entry of the co-coclustering matrix. 
     The middle and right columns display 
     changes in the co-clustering matrix as obtained by the 
     linear approximation and the model refit, respectively.}\label{fig:gene_alpha_coclustering}
\end{figure}


\end{knitrout}





\begin{knitrout}
\definecolor{shadecolor}{rgb}{0.969, 0.969, 0.969}\color{fgcolor}\begin{figure}[!h]

{\centering \includegraphics[width=0.980\linewidth,height=0.314\linewidth]{figure/gene_alpha_coclustering_influence-1} 

}

\caption[The influence function of $g_{ev}$, the sum of the eigenvalues of the coclustering Laplacian matrix]{The influence function of $g_{ev}$, the sum of the eigenvalues of the coclustering Laplacian matrix. }\label{fig:gene_alpha_coclustering_influence}
\end{figure}


\end{knitrout}





\begin{knitrout}
\definecolor{shadecolor}{rgb}{0.969, 0.969, 0.969}\color{fgcolor}\begin{figure}[!h]

{\centering \includegraphics[width=0.980\linewidth,height=0.784\linewidth]{figure/gene_fpert_coclustering-1} 

}

\caption[Effect on the co-clustering matrix after a functional perturbation.
     log-phi (top left, in grey) is set to a Gaussian p.d.f]{Effect on the co-clustering matrix after a functional perturbation.
     log-phi (top left, in grey) is set to a Gaussian p.d.f. centered at mu = -4.2, and scaled to have L-infinity norm equal to two.
    The chosen log-phi roughly corresponds to a positive bump in the influence function of $g_{ev}$.}\label{fig:gene_fpert_coclustering}
\end{figure}


\end{knitrout}


\subsection{Genetic admixture modeling with STRUCTURE}


\section{Conclusion}
This concludes.

\newpage

\small{
%%%%%%%%%%%%%%%%%%%%%%%%%%%%%%%
%%%%%%%%%%%%%%%%%
{\bf Acknowledgments}:
% %%%%%%%%%%%%%%%%%%
%%%%%%%%%%%%%%%%%%%%%%%%%%%%%%%%%
Ryan Giordano's research was funded in full by the
Gordon and Betty Moore Foundation through Grant GBMF3834 and by the Alfred P. Sloan Foundation through Grant 2013-10-27 to the University of California, Berkeley. Runjing Liu's research was funded by the NSF graduate research fellowship. Tamara Broderick's research is supported by an NSF CAREER Award and an ARO YIP
Award. This research is supported in part by the DARPA program on
Lifelong Learning Machines.
}


\bibliography{references}
\bibliographystyle{plainnat}

\newpage
\onecolumn
\appendix

\section{Supplemental results}

\end{document}
