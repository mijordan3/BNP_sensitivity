% \documentclass[ba,preprint]{imsart}% use this for supplement article
\documentclass[ba]{imsart}
%
\pubyear{2021}
\volume{TBA}
\issue{TBA}
% \doi{0000}
%\arxiv{}
\firstpage{1}
\lastpage{1}

\usepackage{amsthm}
\usepackage{amsmath}
\usepackage{natbib}
\usepackage[colorlinks,citecolor=blue,urlcolor=blue,filecolor=blue,backref=page]{hyperref}
\usepackage{graphicx}


% my added packages: copied from the swiss IJ paper
\usepackage{microtype}
\usepackage{graphicx}
\usepackage{subfigure}
\usepackage{booktabs} % for professional tables
\usepackage{siunitx}
\usepackage{hyperref}
\usepackage{xargs}[2008/03/08]
\usepackage{xfrac}

% adding line numbers.
% comment this out when done
% adding line numbers also somewhat messes up the
% formatting on the first page; it looks fine
% without line numbers, curiously.

% \usepackage{lineno}
% \linenumbers

% For inline enumerate:
% https://en.wikibooks.org/wiki/LaTeX/List_Structures
\usepackage{blindtext}
\usepackage[inline]{enumitem}

% Documentation
% http://ftp.math.purdue.edu/mirrors/ctan.org/macros/latex/contrib/refstyle/refstyle.pdf
\usepackage{refstyle}
\usepackage{varioref} % Use refstyle instead of varioref directly.

\usepackage{amssymb}
\usepackage{amsfonts}
\usepackage{mathtools}
\usepackage{colonequals}
\usepackage{algpseudocode, algorithm} %typical alg typesetting packages
\usepackage{xargs} % For def with default arguments

% Math scripts
\usepackage{mathrsfs}  % For script fonts
%\usepackage{dutchcal} % Lower-case mathcal
%\usepackage{mathbbol}  % For lower-case bold fonts

% \DeclareFontFamily{OT1}{pzc}{}
% \DeclareFontShape{OT1}{pzc}{m}{it}{<-> s * [1.10] pzcmi7t}{}
% \DeclareMathAlphabet{\mathpzc}{OT1}{pzc}{m}{it}

\usepackage{listings}
\usepackage{pdfpages}

% added packages for todo notes
% pass in option ``disable" to remove notes
\usepackage[textsize=tiny]{todonotes}

% This picks up the knitr boilerplate, allowing us to \input partial knitr
% documents.
\input{_knitr_header.tex}

\startlocaldefs
% This defines math macros.
\def\mbe{\mathbb{E}}%
\def\ind#1{\mathbb{I}\left(#1\right)}
\def\evalat#1#2{\left.#1\right|_{#2}}
\def\fracat#1#2#3{\left.\frac{#1}{#2}\right\vert_{#3}}
\def\iid{\overset{iid}{\sim}}
\def\expect#1#2{\underset{#1}{\mathbb{E}}\left[#2\right]}

\DeclareMathOperator*{\argmax}{\mathrm{argmax}}
\DeclareMathOperator*{\argmin}{\mathrm{argmin}}

% Variational parameters
\def\etaopt{\hat\eta}
\def\x{x}   % Data
\def\t{t}   % Generic priur parameter
\def\z{z}   % Local posterior parameters
\def\q{q}   % VB dist
\def\logp{\ell}   % VB dist
\def\KL#1{\mathrm{KL}\left(#1\right)}   % KL divergence

% Dimensions
\def\etadim{D_{\eta}}
\def\thetadim{D_{\theta}}

% Domains
\def\etadom{\Omega_{\eta}}
\def\thetadom{\Omega_{\theta}}
\def\tdom{\Omega_{\t}}

\def\K{K}   % Number of components

\def\mathand{\quad\quad\textrm{and}\quad\quad}%
\def\mathwhere{\quad\quad\textrm{where}\quad\quad}%


% This specifies the formatting for references (sections, theorems, etc.)

%%%%%%%%%%%%%%%%%%%%
% amsthm commands

\theoremstyle{plain}
\newtheorem{lem}{Lemma}
\newtheorem{thm}{Theorem}
\newtheorem{prop}{Proposition}
\newtheorem{cond}{Condition}
\newtheorem{assu}{Assumption}
\newtheorem{cor}{Corollary}
\newtheorem{conj}{Conjecture}

%\theoremstyle{definition}
% \newtheorem{defn}{Definition}
%\newtheorem{ex}{Example}

% Example environment with a terminating symbol.
% https://tex.stackexchange.com/questions/16453/denoting-the-end-of-example-remark
\theoremstyle{definition}
\newtheorem{examplex}{Example}
\newenvironment{ex}
  {\pushQED{\qed}\renewcommand{\qedsymbol}{$\triangle$}\examplex}
  {\popQED\endexamplex}

\theoremstyle{definition}
\newtheorem{defnx}{Definition}
\newenvironment{defn}
    {\pushQED{\qed}\renewcommand{\qedsymbol}{$\boxdot$}\defnx}
    {\popQED\enddefnx}


\newcommand{\seeproof}[1]{(See \proofref{#1} \proofpageref[vref]{#1}.)}
\newcommand{\proofof}[1]{\noindent{\bf Proof of #1.}}


%%%%%%%%%%%%%%%%%%%%
% refstyle commands

\newref{event}{
    name=Event~, %
    names=Events~, %
    Name=Event~,
    Names=Events~
    }

\newref{item}{
    name=Item~, %
    names=Items~, %
    Name=Item~,
    Names=Items~
    }

\newref{fig}{
    name=Figure~, %
    Name=Figure~
    }

\newref{tab}{
    name=Table~, %
    Name=Table~
    }

\newref{sec}{
    name=Section~, %
    Name=Section~,
    names=Sections~,
    Names=Sections~,
    }

\newref{app}{
    name=Appendix~, %
    Name=Appendix~,
    names=Appendices~,
    Names=Appendices~,
    }

\newref{eq}{
    name=Eq.~, %
    Name=Eq.~,
    names=Eqs.~, %
    Names=Eqs.~
    }

\newref{fig}{
    name=Figure~, %
    Name=Figure~,
    names=Figures~, %
    Names=Figures~,
    }

\newref{def}{
    name=Definition~, %
    Name=Definition~,
    names=Definitions~, %
    Names=Definitions~
    }

\newref{assu}{
    name=Assumption~, %
    Name=Assumption~,
    names=Assumptions~, %
    Names=Assumptions~,
    }

\newref{cond}{
    name=Condition~, %
    Name=Condition~,
    names=Conditions~, %
    Names=Conditions~
    }

\newref{prop}{
    name=Proposition~, %
    Name=Proposition~,
    names=Propositions~, %
    Names=Propositions~
    }

\newref{lem}{
    name=Lemma~, %
    Name=Lemma~,
    names=Lemmas~, %
    Names=Lemmas~
    }

\newref{ex}{
    name=Example~, %
    Name=Example~,
    names=Examples~,
    Names=Examples
    }

\newref{cory}{
    name=Corollary~, %
    Name=Corollary~
    }

\newref{thm}{
    name=Theorem~, %
    Name=Theorem~,
    names=Theorems~, %
    Names=Theorems~
    }

\newref{proof}{
    name=Proof~, %
    Name=Proof~
    }

\newref{conj}{
    name=Conjecture~, %
    Name=Conjecture~
    }

\newref{algr}{
    name=Algorithm~, %
    Name=Algorithm~,
    names=Algorithms~, %
    Names=Algorithms~,
    }


%% refstyle examples:
% \Secref[vref]{introduction} contains \secref{introduction}.
% \Secref[vref]{ack} does not contain \secref{introduction}.
%
% \begin{align}
%     x=y \eqlabel{myeq}
% \end{align}
%

\endlocaldefs

\begin{document}

% Output of the knitr file defining certain simulated figures.
%%%%%%%%%%%%%%%%%%%%%%%%%%%%%%%%%%%%%%
%%%%%%%%%%%%%%%%%%%%%%%%%%%%%%%%%%%%%%
% Do not edit the TeX file your work
% will be overwritten.  Edit the RnW
% file instead.
%%%%%%%%%%%%%%%%%%%%%%%%%%%%%%%%%%%%%%
%%%%%%%%%%%%%%%%%%%%%%%%%%%%%%%%%%%%%%



\newcommand{\SimPathologicalRTwoFig}{

\begin{knitrout}
\definecolor{shadecolor}{rgb}{0.969, 0.969, 0.969}\color{fgcolor}\begin{figure}[!h]

{\centering \includegraphics[width=0.980\linewidth,height=0.784\linewidth]{figure/r2_pathological-1} 

}

\caption[A plot of $f(x_1, x_2)$ from \exref{r2_pathological}]{A plot of $f(x_1, x_2)$ from \exref{r2_pathological}.}\label{fig:r2_pathological}
\end{figure}


\end{knitrout}
}


\newcommand{\SimPositivePertFig}{

\begin{knitrout}
\definecolor{shadecolor}{rgb}{0.969, 0.969, 0.969}\color{fgcolor}\begin{figure}[!h]

{\centering \includegraphics[width=0.980\linewidth,height=0.784\linewidth]{figure/positive_pert-1} 

}

\caption{A plot of the perturbations from \exref{positive_pert_large} with $p=2$ and $\epsilon=0.05$.  Positive $\phi$ can only add mass, so to remove a small amount of mass requires adding mass everywhere else and re-normalizing, resulting in a large perturbation according to $\norm{\cdot}_p$.}\label{fig:positive_pert}
\end{figure}


\end{knitrout}
}


\newcommand{\FunctionPathsFig}{

\begin{knitrout}
\definecolor{shadecolor}{rgb}{0.969, 0.969, 0.969}\color{fgcolor}\begin{figure}[!h]

{\centering \includegraphics[width=0.980\linewidth,height=0.470\linewidth]{figure/path-1} 

}

\caption[Multiplicative and linear mixture paths between two densities]{Multiplicative and linear mixture paths between two densities.}\label{fig:path}
\end{figure}


\end{knitrout}
}


\newcommand{\FunctionPathsMultFig}{

\begin{knitrout}
\definecolor{shadecolor}{rgb}{0.969, 0.969, 0.969}\color{fgcolor}\begin{figure}[!h]

{\centering \includegraphics[width=0.980\linewidth,height=0.274\linewidth]{figure/mult_path-1} 

}

\caption[Multiplicative mixture paths between two densities]{Multiplicative mixture paths between two densities.}\label{fig:mult_path}
\end{figure}


\end{knitrout}
}


\newcommand{\FunctionPathsLinFig}{

\begin{knitrout}
\definecolor{shadecolor}{rgb}{0.969, 0.969, 0.969}\color{fgcolor}\begin{figure}[!h]

{\centering \includegraphics[width=0.980\linewidth,height=0.274\linewidth]{figure/lin_path-1} 

}

\caption[Linear mixture paths between two densities]{Linear mixture paths between two densities.}\label{fig:lin_path}
\end{figure}


\end{knitrout}
}


\newcommand{\FunctionBallFig}{

\begin{knitrout}
\definecolor{shadecolor}{rgb}{0.969, 0.969, 0.969}\color{fgcolor}\begin{figure}[!h]

{\centering \includegraphics[width=0.980\linewidth,height=0.274\linewidth]{figure/func_ball-1} 

}

\caption[An $\linf{\cdot}$ ball]{An $\linf{\cdot}$ ball.}\label{fig:func_ball}
\end{figure}


\end{knitrout}
}



\newcommand{\FunctionDistFig}{

\begin{knitrout}
\definecolor{shadecolor}{rgb}{0.969, 0.969, 0.969}\color{fgcolor}\begin{figure}[!h]

{\centering \includegraphics[width=0.980\linewidth,height=0.470\linewidth]{figure/func_dist-1} 

}

\caption[Two densities which are distant according to KL divergence and $\norminf{\cdot}$ but close according to $\norm{\cdot}_p$ for $p \in [1, \infty)$]{Two densities which are distant according to KL divergence and $\norminf{\cdot}$ but close according to $\norm{\cdot}_p$ for $p \in [1, \infty)$.}\label{fig:func_dist}
\end{figure}


\end{knitrout}
}


%% *** Frontmatter ***

\begin{frontmatter}
\title{Evaluating Sensitivity to the Stick Breaking Prior in Bayesian Nonparametrics}
%\title{\support{}}
\runtitle{BNP sensitivity}

\begin{aug}
\author{\fnms{Ryan} \snm{Giordano}\thanksref{addr1,t1}},
\author{\fnms{Runjing} \snm{Liu}\thanksref{addr2,t1}},
\author{\fnms{Michael I.} \snm{Jordan}\thanksref{addr2}},
\and
\author{\fnms{Tamara} \snm{Broderick}\thanksref{addr1}}
%\author{\fnms{<firstname>} \snm{<surname>}\thanksref{}\ead[label=e1]{}}
%\and
%\author{\fnms{} \snm{}}

\runauthor{}

\address[addr1]{Department of EECS, MIT
77 Massachusetts Ave., 38-401
Cambridge, MA 02139}

\address[addr2]{Department of Statistics, UC Berkeley
367 Evans Hall, UC Berkeley
Berkeley, CA 94720}

\thankstext{t1}{Equal contribution. }

\end{aug}

\begin{abstract}
A Bayesian nonparametric (BNP) approach to clustering treats the number of
distinct clusters in a data set as a random quantity by modeling an infinite
number of components, allowing for posterior inference without specifying the
number of components {\em a priori}.  A BNP model requires the specification of
a prior distribution on an infinite number of component probabilities, and some
posterior quantities may be sensitive to the prior specification. Since a range
of BNP priors are typically plausible in a given modeling context, it is
important in practice to establish the the sensitivity of conclusions to
possibly arbitrary prior choices.  In principle, one could re-estimate the
posterior for a range of prior choices, but in practice this is computationally
prohibitive, due to the high cost of even a single posterior approximation and
the large number of potential choices for the prior.

We circumvent this difficulty by deriving local sensitivity measures based on
Taylor series approximations for a truncated variational Bayes (VB)
approximation based on the Kullback-Leibler divergence, both for parametric
perturbations (e.g., to the concetration parameter of a $\gem$ prior) and for
nonparametric perturbations to a stick-breaking density.  In constrast to
previous work on Bayesian local sensitivity based on Markov chain Monte Carlo,
VB sensitivity measures can be computed automatically and efficiently.  We state
general conditions under which a VB approximation is continuously
differentiable, and show that nonparametric perturbations can be expressed as an
integral against an \textit{influence function} which summarizes the first-order
effect of changes to the prior density.   Amongst a general class of
nonparmetric prior perturbations corresponding to the $L_p$ spaces, we show that
the VB optimum is Fr{\'e}chet differentiable with respect to the prior density
only for $p=\infty$.  We validate the ability of our local approximation to
usefully extrapolate to alternative priors on several real-world datasets and
show that it can be an accurate way to quickly assess robustness to both
parametric and non-parametric prior perturbations.

\end{abstract}

%% ** Keywords **
\begin{keyword}%[class=MSC]
\kwd{Dirichlet Process}
\kwd{Stick breaking}
\kwd{Local robustness}
\kwd{Variational Bayes}
\kwd{Fr{\'e}chet differentiability}
\kwd{fastSTRUCTURE}
%\kwd[]{}
\end{keyword}

\end{frontmatter}

%% ** Mainmatter **

\section{Introduction}\seclabel{introduction}

A central question in many probabilistic clustering problems is how many
distinct clusters are present in a particular dataset. Bayesian nonparametrics
(BNP) addresses this question by placing a generative process on cluster
assignment, making the number of distinct clusters present amenable to Bayesian
inference.  However, like all Bayesian approaches, BNP requires the
specification of a prior, and this prior may favor a greater or fewer number of
distinct clusters. In practice, it is important to quantitatively establish that
the prior is not too informative, particularly when---as is often the case in
BNP---the particular form of the prior is chosen for mathematical convenience
rather than because of a considered subjective belief.

We derive local sensitivity measures for a truncated variational Bayes (VB)
approximation based on the Kullback-Leibler (KL) divergence. Local sensitivity
measures approximate the nonlinear dependence of a VB optimum on prior
parameters using a local Taylor series approximation
\citep{gustafson:1996:localposterior, giordano:2017:covariances}. Using a
stick-breaking representation of a Dirichlet process, we consider perturbations
both to the scalar concentration parameter and to the functional form of the
stick-breaking distribution. As far as the authors are aware, ours is the first
analysis of the local sensitivity of BNP posteriors when using a VB
approximation.

Unlike previous work on local Bayesian sensitivity for BNP
\citep{Basu:2000:BNP_robustness}, we pay special attention to the ability of our
sensitivity measures to \emph{extrapolate} to different priors, rather than
treating the sensitivity as a measure of robustness \textit{per se}.
Extrapolation motivates the use of multiplicative perturbations to the
functional form of the prior, as the KL divergence is then linear in the
perturbation. Additionally, we linearly approximate only the computationally
intensive part of inference---the optimization of the global parameters---and
retain the non-linearity of easily computed quantities.

We apply our methods to estimate sensitivity to the BNP prior specification of
the expected number of distinct clusters present the Iris dataset
\citep{iris_data_anderson, iris_data_fisher}.  We evaluate the accuracy of our
approximations by comparing to the much more expensive process of re-fitting the
model.



\section{The model and variational approximation}
\seclabel{model}
    \subsection{A stick-breaking model for clustering}
    \seclabel{model_bnp}
    We will consider discrete Bayesian nonparametric (BNP) generative models which
draw data points $\x_n$ from one of a certain number of components indexed by $\k =
1, \ldots, \kmax$, where in principle $\kmax$ might be infinity.
Each component is characterized by a vector $\beta_\k \in \betadom \subseteq
\mathbb{R}^{\betadim}$, with $\p(\x_n \vert \beta_\k)$ denoting the
distribution of data arising from  component $\k$. We will model the $\beta_\k$
as arising IID from a known prior $\beta_\k \iid \pbeta(\beta_\k)$, and write
$\beta = (\beta_1, \beta_2, \ldots)$.

Assignment of data point $\n$ to a mixture component is represented by a vector
$\z_\n = (\z_{\n1}, \ldots, \z_{\n\kmax})$,
where $\z_{\n\k} = 1$ for exactly one $\k$ and $0$ otherwise.
With $\z_\n$ defined in
this way, we can write
%
\begin{align*}
%
\p(\x_n \vert \z_\n, \beta) =
    \prod_{k=1}^\kmax \p(\x_n \vert \beta_\k)^{\z_{\n\k}}.
%
\end{align*}


The assignments $\z_{\n}$ are drawn according to the following
``stick-breaking process.''  Fix a density $\pstick(\cdot)$, with respect to the
Lebesgue measure, over stick-breaking proportions $\nuk \in (0, 1)$ and
draw $\nuk\iid\pstick(\nuk)$ for $\k=1,\ldots,\kmax - 1$.  If $\kmax = \infty$, there are an
infinite number of IID sticks and the $\kmax - 1$ can be ignored; if $\kmax <
\infty$, then we set $\nu_{\kmax} = 1$ for notational convenience below.  Given
these stick lengths, we compute indicator probabilities by
%
\begin{align*}
%
% \pi_\k := \begin{cases}
% \nuk \prod_{\k' < \k} (1 - \nu_{\k'}) & \textrm{For }k < \kmax
%     \textrm{ (all }k\textrm{ when }\kmax = \infty\textrm{)}\\
% \prod_{\k' < \k} (1 - \nu_{\k'}). & \textrm{For }k = \kmax \\
% \end{cases}
\pi_\k := \nuk \prod_{\k' < \k} (1 - \nu_{\k'})
%
\end{align*}
%
where the empty product is taken to be equal to $1$. The convention $\nu_{\kmax} =
1$ allows us to use the same formula for all $\pi_\k$ when $\kmax < \infty$.
Write $\nu := (\nu_1, \ldots, \nu_{\kmax})$ for the vector of all stick lengths
and $\pi := (\pi_1, \ldots, \pi_\kmax)$ for the corresponding vector of
probabilities. By construction, $\sum_{\k=1}^{\kmax} \pi_\k = 1$ even when $\kmax =
\infty$.  We draw $\z_\n$ according to
%
\begin{align*}
%
% \p(\z_{\n\k} = 1 \vert \pi) ={}& \pi_\k \\
% \p(\z_\n \vert \pi) ={}&
%     \ind{\sum_{\k=1}^{\kmax} \z_{\n\k} = 1}
%     \prod_{k=1}^{\kmax} \pi_\k^{\z_{\n\k}}.
\p(\z_\n \vert \pi) ={}&
    \prod_{k=1}^{\kmax} \pi_\k^{\z_{\n\k}}.
%
\end{align*}
%
Since $\pi$ is a deterministic function of $\nu$, we can also write
$\p(\z_\n \vert \nu)$ with no ambiguity.

The stick-breaking distribution $\pstick$ can be thought of as inducing a
a distribution on the vector of probabilities $\pi$. Different
stick-breaking distributions will favor different indicator probabilities
with different implied degrees of concentration.
A particularly common choice for
$\pstick(\nuk)$ is the $\mathrm{Beta}(\nuk \vert 1, \alpha)$ distribution,
%
\begin{align*}
%
\mathrm{Beta}(\nuk \vert 1, \alpha) =
    \frac{\Gamma(1 + \alpha) (1 - \nuk)^{\alpha - 1}}
         {\Gamma(\alpha)}.
%
\end{align*}
%
When $\pstick$ is Beta distributed
and $\kmax=\infty$, the resulting distribution on $\pi$  is known as the
$\textit{GEM distribution}$, and we write $\pi \sim \mathrm{GEM}(\alpha)$.

The GEM distribution is closely related to the Dirichlet process.
When $\pi \sim \mathrm{GEM}(\alpha)$ and
$\beta_\k \iid \pbeta(\beta_\k)$,
the random measure on $\betadom$
\begin{align*}
  \sum_{\k = 1}^\infty \pi_\k\delta_{\beta_\k},
\end{align*}
with $\delta_{\beta_\k}$ denoting point mass at $\beta_\k$,
is a draw from a Dirichlet process with concentration parameter $\alpha$
and base measure $\pbeta$.

We keep our notation for stick-breaking distributions $\pstick$ general however,
because in our sensitivity analysis,
we will consider stick-breaking distributions that are outside the
family of Beta distributions.

In this notation, we can write the joint distribution of
the data and parameters as:
%
\begin{align}\eqlabel{bnp_model}
%
% \logp(\x, \beta, \z, \nu) ={}&
%     \sum_{n=1}^N \sum_{k=1}^{\kmax}
%         \z_{\n\k} \left(
%             \logp(\x_n \vert \beta_\k) + \log \pi_\k
%         \right) +
% \nonumber \\ {}&
%     \sum_{k=1}^{\kmax} \left(
%         \log \pstick(\nuk) + \logp(\beta_\k)
%     \right).
\logp(\x, \beta, \z, \nu) =&
\sum_{n=1}^N \sum_{k=1}^{\kmax}
    \z_{\n\k} \left(
        \logp(\x_n \vert \beta_\k) + \log \pi_\k
    \right)
\nonumber\\
   & +
    \sum_{k=1}^{\kmax} \left(
        \log \pstick(\nuk) + \log \pbeta(\beta_\k)
    \right).
%
\end{align}
%

%%%%%%%%%%%%%%%%%%%%%%%%%%%%%%%%%%%%%%%%%%%%%%%%%%%%%%%%%%%%%%%%%%%%%%%%%%%%%%%
%%%%%%%%%%%%%%%%%%%%%%%%%%%%%%%%%%%%%%%%%%%%%%%%%%%%%%%%%%%%%%%%%%%%%%%%%%%%%%%
\begin{ex}[Gaussian mixture model]\exlabel{iris_bnp_process}
%
The observations are vectors $\x_\n \in \mathbb{R}^\d$,
and we model each component with multivariate Gaussians.
In this case, $\beta_\k = (\mu_k, \Sigma_\k)$,
where $\mu_\k \in \mathbb{R}^4$, $\Sigma_\k$ is a $\d\times\d$ positive
definite covariance matrix, and
%
\begin{align*}
%
\p(\x_\n \vert \beta_\k) ={}& \normdist{\x_n \vert \mu_\k, \Sigma_\k} \\
\logp(\x_\n \vert \beta_\k) ={}&
    -\frac{1}{2}(\x_n - \mu_k)^T \Sigma_\k^{-1} (\x_n - \mu_k)
    -\frac{1}{2} \log |\Sigma_\k| + \const.\\
    & \constdesc{\beta_\k}
%
\end{align*}
Below, we fit a
Gaussian mixture model (GMM) to the Fisher's iris data set CITE.
Each observation is an iris flower with
four measurements:
sepal length, sepal width, petal length, and petal width.
The components in this model can be interpreted as latent iris species;
the inferential goal is to estimate $\z_\n$ and thus assign each
observed flower to an iris species.

% For the iris data, we might imagine that each cluster corresponds to a different
% species with a different distribution of flower dimensions.  The BNP model
% implies that there are a potentially infinite number of differet iris species
% that we might observe.  Then $\z_{\n\k} = 1$ would mean that observation $\n$
% was a member of species $\k$, and $\sum_{k=1}^\kmax \ind{ \sum_{n=1}^{N}
% \z_{\n\k} > 1}$ is the number of distinct species observed in our particular
% dataset.
%
\end{ex}

We started with a Gaussian mixture model
because it is the most generic model that fits cleanly into the
generative process culminating in \eqref{bnp_model}.
The next two examples on real data sets require more careful modeling considerations,
and we adjust the factorization in in \eqref{bnp_model} to suit our purposes.

%%%%%%%%%%%%%%%%%%%%%%%%%%%%%%%%%%%%%%%%%%%%%%%%%%%%%%%%%%%%%%%%%%%%%%%%%%%%%%%
%%%%%%%%%%%%%%%%%%%%%%%%%%%%%%%%%%%%%%%%%%%%%%%%%%%%%%%%%%%%%%%%%%%%%%%%%%%%%%%

\begin{ex}[Regression mixture model]\exlabel{mice_bnp_process}
We cluster time-course gene expression data using a BNP model.
An observation $\x_\n\in\mathbb{R}^\ntimepoints$ is a vector of expression levels at $\ntimepoints$
time points.
Let $\regmatrix$ be a $\ntimepoints \times \d$ regressor matrix;
in our case, we use a cubic B-spline, so the entry $ij$ of $\regmatrix$
is the $j$-th basis vector evaluated at the $i$-th time point (\secref{}).

Each cluster is characterized by a vector of regression coefficients
$\mu_\k$ and a variance $\tau^{-1}_\k$, so
in this model, $\beta_k = (\mu_\k, \tau_\k)$.
The distribution of the data arising from cluster $k$ is
\begin{align*}
\p(\x_\n | \beta_\k, \b_\n) =
\normdist{\x_\n | \regmatrix\mu_\k + \b_\n,
\tau_\k^{-1}I_{\ntimepoints \times \ntimepoints}},
\end{align*}
%
where $\b_{n}$ is a gene-specific additive offset.
We include the additive offset because we
are interested in clustering gene expressions based on their patterns over time,
not their absolute level.

The joint distribution can be written in the same form as~\eqref{bnp_model},
except that the conditional log-likelihood also conditions on $\b_n$,
and we also include an additional prior term:
\begin{align*}
\logp(\x, \beta, \z, \nu) =
    \sum_{n=1}^N \sum_{k=1}^{\kmax}&
        \z_{\n\k} \left(
            \logp(\x_n \vert \beta_\k, \b_n) + \logp(\b_n) + \log \pi_\k
        \right) +\notag \\
    & \quad \ldots + \sum_{k=1}^{\kmax} \left(
        \log \pstick(\nuk) + \log \pbeta(\beta_\k)
    \right).
\end{align*}

\todo{kmax in these examples should all be infinity}
\end{ex}

%%%%%%%%%%%%%%%%%%%%%%%%%%%%%%%%%%%%%%%%%%%%%%%%%%%%%%%%%%%%%%%%%%%%%%%%%%%%%%%
%%%%%%%%%%%%%%%%%%%%%%%%%%%%%%%%%%%%%%%%%%%%%%%%%%%%%%%%%%%%%%%%%%%%%%%%%%%%%%%

Our last example is a topic model applied to genetic data.
Genotypes at selected genetic markers of individuals take the place of
words in a document; in lieu of inferring ``topics," we infer latent populations.

\begin{ex}[A topic model for population structure]\exlabel{structure_bnp_process}

The data set consists of $\nindiv$ individuals genotyped at $\nloci$ loci.
For diploid species, there are two observations at each loci.
Let $\x_{\n\l\i}\in\{1, \ldots, J_\l\}$ be the genotype for individual $\n$ at locus $\l$ and chromosome $\i$,
where $J_\l$ is the number of possible genotypes at locus $\l$
(for example, if the measurements are single nucleotides, either A, T, C or G,
then $J_\l = 4$ for all $\l$).

Each latent population is characterized by the collection
$\beta_k = (\latentpop_{\k1}, \ldots, \latentpop_{\k\nloci})$ where
$\latentpop_{\k\l}\in\Delta^{J_\l - 1}$ are the latent frequencies for the $J_l$
possible alleles at locus $\l$ under population $\k$.
The distribution of $\x_{\n\l\i}\in\{1, \ldots, J_\l\}$ arising from population $\k$ is
\begin{align*}
\p(\x_{\n\l\i} \vert \latentpop_{\k}) =
\categoricaldist{\x_{\n\l\i}\vert \latentpop_{\k\l}}.
\end{align*}


Unlike the previous data examples, each individual now has their own stick-breaking process. Draw sticks
\begin{align*}
\nu_{\n\k} \iid \pstick(\nu_{\n\k}) \quad \forall \n = 1, ..., \nindiv; \k = 1, 2, \ldots.
\end{align*}
In this application,
we call the mixture weights
$\latentadmix_{\n} = (\latentadmix_{\n1}, \latentadmix_{\n2}, \ldots)$ the
\textit{admixture} of individual $\n$.
These are formed by the usual stick-breaking construction,
\begin{align*}
\latentadmix_{\n\k} = \nu_{\n\k} \prod_{\k' < \k} (1 - \nu_{\n\k'}).
\end{align*}
%
The latent population membership $\z_{\n\l\i}$ of the observed genotype
$\x_{\n\l\i}$ is then drawn according to the usual multinomial distribution
\begin{align*}
p(\z_{\n\l\i} | \latentadmix_\n) = \prod_{k=1}^{\kmax} \latentadmix_{\n\k}^{\z_{\n\l\i\k}}.
\end{align*}

The joint log-likelihood decomposes as
\begin{align*}
\logp(\x, \latentpop, \z, \nu) &=
\sum_{\n=1}^\nindiv \sum_{\l=1}^\nloci \sum_{i = 1}^2 \sum_{\k=1}^{\kmax}
        \z_{\n\l\i\k} \left(
            \logp(\x_{\n\l\i} \vert \latentpop_{\k}) + \log \pi_{\n\k}
        \right)
\nonumber\\&
    \quad +
    \sum_{\n=1}^\nindiv \sum_{k=1}^{\kmax} \log \pstick(\nu_{\n\k})
    + \sum_{k=1}^{\kmax} \log \pbeta(\latentpop_{\k}).
\end{align*}
\end{ex}

This model is identical to STRUCTURE,
a model proposed in \citet{pritchard:2000:structure, raj:2014:faststructure},
except that we replace the Dirichlet prior in STRUCTURE
with an infinite stick-breaking process.
The result is a model similar to a heirchical Dirichlet process CITE,
but without the top-level Dirichlet process.


%%%%%%%%%%%%%%%%%%%%%%%%%%%%%%%%%%%%%%%%%%%%%%%%%%%%%%%%%%%%%%%%%%%%%%%%%%%%%%%


    \subsection{Variational approximation}
    \seclabel{model_vb}
    The posterior of a BNP model is difficult to compute for two reasons: because
the number of components is (countably) infinite, and because the posterior
normalization constant is intractible.  To address these problems, we follow
\citet{blei:2006:vi_for_dp} and form a mean field truncated variational
approximation to the posterior.

Let $\zeta$ denote the full vector of unkown posterior variables. For example,
in the GMM model of \exref{iris_bnp_process}, $\zeta := (\beta, \z, \nu)$.  The
exact posterior distribution $\p(\zeta \vert \x)$ is intractable. Variational
Bayes (VB) is an approach that seeks an approximate posterior through solving a
numerical optimization problem \citep{jordan:1999:vi,
wainwright:2008:graphical_models, blei:2017:vi_review}.

VB specifies a family of approximating distributions $\q(\zeta \vert \eta)$
parameterized by a finite-dimensional vector $\eta \in \etadom \subseteq
\mathbb{R}^{\etadim}$ and solves for $\q(\zeta\vert\etaopt)$ that is closest to
the posterior $\p(\zeta \vert \x)$ according to a divergence measure on
posterior distributions. We will make the common choice of Kullback-Leibler (KL)
divergence:
%
\begin{align}\eqlabel{kl_def}
%
\KL{\q(\zeta \vert \eta) || \p(\zeta \vert \x)}
={}&    \expect{\q(\zeta \vert \eta)}{
        \log \q(\zeta \vert \eta) - \logp(\x, \zeta)} + \logp(\x).
%
\end{align}
%
As we discuss below, we will choose $\q(\zeta \vert \eta)$ so that we can easily
approximate the above expectation with respect to $\q(\zeta \vert \eta)$ as a
closed-form function of $\eta$.  We will write $\KL{\eta}$ for our
approximation, choosing the optimal variational parameter $\etaopt$ to satisfy
%
\begin{align}\eqlabel{vb_optimization}
%
\etaopt :={} \argmin_{\eta \in \etadom} \KL{\eta} \mathwhere
\KL{\eta} \approx{} \KL{\q(\zeta \vert \eta) || \p(\zeta \vert \x)},
%
\end{align}
%
where the tractable objective function $\KL{\eta}$ may not be an exact KL
divergence. Notice that the intractable $\logp(\x)$ term does not depend on
$\eta$, and so can be neglected in the objective function $\KL{\eta}$.

In practice, forming an approximating posterior for BNP can be challenging since
the latent variables $\nu$ and $\beta$ are (countably) infinite dimensional. We
would like to keep dimension of the variational parameter $\eta$ finite in order
for the optimization in \eqref{vb_optimization} to be tractable. In the present
paper, we will follow \citet{blei:2006:vi_for_dp} and use a truncated
stick-breaking representation in the variational distribution. We choose a
truncation parameter $\kmax$ large but finite,
%
\footnote{For $\k > \kmax$, $\q(\nu_\k)$ is effectively a point mass  but
$\p(\nu_\k \vert \x)$ is dominated by the Lebesgue measure.  So the KL
divergence $\KL{\q(\nu_\k) || \p(\nu_\k \vert \x)}$ is not well-defined, even
though $\q(\nu_\k)$ does form a sensible approximation to $\p(\nu_\k \vert \x)$
in measures of posterior divergence such as the Wasserstein distance.  This is
one sense in which our tractable objective function $\KL{\eta}$ is not a proper
KL divergence. }
%
and we set $\q(\nu_\k = 1 | \eta) =
1$ for all $\k > \kmax$. This implies that under $\q$, $\pi_\k = 0$ with
probability one for all $\k > \kmax$ (\eqref{stick_breaking}). Correspondingly,
we also set $\q(\z_{\n\k} = 0 | \eta) = 1$ for $\k > \kmax$.

Notice that only our variational approximation is truncated---the model
(\eqref{bnp_model}) itself is not finite. We set $\kmax$ large enough in our
variational approximation to ensure that a large proportion of the components
are unoccupied with high probability under $\q$, in which case the truncation
approximates the fully nonparametric model with $\kmax = \infty$.

For the generic BNP mixture model in \eqref{bnp_model}, we use a mean-field
variational approximating family of the following form:
%
\begin{align}\eqlabel{vb_mf}
%
\q(\zeta \vert \eta) =
    \left( \prod_{\k=1}^{\kmax - 1} \q(\nuk \vert \eta) \right)
    \left( \prod_{\k=1}^{\kmax} \q(\beta_\k \vert \eta) \right)
    \left( \prod_{\n=1}^{\N} \q(\z_{\n} \vert \eta) \right).
%
\end{align}
%
Because $\pi_\k = 0$ for all $\k > \kmax$, we can ignore the latent variables
$\beta_\k$ for $\k > \kmax$ in defining our variational approximation.

Given an expected posterior quantity of interest as in
\exref{insample_nclusters_simple}, we can simply approximate it with the
corresponding variational expectation.

%%%%%%%%%%%%%%%%%%%%%%%%%%%%%%%%%%%%%%%%%%%%%%%%%%%%%%%%%%%%%%%%%%%%%%%%%%
%%%%%%%%%%%%%%%%%%%%%%%%%%%%%%%%%%%%%%%%%%%%%%%%%%%%%%%%%%%%%%%%%%%%%%%%%%
\begin{ex}\exlabel{insample_nclusters_simple}
%
Given a solution $\etaopt$ to \eqref{vb_optimization}, the intractable posterior
expectation in \exref{insample_nclusters_simple} can be approximated as
a funciton of $\etaopt$:
%
\begin{align*}
%
\g(\etaopt) :={}
    \expect{\q(\z\vert\etaopt)}{\nclusters_0(\z)} \approx
    \expect{\p(\z\vert\x)}{\nclusters_0(\z)}.
%
\end{align*}
%
In fact, given the mean field factorization of \eqref{vb_mf}, the function
$\g(\etaopt)$ has a particularly simple form.  Since each $\z_{\n\k}$ is either
zero or one, $\sum_{\n=1}^N \z_{\n\k} > 0$ if and only if $\prod_{\n=1}^\N (1 -
\z_{\n\k}) = 0$, so we can re-write the quantity of interest as
%
\begin{align*}
%
\nclusters_0(\z) ={}&
    \sumk \left(1 -  \prod_{\n=1}^\N (1 - \z_{\n\k})\right).
%
\end{align*}
%
Then, by the mean field assumption,
%
\begin{align*}
%
g(\etaopt) ={}
\expect{\q(\z \vert \etaopt)}{\nclusters_0(\z)} ={}
    \sumk \left(1 -  \prod_{\n=1}^\N
        (1 - \expect{\q(\z \vert \etaopt)}{\z_{\n\k}})\right).
%
\end{align*}
%
It will be useful later to observe that the map $\eta \mapsto \g(\eta)$ is
differentiable whenever $\eta \mapsto \expect{\q(\z \vert \eta)}{\z_{\n\k}}$
is differentiable.
%
\end{ex}
%%%%%%%%%%%%%%%%%%%%%%%%%%%%%%%%%%%%%%%%%%%%%%%%%%%%%%%%%%%%%%%%%%%%%%%%%%

We will discuss more details of the VB approximation, including the form of the
mean field factors and the evaluation of $\KL{\eta}$ in more detail below in
\secref{computing_sensitivity}, as well as in the experiments of
\secref{results}.


    % \subsection{Robustness and the Taylor series approximation}
    % \seclabel{model_taylor}
    % Typically, many different choices for $\pstick$ may be {\em a priori}
reasonable.  A particularly common choice for $\pstick$ is the
$\mathrm{Beta}(\nuk \vert 1, \alpha)$ density, which we write as
%
\begin{align*}
%
\pstick(\nuk \vert \alpha) :=
\mathrm{Beta}(\nuk \vert 1, \alpha) =
    \frac{\Gamma(1 + \alpha) (1 - \nuk)^{\alpha - 1}}
         {\Gamma(\alpha)},
%
\end{align*}
%
When $\pstick$ is $\mathrm{Beta}(\nuk \vert 1, \alpha)$, the resulting
distribution on $\pi$ is known as the $\textit{GEM distribution}$, and we write
$\pi \sim \mathrm{GEM}(\alpha)$.
%
The GEM distribution is closely related to the Dirichlet process (DP).
Define a measure on $\betadom$ as
%
\begin{align*}
  \mathcal{M} = \sum_{\k = 1}^\infty \pi_\k\delta_{\beta_\k},
\end{align*}
%
which places atoms at points $\beta_k$ with weight $\pi_\k$. When $\pi \sim
\mathrm{GEM}(\alpha)$ and $\beta_\k \iid \pbetaprior(\beta_\k)$, $\mathcal{M}$
is a random measure is distributed according to Dirichlet process with
concentration parameter $\alpha$ and base measure $\pbetaprior$
\citep{ferguson:1973:bayesian, sethuraman:1994:constructivedp}.

Typically, the concentration parameter $\alpha$ is not known in advance.
Rather, $\alpha$ may {\em a priori} plausibly lie within some reasonable range.
Since the prior $\pstick(\nuk \vert \alpha)$ depends on $\alpha$, the posterior
the posterior expectation $\expect{\p(\z \vert \x, \alpha)}{\nclusters_0(\z)}$
depends on $\alpha$ as well.  If $\expect{\p(\z \vert \x,
\alpha)}{\nclusters_0(\z)}$ varies meaningfully as $\alpha$ varies over its
plausible values, then the quantity of interest $\expect{\p(\z \vert \x,
\alpha)}{\nclusters_0(\z)}$ is not robust to the choice of $\alpha$.

In practice, one chooses some ``base value,'' $\alpha_0$, and runs a
computationally expensive posterior approximation procedure such variational
Bayes (VB), giving an approximate value for $\expect{\p(\z \vert \x,
\alpha_0)}{\nclusters_0(\z)}$.


More generally, there may be no {\em a priori} reason to believe that $\pstick$
lies in the Beta family at all (other than computational convenience). By
$\pbase$ and $\palt$ denote two candidate stick-breaking densities, we can
. Suppose that
parameterizing one-dimensional paths in the space of prior densities.  Let
$\pbase$


%
%
%
%
% Even if one is willing to restrict However, there is typically no {\em a priori}
% reason to assume that $\pi \sim \mathrm{GEM}(\alpha)$ is a realistic summary of
% our prior beliefs for any particular $\alpha$.
%
%
% We keep the generic notation $\pstick$ for stick-breaking distributions because
% in our sensitivity analysis, we will consider stick-breaking distributions that
% are outside the family of Beta distributions.


\section{A local approximation for sensitivity}
\seclabel{local_sensitivity}

When, then, is $\etaopt(\t)$ continuously differentiable?  We will now state
some sufficient conditions under which we can apply the implicit function
theorem (e.g., \citet{krantz:2012:implicit}) to prove the continuous
differentiability of $\etaopt(\t)$.

The first key assumption, \assuref{dist_fun_nice}, states sufficient conditions
for which we can apply the dominated convergence theorem to variational
expectations of some generic function $\psi(\theta, \t)$, allowing us to
translate continuity of the variational and model densitites into continuity of
the variational objective.  The details can be found in \lemref{logq_derivs,
logq_continuous} of \appref{cont_lemmas}.
%
In case \assuref{dist_fun_nice} seems forbidding, observe that, in lieu of
\assuref{dist_fun_nice} we might equivalently have said that we can exchange
limits and variational expectations whenever needed.  \Assuref{dist_fun_nice} is
simply a precise catalogue of what is needed.

% %%%%%%%%%%%%%%%%%%%%%%%%%%%%%%%%%%%%%%%%%%%%%%%%%%%%%%%%%%%%%%%%%%%%%%%%%%%%
% %%%%%%%%%%%%%%%%%%%%%%%%%%%%%%%%%%%%%%%%%%%%%%%%%%%%%%%%%%%%%%%%%%%%%%%%%%%%
% \begin{lem}
% %
% Under \assuref{dist_fun_nice}, we can exchange the order of integration
% and differentiation in
% %
% \begin{align*}
% %
% \fracat{\partial\expect{\q(\theta \vert \eta)}{\psi(\theta, \t)}}
%        {\partial\eta}{\etaopt, \t=0}
% \textrm{, }
% \fracat{\partial^2\expect{\q(\theta \vert \eta)}{\psi(\theta, \t)}}
%        {\partial\eta \partial\eta}{\etaopt, \t=0}
% \textrm{, and }
% \fracat{\partial^2\expect{\q(\theta \vert \eta)}{\psi(\theta, \t)}}
%        {\partial\eta \partial \t}{\etaopt, \t=0}.
% %
% \end{align*}
% %
% (See also \lemref{logq_derivs, logq_continuous} of \appref{cont_lemmas}
% for a more detailed statement.)
% %
% \end{lem}
% %%%%%%%%%%%%%%%%%%%%%%%%%%%%%%%%%%%%%%%%%%%%%%%%%%%%%%%%%%%%%%%%%%%%%%%%%%%%
%


% Finally, in \assuref{q_stick_regular} we draw the needed connection between the
% class of prior perturbations and the variational approximation.
%
% %%%%%%%%%%%%%%%%%%%%%%%%%%%%%%%%%%%%%%%%%%%%%%%%%%%%%%%%%%%%%%%%%%%%%%%%%%%%
% %%%%%%%%%%%%%%%%%%%%%%%%%%%%%%%%%%%%%%%%%%%%%%%%%%%%%%%%%%%%%%%%%%%%%%%%%%%%
% \begin{assu}\assulabel{q_stick_regular}
% %
% Under \defref{prior_t}, assume that the variational densities $\q(\theta \vert
% \eta)$ satisfy \assuref{dist_fun_nice} with both $\psi(\theta, \t) \equiv 1$ (no
% $\theta$ dependence) and with and $\psi(\theta, \t) = \log \p(\theta \vert \t) -
% \log \p(\theta \vert \t=0)$.
% %
% \end{assu}
% %%%%%%%%%%%%%%%%%%%%%%%%%%%%%%%%%%%%%%%%%%%%%%%%%%%%%%%%%%%%%%%%%%%%%%%%%%%%


We conclude this section by showing that \thmref{etat_deriv} applies to the
setting of BNP stick-breaking.

%%%%%%%%%%%%%%%%%%%%%%%%%%%%%%%%%%%%%%%%%%%%%%%%%%%%%%%%%%%%%%%%%%%%%%%%%%%%
%%%%%%%%%%%%%%%%%%%%%%%%%%%%%%%%%%%%%%%%%%%%%%%%%%%%%%%%%%%%%%%%%%%%%%%%%%%%
\begin{lem}\lemlabel{normal_q_is_regular}
%
In the setting of \assuref{dist_fun_nice}, let $\theta \in \mathbb{R}$, let
$\mu$ be the Lebesgue measure, and let $\q(\theta \vert \eta) = \normdist{\theta
\vert \eta}$ be a normal distribution parameterized by its natural exponential
family parameters.

Let $\sigma(\eta)$ denote the standard deviation of the normal distribution.
Fix $\eta_0$ such that $\sigma(\eta_0) > 0$, and let $\ball_\eta$ be an open set
containing $\eta_0$ such a that $\sigma_{max} := \sup_{\eta \in \ball_\eta}
\sigma(\eta) < \infty$.

If there exists a neighborhood $\ball_\t$ of $\t_0$ such that $\abs{\psi(\theta,
\t)}$ and $\abs{\psigrad{\theta, \t}}$ are uniformly bounded by some multiple of
$\exp(-\abs{\theta})$ for all $\t \in \ball_\t$, then $\q$ and $\psi$ satisfy
\assuref{dist_fun_nice}.

\begin{proof}
%
By properties of the exponential family,
%
\begin{align*}
%
\lqgrad{\theta, \eta} ={} (\theta, \theta^2)^T \mathand&
\lqhess{\theta, \eta} ={} 0_{2\times2} \Rightarrow\\
%
\norm{\lqgrad{\theta, \eta}}_2^2 ={} \theta^2 + \theta^4 \mathand&
\norm{\lqhess{\theta, \eta}}_2 ={} 0.
%
\end{align*}
%
Let $\ballclosed_\eta$ denote the closure of $\ball_\eta$, and let
%
\begin{align*}
%
\eta^* := \argmax_{\eta \in \ballclosed_\eta}
    \expect{\q(\theta \vert \eta)}{\exp(\abs{\theta})}.
%
\end{align*}
%
By standard properties of the normal and the boundedness of $\sigma(\eta)$, the
right hand side of the preceding display is finite.
%
Then
%
\begin{align*}
\int \q(\theta \vert \eta) \psi(\theta, \t) \mu(d \theta) \le{}&
    \left( \sup_{\theta} \sup_{\t \in \ball_\t}
        \abs{\psi(\theta, \t)} \exp(-\abs{\theta}) \right)
    \int \q(\theta \vert \eta) \exp(\theta) \mu(d \theta)
%
\\\le{}&
    \const
    \expect{\q(\theta \vert \eta^*)}{\exp(\abs{\theta})}.
    \quad\constdesc{\eta, \t}
%
\end{align*}
%
Therefore, for \assuitemref{dist_fun_nice}{fundom}, we can take $M(\theta)
\propto \q(\theta \vert \eta^*) \exp(\abs{\theta})$. The other terms follow
similarly, since each multiplier of $\q(\theta \vert \eta)$ is dominated by
$\exp(-\abs{\theta})$.  The final $M(\theta)$ simply takes the largest
of the five constants.
%
\end{proof}
%
\end{lem}
%%%%%%%%%%%%%%%%%%%%%%%%%%%%%%%%%%%%%%%%%%%%%%%%%%%%%%%%%%%%%%%%%%%%%%%%%%%%


%%%%%%%%%%%%%%%%%%%%%%%%%%%%%%%%%%%%%%%%%%%%%%%%%%%%%%%%%%%%%%%%%%%%%%%%%%%%
%%%%%%%%%%%%%%%%%%%%%%%%%%%%%%%%%%%%%%%%%%%%%%%%%%%%%%%%%%%%%%%%%%%%%%%%%%%%
\begin{ex}\exlabel{gem_pert_ok}
%
To analyze \exref{alpha_perturbation}, we take $\theta$
in \lemref{normal_q_is_regular}
be the unconstrained stick-breaking proportion $\lnuk$, which
recall from \secref{stick_expectations}
was normally distributed under $\q$.
Let $\mu$ be the Lebesgue measure on $\mathbb{R}$.

In \exref{alpha_perturbation},
the parameterization of the stick-breaking distribution was given by
\begin{align*}
  \log \pstick(\nuk \vert \t) - \log \pstick(\nuk \vert \t=0) ={}&
  \t \log(1 - \nuk).
\end{align*}
%
Expressing the perturbation in terms of
$\lnuk$,
%
\begin{align*}
%
\log \pstick(\lnuk \vert \t) - \log \pstick(\lnuk \vert \t=0) ={}&
\t \log\left(1 - \frac{\exp(\lnuk)}{1 + \exp(\lnuk)}\right)
\\={}&
-\t \log\left(1 + \exp(\lnuk)\right).
%
\end{align*}
%
So, by \lemref{normal_q_is_regular}, \assuref{q_stick_regular} is satisfied with
the VB approximation given in \secref{stick_expectations} and the parametric
perturbation given in \exref{alpha_perturbation}.
%
\end{ex}
%%%%%%%%%%%%%%%%%%%%%%%%%%%%%%%%%%%%%%%%%%%%%%%%%%%%%%%%%%%%%%%%%%%%%%%%%%%%



%%%%%%%%%%%%%%%%%%%%%%%%%%%%%%%%%%%%%%%%%%%%%%%%%%%%%%%%%%%%%%%%%%%%%%%%%%%%
%%%%%%%%%%%%%%%%%%%%%%%%%%%%%%%%%%%%%%%%%%%%%%%%%%%%%%%%%%%%%%%%%%%%%%%%%%%%

\begin{cor}\corlabel{gem_approximation_ok}
%
For the variational approximation of \secref{model_vb} and perturbation
given in \exref{alpha_perturbation}, $\alpha \mapsto \etaopt(\alpha)$
is continuously differentiable.
%
\begin{proof}
%
We have already shown in \exref{gem_pert_ok} that \assuref{q_stick_regular} is
satisfied.  Given that the variational approximations to $\p(\z \vert \x, \beta,
\nu)$ and $\p(\beta \vert \x, \nu, \z)$ are conjugate exponential family
approximations, $\eta \mapsto \KL{\eta, 0}$ is continuous.  It remains only to
numerically find $\etaopt$ and verify \assuitemref{kl_opt_ok}{kl_hess}, i.e.
that the Hessian is positive definite at the optimum.
%
\end{proof}
%
\end{cor}

%%%%%%%%%%%%%%%%%%%%%%%%%%%%%%%%%%%%%%%%%%%%%%%%%%%%%%%%%%%%%%%%%%%%%%%%%%%%

% We conclude this section with a brief remark about computing the expectation
% $\crosshessian$ in our BNP sensitivity analysis.
% We are interested in sensitivity to the stick-breaking distribution,
% so only the prior terms on stick-breaking proportions
% $\nu = (\nu_1, ..., \nu_{\kmax - 1})$ depends on $t$.
% Because the elements of $\nu$ fully factorize
% under both the prior and the variational distributions,
% $\crosshessian$ decomposes as
% \begin{align}
%   \crosshessian &=
%   \sum_{\k=1}^{\kmax - 1}
%           \expect{\q(\nuk \vert \eta)}
%                  {
%                  \lqgrad{\nuk \vert \etaopt}
%                  \fracat{\log \pstick(\nuk \vert \t)}{\partial \t}{\t = 0}
%                  } \notag\\
%   &= \sum_{\k=1}^{\kmax - 1}
%          \evalat{\nabla_\eta \expect{\q(\nuk \vert \eta)}
%                 {
%                 \fracat{\log \pstick(\nuk \vert \t)}{\partial \t}{\t = 0}
%                 }}{\eta = \etaopt(0)},
% \eqlabel{sens_mixed_partial}
% \end{align}
% where we assumed that $\q(\theta \vert \eta)$ is normalized, so
% $\lqgradbar{\theta \vert \etaopt} = \lqgrad{\theta \vert \etaopt}$,
% and that the assumptions of \thmref{etat_deriv} hold, so we
% can freely exchange derivatives with expectations.
%
% We approximate the expectation using GH quadrature (\eqref{gh_integral}),
% with
% $f(\nu_k) = \fracat{\log \pstick(\nuk \vert \t)}{\partial \t}{\t = 0}$.
% In all the functional forms for
% $\t \mapsto \pstick(\nuk \vert \t)$ considered below,
% $f(\nu_k)$ can be provided in either closed-form or computed with automatic differentiation.
% The resulting GH approximation is a deterministic function of $\eta$,
% and thus the gradient in \eqref{sens_mixed_partial} can be computed
% with another application of automatic differentiation.
% Note that $\crosshessian$ is sparse in \eqref{sens_mixed_partial}:
% it is zero for all entries of
% $\eta$ other than those that parameterize the sticks.


\section{The influence function and worst-case functional perturbations}
\seclabel{influence_function}
In this section, we show how to find worst-case functional perturbations to the
stick form. In particular, we start by  to motivating an norm measuring the size
of a functional perturbation $\phi$. We then show how to compute an influence
function to summarize the effect of different choices of $\phi$. We prove that,
for multiplicative perturbations and the $\infty$-norm, the linear approximation
is uniformly good. Finally, we show that this uniformly good approximation is
unique among many alternative choices of functional perturbation.

% %
% \noindent \textbf{An infinity norm.}
% First, we define $\norminf{\cdot}$ and a corresponding ball.
% Let $\mu$ be a probability measure on $[0,1]$. Then
% \begin{equation} \eqlabel{infty_norm}
% 	\norminf{\phi} := \esssup_{\nu_0 \sim \mu} \abs{\phi(\nu_0)},
% 	\quad \ball_\phi(\delta) := \left\{ \phi: \norminf{\phi} <
% \delta \right\}.
% \end{equation}
% %
%%
\noindent \textbf{The influence function.}
%
Next we define the influence function $\infl$ and discuss its usefulness for
understanding the effect of functional perturbations $\phi$.
\todo{The experiments use $\mathbb{R}$ as the space, not $[0,1]$.  So it
would be good to be explicit about the densitites.}
%
\begin{cor}\corylabel{etafun_deriv_form_stick}
%
Under the conditions of \thmref{bnp_deriv}, with $\norminf{\phi} < \infty$ and
$\varepsilon = \t$, let $\g(\eta): \etadom \mapsto \mathbb{R}$ denote a
continuously differentiable real-valued function of interest.  Define the
\emph{influence function} $\infl: [0,1] \mapsto \mathbb{R}$:
%
\begin{align} \eqlabel{infl_defn_bnp}
%
\infl(\cdot) :={}&
    - \sum_{k=1}^{\kmax-1} \fracat{d g(\eta)}{ d \eta^T}{\etaopt} \hessopt^{-1}
        \lqgradbark{\cdot \vert \etaopt}
        \qk(\cdot \vert \etaopt),
%
\end{align} where $\lqgradbark{\cdot \vert \etaopt}$ and $\qk(\cdot \vert
\etaopt)$ replace $\q(\zeta \vert \eta)$ with just the factor of $\q$ for
$\nu_k$.
%
Then the derivative \eqref{bnp_vb_eta_sens} can be written as
%
\begin{align} \eqlabel{vb_eta_infl_sens_bnp}
%
\fracat{d \g(\etaopt(\t))}{d \t}{0} ={}&
    \int_0^1 \infl(\nu_0) \phi(\nu_0) d\nu_0.
%
\end{align}
\end{cor}
%
\begin{proof}
%
The form of the influence function is given by gathering terms in
\eqref{bnp_vb_eta_sens} and re-writing the variational expectation as an
integral over $[0,1]$. We establish an analogous general result for mean-field
VB approximations in \coryref{etafun_deriv_form} of
\appref{diffable_nonparametric}, specializing to the BNP case in
\exref{infl_univariate} of \appref{diffable_nonparametric}.
%
\end{proof}

Although we will shortly use the influence function to find the formal
worst-case choice of $\phi$, we can also use the influence function to
informally choose influential prior perturbations. In our experiments in
\secref{results}, we show how to do this by choosing $\phi$ that align with
particularly high-magnitude positive or negative values of the influence
function; this alignment will ensure a large positive or negative gradient and
hence a large change.

%%
\noindent \textbf{Worst-case functional perturbations.}
%
With \coryref{etafun_deriv_form_stick} in hand, we can find the worst-case
choice of $\phi \in \ball_\phi(\delta)$.

\LinfExamplesFig{}

First, we find the VB analogue of \citet[Result 11]{gustafson:1996:local}.
Second, we justify using the influence function for finding the worst case by
establishing uniform quality of the linear approximation for sufficiently small
$\ball_\phi(\delta)$.

\begin{cor}\corylabel{etafun_worst_case_stick}
%
Under the conditions of \coryref{etafun_deriv_form_stick},
%
\begin{align*}
%
\sup_{\phi \in \ball_\phi(\delta)}
    \fracat{d g(\etaopt(\t))}{d \t}{0} =
        \delta \int \abs{\infl(\nu_0)} \mu(d\nu_0),
%
\end{align*}
%
and the supremum is achieved at the perturbation
$\phi^*(\cdot) = \delta \, \mathrm{sign}\left(\infl(\cdot)\right)$.
%
\end{cor}
%
\begin{proof}
%
The result follows immediately from applying H{\"o}lder's inequality to
\eqref{vb_eta_infl_sens_bnp}. We establish a similar but much more general
result for mean-field VB approximations with general choices of model and
parameters in \coryref{etafun_worst_case} of \appref{diffable_worst_case}. The
present result is a special case using \exref{infl_univariate} of
\appref{diffable_worst_case}.
%
\end{proof}

To justify using linear approximations to explore the unit ball
$\ball_\phi(\delta)$ and find the worst case, we require a stronger result than
\coryref{etafun_deriv_form}. In particular, \coryref{etafun_deriv_form} states
only that, for a {\em particular} direction $\phi$, $\t \mapsto \etaopt(\t)$ is
continuously differentiable.  Since $\t \phi \in \ball_\phi(\t \norminf{\phi})$,
\coryref{etafun_deriv_form} implies only that, for a fixed $\phi$, one can make
$\t$ sufficiently small so that the error $\abs{\etaopt(\t) - \etalin(\t)}$ goes
to zero faster than $\t$. But, if we write $\etaopt(\t\phi)$ and $\etalin(\t
\phi)$ to make the dependence on $\phi$ explicit, then
\coryref{etafun_deriv_form} does not imply that, for a fixed $\delta$ (no matter
how small), the worst-case error $\sup_{\phi \in \ball_\phi(\delta)}
\abs{\etaopt(\phi) - \etalin(\phi)}$ is bounded, much less that it goes to zero
faster than $\delta$.

To be able to apply \coryref{etafun_worst_case} to find the worst-case
perturbation $\phi$, we need to establish that the approximation is sufficiently
good over all $\phi$ of interest. Observe that $\phi$ is a member of the Banach
space $L_\infty$ \citep[Theorem 5.2.1]{dudley:2018:real}.  We require that the
map $\phi \mapsto \etaopt(\phi)$, which maps $L_\infty$ to $\mathbb{R}^\etadim$,
admits a uniformly good linear approximation amongst $\phi \in
\ball_\phi(\delta)$. In other words, we require $\phi \mapsto \etaopt(\phi)$ to
be Fr{\'e}chet differentiable, as we now formally define.

\begin{defn}\deflabel{diffable_classes}
    (Fr{\'e}chet differentiability,
    \citep[Definition 4.5]{zeidler:2013:functional})
%
Let $B_1$ and $B_2$ denote Banach spaces, and let $\ball_1 \subseteq B_1$ define
an open neighborhood of $\phi_0 \in B_1$.
%
A function $f: \ball_1 \mapsto B_2$ is {\em Fr{\'echet} differentiable} (also
known as boundedly differentiable) at $\phi_0$ if there exists a  bounded linear
operator, $f^{\mathrm{lin}}: B_1 \mapsto B_2$, such that
%
\begin{align*}
%
\lim_{t \rightarrow 0}
    \sup_{\phi: \norm{\phi - \phi_0} = 1}
    \frac{f(\phi) - f(\phi_0) -
          f^{\mathrm{lin}}(t (\phi - \phi_0))
         }{t} \rightarrow 0.
%
\end{align*}
%
\end{defn}

By \citep[Proposition 4.8]{zeidler:2013:functional}, if a function is
Fr{\'e}chet differentiable, then the linear operator $f^{\mathrm{lin}}$ is given
precisely by the directional derivative $d f(t (\phi - \phi_0)) / d t$. Thus, if
$\phi \mapsto \etaopt(\phi)$ is Fr{\'e}chet differentiable, its derivative is
given by \coryref{etafun_deriv_form}.  Fr{\'e}chet differentiability guarantees
that the error of the linear approximation given by \coryref{etafun_deriv_form}
does not blow up in the ball $\ball_\phi(\delta)$.

We emphasize that Fr{\'e}chet differentiability is neither sufficient nor
necessary for a derivative to be useful.  For example, it is possible in
principle for a function to be Fr{\'e}chet differentiable but still have a very
large finite second derivative, and so fail to extrapolate meaningfully to any
alternatives one cares about.  Conversely, if a function fails to be Fr{\'e}chet
differentiable, the derivative may still perform well in particular directions,
including that chosen by \coryref{etafun_worst_case}.  Nevertheless, Fr{\'e}chet
differentiability is a strong local result, and provides some assurance that one
can use results such as \coryref{etafun_worst_case} without uncovering
pathological behavior.

Finally, then, we prove that our perturbation here is Fr{\'e}chet differentiable.

%%%%%%%%%%%%%%%%%%%%%%%%%%%%%%%%%%%%%%%%%%%%%%%%
\begin{thm}\thmlabel{eta_phi_deriv_stick}
%
Under the conditions of \thmref{bnp_deriv}, the map $\phi \mapsto \etaopt(\phi)$
is well-defined and continuously Fr{\'e}chet differentiable in a neighborhood of
the zero function as a map from $\linf$ to $\mathbb{R}^\etadim$,
with the derivative given in \coryref{etafun_deriv_form}. \end{thm}
%
\begin{proof}
%
Our result here is a special case of a general result for VB approximations
based on KL divergence given in \thmref{eta_phi_deriv} of
\appref{diffable_worst_case}.
%
\end{proof}

%%
\noindent \textbf{Many other functional perturbations and norms are not Fr{\'e}chet differentiable.}
%
So far we have focused on the multiplicative functional perturbations in
\eqref{mult_perturbation} combined with the infinity norm in \eqref{infty_norm}.
We now ask whether we could perform a similar analysis for other functional
perturbations. We show that, of the perturbations proposed by
\citet{gustafson:1996:local}, only multiplicative perturbations yield
Fr{\'e}chet differentiable VB optima.

Specifically, \citet{gustafson:1996:local} examines general perturbations, from
initial prior $\pbase$ to alternative $\palt$, that take the following form --
with $\theta$ a parameter $\theta \in \thetadom \subseteq
\mathbb{R}^{\thetadim}$ and $p \in [1, \infty)$:
%
\begin{align}\eqlabel{p_pert_simple_bnp}
%
\ptil(\theta \vert \tp) :=
    \left((1 - \tp)\pbase(\theta)^{1/p} +
    \tp \frac{1}{p}\palt(\theta)^{1/p} \right)^{p}.
%
\end{align}
%
Again, let $\phi$ represent the perturbation size, now with:
%
\begin{align}\eqlabel{phi_lp_norm_bnp}
%
\phi(\theta \vert \palt, p) :={}
    \palt(\theta)^{1/p} - \pbase(\theta)^{1/p} \mathand
\norm{\phi}_p :={} \left(\int \abs{\phi(\theta)}^p \right)^{1/p}.
%
\end{align}
%
The limit $p \rightarrow \infty$ recovers our multiplicative perturbation in
\eqref{mult_perturbation} with infinity norm in \eqref{infty_norm}. The choice
$p=0$ recovers a purely additive perturbation.

Our next theorem shows that the KL is discontinuous for $p < \infty$.
Since Fr{\'e}chet differentiability implies continuity \citep[Proposition 4.8
(d)]{zeidler:2013:functional}, \thmref{kl_discontinuous} implies that it is
impossible to derive an analogue of \thmref{eta_phi_deriv} for perturbations of
the form \eqref{p_pert_simple_bnp} with the norms \eqref{phi_lp_norm_bnp}.
%
\begin{thm}\thmlabel{kl_discontinuous_main}
%
Let $\mu$ denote a measure on $\thetadom$ that is absolutely continuous with
respect to the Lebesgue measure, and let $\q(\theta)$ and $\pbase(\theta)$
denote densities with respect to $\mu$.  Without loss of generality, assume that
$\q(\theta) > 0$ on $\thetadom$.  Assume that $\KL{\q(\theta) ||
\pbase(\theta)}$ is well-defined and finite.

Then, for any $\epsilon > 0$ and any $M > 0$, we can find a density
$\palt(\theta)$ such that $\norm{\phi(\theta \vert \palt, p)}_p < \epsilon$ but
$\abs{\KL{q(\theta) || \palt(\theta)} - \KL{q(\theta) || \pbase(\theta)}} > M$.
%
\end{thm}
%
See \appref{diffable_lp} for a proof.
% \thmref{kl_discontinuous}

Recall from \secref{local_sensitivity} (and \exref{beta_inf_norm} of
\appref{diffable_nonparametric}) that there exist priors that cannot be formed
from \eqref{mult_perturbation} using $\phi$ with $\norminf{\phi} < \infty$. In
light of the proof of \thmref{kl_discontinuous_main}, the limited expressiveness
of multiplicative perturbations with the $\norminf{\cdot}$ norm looks like a
feature rather than a bug.
% The KL divergence that defines a variational objective cannot handle
% prior densities that are too close to zero.  The $\norminf{\cdot}$ norm
% considers such densities to be ``distant'' from $\pbase$, whereas the
% more permissive $\norm{\cdot}_p$ norms do not.
Consider \figref{linf_examples}, which illustrates the tradeoffs between the
various norms.  The two blue and red densities are far from one another
according to KL divergence since the red density takes values that are nearly
zero where the blue density has nonzero mass. They are also distant in
$\norminf{\cdot}$ since it takes a large multiplicative change to turn the
nonzero blue density into the nearly zero red density. However, the two
densities are close in $\norm{\cdot}_{p}$ since the region where the red density
is nearly zero has a small measure. In order for VB approximations to be
continuous (a necessary condition for Fr{\'e}chet differentiability), one must
consider a topology on priors that is no coarser than the topology induced by KL
divergence.  But since valid priors can take values close to zero, a sacrifice
in expressiveness of the neighborhood of zero must be made in order to induce a
topology that works with KL divergence. Multiplicative changes and the
$\norminf{\cdot}$ norm make such a tradeoff in a natural, easy-to-understand
way.

%\FunctionDistFig{}

In this sense, VB approximations based on KL divergence are inherently
non-robust to priors that ablate mass nearly to zero.  No parameterization of
the space of priors will relieve this non-robustness.  Only by basing
variational approximations on divergences other than KL will this non-robustness
be alleviated.


\section{Fast computation of the sensitivity}
\seclabel{computing_sensitivity}
A principal challenge of computing the sensitivity efficiently is the
high-dimensionality of the parameter $\zeta$ and hence variational parameters
$\eta$. In particular, we have seen that, in our BNP stick-breaking model,
$\zeta$ and $\eta$ both grow linearly with the number of data points $N$. This
growth leads to two potential major computational challenges: (1) a
high-dimensional optimization problem to extremize the VB objective and (2)
computing and inverting the Hessian $\hessopt$. Here we show how we can use
special structure in the model to reduce to low-dimensional problems and thereby
enjoy efficient computation.

%%
\noindent \textbf{Global and local parameters.} In both cases, the key to
reducing to a lower-dimensional problem is separating \emph{global} and
\emph{local} parameters. Global variables are common to all data points. Local
variables are unique to each data point. For instance, in a Gaussian (or other
typical) mixture model, the stick-breaking proportions $\nu$ and component
parameters $\beta$ are global. But the cluster assignment parameters $z$ are
local.

Let $\gamma$ denote the collection of global parameters. Since we use mean-field
VB, these parameters have their own variational parameters, which we denote
$\etaglob$. Similarly, let $\ell$ denote the local parameters and let
$\etalocal$ be the corresponding local variational parameters.

%%
\noindent \textbf{Reducing to optimization over the global variational
parameters.} We next show how to reduce the potentially high-dimensional
optimization problem over all of $\eta$ to optimizing over just the global
variational parameters $\etaglob$.

In all models we will consider, the conditional posterior $\p(\z \vert
\gamma,\x)$ has a tractable closed form.  Since we choose a conjugate mean field
approximating family for $\q(\z \vert \eta)$, the optimal local variational
parameters $\etaoptlocal$ can be written as a closed-form function of the global
variational parameters $\etaglob$. Let $\etaoptlocal(\eta_\gamma; \t)$ denote
this mapping, so that
%
\begin{align}\eqlabel{local_eta_optim}
\etaoptlocal(\etaglob; \t) :=
    \argmin_{\etalocal} \KL{(\eta_\gamma, \etalocal), \t}.
\end{align}
%
%When $\t=0$, we will write $\etaoptlocal(\etaglob; \t=0) =
%\etaoptlocal(\etaglob)$.
In \exref{qz_optimality} (\appref{gmm_global_local_vb}), we illustrate with a
Gaussian mixture model example.
%
Using \eqref{local_eta_optim}, we can rewrite our objective as a
function of the global parameters.  Define
%
\begin{align*}
\KLglobal(\etaglob, \t) :=
    \mathrm{KL}\Big((\etaglob, \etaoptlocal(\etaglob; \t)), \t\Big).
\end{align*}
%
The $\etaoptglob(\t)$ that minimizes $\KLglobal(\etaglob, \t)$ is the same as
the corresponding sub-vector of the $\etaopt(\t)$ that minimizes $\KL{\eta,
\t}$.  %Therefore we can use the objective function $\KLglobal(\etaglob, \t)$ as
%a numerical surrogate for $\KL{\eta, \t}$ when optimizing or computing
%derivatives.

Rather than optimizing the $\KL{\eta}$ over all variational parameters then, we
numerically optimize $\KLglobal$, which is a function only of the relatively
low-dimensional global parameters.  To minimize $\KLglobal(\etaglob)$ in
practice, we run the BFGS algorithm with a loose convergence tolerance followed
by trust-region Newton conjugate gradient
\citep[Chapter~7]{nocedal:2006:numerical} to find a high-quality optimum. After
the optimization terminates at an optimal $\etaoptglob$, the optimal local
parameters $\etaoptlocal$ can be set in closed form to produce the entire vector
of optimal variational parameters $\etaopt = (\etaoptglob, \etaoptlocal)$.

\subsection{Computing and inverting the Hessian} Since the dimension $\etadim$
of $\eta$ scales with $N$, we can quickly reach cases where inverting or even
instantiating a dense matrix of size $\etadim \times \etadim$ in memory would be
prohibitive. The key to efficient computation is that $\hessopt$ is not dense;
we will again exploit structure inherent in the global/local decomposition.

For generic variables $a$ and $b$, let $\hess{ab}$ denote the sub-matrix
$\evalat{\partial^2 \KL{\eta} / \partial \eta_a \eta_b^T}{\etaopt}$, the Hessian
with respect to the variational parameters governing $a$ and $b$. We decompose
the Hessian matrix $\hessopt$ into four blocks according to the global/local
decomposition:
%
\begin{align*}
%
\hessopt =
\fracat{\partial^2 \KL{\eta}}
       {\partial \eta \partial \eta^T}
       {\etaopt} ={}&
\left(
\begin{array}{cc}
   \hess{\gamma\gamma} & \hess{\gamma\ell} \\
   \hess{\ell\gamma}     & \hess{\ell\ell} \\
\end{array}
\right).
%
\end{align*}
%
Similarly, let $\crosshessian_\gamma$ be the components of $\crosshessian$
corresponding to the variational parameters $\etaglob$.  The local components,
$\crosshessian_\ell$, are zero since no local variables enter the expectation in
\eqref{bnp_vb_crosshessian} when we are perturbing the stick-breaking
distribution.
%
%We can thus write
%\begin{align*}
%  \crosshessian = \left( \begin{array}{c} \crosshessian_\gamma \\ 0 \end{array}\right).
  %
%\end{align*}

In this notation,
%
\begin{align} \eqlabel{global_local_derivative_breakdown}
%
\fracat{d \etaopt(\t)}{d \t}{t = 0} ={}&
-\left(
\begin{array}{cc}
   \hess{\gamma\gamma} & \hess{\gamma\ell} \\
   \hess{\ell\gamma}     & \hess{\ell\ell} \\
\end{array}
\right)^{-1}
\left( \begin{array}{c} \crosshessian_\gamma \\ 0 \end{array}\right).
%
\end{align}
%
Applying the Schur complement and focusing on the global parameters (see
\appref{more_hessian} for more details), we find
%
\begin{align}\eqlabel{global_sens}
  \fracat{d \etaopt_\gamma(\t)}{d \t}{t = 0} &=
  - \hessopt_\gamma^{-1}\crosshessian_\gamma
  \mathwhere
  \hessopt_\gamma := \left(\hess{\gamma\gamma} -
        \hess{\gamma\ell} \hess{\ell\ell}^{-1} \hess{\ell\gamma}\right),
\end{align}
%
In our model, $\hess{\ell\ell}$ is block diagonal, and the size of
$\hess{\gamma\gamma}$ is relatively small. Thus, each term of $\hessopt_\gamma$
can be tractably computed, stored in memory, and inverted, even on very large
datasets. While the Schur complement calculation is illustrative, we can get the
same benefits directly from automatic differentiation; see \appref{more_hessian}
for details.

In our BNP applications, it is not cost-effective to form and factorize
$\hessopt$ in memory.  Instead, we numerically solve linear systems of the form
$\hessopt^{-1} v$ using the conjugate gradient (CG) algorithm \citep[Chapter
5]{nocedal:2006:numerical}, which requires only Hessian-vector products that are
readily available through automatic differentiation.

%%
\noindent \textbf{A linear approximation only in the global variational parameters}.
%
With the tools above, we can separate out the linear approximation in the global
parameters and then directly compute the local parameters. In particular, we
compute
%
\begin{align}\eqlabel{global_lin_approx}
  \etalin_\gamma(\t) := \etaopt_\gamma +
  \fracat{d \etaopt_\gamma(\t)}{d \t}{\t=0} \t .
\end{align}
%
and then use $\etaoptlocal(\etaglob)$ e.g.\ in computing our quantity of
interest. We give an example for the expected number of clusters in
\appref{vb_insample_nclusters_example}.  In all our experiments, we use
\eqref{global_lin_approx} in this way.


\section{Experimental Results}
\seclabel{results}
We next evaluate our sensitivity approximations on three real data sets,
each with a different model using stick-breaking.
We find that our approximations largely agree with ground truth obtained by re-running the VB optimization, but
with the evaluation of our derivative an order of magnitude faster than
re-optimizing for a given perturbation.


    \subsection{Gaussian mixture modeling on iris data}
    \seclabel{results_iris}
    %%%%%%%%%%%%%%%%%%%%%%%%%%%%%%%%%%%%%%
%%%%%%%%%%%%%%%%%%%%%%%%%%%%%%%%%%%%%%
% Do not edit the TeX file your work
% will be overwritten.  Edit the RnW
% file instead.
%%%%%%%%%%%%%%%%%%%%%%%%%%%%%%%%%%%%%%
%%%%%%%%%%%%%%%%%%%%%%%%%%%%%%%%%%%%%%



We perform a clustering analysis of Fisher's iris data set
\cite{anderson:1936:iris}. Here each  data point (with $N=150$ total points)
represents $d=4$ measurements of a particular flower, from one of three iris
species. We use a standard Gaussian mixture model with conjugate
Gaussian-Wishart prior for the component parameters (detailed in
\appref{app_iris}) and a mean-field VB approximation with truncation parameter
\todo{Do we want to make it clear that $\gclustersabbr$ is not the same as
the example given in \secref{model_vb}, but rather the version that linearizes
only the global parameters?}
$\kmax = 15$. We consider two quantities of interest: (1) $\gclustersabbr$, the
posterior expected number of clusters among the $N$ observed data points, and
(2) $\gclusterspred$, the posterior predictive expected number of clusters in
$N$ new (i.e.\ as-yet-unseen) data points. We set the base stick-breaking prior
$\pbase(\nuk)$ to be the standard $\betadist{\nuk \vert 1,\alpha}$ distribution
with $\alpha = \alpha_0 = 2$. Under the base stick-breaking prior with
$\alpha_0$, the posterior expected number of clusters matches the three iris
species; see also \figref{iris_fit} in \appref{app_iris} for an illustration.

%%
\noindent \textbf{Sensitivity to the concentration parameter} We approximate the
changes in the quantities of interest as $\alpha$ varies over $\alpha\in[0.1,
4.0]$, which corresponds to an a priori expected number of clusters among $N$
data points in $[1.5,15]$ (\appref{app_beta_prior}). Over this range, the shape
of a $\betadist{1,\alpha}$ density varies considerably, as shown in
\figref{beta_priors} in \appref{app_beta_prior}.


\begin{knitrout}
\definecolor{shadecolor}{rgb}{0.969, 0.969, 0.969}\color{fgcolor}\begin{figure}[!h]

{\centering \includegraphics[width=0.784\linewidth,height=0.439\linewidth]{figure/iris_alpha_sens-1} 

}

\caption[The expected number of clusters in the original data set ($\gclustersabbr$, left) and in a new data set of size $N$ ($\gclusterspredabbr$, right) as $\alpha$ varies in the GMM fit of the iris data]{The expected number of clusters in the original data set ($\gclustersabbr$, left) and in a new data set of size $N$ ($\gclusterspredabbr$, right) as $\alpha$ varies in the GMM fit of the iris data. We formed the linear approximation at $\alpha_0=2$.}\label{fig:iris_alpha_sens}
\end{figure}


\end{knitrout}

\Figref{iris_alpha_sens} compares our linear approximation to ground truth on
the two quantities of interest as $\alpha$ varies. Over this range of $\alpha$,
the posterior expected number of clusters in the observed data is quite robust;
it remains nearly constant at three. The posterior predictive expected number of
clusters in $N$ new data points is less robust; it ranges roughly from
3.0 to 5.6 expected species.  Our approximation captures this
qualitative behavior. As expected, the approximation is least accurate furthest
from the $\alpha_0$, where the Taylor series is centered.

%%
\noindent \textbf{Sensitivity to functional perturbations.} Insensitivity of the
expected number of clusters $\gclustersabbr$ to $\alpha$ does not rule out
sensitivity to other prior perturbations. We now check how our approximation
fares for the multiplicative perturbations in \eqref{mult_perturbation}. We
consider perturbations $\phi$ that are Gaussian bumps in logit stick space, with
each perturbation centered at a different location on the real line.  Each row
of \figref{iris_fsens} corresponds to a different $\phi$. Each $\phi$ is shown
in gray in the leftmost plot of its row. The middle column of
\figref{iris_fsens} shows the stick-breaking prior $\p(\nuk \vert \phi)$ induced
by the corresponding $\phi$.


\begin{knitrout}
\definecolor{shadecolor}{rgb}{0.969, 0.969, 0.969}\color{fgcolor}\begin{figure}[!h]

{\centering \includegraphics[width=0.980\linewidth,height=0.862\linewidth]{figure/iris_fsens-1} 

}

\caption{Sensitivity of
        the expected number of in-sample clusters
        in the iris data set
        to three multiplicative perturbations each with $\norminf{\phi} = 1$.
        (Left) The multiplicative perturbation $\phi$ is in grey.
        The influence function $\Psi$, scaled so $\norminf{\Psi}=1$, is in purple.
        (Middle) The initial $\pbase(\nuk)$ (light blue)
        and alternative $\palt(\nuk)$ (dark blue) priors. 
        (Right) The effect of the perturbation
        on the change in expected number of in-sample clusters
        for $t \in[0, 1]$.}\label{fig:iris_fsens}
\end{figure}


\end{knitrout}

The rightmost column of \figref{iris_fsens}
shows the changes produced by the $\phi$ perturbation for that row.
We see that our approximation captures the qualitative behavior of the exact
changes.

We also see in this example that we can use the influence function to predict
the effect of functional changes to the stick-breaking prior. In the leftmost
column, we plot  in purple the influence function in the logit space.
\footnote{\coryref{etafun_deriv_form_stick} expresses the influence function in
the stick domain $[0,1]$, but, for visualization, it is easier to express the
influence function in the logit stick domain $\mathbb{R}$.  The general result
\coryref{etafun_deriv_form} in \appref{diffable_nonparametric} accomodates such
transformations.}
%
According to \coryref{etafun_deriv_form}, the sign and magnitude of the effect
of a perturbation should be determined by its integral against the influence
function.  Thus, when $\phi$ lines up with a negative part of $\infl$, as in the
first row, we expect the change to be negative.  Similarly, we expect the
perturbation of the bottom row to produce a positive change, and the middle row,
in which $\phi$ overlaps with both negative and positive parts of the influence
function, to produce a relatively small change. We see this intuition borne out
in the rightmost column.


\begin{knitrout}
\definecolor{shadecolor}{rgb}{0.969, 0.969, 0.969}\color{fgcolor}\begin{figure}[!h]

{\centering \includegraphics[width=0.980\linewidth,height=0.412\linewidth]{figure/iris_worstcase-1} 

}

\caption[Sensitivity of
        the expected number of in-sample clusters in the iris data set
        to the worst-case multiplicative perturbation with
        $\norminf{\phi} = 1$]{Sensitivity of
        the expected number of in-sample clusters in the iris data set
        to the worst-case multiplicative perturbation with
        $\norminf{\phi} = 1$.}\label{fig:iris_worstcase}
\end{figure}


\end{knitrout}

%%
\noindent \textbf{Worst-case functional perturbation.}
Finally, \figref{iris_worstcase} shows the worst-case multiplicative perturbation with $\norminf{\phi} = 1$, as given by \coryref{etafun_worst_case},
along with its effect on the prior and $\gclustersabbr$.
As expected, this worst-case perturbation
has a much larger effect on $\gclustersabbr$
compared to the other unit-norm perturbations in \figref{iris_fsens}.
However, even with the worst-case perturbation---which results in an unreasonably shaped prior density---the change in $\gclustersabbr$
is still small. We conclude that $\gclustersabbr$ appears
to be a robust quantity for this model and dataset.


%%%%%%%%%%
% these figures are moved to the appendix
%%%%%%%%%%
\newcommand{\BetaPriorsEx}{

\begin{knitrout}
\definecolor{shadecolor}{rgb}{0.969, 0.969, 0.969}\color{fgcolor}\begin{figure}[!h]

{\centering \includegraphics[width=0.980\linewidth,height=0.392\linewidth]{figure/beta_priors-1} 

}

\caption[Probability density functions of $\text{Beta}(1, \alpha)$ distributions, under various $\alpha$ considered for the iris data set]{Probability density functions of $\text{Beta}(1, \alpha)$ distributions, under various $\alpha$ considered for the iris data set.}\label{fig:beta_priors}
\end{figure}


\end{knitrout}
}

\newcommand{\IrisInitFit}{

\begin{knitrout}
\definecolor{shadecolor}{rgb}{0.969, 0.969, 0.969}\color{fgcolor}\begin{figure}[!h]

{\centering \includegraphics[width=0.588\linewidth,height=0.470\linewidth]{figure/iris_fit-1} 

}

\caption[The iris data in principal component space and
                      GMM fit at $\alpha = 2$]{The iris data in principal component space and
                      GMM fit at $\alpha = 2$. Colors denote inferred memberships and
                      ellipses represent estimated covariances. }\label{fig:iris_fit}
\end{figure}


\end{knitrout}
}


    \subsection{Regression mixture modeling}
    \seclabel{results_mice}
    %%%%%%%%%%%%%%%%%%%%%%%%%%%%%%%%%%%%%%
%%%%%%%%%%%%%%%%%%%%%%%%%%%%%%%%%%%%%%
% Do not edit the TeX file your work
% will be overwritten.  Edit the RnW
% file instead.
%%%%%%%%%%%%%%%%%%%%%%%%%%%%%%%%%%%%%%
%%%%%%%%%%%%%%%%%%%%%%%%%%%%%%%%%%%%%%



We next check our approximation on a more complex clustering task: clustering
time series, with a co-clustering matrix (and summaries thereof) as the quantity of interest.

%%
\noindent \textbf{Data and model.}
We use a publicly available data set of mice gene expression
\citep{shoemaker:2015:ultrasensitive}. Mice were infected with influenza virus,
and expression levels of a set of genes were assessed at 14 time points after
infection. Three measurements were taken at each time point (called biological
replicates), for a total of $\ntimepoints = 42$ measurements per gene.

The goal of the analysis is to cluster the time-course gene expression data
under the assumption that genes with similar time-course behavior may
have similar function.
Clustering gene expressions is often used for exploratory analysis
and is a first step before further downstream investigation.
It is important, therefore, to ascertain the
stability of the discovered clusters.

The left plot of \figref{example_genes} in \appref{app_mice}
shows the measurements of a single gene over time.
We model each gene as belonging to a latent component,
where each component defines a smooth expression curve over time.
Then, observations are drawn by adding i.i.d.\ noise to the smoothed
curve along with a gene-specific offset.
Following \citet{Luan:2003:clustering}, we construct the smoothers
using cubic B-splines.

Let $\x_\n\in\mathbb{R}^\ntimepoints$ be measurements of gene $\n$ at
$\ntimepoints$ time points. Let $\regmatrix$ be the $\ntimepoints \times \d$
B-spline regressor matrix, so that the $ij$-th entry of $\regmatrix$ is the
$j$-th B-spline basis vector evaluated at the $i$-th time point. The right plot
of \figref{example_genes} in \appref{app_mice} shows the B-spline basis.
The distribution of the data
arising from component $k$ is
%
\begin{align}\eqlabel{mice_model}
\p(\x_\n | \beta_\k, \b_\n) =
\normdist{\x_\n | \regmatrix\mu_\k + \b_\n,
\tau_\k^{-1}I_{\ntimepoints \times \ntimepoints}},
\end{align}
%
where $\b_\n$ is a gene-specific additive offset and $I$ is the identity matrix.
We include the additive offset because we
are interested in clustering gene expressions based on their patterns over time,
not their absolute level.
In this model, the component-specific parameters are $\beta_\k = (\mu_\k, \tau_\k)$,
the regression coefficients and the inverse noise variance.
The component frequencies are determined by stick-breaking according to $\nu$, and 
cluster assignments $z$ are drawn as in \secref{model_bnp}.
We use $\p(\nu_\k\vert\alpha_0) = \betadist{\nu_\k \vert 1, \alpha_0}$
with $\alpha_0 = 6$ as the initial stick-breaking prior.
%
%The mixture weights $\pi$ and cluster assignments $\z$ are drawn from the
%stick-breaking process described in \secref{model_bnp}.

Our variational approximation factorizes similarly to \eqref{vb_mf}
except with an additional factor for the additive shift.
In our variational approximation, we also make a simplification by letting
$\q(\beta_\k \vert \eta) = \delta (\beta_k \vert \eta)$,
where $\delta(\cdot \vert \eta)$ denotes a point mass at a parameterized location.
See \appref{app_mice} for further details concerning the model and
variational approximation.

%%
\noindent \textbf{Quantity of interest: the co-clustering matrix and summaries.}
In this application,
we are particularly interested in which genes cluster together,
so we focus on the posterior co-clustering
matrix.  Let $\gcoclustering(\eta)\in\mathbb{R}^{\N\times\N}$ denote the matrix
whose $(i,j)$-th entry is the posterior probability that gene $i$ belongs to the
same cluster as gene $j$, given by
%
\begin{align*}
%
[\gcoclustering(\eta)]_{ij} =
\expect{\q(\z\vert\eta)}{\ind{\z_{i} = \z_{j}}}  =
\begin{cases}
\sum_{k=1}^{\kmax}\left(\expect{\q(\z_i\vert\eta)}{\z_{ik}}
\expect{\q(\z_j\vert\eta)}{\z_{jk}}\right)
& \text{for } i \not= j\\
1 & \text{for } i = j
\end{cases}.
%
\end{align*}
\figref{gene_initial_coclustering} shows the inferred
co-clustering matrix at $\alpha_0$.

\newcommand{\MiceExampleGenes}{

\begin{knitrout}
\definecolor{shadecolor}{rgb}{0.969, 0.969, 0.969}\color{fgcolor}\begin{figure}[!h]

{\centering \includegraphics[width=0.980\linewidth,height=0.392\linewidth]{figure/example_genes-1} 

}

\caption[(Left) An example gene and its expression measured at 14 unique time points
    with three biological replicates at each time point.
     (Right) The cubic B-spline basis with 7 degrees of freedom,
    along with three indicator functions for the last three time points,
    $\timeindx = 72, 120, 168$]{(Left) An example gene and its expression measured at 14 unique time points
    with three biological replicates at each time point.
     (Right) The cubic B-spline basis with 7 degrees of freedom,
    along with three indicator functions for the last three time points,
    $\timeindx = 72, 120, 168$.}\label{fig:example_genes}
\end{figure}


\end{knitrout}
}
%

\newcommand{\MiceSmoothers}{
% moving this to appendix

\begin{knitrout}
\definecolor{shadecolor}{rgb}{0.969, 0.969, 0.969}\color{fgcolor}\begin{figure}[!h]

{\centering \includegraphics[width=0.980\linewidth,height=0.627\linewidth]{figure/gene_centroids-1} 

}

\caption[Inferred clusters in the mice gene expression dataset.
    Shown are the twelve most occupied clusters.
    In blue, the inferred cluster centroid.
    In grey, gene expressions averaged over replicates and
    shifted by their inferred intercepts]{Inferred clusters in the mice gene expression dataset.
    Shown are the twelve most occupied clusters.
    In blue, the inferred cluster centroid.
    In grey, gene expressions averaged over replicates and
    shifted by their inferred intercepts. }\label{fig:gene_centroids}
\end{figure}


\end{knitrout}
}



\begin{knitrout}
\definecolor{shadecolor}{rgb}{0.969, 0.969, 0.969}\color{fgcolor}\begin{figure}[!h]

{\centering \includegraphics[width=0.588\linewidth,height=0.470\linewidth]{figure/gene_initial_coclustering-1} 

}

\caption[The inferred co-clustering matrix of gene expressions at $\alpha_0 = 6.$ ]{The inferred co-clustering matrix of gene expressions at $\alpha_0 = 6.$ }\label{fig:gene_initial_coclustering}
\end{figure}


\end{knitrout}

Below, we will use the influence function (\coryref{etafun_worst_case}) to try
and find a perturbation that produces large changes in the co-clustering matrix.
To compute the worst-case perturbation, we must choose a univariate summary of
the $\ngenes^2$-dimensional co-clustering matrix whose derivative we wish to
extremize. We use the sum of the eigenvalues of the symmetrically normalized
graph Laplacian, as given by
%
\begin{align*} \laplacianevsum(\eta) = \text{Tr}\left( I - D(\eta)^{-1/2}
\gcoclustering(\eta) D(\eta)^{-1/2} \right), \end{align*}
%
where $D(\eta)^{-1/2}$ is the diagonal matrix with entries $d_i =
\sum_{j=1}^{\ngenes}[\gcoclustering(\eta)]_{ij}$. The quantity $\laplacianevsum$
is differentiable, and has close connection with the number of distinct
components in a graph~\citep{luxburg:2007:spectralcluster}. We expect that prior
perturbations that produce large changes in $\laplacianevsum$ will also produce large changes in the full co-clustering matrix.

%%
\noindent \textbf{Sensitivity to the concentration parameter.}
We first evaluate the sensitivity of the co-clustering matrix $\gcoclustering$
to the choice of $\alpha$ in the stick-breaking prior.

We use the linear approximation to extrapolate the co-clustering matrix
under prior parameters
$\alpha = 0.1$ and $\alpha = 12$.
The a priori expected number of clusters in the original data at these values 
is 2 and 50, respectively. \textcolor{red}{what is it at $\alpha_0$?}
%At $\alpha = alpha_pert1$, only two clusters are expected
%{\em a priori} under the GEM prior;
%at $\alpha = alpha_pert2$, more than fifty are expected.
Despite this wide prior range, the change in the posterior
co-clustering matrix for each $\alpha$ is minuscule (\figref{gene_alpha_coclustering}). The
largest absolute changes in the co-clustering matrix is of order $10^{-2}$.
Refitting the approximate posterior at $\alpha = 0.1$ and
$\alpha = 12$ confirms the insensitivity predicted by the
linearized variational global parameters. Beyond capturing insensitivity, the
linearized parameters were also able to capture the sign and size of the
changes in the individual entries of the co-clustering matrix, even though these
changes are small.


\begin{knitrout}
\definecolor{shadecolor}{rgb}{0.969, 0.969, 0.969}\color{fgcolor}\begin{figure}[!h]

{\centering \includegraphics[width=0.980\linewidth,height=0.784\linewidth]{figure/gene_alpha_coclustering-1} 

}

\caption[Differences in the
     co-clustering matrix at $\alpha = 0.1$ (top row)
     and $\alpha = 12$ (bottom row),
     relative to the co-clustering matrix at $\alpha_0 = 6$.
     (Left) A scatter plot of differences under the linear approximation
     against differences after refitting.
     Each point represents an entry of the co-clustering matrix.
     Note the scales of the axes:
     the largest change in an entry of the co-clustering matrix is
     $\approx 0.03$.
     (Middle) Sign changes in the co-clustering matrix observed after refitting,
     ignoring the magnitude of the change.
     (Right) Sign changes under the linearly approximated variational
     parameters.
     For visualization, changes with absolute value $< 10^{-5}$ are not colored]{Differences in the
     co-clustering matrix at $\alpha = 0.1$ (top row)
     and $\alpha = 12$ (bottom row),
     relative to the co-clustering matrix at $\alpha_0 = 6$.
     (Left) A scatter plot of differences under the linear approximation
     against differences after refitting.
     Each point represents an entry of the co-clustering matrix.
     Note the scales of the axes:
     the largest change in an entry of the co-clustering matrix is
     $\approx 0.03$.
     (Middle) Sign changes in the co-clustering matrix observed after refitting,
     ignoring the magnitude of the change.
     (Right) Sign changes under the linearly approximated variational
     parameters.
     For visualization, changes with absolute value $< 10^{-5}$ are not colored. }\label{fig:gene_alpha_coclustering}
\end{figure}


\end{knitrout}

%%
\noindent \textbf{Sensitivity to functional perturbations.}
We now investigate sensitivity of the co-clustering matrix
to deviations from the beta prior.
In \figref{gene_fpert_coclustering}, we use the influence function for
$\laplacianevsum$ to construct a nonparametric prior perturbation that we
expect to have a large, positive effect.  The resulting prior does indeed
produce changes an order of magnitude larger than those produced by $\alpha$
perturbations shown in \figref{gene_alpha_coclustering}, and our approximation
is again able to capture the qualitative changes.
The influence function is also able to
explain why $\alpha$ perturbations were unable to produce large changes in this
case: \figref{alpha_pert_logphi} shows that changing $\alpha$ (as in
\exref{beta_inf_norm}) induces large changes in the prior only where the
influence function is small.



\begin{knitrout}
\definecolor{shadecolor}{rgb}{0.969, 0.969, 0.969}\color{fgcolor}\begin{figure}[!h]

{\centering \includegraphics[width=0.980\linewidth,height=0.823\linewidth]{figure/gene_fpert_coclustering-1} 

}

\caption[Effect on the co-clustering matrix of a multiplicative functional
     perturbation.
     (Top left) The perturbation $\phi$ is in grey,
     and the influence function is in purple.
     (Top right) The effect of this perturbation on the prior density.
     (Bottom) The effect of this perturbation on
    the co-clustering matrix.
    Note the scale of the scatter plot axes compared with
    the scatter plots in \figref{gene_alpha_coclustering}]{Effect on the co-clustering matrix of a multiplicative functional
     perturbation.
     (Top left) The perturbation $\phi$ is in grey,
     and the influence function is in purple.
     (Top right) The effect of this perturbation on the prior density.
     (Bottom) The effect of this perturbation on
    the co-clustering matrix.
    Note the scale of the scatter plot axes compared with
    the scatter plots in \figref{gene_alpha_coclustering}. }\label{fig:gene_fpert_coclustering}
\end{figure}


\end{knitrout}


\begin{knitrout}
\definecolor{shadecolor}{rgb}{0.969, 0.969, 0.969}\color{fgcolor}\begin{figure}[!h]

{\centering \includegraphics[width=0.882\linewidth,height=0.423\linewidth]{figure/alpha_pert_logphi-1} 

}

\caption[The multiplicative perturbations $\phi_\alpha(\cdot)$ that
    corresponds to decreasing (left) or increasing (right)
    the $\alpha$ parameter]{The multiplicative perturbations $\phi_\alpha(\cdot)$ that
    corresponds to decreasing (left) or increasing (right)
    the $\alpha$ parameter. }\label{fig:alpha_pert_logphi}
\end{figure}


\end{knitrout}

However, even with the (unreasonable-looking) selected functional perturbation,
the size of the differences in the co-clustering matrix remains modest. It is
unlikely that any scientific conclusions derived from the co-clustering matrix would have
changed after the functional perturbation. The co-clustering matrix appears
robust to perturbations in the stick-breaking distribution.


    \subsection{Genetic admixture modeling with fastSTRUCTURE}
    \seclabel{results_structure}
    \input{experiments_structure.tex}

    \subsection{Computation time}
    \seclabel{compute_time}
    %%%%%%%%%%%%%%%%%%%%%%%%%%%%%%%%%%%%%%
%%%%%%%%%%%%%%%%%%%%%%%%%%%%%%%%%%%%%%
% Do not edit the TeX file your work
% will be overwritten.  Edit the RnW
% file instead.
%%%%%%%%%%%%%%%%%%%%%%%%%%%%%%%%%%%%%%
%%%%%%%%%%%%%%%%%%%%%%%%%%%%%%%%%%%%%%



The data sets we considered in our experiments had varying degrees
of complexity, 
and the computational of cost of fitting the variational approximation 
thus also varies accordingly. 
However, in each experiment, the cost of forming the linear approximation 
is rougly an order of magnitude faster than refitting (\tabref{timing_table})

For example, in the thrush data and topic model,
the initial fit took seven seconds, with subsequent refits
(which recall were warm-started with the initial fit) taking
between five and ten seconds.
Inverting the Hessian to form the linear approximation takes less than a second.
In all the examples, after forming the linear approximation, 
extrapolating to a new $\alpha$ or a 
perturbation $\phi$ takes \sfrac{1}{1000}th of a second. 


In each data set and model we consider, the the speed of the linear
approximation allows us to quickly explore a wide range of possible prior perturbations. 
However, even with the speed of our approximation, refitting for all possible perturbations is impossible.
The influence function, which is also cheap to compute, 
can thus provide a guide to uncover influential perturbations, 
at which we can then refit the model. 



% For example, in the mice data and regression model, 
% the initial fit took 30 seconds, with subsequent refits
% (which recall were warm-started with the initial fit) taking 
% between 20 and 30 seconds. 
% Inverting the Hessian to form the linear approximation three seconds. 
% Computing the influence function also takes less than a second. 





\begin{table}[tb]
\centering
\caption{Compute time in seconds of various quantities on each data set. 
Reported times for $\etaopt(\alpha)$ and $\etalin(\alpha)$ are 
median times over the set of considered $\alpha$'s.
The reported influence function time is the time required to 
evaluate the influence function on a grid of 1000 points. }
\tablabel{timing_table}
\begin{tabular}{|r|r|r|r|}
    \hline
    & iris & mice  & thrush \\
    \hline
    %%%%%%%%%%%%%%%%%%%%%%%%%%%%%%%%
    % initial fit
    %%%%%%%%%%%%%%%%%%%%%%%%%%%%%%%%
    Initial fit & 
    1 & 
    30 & 
    7 \\
    \hline
    %%%%%%%%%%%%%%%%%%%%%%%%%%%%%%%%
    % hessian solve for alpha
    %%%%%%%%%%%%%%%%%%%%%%%%%%%%%%%%
    Hessian solve for $\alpha$ sensitivity &
    0.02 & 
    3 & 
    0.3 \\
    %%%%%%%%%%%%%%%%%%%%%%%%%%%%%%%%
    % linear approx time for alpha
    %%%%%%%%%%%%%%%%%%%%%%%%%%%%%%%%
    Linear approx. $\etalin(\alpha)$ &
    0.0008 & 
    0.001 & 
    0.0008 \\
    %%%%%%%%%%%%%%%%%%%%%%%%%%%%%%%%
    % refit time for alpha
    %%%%%%%%%%%%%%%%%%%%%%%%%%%%%%%%
    Refits $\etaopt(\alpha)$ &
    0.5 & 
    30 & 
    5 \\
    \hline
    %%%%%%%%%%%%%%%%%%%%%%%%%%%%%%%%
    % influence function
    %%%%%%%%%%%%%%%%%%%%%%%%%%%%%%%%
    The influence function &
    0.09 &
    3 &
    0.6 \\
    %%%%%%%%%%%%%%%%%%%%%%%%%%%%%%%%
    % hessian solve for functional perturbation
    %%%%%%%%%%%%%%%%%%%%%%%%%%%%%%%%
    Hessian solve for $\phi$ &
    0.02 & 
    3 & 
    0.4\\
    %%%%%%%%%%%%%%%%%%%%%%%%%%%%%%%%
    % linear approx for functional perturbation
    %%%%%%%%%%%%%%%%%%%%%%%%%%%%%%%%
    Linear approx. $\etalin(\phi)$ &
    0.001 & 
    0.001 & 
    0.0008 \\
    %%%%%%%%%%%%%%%%%%%%%%%%%%%%%%%%
    % refit time for functional perturbation
    %%%%%%%%%%%%%%%%%%%%%%%%%%%%%%%%
    Refit $\etalin(\phi)$ &
    0.6 & 
    20 & 
    10 \\
    \hline
\end{tabular}
\end{table}


% BA does not require a conclusion.
% https://projecteuclid.org/journals/bayesian-analysis/author-guidelines
% \section{Conclusion}
% \seclabel{conclusion}
% TODO: write this when the rest of the paper is done.


%%%%%%%%%%%%%%%%%%%%%%%%%%%%%%%%%%%%%%%%%%%%%%
%% Supplementary Material, if any, should   %%
%% be provided in {supplement} environment  %%
%% with title and short description.        %%
%%%%%%%%%%%%%%%%%%%%%%%%%%%%%%%%%%%%%%%%%%%%%%
%\begin{supplement}
%\stitle{???}
%\sdescription{???.}
%\end{supplement}

% ** The bibliograhy **
\bibliographystyle{ba}
\bibliography{references}% place <bib-data-file> in ./bib folder

% ** Acknowledgements **
\begin{acks}[Acknowledgments]
  TODO
\end{acks}


%%%%%%%%%%%%%%%%%%%%%%%%%%%%%%%%%%%%%%%%%%%%%%
%% from my understanding,
%% BA requests that supplementary material should be in a new file.
%% For now, I left it here so all our text is in one place,
%% and will move it later
%%%%%%%%%%%%%%%%%%%%%%%%%%%%%%%%%%%%%%%%%%%%%%

\appendix

% \section{Illustrative examples and review}
% \seclabel{examples}
% We here provide a number of illustrative examples, and review concepts from the literature, to supplement our main text.

%%
\subsection{A Gaussian mixture model with conjugate component distributions}
\applabel{gmm_example}

In this example, we provide the details for a Gaussian mixture model (GMM)
with conjugate component distributions. 

\begin{ex}[Gaussian mixture model]\exlabel{iris_bnp_process}
%
The observations are vectors $\x_\n \in \mathbb{R}^\d$, and we model each
component with a multivariate Gaussian. In this model, $\beta_\k = (\mu_k,
\Lambda_\k)$, where $\mu_\k \in \mathbb{R}^\d$, $\Lambda_\k$ is a $\d\times\d$
positive definite information matrix, and
%
\begin{align*}
%
\p(\x_\n \vert \beta_\k) ={}& \normdist{\x_n \vert \mu_\k, \Lambda_\k^{-1}} \\
\log\p(\x_\n \vert \beta_\k) ={}&
    -\frac{1}{2}(\x_n - \mu_k)^T \Lambda_\k (\x_n - \mu_k)
    + \frac{1}{2} \log |\Lambda_\k| + \const.\\
    & \constdesc{\beta_\k}
%
\end{align*}
We let $\pbetaprior(\beta_\k)$ be the conjugate prior, which in this case is normal-Wishart:
\begin{align*}
  \pbetaprior(\beta_\k) &= \normalwishart{\beta_\k \vert \tau_0, n_0, p_0, V_0}\\
  \log\pbetaprior(\beta_\k) &=
      -\frac{\tau_0}{2}(\mu_\k - \mu_0)^T \Lambda_\k (\mu_\k - \mu_0)\\
      &{} + \frac{n_0 - p_0 - 1}{2} \log |\Lambda_\k| -
      \frac{1}{2} \textrm{Tr}(V_0 \Lambda_\k) + \const,
\end{align*}
where $(\tau_0, n_0, p_0, V_0)$ are fixed prior parameters.
%
For a choice of $\pstick$, and for $\pbetaprior(\beta_\k)$ and $\p(\x_\n \vert
\beta_\k)$ as given above, the posterior $\p(\beta, \z, \nu \vert \x)$ can in
principle be computed by applying Bayes' rule.

In \secref{results}, we fit a GMM to Fisher's iris data set and cluster irises
into latent species based on morphological measurements.  In that case, $\k$
indexes distinct species, $\beta_{\k}$ characterizes the distribution of
morphological measurements for species $\k$, and $\z_{\n\k} = 1$ when
observation $\n$ is a member of species $\k$.
%
\end{ex}

%%
\subsection{A closed form for local variational parameters as a function of global variational parameters}
\applabel{gmm_global_local_vb}

We next illustrate for a Gaussian mixture model (GMM) how we 
can express the local variational parameters as a closed-form function
of the global variational parameters.

\begin{ex}[Optimality of $\etalocal$ in a GMM]\exlabel{qz_optimality}
Recall that under our truncated variational approximation,
the cluster assignment $\z_\n$ is a discrete random variable
over $\kmax$ categories.

Let $\eta_{\z_\n}$ be the categorical parameters in its exponential family
natural parameterization. That is, we let $\eta_{\z_\n} = (\rho_{\n1},
\rho_{\n2}, ..., \rho_{\n(\kmax-1)})$ be an unconstrained vector in
$\mathbb{R}^{\kmax-1}$; in this parameterization, the assignment probabilities
are
%
\begin{align*}
  p_{\n\k} := \expect{\q(\z_\n \vert \etaz)}{\z_{\n\k}} =
  \frac{\exp(\rho_{\n\k})}{1 + \sum_{\k'=1}^{\kmax-1}\exp(\rho_{\n\k})}
\end{align*}
%
We use the exponential family parameterization because we require the optimal
variational parameters $\etaopt$ to be interior to $\etadom$ in
\assuitemref{kl_opt_ok}{kl_opt_interior}.

Fixing $\q(\beta\vert\etabeta)$ and $\q(\nu\vert\etanu)$,
the optimal $\etaopt_{\z_\n}$ must satisfy
%
\begin{align*}
& \q(\z_\n | \etaopt_{\z_\n}) \propto \exp\left(\tilde \rho_{\n\k}\right)\\
& \mathwhere \tilde \rho_{\n\k} := \expect{\q(\beta, \nu \vert \eta)}
       {\log\p(\x_n \vert \beta_\k) + \log \pi_\k}.
\end{align*}
%
See \citet{bishop:2006:PRML} and \citet{blei:2017:vi_review} for details.
To satisfy this optimality condition,
we set the optimal $\etaopt_{\z_\n}$ to be
%
\begin{align*}
%
\etaopt_{\z_\n} = \left(\log\frac{\tilde\rho_{\n1}}{\tilde\rho_{\n\kmax}},
\log\frac{\tilde\rho_{\n2}}{\tilde\rho_{\n\kmax}}, \ldots,
\log\frac{\tilde\rho_{\n(\kmax-1)}}{\tilde\rho_{\n\kmax}}\right).
%
\end{align*}
%
Thus, as long as the expectations $\tilde\rho_{\n\k}$ can be provided
as a closed-form function of
$(\etabeta, \etanu)$, the optimal $\etaopt_{\z_\n}$ can be also be set in closed-form as
a function of $(\etabeta, \etanu)$.
%
\end{ex}

%%
\subsection{More details on computing and inverting the Hessian}
\applabel{more_hessian}

We fill in more details for the efficient computation of the Hessian outlined in
\secref{computing_sensitivity}.

We start from our formula in \eqref{global_local_derivative_breakdown}.
%
\begin{align*}
%
\fracat{d \etaopt(\t)}{d \t}{t = 0} ={}&
-\left(
\begin{array}{cc}
   \hess{\gamma\gamma} & \hess{\gamma\ell} \\
   \hess{\ell\gamma}     & \hess{\ell\ell} \\
\end{array}
\right)^{-1}
\left( \begin{array}{c} \crosshessian_\gamma \\ 0 \end{array}\right),
%
\end{align*}
%
and an application of the Schur complement gives
%
\begin{align*}
%
\fracat{d \etaopt(\t)}{d \t}{t = 0} ={}&
-\left(\begin{array}{c}
I_{\gamma\gamma} \\
\hess{\ell\ell}^{-1} \hess{\ell\gamma}
\end{array}\right)
\left(\hess{\gamma\gamma} -
      \hess{\gamma\ell} \hess{\ell\ell}^{-1} \hess{\ell\gamma}\right)^{-1} \crosshessian_\gamma,
%
\end{align*}
%
where $I_{\gamma\gamma}$ is the identity matrix with
the same dimension as $\eta_\gamma$.
%
Specifically, observe that the sensitivity of the global parameters
is given by
%
\begin{align}\eqlabel{global_sens}
  \fracat{d \etaopt_\gamma(\t)}{d \t}{t = 0} &=
  - \hessopt_\gamma^{-1}\crosshessian_\gamma
  \mathwhere
  \hessopt_\gamma := \left(\hess{\gamma\gamma} -
        \hess{\gamma\ell} \hess{\ell\ell}^{-1} \hess{\ell\gamma}\right),
\end{align}
%
In our model, $\hess{\ell\ell}$ is sparse, and the size of $\hess{\gamma\gamma}$
does not grow with $\N$. Thus, each term of $\hessopt_\gamma$ can be tractably
computed, stored in  memory, and inverted, even on very large datasets.

One can derive the exact same identity using the optimality of
$\etaoptlocal(\eta_\gamma)$.  By applying the chain rule, one can
verify that
%
\begin{align}\eqlabel{global_kl}
\hessopt_{\gamma} &=
    \frac{\partial^2}{\partial\eta_\gamma\partial\eta_\gamma^T}
    \KLglobal(\etaopt_\gamma, 0).
\end{align}
%
In practice, we evaluate $\hessopt_\gamma$ using automatic differentiation and
\eqref{global_kl} rather than the Schur complement.

%%
\subsection{An example using the global/local decomposition}
\applabel{vb_insample_nclusters_example}

In this example, we take advantage of the global/local structure of the BNP problem
when our quantity of interest is the in-sample expected posterior number of clusters. 

\begin{ex}\exlabel{vb_insample_nclusters_globallocal}
%
Let
$\gclustersabbr(\eta)$ denote our variational approximation to
$\expect{\p(\z\vert\x)}{\nclusters(\z)}$.   Using the fact that $\p(\z_\n
\vert \beta, \nu, \x)$ is available in closed form, we can then take
%
\begin{align*}
%
\gclustersabbr(\etaopt) :={}&
    \expect{\q(\beta, \nu \vert\etaopt)}{
        \expect{\p(\z \vert \beta, \nu, \x)}{\nclusters(\z)}
    }
\\\approx{}&
    \expect{\p(\beta, \nu \vert \x)}{
        \expect{\p(\z \vert \beta, \nu, \x)}{\nclusters(\z)}
    }
    = \expect{\p(\z\vert\x)}{\nclusters(\z)} \Rightarrow \\
%
\gclustersabbr(\eta) ={}&
    \sumkm \left(1 -  \prod_{\n=1}^\N
        \left(1 - \expect{\q(\beta, \nu \vert \eta_\beta, \eta_\nu)}
                    {\expect{\p(\z_{\n} \vert \beta, \nu, \x)}{\z_{\n\k}}}
                    \right)\right).
%
\end{align*}
%
In this way, $\gclustersabbr(\eta)$ depends only on $\eta_\beta$ and $\eta_\nu$,
which are much lower-dimensional than $\eta_\z$, and retains nonlinearities in
the map
%
\begin{align*}
%
\eta_\beta, \eta_\nu \mapsto \expect{\q(\beta, \nu \vert \eta_\beta,
\eta_\nu)} {\expect{\p(\z_{\n} \vert \beta, \nu, \x)}{\z_{\n\k}}}.
%
\end{align*}
%
\end{ex}

%%
\subsection{An example using the global/local decomposition}
\applabel{dp_cluster_growth}

To help us understand the effect of the concentration parameter
$\alpha$ we often use the following fact. Under the $\gem$ prior, the {\em a priori} expected
number of distinct clusters in a dataset of size $N$ is given by
%
\begin{align}\eqlabel{prior_num_clusters}
\expect{\p(\z \vert \pi)\p(\pi \vert \alpha)}{\nclusters(\z)} =
\sum_{\n = 1}^\N \frac{\alpha}{\alpha + \n - 1}.
\end{align}
%
See \citep[Equation 75]{jordan:2015:gentleintrodp}. \textcolor{red}{There's a much, much earlier citation for this. E.g.\ the original Blackwell MacQueen paper?}



\section{General differentiability results}
\applabel{diffable_intro}
In this section we state general conditions under which VB optima, as defined by
\eqref{vb_optimization}, are differentiable functions of both parametric and
nonparametric prior perturbations.  The desired results for the BNP model will
follow as special cases of these general results.

We first define our general setup in \defref{prior_t}, and then connect
the general definition to the our BNP problem.

%%%%%%%%%%%%%%%%%%%%%%%%%%%%%%%%%%%%%%%%%%%%%%%%%%%%%%%%%%%%%%%%%%%%%%%%%%%
%%%%%%%%%%%%%%%%%%%%%%%%%%%%%%%%%%%%%%%%%%%%%%%%%%%%%%%%%%%%%%%%%%%%%%%%%%%%
\begin{defn}\deflabel{prior_t}
%
For some parameter $\theta \in \thetadom \subseteq \mathbb{R}^{\thetadim}$, let
$\p(\theta \vert \t)$ denote a class of probability densities relative to
a sigma-finite measure $\mu$, defined for $\t$ in an open set $\ball_\t
\subseteq \mathbb{R}$ containing $0$.  Let $\q(\theta \vert \eta)$ be a
family of approximating densities, also defined relative to $\mu$.

Let the variational objective factorize as
%
\begin{align}
%
\KL{\eta, \t} :={}&
    \KL{\eta} +
    \expect{\q(\theta \vert \eta)}
       {\log \p(\theta \vert \t) - \log \p(\theta \vert \t=0)}           \eqlabel{perturbed_objective}\\
\etaopt(\t) :={}& \argmin_{\eta \in \etadom} \KL{\eta, \t}.
    \eqlabel{perturbed_optimum}
%
\end{align}
%
Let $\etaopt$ with no argument refer to $\etaopt(0)$, the minimizer
of $\KL{\eta}$.
%
\end{defn}
%%%%%%%%%%%%%%%%%%%%%%%%%%%%%%%%%%%%%%%%%%%%%%%%%%%%%%%%%%%%%%%%%%%%%%%%%%%%

By identfiying $\t$ with some hyperparameter (e.g. the concentration parameter,
as in \exref{alpha_perturbation} below), we can use \defref{prior_t} to study
parametric perturbations.  Furthermore, by parameterizing a path through the
space of general densities, \defref{prior_t} will allow us to study
nonparametric perturbations (e.g. \exref{gem_mult_perturbation} below and the
detailed analysis of \secref{diffable_nonparametric, diffable_lp}).  We
can thus study VB prior robustness in general by studying problems of the
form \defref{prior_t}.
%
% As $\etaopt(\t)$ varies, so does our quantity of interest, $\g(\etaopt(\t))$. If
% $\g(\etaopt(\t))$ varies meaningfully as $\t$ ranges over its plausible values,
% we say that a the quantity of interest $\g$ is {\em not robust} to changes of
% $\t$.  In any particular problem, some posterior quantities of interest may be
% robust while others are not.
% %
% Note that non-robustness is to some extent subjective, in that it depends on a
% decision about what a ``reasonable'' range of $\alpha$ might be, as well as how
% much variation in $\g(\etaopt(\alpha))$ is ``acceptable.''  In the present paper
% we will primarily focus on the task of approximating the computation of
% $\g(\etaopt(\alpha))$, leaving the context-specific decision of what priors are
% reasonable up to the the reader.  However, when considering non-parametric
% perturbations to the prior, we will explicitly consider whether the
% nonparametric neighborhoods correspond to intuitively reasonable sets of prior
% perturbations.
%
% In a world with no computational costs, we could simply evaluate
% $\g(\etaopt(\t))$ for a large number of candidate $\t$ and directly
% check for robustness.  Unfortunately, the evaluation of $\etaopt(\t)$
% requires solving an optimization problem, which is often computationally
% expensive for even a single $\t$, much less many different $\t$.  For
% nonparametric perturbations, the problem is even greater, since each $\t$
% explores only a single alternative $\palt(\theta)$.
%
% We propose avoiding the need to repeatedly re-optimize by forming a {\em linear
% approximation} to the map $\alpha \mapsto \etaopt(\alpha)$ as follows. Supposing
% for the moment that the map $\alpha \mapsto \etaopt(\alpha)$ is continuously
% differentiable (we will establish precise conditions below), and that we have
% found $\etaopt(\alpha_0)$ for some ``base value,'' $\alpha_0$.  If we can
% compute the derivative $d \etaopt(\alpha) / d\alpha$ (which is a
% $\etadim$-length vector), we can form the first-order Taylor series
% approximation
% %
% \begin{align*}
% %
% \etaopt(\alpha) \approx \etalin(\alpha) :={}
%     \etaopt(\alpha_0) +
%     \fracat{d \etaopt(\alpha)}{d\alpha}{\alpha_0} (\alpha - \alpha_0).
% %
% \end{align*}
% %
% Given the derivative $d \etaopt(\alpha) / d\alpha$, the cost of evaluating
% $\etalin(\alpha)$ for any $\alpha$ is only that of vector multiplication, not
% that of solving a new optimization problem.
%
% We can then use the approximation $\etalin(\alpha)$ to approximate our quantity
% of interest, $\g(\etaopt(\alpha))$.  For this, we can either use the
% approximation
% %
% \begin{align*}
% %
% \g(\etaopt(\alpha)) \approx{} \g(\etalin(\alpha))
% %
% \end{align*}
% %
% or, when $\eta \mapsto \g(\eta)$ is continuously differentiable, using the chain
% rule to form
% %
% \begin{align*}
% %
% %\g(\etaopt(\alpha)) \approx{}& \g(\etaopt(\alpha)) \quad \textrm{or}\\
% \g(\etaopt(\alpha)) \approx{}&
%     \glin(\alpha) :=
%         \g(\etaopt(\alpha_0)) +
%             \fracat{d \g(\eta)}{ d\eta^T}{\etaopt(\alpha_0)}
%             \fracat{d \etaopt(\alpha)}{d\alpha}{\alpha_0} (\alpha - \alpha_0).
% %
% \end{align*}
% %
% We will discuss the relative advantages of $\g(\etalin(\alpha))$ and
% $\glin(\alpha)$ below.  In short, the advantage of $\g(\etalin(\alpha))$ is that
% the approximation retains nonlinearities in the map from $\eta \mapsto
% \g(\eta)$, but the full linear approximation $\glin(\alpha)$ allows the
% computation of helpful quantities such worst-case nonparametric prior
% perturbations.
%
% In order for the approximation $\etalin(\alpha)$ to be useful, three necessary
% conditions must be met.  First, the map $\alpha \mapsto \etaopt(\alpha)$ must be
% continuously differentiable; if it is not differentiable, then the derivative
% does not even exist, and if it is not continuously differentiable, then small
% $\abs{\alpha - \alpha_0}$ may not guarantee a good approximation to
% $\etaopt(\alpha)$. Second, the derivative $d \etaopt(\alpha) / d\alpha$ must be
% relatively easy to compute---in particular, it must be easier to compute than
% solving a new optimization problem.  Finally, the derivative must {\em
% extrapolate} meaningfully over a meaningful range of alternative values of
% $\alpha$, i.e., the map $\alpha \mapsto \etaopt(\alpha)$ must be shown to be
% smooth enough in practical problems to be useful.
%
% The remainder of the paper addresses these desiderata in turn, both for
% parametric perturbations to the concentration parameter as well as a particular
% family of nonparametric perturbations.  First, we provide theoretical conditions
% under which we have continuously differentiability of the VB optimum in
% \secref{local_sensitivity}, and show that they are satisfied for our BNP model.
% In \secref{computing_sensitivity}, we use the theoretical results of
% \secref{local_sensitivity} to show that the derivative can be computed
% efficiently in the BNP problems we are considering, with an emphasis on how
% modern automatic differentiation tools render many of the computations automatic
% \citep{jax2018github}.  Finally, in \secref{results}, we demonstrate the
% practical usefulness of the approximation on a set of three real-world BNP
% problems of increasing complexity.

%%%%%%%%%%%%%%%%%%%%%%%%%%%%%%%%%%%%%%%%%%%%%%%%%%%%%%%%%%%%%%%%%%%%%%%%%%%%%%%%
%%%%%%%%%%%%%%%%%%%%%%%%%%%%%%%%%%%%%%%%%%%%%%%%%%%%%%%%%%%%%%%%%%%%%%%%%%%%%%%%
\begin{ex}\exlabel{alpha_perturbation}
%
For the BNP model with the GEM prior, take $\theta = (\nu_1, \ldots,
\nu_{\kmax-1})$, $\mu$ to be the Lebesgue measure on $[0,1]^{\kmax-1}$.
Let $\alpha_0$ be some base value of the concentration parameter, and
let $\t$ be $\alpha - \alpha_0$, so that deviations of $\t$ away from
$0$ represent deviations of $\alpha$ away from $\alpha_0$.

Expanding the KL divergence in \eqref{kl_def}, we see that the prior
$\p(\nuk \vert \alpha)$ enters the VB objective in a term of the form
$\sum_{\k=1}^\infty \expect{\q(\nuk \vert \eta)}{\log \p(\nuk \vert \alpha)}$.
Adding and subtracting the this term evaluated at $\alpha_0$ gives
% %
% \begin{align*}
% %
% \MoveEqLeft
% \KL{\q(\zeta \vert \eta) || \p(\zeta \vert \x, \alpha)}
% \\={}&    \expect{\q(\zeta \vert \eta)}{
%         \log \q(\zeta \vert \eta) - \logp(\x, \zeta)} + \logp(\x)
% \\={}&    \expect{\q(\zeta \vert \eta)}{\log \q(\zeta \vert \eta)} -
%         \expect{\q(\zeta \vert \eta)}
%                {\log \p(\x, \beta, \z \vert \nu )}
%                +\logp(\x)
% \\{}& -\sum_{\k=1}^{\kmax - 1}
%             \left(
%                 \expect{\q(\nuk \vert \eta)}{\log \p(\nuk \vert \alpha)} -
%                 \expect{\q(\nuk \vert \eta)}{\log \p(\nuk \vert \alpha_0)} +
%                 \expect{\q(\nuk \vert \eta)}{\log \p(\nuk \vert \alpha_0)}
%              \right)
% \\{}&
% - \sum_{\k = \kmax}^\infty
%     \expect{\q(\zeta \vert \eta)}{\log \p(\nuk \vert \alpha)}.
% %
% \end{align*}
% %
% Recall that the truncated VB approximation ignores the terms
% $\expect{\q(\zeta \vert \eta)}{\log \p(\nuk \vert \alpha)}$ for $\k \ge \kmax$,
% so that we can write the truncated objective as
%
\begin{align*}
%
\KL{\eta, \alpha} = \KL{\eta, \alpha_0}
-\sum_{\k=1}^{\kmax - 1}
            \left(
                \expect{\q(\nuk \vert \eta)}{\log \p(\nuk \vert \alpha)} -
                \expect{\q(\nuk \vert \eta)}{\log \p(\nuk \vert \alpha_0)}
             \right).
%
\end{align*}
%
Plugging in the definition of $\p(\nuk \vert \alpha)$, recognizing that the
normalizing constant does not depend on $\nuk$ and so can be neglected in the
optimization, letting $\KL{\eta} := \KL{\eta, \alpha_0}$, and substituting $\t =
\alpha - \alpha_0$ gives
%
\begin{align*}
%
\KL{\eta, \t} = \KL{\eta, \alpha_0}
-\t \sum_{\k=1}^{\kmax - 1}
    \expect{\q(\nuk \vert \eta)}{\log (1 - \nuk)}.
%
\end{align*}
%
% By the definition of $\p(\nuk \vert \alpha)$,
% %
% \begin{align*}
% %
% \log \p(\nu_1, ..., \nu_\kmax \vert \alpha)
%     ={}& (\alpha - 1) \sum_{k=1}^\kmax \log (1 - \nuk)
%     + \kmax \log \frac{\Gamma(1 + \alpha)}{\Gamma(\alpha)}.
% %
% \end{align*}
% %
% The normalizing constant does not depend on $\nuk$, and so its expectation
% does not depend on $\eta$, and so it can be neglected in the VB objective,
% and we can write
% %
% \begin{align*}
% %
% \KL{\eta, \alpha} = \KL{\eta, \alpha_0}
% -(\alpha - \alpha_0) \sum_{\k=1}^{\kmax - 1}
%     \expect{\q(\nuk \vert \eta)}{\log (1 - \nuk)}.
% %
% \end{align*}
% %
% Identifying $\KL{\eta}$ of \defref{prior_t} with $\KL{\eta, \alpha_0}$ evaluated
% at the base concentration parameter shows that the BNP problem with the GEM
% prior is of the form \eqref{perturbed_objective}, and that
% %
% \begin{align*}
% %
% \psi(\nu, \t) = -\t \sum_{\k=1}^{\kmax - 1}
%     \expect{\q(\nuk \vert \eta)}{\log (1 - \nuk)}.
% %
% \end{align*}
%
\end{ex}
%%%%%%%%%%%%%%%%%%%%%%%%%%%%%%%%%%%%%%%%%%%%%%%%%%%%%%%%%%%%%%%%%%%%%%%%%%%%%%%%



%%%%%%%%%%%%%%%%%%%%%%%%%%%%%%%%%%%%%%%%%%%%%%%%%%%%%%%%%%%%%%%%%%%%%%%%%%%%%%%%
%%%%%%%%%%%%%%%%%%%%%%%%%%%%%%%%%%%%%%%%%%%%%%%%%%%%%%%%%%%%%%%%%%%%%%%%%%%%%%%%
\begin{ex}\exlabel{gem_mult_perturbation}
%
As in \exref{alpha_perturbation}, take $\theta = (\nu_1, \ldots, \nu_{\kmax-1})$
and $\mu$ to be the Lebesgue measure on $[0,1]^{\kmax-1}$. Let $\pbase(\nuk) :=
\betadist{\nuk \vert 1, \alpha_0}$, and let $\palt(\nuk)$ be a distribution, not
in the Beta family, that shifts mass towards zero:
%
\begin{align*}
%
\palt(\nuk) :=
    \frac{\exp(-\nuk)\pbase(\nuk)}{\int \exp(-\nuk')\pbase(\nuk') d\nuk'}.
%
\end{align*}
%
For $\t \in [0,1]$ define the multiplicatively perturbed prior
%
\begin{align*}
%
\p(\nuk \vert \t) :=
    \frac{\palt(\nuk)^{\t} \pbase(\nuk)^{1-\t}}
         {\int \palt(\nuk')^{\t} \pbase(\nuk')^{1-\t} d\nuk'}.
%
\end{align*}
%
When $\t = 0$, $\p(\nuk \vert \t) = \pbase(\nuk)$, when $\t = 1$,
$\p(\nuk \vert \t)  = \palt(\nuk)$, and $\p(\nuk \vert \t)$ varies smoothly
between the two for intermediate values of $\t$.

As in \exref{alpha_perturbation}, up to constants not depending on
$\lnuk$ we can write
%
\begin{align*}
%
\log \p(\nuk \vert \t) - \log \p(\nuk \vert \t=0) ={}&
    -\t \log \pbase(\nuk) + \t \log \palt(\nuk) + \const
\\={}& -\t \nuk + \const \Rightarrow
\\
\KL{\eta, \t} ={}& \KL{\eta} -\t \nuk + \const.
%
\end{align*}
%
Different choices for $\palt(\nuk)$ would give different additive
perturbations to the KL divergence.
%
\end{ex}
%
%%%%%%%%%%%%%%%%%%%%%%%%%%%%%%%%%%%%%%%%%%%%%%%%%%%%%%%%%%%%%%%%%%%%%%%%%%%%%%%%


    \subsection{Parametric prior perturbations}
    \applabel{diffable_parametric}
    We now state conditions under which $\t \mapsto \etaopt(\t)$, as defined by
\defref{prior_t}, is continuously differentiable.  Our key theoretical tool will
be the implicit function theorem (e.g., \citet{krantz:2012:implicit}), applied
to the first-order conditions for the VB optimization problem, and the dominated
converence theorem (e.g., \citet[Theorem 16.8]{billingsley:1986:probability}),
which will allow us to express derivatives of variational expectations in terms
of properties  of other variational expectations.

The derivative can expressed in terms of unnormalized densities, which can
simplify some computation.  To that end, let $\qtil$ and $\ptil$ refer to
potentially unnormalized (but normalizable) versions of the respectively
corresponding $\q$ and $\p$ given in \defref{prior_t}, so that
%
\begin{align*}
%
\q(\theta \vert \eta) :={}
    \frac{\qtil(\theta \vert \eta)}
    {\int \qtil(\theta' \vert \eta) \mu(d\theta')} \mathand
\p(\theta \vert \t) :={}
    \frac{\ptil(\theta \vert \t)}
    {\int \ptil(\theta' \vert \t) \mu(d\theta')}.
%
\end{align*}

For \assuref{kl_opt_ok}, we state some regularity conditions on the ``base
problem'', $\KL{\eta}$.

%%%%%%%%%%%%%%%%%%%%%%%%%%%%%%%%%%%%%%%%%%%%%%%%%%%%%%%%%%%%%%%%%%%%%%%%%%%%
%%%%%%%%%%%%%%%%%%%%%%%%%%%%%%%%%%%%%%%%%%%%%%%%%%%%%%%%%%%%%%%%%%%%%%%%%%%%
\begin{assu}\assulabel{kl_opt_ok}
%
Let the following conditions on the variational approximation hold.
%
%%%%%%%%%%%%%%%%%%%%%%%%%%%%%%%%%%%%%%%%%%%%%%%%%%%%%%%%%%%%%%%%%%%%%%
\begin{enumerate}
%
    \item \itemlabel{kl_diffable} The map $\eta \mapsto \KL{\eta}$ is twice
    continuously differentiable at $\etaopt$.

    \item\itemlabel{kl_hess} The Hessian matrix $\fracat{\partial^2 \KL{\eta}}
    {\partial \eta \partial \eta^T} {\etaopt}$ is non-singular.

    \item \itemlabel{kl_opt_interior} There exists an open ball $\ball_\eta
    \subset \mathbb{R}^\etadim$ such that $\etaopt \in \ball_\eta \subset
    \etadom$.
%
\end{enumerate}
%
\end{assu}
%%%%%%%%%%%%%%%%%%%%%%%%%%%%%%%%%%%%%%%%%%%%%%%%%%%%%%%%%%%%%%%%%%%%%%%%%%%%

Next, we assume that we can exchange the order of integration and
differentiation in variational expectations.

%%%%%%%%%%%%%%%%%%%%%%%%%%%%%%%%%%%%%%%%%%%%%%%%%%%%%%%%%%%%%%%%%%%%%%%%%%%%
%%%%%%%%%%%%%%%%%%%%%%%%%%%%%%%%%%%%%%%%%%%%%%%%%%%%%%%%%%%%%%%%%%%%%%%%%%%%
%
\begin{assu}\assulabel{exchange_order}
%
Assume that the map $\eta \mapsto \qtil(\theta \vert \eta)$ is twice
continuously differentiable, and that the map $\t \mapsto \ptil(\theta \vert
\t)$ is continuously differentiable.

% Which version is better?
% Assume that we can exchange the order of $\q$-integration and differentiation in
% the expression $\expect{\q(\theta \vert \eta)}{\log \ptil(\theta \vert \t)}$ at
% $\eta = \etaopt$ and $\t = 0$ for the derivatives $\partial / \partial \eta$,
% $\partial^2 / \partial \eta^2$, and $\partial^2 / \partial \eta \partial \t$.

Furhter, assume that we can exchange the order of integration and
differentiation in the expressions $\int \qtil(\theta \vert \eta) \log
\ptil(\theta \vert \t) \mu(d\theta)$ and $\int \qtil(\theta \vert \eta)
\mu(d\theta)$ at $\eta = \etaopt$ and $\t = 0$ for the derivatives $\partial /
\partial \eta$, $\partial^2 / \partial \eta^2$, and $\partial^2 / \partial \eta
\partial \t$.
%
\end{assu}
%%%%%%%%%%%%%%%%%%%%%%%%%%%%%%%%%%%%%%%%%%%%%%%%%%%%%%%%%%%%%%%%%%%%%%%%%%%%

In certain cases, one can verify \assuref{exchange_order} directly, such as when
$\expect{\q(\theta \vert \eta)}{\log \ptil(\theta \vert \t)}$ has a closed form.
For more general situations, the following straightforward extention of the
dominated convergence theorem \citep[Theorem 16.8]{billingsley:1986:probability}
is useful.

%%%%%%%%%%%%%%%%%%%%%%%%%%%%%%%%%%%%%%%%%%%%%%%%%%%%%%%%%%%%%%%%%%%%%%%%%%%%
%%%%%%%%%%%%%%%%%%%%%%%%%%%%%%%%%%%%%%%%%%%%%%%%%%%%%%%%%%%%%%%%%%%%%%%%%%%%
%
\todo{Make this into a separate assumption and lemma.}
\begin{lem}\lemlabel{exchange_order}
%
Let $M(\theta) > 0$ be a measurable function with $\int M(\theta) \mu(d\theta) <
\infty$.  Let $f(\theta, \eta, \t)$ denote a function that is continuously
differentiable in $\t$ and measurable in $\theta$ for $\eta, \t \in \ball_\eta
\times \ball_\t$, where $\ball_\eta \times \ball_\t$ is open.

\todo{Include differentiability assumptions here}

Assume that, for all $\eta, \t \in \ball_\eta \times \ball_\t$, $M(\theta)$ is
$\mu$-almost everywhere greater than each of the following functions: $f(\theta,
\eta, \t)$, $\norm{\partial f(\theta, \eta, \t) / \partial \eta}_2$,
$\norm{\partial^2 f(\theta, \eta, \t) / \partial \eta \partial \eta^T}_2$, and
$\norm{\partial^2 f(\theta, \eta, \t) / \partial \eta \partial \t}_2$. Then, at
any $\eta, \t \in \ball_\eta \times \ball_\t$, we can exchange the order of
integration and differentiation in $\int f(\theta, \eta, \t) \mu(d\theta)$ for
the derivatives $\partial / \partial \eta$, $\partial^2 / \partial \eta^2$, and
$\partial^2 / \partial \eta \partial \t$.

\seeproof{exchange_order}
%
\end{lem}
%%%%%%%%%%%%%%%%%%%%%%%%%%%%%%%%%%%%%%%%%%%%%%%%%%%%%%%%%%%%%%%%%%%%%%%%%%%%

%%%%%%%%%%%%%%%%%%%%%%%%%%%%%%%%%%%%%%%%%%%%%%%%%%%%%%%%%%%%%%%%%%%%%%%%%%%%
%%%%%%%%%%%%%%%%%%%%%%%%%%%%%%%%%%%%%%%%%%%%%%%%%%%%%%%%%%%%%%%%%%%%%%%%%%%%
%
\begin{assu}\assulabel{exchange_order_dom}
(Sufficient conditions for \assuref{exchange_order}.)
%
Assume that \lemref{exchange_order} applies with the function $f(\theta, \eta,
\t) = \qtil(\theta \vert \eta) \log \ptil(\theta \vert \t)$ as well as with
$f(\theta, \eta, \t) = \qtil(\theta \vert \eta)$.
%
\end{assu}
%%%%%%%%%%%%%%%%%%%%%%%%%%%%%%%%%%%%%%%%%%%%%%%%%%%%%%%%%%%%%%%%%%%%%%%%%%%%

The advantage of \assuref{exchange_order_dom} over \assuref{exchange_order} is
that the conditions of \assuref{exchange_order_dom} can typically be verified
even when the expectation $\expect{\q(\theta \vert \eta)}{\log \ptil(\theta
\vert \t)}$ does not have a closed form.  In \secref{diffable_concentration}, we
will dicuss how different choices of variational approximations for the stick
lengths lend themselves to either \assuref{exchange_order} of
\assuref{exchange_order_dom}.

We are now in a position to define the quantities that occur in the derivative.

%%%%%%%%%%%%%%%%%%%%%%%%%%%%%%%%%%%%%%%%%%%%%%%%%%%%%%%%%%%%%%%%%%%%%%%%%%%%
%%%%%%%%%%%%%%%%%%%%%%%%%%%%%%%%%%%%%%%%%%%%%%%%%%%%%%%%%%%%%%%%%%%%%%%%%%%%
\begin{defn}\deflabel{deriv_quantities}
%
Under the conditions of \defref{prior_t}, when \assuref{kl_opt_ok,
exchange_order} hold, define
%
\begin{align*}
%
\hessopt :={}& \fracat{\partial^2 \KL{\eta}}
                      {\partial \eta \partial \eta^T}
                      {\etaopt} \mathand \\
%
\lqgradbar{\theta \vert \etaopt} :={}&
    \lqgrad{\theta \vert \etaopt} -
    \expect{\q(\theta \vert \etaopt)}{\lqgrad{\theta \vert \etaopt}}.
%
\end{align*}

% Note that if $\qtil(\theta \vert \eta)$ is already normalized ($\qtil = \q$),
% then $\expect{\q(\theta \vert \eta)}{\lqgrad{\theta \vert \eta}} = 0$ for all
% $\eta$ and $\lqgradbar{\theta \vert \etaopt} = \lqgrad{\theta \vert \etaopt}$.

Further, define

\begin{align*}
%
\crosshessian :={}&
    \fracat{\partial
            \expect{\q(\theta \vert \etaopt)}
                   {\fracat{\partial \log \ptil(\theta \vert \t)}
                           {\partial \t}{\t=0} }
            }
        {\partial \eta}{\etaopt}
={}
    \expect{\q(\theta \vert \etaopt)}{
          \lqgradbar{\theta \vert \etaopt}
          \fracat{\partial \log \ptil(\theta \vert \t)}
                 {\partial \t}{\t=0}},
%
\end{align*}
%
where the final equality follows from differentiating under the integral using
\assuref{exchange_order} (see \lemref{logq_continuous} in \appref{proofs} for
more details).
%
\end{defn}
%%%%%%%%%%%%%%%%%%%%%%%%%%%%%%%%%%%%%%%%%%%%%%%%%%%%%%%%%%%%%%%%%%%%%%%%%%%%


%%%%%%%%%%%%%%%%%%%%%%%%%%%%%%%%%%%%%%%%%%%%%%%%%%%%%%%%%%%%%%%%%%%%%%%%%%%%
%%%%%%%%%%%%%%%%%%%%%%%%%%%%%%%%%%%%%%%%%%%%%%%%%%%%%%%%%%%%%%%%%%%%%%%%%%%%
\begin{thm}\thmlabel{etat_deriv}
%
Under the conditions of \defref{prior_t, deriv_quantities}, let
\assuref{kl_opt_ok, exchange_order} hold.   Then the map $\t \mapsto
\etaopt(\t)$ is continuously differentiable at $\t=0$ with derivative
%
\begin{align}\eqlabel{vb_eta_sens}
%
\fracat{d \etaopt(\t)}{d \t}{0} ={}&
    - \hessopt^{-1} \crosshessian.
%
\end{align}
%
(For a proof, see \appref{proofs} \proofref{etat_deriv}.)
%
\end{thm}
%%%%%%%%%%%%%%%%%%%%%%%%%%%%%%%%%%%%%%%%%%%%%%%%%%%%%%%%%%%%%%%%%%%%%%%%%%%%


    \subsection{Differentiability of BNP models with respect to $\alpha$}
    \applabel{diffable_concentration}
    In this section, we return to the BNP problem and prove carefully that the map
$\alpha \mapsto \etaopt(\alpha)$ satisfies \assuref{kl_opt_ok, exchange_order},
and so the conditions of \thmref{etat_deriv}.  As in \exref{alpha_perturbation},
we will take $\mu$ to be the Lebesgue measure on $[0,1]^{\kmax - 1}$.

Recall from \secref{model_vb} that we take $\q(\nuk \vert \eta)$ to be a normal
distribution on the logit-transformed sticks, $\lnu_\k$.  For the duration of
this section, write $\q(\lnuk \vert \eta) = \normdist{\lnuk \vert \mu_\k,
\sigma^2_\k}$, so that the subvector of $\eta$ parameterizing $\q(\lnuk \vert
\eta)$ is $\etanuk = (\mu_\k, \sigma_\k)$.
%
By the formula for transformation of probability densities,
%
\begin{align*}
%
\q(\nuk \vert \etanuk) =
    \normdist{\log\left(\frac{\nu_\k}{1 - \nu_\k} \right)
        \Big\vert  \mu_\k, \sigma^2_\k}
    \frac{1}{\nuk (1 - \nuk)},
%
\end{align*}
%
where we have used the fact that $\fracat{d \lnu_\k}{ d\nuk}{\nuk} =
\frac{1}{\nuk (1 - \nuk)}$.  Similarly, for any function $f(\nuk)$ of the stick
lengths, we can transform the expecations as $\expect{\q(\nuk \vert
\etanuk)}{f(\nuk)} = \expect{\q(\lnuk \vert \etanuk)}{f\left(
\frac{\exp(\lnuk)}{1 + \exp(\lnuk)}  \right))}$, using the fact that
$\nuk = \frac{\exp(\lnuk)}{1 + \exp(\lnuk)}$.

Differentiability of $\KL{\eta}$ (\assuitemref{kl_opt_ok}{kl_diffable}) is
immediately satisfied for the $\eta$ that parameterize $\q(\beta \vert \eta)$
and $\q(\z \vert \eta)$ by our use of conjugate approximating families and
standard parameterizations.  The stick length density, $\q(\nuk \vert \etanuk)$
is not a standard exponential family
%
\footnote{In this section, we continue to take $\mu$ to be the Lebesgue measure
on $[0,1]$ as in \exref{alpha_perturbation}.  We could have equivalently taken
$\mu$ to be the Lebesgue measure on $\mathbb{R}$ and analyzed $\p(\lnuk \vert
\alpha)$ instead.  Had we done so, the same log Jacobian term $\log (\nuk(1 -
\nuk))$ would have appeared in the $\log \ptil(\lnuk \vert \alpha)$ term rather
than the $\q(\lnuk \vert \etanuk)$ entropy term, and so been part of
\assuref{exchange_order} rather than \assuitemref{kl_opt_ok}{kl_diffable}.  For
essentially this reason, the choice of dominainting measure in \defref{prior_t}
does not matter.}
%
, so we must show that the entropy $\expect{\q(\nuk \vert \etanuk)}{\log \q(\nuk
\vert \etanuk)}$ is  twice continuously differentiable. The entropy is given up
to a constant by
%
\begin{align*}
%
\MoveEqLeft
\expect{\q(\nuk \vert \etanuk)}{\log \q(\nuk \vert \etanuk)}
\\={}&
    \expect{\q(\nuk \vert \etanuk)}
           {\log \normdist{\log\left(\frac{\nu_\k}{1 - \nu_\k} \right)
               \Big\vert  \mu_\k, \sigma^2_\k}} +
    \expect{\q(\nuk \vert \etanuk)}
           {\log \left(\nuk (1 - \nuk)\right)}
% \\={}&
%     \expect{\q(\lnuk \vert \etanuk)}
%            {\log \normdist{\lnuk \Big\vert  \mu_\k, \sigma^2_\k}} +
%     \expect{\q(\lnuk \vert \etanuk)}
%            {\log \frac{\exp(\lnuk)}{1 + \exp(\lnuk)} } -
%     \expect{\q(\lnuk \vert \etanuk)}
%            {\log \frac{1}{1 + \exp(\lnuk)} }
\\={}&
   \expect{\q(\lnuk \vert \etanuk)}
          {\log \normdist{\lnuk \Big\vert  \mu_\k, \sigma^2_\k}} +
   \expect{\q(\lnuk \vert \etanuk)}{\lnuk}
\\={}&
    \frac{1}{2} \log \sigma^2_\k + \mu_\k + \const,
%
\end{align*}
%
which is twice continuously differentiable by inspection.
%
Indeed, \assuitemref{kl_opt_ok}{kl_diffable} is typically satisfied in VB
problems; when it is not, many black-box optimization methods also do not apply.

Non-singularity of the Hessian matrix $\hessopt$
(\assuitemref{kl_opt_ok}{kl_hess}) is satisfied whenever $\etaopt$ is at a local
optimum of $\KL{\eta}$.  In practice, we compute $\etaopt$ and (approximately)
check \assuitemref{kl_opt_ok}{kl_hess} numerically as part of computing the
sensitivity $\hessopt^{-1} \crosshessian$.  As with
\assuitemref{kl_opt_ok}{kl_diffable}, if \assuitemref{kl_opt_ok}{kl_hess} is
violated, then the user will probably have difficulty optimizing $\KL{\eta}$.

\assuitemref{kl_opt_ok}{kl_opt_interior} essentially requires that $\KL{\eta}$
be well-defined in an $\mathbb{R}^\etadim$ neighborhood of $\etaopt$, and can
require some care in choosing the parameterization $\eta$.  As an example of a
parameterization that would violate \assuitemref{kl_opt_ok}{kl_opt_interior},
consider parametrizing $\q(\z_{\n} \vert \eta)$ by the $\kmax$ expectations
$m_\k := \expect{\q(\z_{\n} \vert \eta)}{\z_{\n\k}}$.  The set $(m_1, \ldots,
m_\kmax)$ completely specify $\q(\z_{\n} \vert \eta)$, but violate
\assuitemref{kl_opt_ok}{kl_opt_interior}, since any valid parametization
satisfies $\sum_{\k=1}^\kmax m_\k = 1$, and so no open ball in
$\mathbb{R}^\etadim$ can be contained in $\etadom$.  However,
\assuitemref{kl_opt_ok}{kl_opt_interior} is satisfied we use an {\em
unconstrained parameterization} for $\q(\zeta \vert \eta)$.   Unconstrained
parameterizations of variational distributions allow the use of unconstrained
optimization for variational inference and are a good practice when available
\citep{kucukelbir:2016:advi}.  For details on our parameterizations, see
\appref{app_vb_details}.

Verifying \assuref{exchange_order} is the principal technical challenge of
satisfying the conditions of \thmref{etat_deriv}. Recall from
\exref{alpha_perturbation} that $\log \ptil(\nuk \vert \t) = t \log (1 - \nuk)$,
so we need to establish \assuref{exchange_order} for
%
\begin{align*}
%
-\expect{\q(\nuk \vert \etanuk)}{t \log (1 - \nuk)} =
% -\expect{\q(\lnuk \vert \etanuk)}
%        {t \log (1 - \frac{\exp(\lnuk)}{1 + \exp(\lnuk)})} =
\expect{\q(\lnuk \vert \etanuk)}
      {t \log (1 + \exp(\lnuk))}.
%
\end{align*}
%
Since the preceding equality holds for all $\t$ and $\etanuk$, it suffices to
establish that we can exchange the order of integration and differentiation for
the right hand side.  Note that derivatives with respect to any components of
$\eta$ other than $\etanuk$ are zero and so \assuref{exchange_order} is
trivially satisfied.

The following lemma establishes a general condition for the ability
to exchange normal expectations and the needed derivatives.

% \begin{align}\eqlabel{lnuk_derivatives}
% %
% \fracat{d \lnu_\k}{ d\nuk}{\nuk} ={}
% %     \frac{1-\nuk}{\nuk}
% %     \left(\frac{1}{1 - \nuk} + \frac{\nuk}{(1 - \nuk)^2} \right)
% % \\={}& \frac{1}{\nuk} + \frac{1}{1 - \nuk}
% % \\={}&
%     \frac{1}{\nuk (1 - \nuk)} \mathand
% %
% \fracat{d \nuk}{ d\lnuk}{\lnuk} ={}
%     \frac{\exp(\lnuk)}{(1 + \exp(\lnuk))^2}.
% %
% \end{align}


%%%%%%%%%%%%%%%%%%%%%%%%%%%%%%%%%%%%%%%%%%%%%%%%%%%%%%%%%%%%%%%%%%%%%%%%%%%%
%%%%%%%%%%%%%%%%%%%%%%%%%%%%%%%%%%%%%%%%%%%%%%%%%%%%%%%%%%%%%%%%%%%%%%%%%%%%
\begin{lem}\lemlabel{normal_q_is_regular}
%
Let $\mu$ denote the Lebesgue measure on $\mathbb{R}$. Let $\eta = (\mu,
\sigma)$, let $\normdist{\theta \vert \eta}$ denote the corresponding normal
density with respect to $\mu$, and and let $\ball_\eta$ denote an open ball in
$\mathbb{R}^2$ such that $0 < \sigma < \infty$ for all $\eta \in \ball_\eta$.
%
Let $\ball_\t$ denote an open ball in $\mathbb{R}$, and let $\psi(\theta, \t)$
be a function such that $\theta \mapsto \psi(\theta, \t)$ is $\mu$-measurable
for all $\t \in \ball_\t$.

If there exists a constant $C > 0$ such that
%
\begin{align*}
%
\sup_{\t \in \ball_\t} \abs{\psi(\theta, \t)}
    \le C \exp(\abs{\theta})
\mathand
\sup_{\t \in \ball_\t}
    \abs{\frac{\partial \psi(\theta, \t)}{\partial \t}}
    \le C \exp(\abs{\theta}),
%
\end{align*}
%
then one can exchange the order of expectation and differentation in the
expression $\expect{\normdist{\theta \vert \eta}}{\psi(\theta, \t)}$ for the
derivatives $\partial / \partial \eta$, $\partial^2 / \partial \eta \partial
\eta$, and $\partial^2 / \partial \eta  \partial \t$, eavluated at any $\t \in
\ball_\t$ and $\eta \in \ball_\eta$.

\seeproof{normal_q_is_regular}
\end{lem}
%%%%%%%%%%%%%%%%%%%%%%%%%%%%%%%%%%%%%%%%%%%%%%%%%%%%%%%%%%%%%%%%%%%%%%%%%%%%

Since $\log (1 + \exp(\lnuk)) \exp(-\abs{\lnuk})  < \infty$ for all $\lnuk \in
\mathbb{R}$, the conditions of \lemref{normal_q_is_regular} are satisfied for
any ball $\ball_\eta$ such that the variational variances are all finite, and
\assuref{exchange_order} is satsified.

It follows that \thmref{etat_deriv} applies to the map $\alpha \mapsto
\etaopt(\alpha)$.


    \subsection{Nonparametric prior perturbations}
    \applabel{diffable_nonparametric}
    In the previous section, we showed that we can differentiate the VB optimum with
respect to $\alpha$ in the $\gem$ prior, which we can use to form a Taylor
series to how $\etaopt(\alpha)$ varies within the $\gem$ family.  However, there
is typically no {\em a priori} reason to believe that the stick breaking prior
lies within the parametric Beta family.  We now show how, by parameterizing a
path between two arbitrary densities, we can apply \thmref{etat_deriv} to
nonparametric perturbations.

Again let us return to the abstract setting of \defref{prior_t}. Let us fix an
initial prior density, $\pbase(\theta)$, at which we have computed a VB
approximation, and suppose we wish to ask what the variational optimum would
have been had we used some alternative prior density, $\palt(\theta)$. For
example, in the BNP setting, one might take $\pbase(\theta)$ to be
$\betadist{\nuk \vert \alpha_0}$, and $\palt(\theta)$ to be some generic
function of $\theta$ outside the Beta family. Let us write $\etaopt(\pbase)$ and
$\etaopt(\palt)$ for these two approximations, respectively, so we are
interested in quantifying the change $\g(\etaopt(\palt)) - \g(\etaopt(\pbase))$.
If this change is large, we say that our quantity of interest is not robust to
replacing $\pbase$ with $\palt$.

To approximately assess robustness using the local sensitivity approach of
\secref{local_sensitivity}, we must somehow define a continuous path from
$\pbase(\theta)$ to $\palt(\theta)$ parameterized, say, by $\t \in [0, 1]$. One
way to do so is to define a multiplicative path
%
\begin{align}
%
\log \ptil(\theta \vert \t) ={}&
    (1 - \t)\log \pbase(\theta) + \t \log \palt(\theta).
        \eqlabel{mult_pert_simple}
%
\end{align}
%
Under \eqref{mult_pert_simple}, when $\t=0$, $\p(\theta \vert \t) =
\pbase(\theta)$, when $\t=1$, $\p(\theta \vert \t, \pbase, \palt) =
\palt(\theta)$, and $\t \in (0,1)$ smoothly parameterizes a path between the
two.  If we can verify that \thmref{etat_deriv} applies to the perturbation
given in \eqref{mult_pert_simple}, then, just as in the parametric case, we can
form the Taylor series approximation,
%
\begin{align*}
%
\etaopt(\palt) \approx
    \etaopt(\pbase) + \fracat{d \etaopt(\t)}{d\t}{\t=0} (1 - 0).
%
\end{align*}

Our first task is then to state conditions under which \thmref{etat_deriv}
applies to \eqref{mult_pert_simple}.  In \eqref{mult_pert_simple} we have
assumed that $\palt$ is a density, but it will be more convenient to observe
that, when $\palt \ll \pbase$, we can re-write
%
\begin{align*}
%
\log \ptil(\theta \vert \t) ={}&
    \log \pbase(\theta) +
        \t \log \frac{\palttil(\theta)}{\pbasetil(\theta)} +
        \const. & \constdesc{\theta}
%
\end{align*}
%
Defining the generic function $\phi(\theta) := \log
\frac{\palttil(\theta)}{\pbasetil(\theta)}$ motivates consideration of
perturbations of the form $\log \ptil(\theta \vert \t) = \pbase(\theta) + \t
\phi(\theta)$, where $\phi(\theta)$ is some generic measurable function. We can
then ask what $\phi$ give rise to valid densities as well as differentiable maps
$\t \mapsto \etaopt(\t)$.

%%%%%%%%%%%%%%%%%%%%%%%%%%%%%%%%%%%%%%%%%%%%%%%%%%%%%%%%%%%%%%%%%%%%%%%%%
%%%%%%%%%%%%%%%%%%%%%%%%%%%%%%%%%%%%%%%%%%%%%%%%%%%%%%%%%%%%%%%%%%%%%%%%%
\begin{defn}\deflabel{prior_nl_pert}
%
Let $\mu$ denote a measure and fix $\pbase(\theta)$, a density with respect to
$\mu$.  Assume that $\pbase(\theta) > 0$ on $\thetadom$. For any measurable
$\phi: \thetadom \mapsto \mathbb{R}$ for which the expressions are well-defined,
let
%
\begin{align*}
\ptil(\theta \vert \phi) :={}& \pbase(\theta)\exp(\phi(\theta)).
%
\end{align*}
%
As usual, when $0 < \int \ptil(\theta \vert \phi) \mu(d\theta) < \infty$, we let
$\p(\theta \vert \phi)$ be the normalized version of $\ptil(\theta \vert \phi)$.
Further, define the norm $\norminf{\phi} := \esssup_{\theta \sim \mu}
\abs{\phi(\theta)}$, and let $\ball_\phi(\delta) := \left\{ \phi: \norminf{\phi} <
\delta \right\}$.
\todo{Tamara rightly points out that there is a need for $\mu$ and
$\pbase$ to be mutually absolutely continuous, where $\pbase$ is fixed.
This is awkward as $\pbase$ is now defined as a density, but it is actually
the distribution induced by $\pbase$ that is the fundamental object.}
%
\end{defn}
%
%%%%%%%%%%%%%%%%%%%%%%%%%%%%%%%%%%%%%%%%%%%%%%%%%%%%%%%%%%%%%%%%%%%%%%%%%

The class of perturbations defined in \defref{prior_nl_pert} are one of the
family of ``nonlinear'' functional perturbations given by
\citet{gustafson:1996:local}, though we deviate from
\citet{gustafson:1996:local} by allowing $\phi$ to take on negative values. The
following result, which motivates the use of the $\norminf{\cdot}$ norm to
measure the ``size'' of a perturbation $\phi$, is only a minor modification of
the corresponding result from \citet{gustafson:1996:local} to allow negative
perturbations.

%%%%%%%%%%%%%%%%%%%%%%%%%%%%%%%%%%%%%%%%%%%%%%%%%%%%%%%%%%%%%%%%%%%%%%%%%%%
%%%%%%%%%%%%%%%%%%%%%%%%%%%%%%%%%%%%%%%%%%%%%%%%%%%%%%%%%%%%%%%%%%%%%%%%%%%
\begin{lem}\lemlabel{pert_invariance}
%
(\citet{gustafson:1996:local})
%
Fix the quantities given in \defref{prior_nl_pert}.  For a fixed probability
measure $\palt \ll \mu$, let $\phi(\theta \vert \palt) := \log \palt(\theta) /
\pbase(\theta)$.  Then $\palt \mapsto \norminf{\phi(\cdot \vert \palt)}$ is a
norm, does not depend on $\mu$, and is invariant to invertible transformations
of $\theta$.
\todo{Make sure this works with the corrected \defref{prior_nl_pert} }

Furthermore, for any $\phi$ with $\norminf{\phi} < \infty$, the quantity
$\ptil(\theta \vert \phi)$ gives rise to a valid prior, in the sense that
$\ptil(\theta \vert \phi) \ge 0$ $\mu$-almost everywhere, and
$0 < \int \ptil(\theta \vert \phi) \mu(d\theta) < \infty$.
%
\seeproof{pert_invariance}
%
\end{lem}
%%%%%%%%%%%%%%%%%%%%%%%%%%%%%%%%%%%%%%%%%%%%%%%%%%%%%%%%%%%%%%%%%%%%%%%%%%%

The set of priors $\left\{\p(\theta \vert \phi) : \phi \in
\ball_\phi(\delta)\right\}$ live in a multiplicative band around the original
prior, $\pbase$, as shown in \figref{func_ball}. Although
\lemref{pert_invariance} proves that every $\phi$ with $\norminf{\phi}$ is a
valid prior, the converse is not true, and the Beta prior perturbation of
\exref{alpha_perturbation} is a counterexample.


%%%%%%%%%%%%%%%%%%%%%%%%%%%%%%%%%%%%%%%%%%%%%%%%%%%%%%%%%%%%%%%%%%%%%%%%%
%%%%%%%%%%%%%%%%%%%%%%%%%%%%%%%%%%%%%%%%%%%%%%%%%%%%%%%%%%%%%%%%%%%%%%%%%%%
\begin{ex}\exlabel{beta_inf_norm}
%
Take $\mu$ to be the Lebesgue measure on $[0,1]$, let $\pbase(\theta) =
\betadist{\theta \vert 1, \alpha_0}$ and $\palt(\theta) = \betadist{\theta \vert
1, \alpha_1}$ for $\alpha_0 \ne \alpha_1$.  Taking
$\phi(\theta) = (\alpha_1 - \alpha_0) \log(1 - \theta)$ parameterizes
a path from $\pbase$ to $\palt$ as in \eqref{mult_pert_simple}, and
%
\begin{align*}
%
\norminf{\phi} =
    \abs{\alpha_1 - \alpha_0} \sup_{\theta \in [0,1]} \abs{\log(1 - \theta)} =
    \infty.
%
\end{align*}
%
Therefore, in general, there exist valid priors that cannot be expressed by
\defref{prior_nl_pert} with $\phi$ with $\norminf{\phi} < \infty$.
%
\end{ex}
%%%%%%%%%%%%%%%%%%%%%%%%%%%%%%%%%%%%%%%%%%%%%%%%%%%%%%%%%%%%%%%%%%%%%%%%%%%%

We now show that, when $\norminf{\phi} < \infty$, we can apply
\thmref{etat_deriv}.  We still require the following assumption on the VB
density, which is strictly weaker than \assuref{exchange_order_dom}.

%%%%%%%%%%%%%%%%%%%%%%%%%%%%%%%%%%%%%%%%%%%%%%%%%%%%%%%%%%%%%%%%%%%%%%%%%
%%%%%%%%%%%%%%%%%%%%%%%%%%%%%%%%%%%%%%%%%%%%%%%%%%%%%%%%%%%%%%%%%%%%%%%%%
\begin{assu}\assulabel{exchange_order_q}
%
Assume that \assuref{exchange_order_f} applies with the function $f(\theta,
\eta, \t) = \q(\theta \vert \eta)$ (no $\t$ dependence).
%
\end{assu}
%%%%%%%%%%%%%%%%%%%%%%%%%%%%%%%%%%%%%%%%%%%%%%%%%%%%%%%%%%%%%%%%%%%%%%%%%

%%%%%%%%%%%%%%%%%%%%%%%%%%%%%%%%%%%%%%%%%%%%%%%%%%%%%%%%%%%%%%%%%%%%%%%%%
%%%%%%%%%%%%%%%%%%%%%%%%%%%%%%%%%%%%%%%%%%%%%%%%%%%%%%%%%%%%%%%%%%%%%%%%%

\begin{cor}\corylabel{etafun_deriv_form}
%
Fix the quantities given in \defref{prior_nl_pert}, and let \assuref{kl_opt_ok,
exchange_order_q} hold. Let $g(\eta): \etadom \mapsto \mathbb{R}$ denote a
continuously differentiable real-valued function of interest.  Define the
``influence function'' $\infl: \thetadom \mapsto \mathbb{R}$:
%
\begin{align}\eqlabel{infl_defn}
%
\infl(\theta) :={}&
    - \fracat{d g(\eta)}{ d \eta^T}{\etaopt} \hessopt^{-1}
        \lqgradbar{\theta \vert \etaopt}
        \q(\theta \vert \etaopt).
%
\end{align}
%
Then, if $\norminf{\phi} < \infty$, the map $\t \mapsto g(\etaopt(\t \phi))$ is
continuously differentiable at $\t=0$ with derivative
%
\begin{align}\eqlabel{vb_eta_infl_sens}
%
\fracat{d g(\etaopt(\t \phi))}{d \t}{0} ={}&
    \int \infl(\theta) \phi(\theta) \mu(d\theta).
%
\end{align}
%
\begin{proof}
%
It suffices to show that \assuref{exchange_order_q} implies
\assuref{exchange_order} for the perturbation given in \defref{prior_nl_pert}
when $\norminf{\phi} < \infty$.  Observe that $\log \ptil(\theta \vert \t) = \t
\phi(\theta)$, so, for any $f(\theta, \eta, \t)$ that satisfies the conditions
of \assuref{exchange_order_f},
%
%\begin{align*}
%
$\phi(\theta) f(\theta, \eta, \t) \le \norminf{\phi} M(\theta)$.
%
%\end{align*}
%
Therefore \assuref{exchange_order_f} is satisfied by $\phi(\theta) f(\theta,
\eta, \t)$ as well.  It follows that \assuref{exchange_order_q} $\Rightarrow$
\assuref{exchange_order_dom} $\Rightarrow$ \assuref{exchange_order}.
%
The form of the influence function is then given by gathering terms in
\eqref{vb_eta_sens}.
%
\end{proof}
%
\end{cor}

%%%%%%%%%%%%%%%%%%%%%%%%%%%%%%%%%%%%%%%%%%%%%%%%%%%%%%%%%%%%%%%%%%%%%%%%%

The influence function can be a useful summary of the effect of making generic
changes to the prior density, as we will show in the experiments of
\secref{results}.  For visualization, it can be useful to reduce the dimension
of the domain of the influence function, as we discuss in the following example.

%%%%%%%%%%%%%%%%%%%%%%%%%%%%%%%%%%%%%%%%%%%%%%%%%%%%%%%%%%%%%%%%%%%%%%%%%
%%%%%%%%%%%%%%%%%%%%%%%%%%%%%%%%%%%%%%%%%%%%%%%%%%%%%%%%%%%%%%%%%%%%%%%%%
\begin{ex}\exlabel{infl_univariate}
%
In the BNP example, we are perturbing each of the sticks, so we take $\theta \in
[0,1]^{\kmax - 1}$.  Formally, $\phi: [0,1]^{\kmax - 1} \mapsto \mathbb{R}$ can
express different perturbations for the density of each of the $\kmax - 1$
sticks.  However, when we describe ``changing the stick breaking density,'' we
mean changing each stick's prior density in the same way.

To represent perturbing all the sticks simultaneously, take some univariate
perturbation $\phi_{u}: [0,1] \mapsto \mathbb{R}$, and set $\phi(\nu_1, \ldots,
\nu_{\kmax - 1}) = \sum_{\k=1}^{\kmax - 1} \phi_{u}(\nuk)$. By linearity of the
derivative \coryref{etafun_deriv_form},
%
\begin{align*}
%
\fracat{d g(\etaopt(\t \phi))}{d \t}{0} ={}&
    \int \infl(\theta) \left(
        \sum_{\k=1}^{\kmax - 1} \phi_{u}(\nuk) \right)
    d\nu_1 \ldots d \nu_{\kmax - 1}.
%
\end{align*}
%
By definition, $\expect{\q(\theta \vert \etaopt)}{\lqgradbar{\theta \vert
\etaopt}} = 0$, so $\int \infl(\theta) \mu(d\theta) = 0$.  By the mean field
assumption, $\infl(\nu_1, \ldots, \nu_{\kmax - 1}) = \prod_{\k=1}^{\kmax - 1}
\infl_\k(\nuk)$, where $\infl_\k(\nuk)$ is derived from \eqref{infl_defn} but
using $\theta = \nuk$.  Letting $\nu_0 \in [0,1]$ denote the variable of
integration and plugging in the preceding observations gives
%
\begin{align*}
%
\int \infl(\theta) \phi(\theta) \mu(d\theta) =
    \int_0^1 \left(\sum_{\k=1}^{\kmax - 1} \infl_k(\nu_0) \right)
        \phi_{u}(\nu_0) d \nu_0.
%
\end{align*}
%
Thus we can say that the influence function for perturbing all the stick
breaking densities simultaneously is given by the sum of the
individual sticks' influence functions, which maps $[0,1] \mapsto \mathbb{R}$.
%
\end{ex}
%%%%%%%%%%%%%%%%%%%%%%%%%%%%%%%%%%%%%%%%%%%%%%%%%%%%%%%%%%%%%%%%%%%%%%%%%


    \subsection{Worst-case prior perturbations and Fr{\'e}chet differentiability}
    \applabel{diffable_worst_case}
    As we saw in \corref{etafun_deriv_form}, the derivative of perturbations given
by \defref{prior_nl_pert} takes the form of an integral of the influence
function against the perturbation.  It is natural to use the influence function
to {\em explore} the space of priors, e.g., to find alternative priors with
large influence but small $\norminf{\phi}$.  Consider as an example the
following corollary is an example, which is the VB analogue of \citet[Result
11]{gustafson:1996:local}.

% , where the integrand is known as the ``influence function.'' In
% addition to providing an interpretable summary of the effect of different
% perturbations, the influence function motivates the consideration of
% ``worst-case'' prior perturbations which maximize the influence subject to a
% size constraint.  To justify this using the influence function in this way, we
% prove that the VB optimum is in fact Fr{\'e}chet differentiable as a function of
% the perturbation.

%%%%%%%%%%%%%%%%%%%%%%%%%%%%%%%%%%%%%%%%%%%%%%%%%%%%%%%%%%%%%%%%%%%%%%%%%
%%%%%%%%%%%%%%%%%%%%%%%%%%%%%%%%%%%%%%%%%%%%%%%%%%%%%%%%%%%%%%%%%%%%%%%%%

\begin{cor}\corlabel{etafun_worst_case}
%
The ``worst-case'' derivative in $\ball_\phi(\delta)$ is given by
%
\begin{align*}
%
\sup_{\phi \in \ball_\phi(\delta)}
    \fracat{d g(\etaopt(\t \phi))}{d \t}{0} =
        \delta \int \abs{\infl(\theta)} \mu(d\theta),
%
\end{align*}
%
which is acheived at the perturbation
$\phi^*(\theta) = \delta \, \mathrm{sign}\left(\infl(\theta)\right)$.
%
\begin{proof}
%
The result follows immediately from applying H{\"o}lder's inequality
(\citet[Theorem 5.1.2]{dudley:2018:real} and subsequent disscussion)
to \eqref{vb_eta_infl_sens}.
%
\end{proof}
%
\end{cor}

%%%%%%%%%%%%%%%%%%%%%%%%%%%%%%%%%%%%%%%%%%%%%%%%%%%%%%%%%%%%%%%%%%%%%%%%%

However, \corref{etafun_deriv_form} does not precisely justify using
the influence function as in \corref{etafun_worst_case}.


In turn, \corref{etafun_worst_case} motivates the question of whether
\eqref{vb_eta_infl_sens} provides a uniformly good linear approximation to
$\etaopt(\t)$ in a neighborhood of $0$: that is, whether $\phi \mapsto
\etaopt(\phi)$ is Fr{\'e}chet differentiable. By \citep[Theorem
5.2.1]{dudley:2018:real}, $L_\infty$ is a Banach space.

%%%%%%%%%%%%%%%%%%%%%%%%%%%%%%%%%%%%%%%%%%%%%%%%%%%%%%%%%%%%%%%%%%%%%%%%%%%
%%%%%%%%%%%%%%%%%%%%%%%%%%%%%%%%%%%%%%%%%%%%%%%%%%%%%%%%%%%%%%%%%%%%%%%%%%%
\begin{defn}\deflabel{diffable_classes}
    (\citep[Definition 4.5]{zeidler:2013:functional})
%
Let $B_1$ and $B_2$ denote Banach spaces, and let $\ball_1 \subseteq B_1$ define
an open neighborhood of $\phi_0 \in B_1$.  Fix a function $f: \ball_1
\mapsto B_2$.
%
A function $f$ is {\em Fr{\'echet} differentiable} (also known as boundedly
differentiable) at $\phi_0$ if
%
\begin{align*}
%
\lim_{t \rightarrow 0}
    \sup_{\phi: \norm{\phi - \phi_0} = 1}
    \frac{f(\phi) - f(\phi_0) -
          f^{\mathrm{lin}}(t (\phi - \phi_0))
         }{t} \rightarrow 0.
%
\end{align*}
%
\end{defn}
%%%%%%%%%%%%%%%%%%%%%%%%%%%%%%%%%%%%%%%%%%%%%%%%%%%%%%%%%%%%%%%%%%%%%%%%%%%

%%%%%%%%%%%%%%%%%%%%%%%%%%%%%%%%%%%%%%%%%%%%%%%%%%%%%%%%%%%%%%%%%%%%%%%%%%%%
%%%%%%%%%%%%%%%%%%%%%%%%%%%%%%%%%%%%%%%%%%%%%%%%%%%%%%%%%%%%%%%%%%%%%%%%%%%%
\begin{thm}\thmlabel{eta_phi_deriv}
%
Let \assuref{kl_opt_ok, exchange_order_q} hold. Then the map $\phi \mapsto
\etaopt(\phi)$ is well-defined and continuously Fr{\'e}chet differentiable in a
neighborhood of $\phiz$ as a map from $\lp{\mu,\infty}$ to $\mathbb{R}^\etadim$,
with the derivative given in \corref{etafun_deriv_form}.

(For a proof, see \appref{proofs} \proofref{eta_phi_deriv}.)

\end{thm}
%%%%%%%%%%%%%%%%%%%%%%%%%%%%%%%%%%%%%%%%%%%%%%%%%%%%%%%%%%%%%%%%%%%%%%%%%%%%


    \subsection{Other nonparametric prior perturbations}
    \applabel{diffable_lp}
    In \secref{diffable_nonparametric} we considered multiplicative perturbations to
the prior density.  One might ask whether one could consider other paths through
the space of priors, such as additive perturbations.  In this section, we
briefly consider a broader class of nonlinear perturbations investigated by
\citet{gustafson:1996:local}, of which additive and multiplicative perturbations
are special cases, and show that, within this class, only multiplicative
perturbations are compatible with KL divergence.

As in \secref{diffable_nonparametric}, suppose we have a base prior $\pbase$ and
an alternative $\palt$, and that we wish to parameterize a continuous path
between them.  We will do so as follows. For some $p \in [1, \infty)$, let
%
\begin{align}\eqlabel{p_pert_simple}
%
\ptil(\theta \vert \tp) :=
    \left((1 - \tp)\pbase(\theta)^{1/p} +
    \tp \frac{1}{p}\palt(\theta)^{1/p} \right)^{p}.
%
\end{align}
%
As with \eqref{mult_pert_simple}, $\p(\theta \vert \tp = 0) = \pbase(\theta)$,
$\p(\theta \vert \tp = 1) = \palt(\theta)$, and $\p(\theta \vert \tp)$ moves
continously between the two in $\tp \in (0, 1)$.  When $p = 1$,
$\ptil(\theta \vert \tp)$ defines an ``additive perturbation,'' and
the limit as $p \rightarrow \infty$ gives the multiplicative perturbation
of \eqref{mult_pert_simple}.
%
\citet{gustafson:1996:local} proves that a result analogous to
\eqref{p_pert_simple} for \eqref{p_pert_simple}, where the
$\norminf{\cdot}$ norm is repalaced by
%
\begin{align}\eqlabel{phi_lp_norm}
%
\phi(\theta \vert \palt, p) :={}
    \palt(\theta)^{1/p} - \pbase(\theta)^{1/p} \mathand
\norm{\phi}_p :={} \left(\int \abs{\phi(\theta)}^p \right)^{1/p}.
%
\end{align}
%
We refer the reader to \citet{gustafson:1996:local} for details.  For our
present discussion, what matters is that the use of the perturbation in
\eqref{p_pert_simple} strongly motivates the use of the norm $\norm{\phi(\theta
\vert \palt, p)}_p$ when forming, for example, worst-case perturbations as in
\corref{etafun_worst_case}.

Though the $\norm{\phi(\theta \vert \palt, p)}_p$ norm does not appear to cause
major difficulties for the full Bayesian posterior, this norm is not compatible
with KL divergence, in the sense that KL divergence is {\em discontinuous} in
this norm.  Prior changes that are arbitrarily small according to
$\norm{\phi(\theta \vert \palt, p)}_p$ can induce arbitrarily large changes in
the KL divergence \eqref{kl_def}, and so (in genreal) arbitrarily large
changes in its optimum.  The problem is best illustrated with an
example.

%%%%%%%%%%%%%%%%%%%%%%%%%%%%%%%%%%%%%%%%%%%%%%%%%%%%%%%%%%%%%%%%%%%%%%%%%
%%%%%%%%%%%%%%%%%%%%%%%%%%%%%%%%%%%%%%%%%%%%%%%%%%%%%%%%%%%%%%%%%%%%%%%%%
\begin{thm}\thmlabel{kl_discontinuous}
%
Let $\mu$ denote a measure on $\thetadom$ that is absolutely continous
with respect to the Lebesgue measure, and let $\q(\theta)$ and
$\pbase(\theta)$ denote densities with respect to $\mu$.  Without loss of
generality, assume that $\q(\theta) > 0$ on $\thetadom$.  Assume that
$\KL{q(\theta) || \pbase(\theta)}$ is well-defined and finite.

Then, for any $\epsilon > 0$ and any $M > 0$, we can find a density
$\palt(\theta)$ such that $\norm{\phi(\theta \vert \palt, p)}_p < \epsilon$ but
$\abs{\KL{q(\theta) || \palt(\theta)} - \KL{q(\theta) || \pbase(\theta)}} > M$.

\begin{proof}
%
The proof will follow by perturbing $\pbase(\theta)$ to zero in a small
interval.  By making the interval narrow, we can make $\norm{\phi(\theta \vert
\palt, p)}_p$ small, but by making the $\palt(\theta)$ sufficiently close to
zero, we can make the KL divergence difference large irresective of how narrow
the interval is.

First, observe that
%
\begin{align*}
%
\KL{q(\theta) || \palt(\theta)} -
\KL{q(\theta) || \pbase(\theta)} ={}&
\expect{\q(\theta)}{\log \frac{\palt(\theta)}{\pbase(\theta)}}.
%
\end{align*}

For any set $S$ with $\pbase(S) = \epsilon$, define
%
\begin{align*}
%
\palt(\theta \vert S, \delta) :=
    \frac{\delta^{\ind{\theta \in S}}}{1 + \epsilon(1 - \delta)}.
%
\end{align*}
%
Then $\palt(\theta \vert S, \delta)$ is a valid density, and
%
\begin{align*}
%
\KL{q(\theta) || \palt(\theta)} - \KL{q(\theta) || \pbase(\theta)}
    ={}& \q(S) \log \delta - \log\left( 1 + \epsilon(1 - \delta) \right).
%
\end{align*}
%
By \eqref{phi_lp_norm},
%
\begin{align*}
%
\phi(\theta \vert \palt, p) ={}&
    \pbase(\theta)^{1/p} \left(
        \frac{\left(\delta^{1/p}\right)^{\ind{\theta \in S}}}
             {\left( 1 + \epsilon(1 - \delta) \right)^{1/p}} - 1 \right) \mathand\\
\norm{\phi(\theta \vert \palt, p)}_p^p ={}&
\epsilon \left(
   \frac{\left(\delta^{1/p}\right)}
        {\left( 1 + \epsilon(1 - \delta) \right)^{1/p}} - 1 \right) +
(1 - \epsilon) \left(
   \frac{1}
        {\left( 1 + \epsilon(1 - \delta) \right)^{1/p}} - 1 \right).
%
\end{align*}

Since $\mu$ is absolutely continuous with respect to the Lebesgue measure, there
exists a sequence $\epsilon_n \rightarrow 0$ with $\epsilon_n > 0$ and a
sequence of corresponding sets $S_n$ such that $\pbase(S_n) = \epsilon_n$. (See
\lemref{continuity_partition} for a proof of this fact, which is a
straightforward consequence of \citet[Proposition 15.5]{nielsen:1997:measure}
and the continuity of the Lebesgue measure.) Since $\q(\theta) > 0$ on
$\thetadom$, $\q(S_n) > 0$ for all $n$.  Since $\KL{\q(\theta) ||
\pbase(\theta)}$ is finite, we must have $\lim_{n \rightarrow} \q(S_n) = 0$.

Take $\delta_n  = \exp(-1 / (\q(S_n)^2))$, and take $\palt(\theta) =
\palt(\theta \vert S_n, \delta_n)$.  Then $\epsilon_n (1 - \delta_n) \rightarrow
0$, and $\q(S_n)\log \delta_n = -1 / \q(S_n)$, so
%
\begin{align*}
%
\abs{\KL{q(\theta) || \palt(\theta \vert S_n, \delta_n)} -
    \KL{q(\theta) || \pbase(\theta)}} \rightarrow{}& \infty, \quad \textrm{but}\\
%
\norm{\phi(\theta \vert \palt(\cdot \vert S_n, \delta_n), p)}_p^p
    \rightarrow{}& 0.
%
\end{align*}
%
Thus, for sufficiently large $n$, the conclusion follows.
%
\end{proof}
%
\end{thm}
%%%%%%%%%%%%%%%%%%%%%%%%%%%%%%%%%%%%%%%%%%%%%%%%%%%%%%%%%%%%%%%%%%%%%%%%%

Since Fr{\'e}chet differentiability implies continuity,
\thmref{kl_discontinuous} shows that it is impossible to derive an analogue of
\thmref{eta_phi_deriv} for perturbations of the form \eqref{p_pert_simple}
with the norms \eqref{phi_lp_norm}.

Recall \exref{beta_inf_norm}, where we showed that there exist valid priors for
which $\norminf{\phi} < \infty$.  Viewed in light of the proof of
\thmref{kl_discontinuous}, the limited expressiveness of the $\norminf{\cdot}$
norm looks like a feature, rather than a bug.
% The KL divergence that defines a variational objective cannot handle
% prior densities that are too close to zero.  The $\norminf{\cdot}$ norm
% considers such densities to be ``distant'' from $\pbase$, whereas the
% more permissive $\norm{\cdot}_p$ norms do not.
The situation is illustrated in  \figref{func_dist}.  The two shown densities
are far from one another according to KL divergence (from blue to red) since the
red density has nearly zero mass where the blue does not.  They are also distant
from one another in $\norminf{\cdot}$, since it takes a large multiplicative
change to turn a nonzero number into a nearly zero number.  However, the two
densities are close in $\norm{\cdot}_{p}$, since the region where the red
density is nearly zero has a small measure. In order for VB approximations to be
continuous (a necessary condition for Fr{\'e}chet differentiability), one must
consider a topology on priors that is no coarser than the topology induced by KL
divergence.  But since valid priors can take values close to zero, a sacrifice
in expressiveness of the neighborhood of zero must be made in order to induce a
topology that works with KL divergence.  Multiplicative changes and the
$\norminf{\cdot}$ norm make such a tradeoff in a natural, easy-to-understand
way.

In this sense, VB approximations based on KL divergence are inherently
non-robust to priors that ablate mass nearly to zero.  No parameterization of
the space of priors will relieve this non-robustness.  Only by basing
variational approximations on divergences other than KL will this non-robustness
be alleviated.

\FunctionDistFig{}


\section{Proofs}\applabel{proofs}
For now let's leave the proofs in the main text until everything looks good.


\section{Positive Perturbations Are Counterintuitive}\applabel{positive_pert}

% Suppose $p < \infty$ and we wish to choose a $\phi$ to change $\pbase$ into
% $\palt$, for which \defref{prior_nl_pert} gives that we must choose
% %
% \begin{align}\eqlabel{phi_for_palt}
% %
% \phi(\theta) = \alpha \palt(\theta)^{1/p} - \pbase(\theta)^{1/p}
%     \mathtxt{where}\alpha > 0.
% %
% \end{align}
%
As long as $\palt(\nu) > 0$ whenever $\pbase(\nu) >
0$ (a condition which is certainly not always satisfied), one can choose
%
\begin{align*}
%
\alpha = \sup_{\nu} \left(\frac{\pbase(\nu)}{\palt(\nu)}\right)^{1/p}
%
\end{align*}
%
to guarantee that $\phi(\nu)$ is non-negative.  However, by doing so, one might
have to create a ``large'' perturbation, according to the $\norm{\cdot}_p$,
as the following \exref{positive_pert_large} demonstrates.

% %%%%%%%%%%%%%%%%%%%%%%%%%%%%%%%%%%%%%%%%%%%%%%%%%%%%%%%%%%%%%%%%%%%%%%%%%
% \begin{ex}
% %
% Let $\pbase(\nu) = \frac{4}{3}\ind{\frac{1}{4} \le \nu \le 1}$ and $\palt(\nu) =
% \frac{4}{3}\ind{0 \le \nu \le \frac{3}{4}}$.  Then, for any $\alpha > 0$, any $1
% \le p < \infty$, and for $\phi$ as given in \eqref{phi_for_palt}, $\phi(7/8) <
% 0$.  So there is no value of $\alpha$ that can transform $\pbase$ into $\palt$
% using only a positive perturbation.
% %
% \end{ex}
%%%%%%%%%%%%%%%%%%%%%%%%%%%%%%%%%%%%%%%%%%%%%%%%%%%%%%%%%%%%%%%%%%%%%%%%%
%%%%%%%%%%%%%%%%%%%%%%%%%%%%%%%%%%%%%%%%%%%%%%%%%%%%%%%%%%%%%%%%%%%%%%%%%

\SimPositivePertFig

%%%%%%%%%%%%%%%%%%%%%%%%%%%%%%%%%%%%%%%%%%%%%%%%%%%%%%%%%%%%%%%%%%%%%%%%%
\begin{ex}\exlabel{positive_pert_large}
%

%%%%%%%%%%%%%%%%%%%%%%%%%%%%%%%%%%%%%%%%%%%%%%%%%%%%%%%%%%%%%%%%%%%%%%%%%
%%%%%%%%%%%%%%%%%%%%%%%%%%%%%%%%%%%%%%%%%%%%%%%%%%%%%%%%%%%%%%%%%%%%%%%%%
% \begin{figure}[h!]
%
% \includegraphics[width=0.980\linewidth,height=0.980\linewidth]{static_images/positive_phi_example.png}
% %
% \caption{A plot of the perturbations from \exref{positive_pert_large}
% with $p=2$ and $\epsilon=0.05$.  Positive $\phi$ can only add mass, so to remove
% a small amount of mass requires adding mass everywhere else and re-normalizing,
% resulting in a large perturbation according to $\norm{\cdot}_p$.}
% %
% \figlabel{positive_pert_large}
% \centering
% \end{figure}
%%%%%%%%%%%%%%%%%%%%%%%%%%%%%%%%%%%%%%%%%%%%%%%%%%%%%%%%%%%%%%%%%%%%%%%%%

Let $\pbase(\nu) = \ind{0 \le \nu \le 1}$.  For some $\delta > 0$ and $0 <
\epsilon \ll 1$, let
%
\begin{align*}
%
\palt(\nu) :={}&
    \left(\frac{1-\delta \epsilon}{1 - \epsilon} \right)
        \ind{\epsilon \le \nu \le 1} +
    \delta \ind{0 \le \nu \le \epsilon}.
%
\end{align*}
%
% where the final approximation is due to the smallness of $\epsilon$.
Then (PHI FOR PALT) gives, for some $\alpha$,
%
\begin{align*}
%
\phi ={}&
    \left( \alpha\left(\frac{1-\delta \epsilon}{1-\epsilon} \right)^{1/p}
        - 1
    \right)
        \ind{\epsilon \le \nu \le 1} +
    \left(\alpha \delta^{1/p} - 1 \right) \ind{0 \le \nu \le \epsilon}.
%
%
\end{align*}
%
And
%
\begin{align*}
%
\norm{\phi}_p ={}&
    \left( \alpha\left(\frac{1-\delta \epsilon}{1-\epsilon} \right)^{1/p} - 1
    \right) (1- \epsilon) +
    \left(\alpha \delta^{1/p} - 1 \right) \epsilon.
%
\end{align*}
%
For $\phi$ to be positive, we require
%
\begin{align*}
%
\alpha^p \ge \frac{1 - \epsilon}{1 - \delta \epsilon}
    \mathtxt{and}
\alpha^p \ge \frac{1}{\delta}.
%
\end{align*}

First, let us consider adding a small amount of prior mass, taking $\delta = 2 -
\epsilon$; let the corresponding perturbation be $\phi^+$.  For $\delta > 1$,
then we achieve $\phi \ge 0$ by taking $\alpha^p = \frac{1 - \epsilon}{1 -
\delta \epsilon}$.  Using the fact that $\epsilon \ll 1$ and keeping only
leading-order terms,
%
\begin{align*}
%
\frac{1-\epsilon}{1 - \delta \epsilon} \approx{}&
    (1- \epsilon)(1 + \delta \epsilon)
\\\approx{}& 1 + (\delta - 1) \epsilon
\\\approx{}& 1 + \epsilon,
%
\end{align*}
%
so
%
\begin{align*}
%
\norm{\phi^+}_p  ={}&
    \left(\alpha \delta^{1/p} - 1 \right) \epsilon
\\\approx{}&
    \left(
        \left( \left(1 + \epsilon\right) \left(2 - \epsilon \right)\right)^{1/p}
        - 1 \right) \epsilon
\\\approx{}&
%
\left( 2^{1/p} - 1 \right) \epsilon.
%
\end{align*}
%

Next, consider removing the same amount of mass with the symmetric change
$\delta = \epsilon$, letting $\phi^-$ be the corresponding perturbation. Then we
can ensure that $\phi(\nu) \ge 0$ with $\alpha^p \ge \epsilon^{-1}$, and
$\epsilon \ll 1$ gives
%
\begin{align*}
%
\frac{1-\delta\epsilon}{1 - \epsilon} \approx{}& 1- \epsilon,
%
\end{align*}
%
and
%
\begin{align*}
%
\norm{\phi^-}_p  ={}&
    \left( \alpha\left(\frac{1-\delta \epsilon}{1-\epsilon} \right)^{1/p} - 1
    \right) (1- \epsilon)
\\\approx{}&
\left(\left(\frac{1- \epsilon}{\epsilon}  \right)^{1/p} - 1\right)(1 - \epsilon)
%
\\\approx{}&
    \left( \frac{1}{\epsilon}\right)^{1/p}.
%
\end{align*}

Since $\epsilon$ is small, $\norm{\phi^-}_p \approx \left(
\frac{1}{\epsilon}\right)^{1/p} \gg \norm{\phi^+}_p \approx \left( 2^{1/p} - 1
\right) \epsilon$, despite the two perturbations respectively removing and
adding the same amount of arbitrarily small probability mass.

\end{ex}
%%%%%%%%%%%%%%%%%%%%%%%%%%%%%%%%%%%%%%%%%%%%%%%%%%%%%%%%%%%%%%%%%%%%%%%%%


\section{Computational details}

\subsection{The optimal local parameters}
\applabel{gmm_global_local_vb}
In all models we consider,
the optimal local variational parameters $\etaoptlocal$ can be written
as a closed-form function of the global variational parameters $\etaglob$.
Let $\etaoptlocal(\eta_\gamma; \t)$ denote this mapping; that is,
\begin{align*}
  \etaoptlocal(\etaglob; \t) := \argmin_{\etalocal} \KL{(\eta_\gamma, \etalocal), \t}.
\end{align*}

The next example details this mapping for the Gaussian mixture model.

\begin{ex}[Optimalility of $\etalocal$ in a GMM]\exlabel{qz_optimality}
Recall that under our truncated variational approximation,
the cluster assignment $\z_\n$ is a discrete random variable
over $\kmax$ categories.

Let $\eta_{\z_\n}$ be the categorical parameters in its
exponential family natural parameterization.
That is, we let $\eta_{\z_\n} = (\rho_{\n1}, \rho_{\n2}, ..., \rho_{\n(\kmax-1)})$
be an unconstrained vector in $\mathbb{R}^{\kmax-1}$;
in this parameterization, the assignment probabilities are
%
\begin{align*}
  p_{\n\k} := \expect{\q(\z_\n \vert \etaz)}{\z_{\n\k}} =
  \frac{\exp(\rho_{\n\k})}{1 + \sum_{\k'=1}^{\kmax-1}\exp(\rho_{\n\k})}
\end{align*}
%
We use the exponential family parameterization because
we require the optimal variational parameters $\etaopt$
to be interior to $\etadom$ in \thmref{etat_deriv}.
In the mean parameterization,
$\sum_{\k=1}^\kmax p_{\n\k} = 1$, so the
optimal mean parameters $\hat p_{\n}$ cannot be
interior to $\Delta^{\kmax - 1}$.
On the other hand, $\eta_{\z_\n}$ as defined
is unconstrained in $\mathbb{R}^{\kmax - 1}$.

Fixing $\q(\beta\vert\etabeta)$ and $\q(\nu\vert\etanu)$,
the optimal $\etaopt_{\z_\n}$ must satisfy
%
\begin{align*}
& \q(\z_\n | \etaopt_{\z_\n}) \propto \exp\left(\tilde \rho_{\n\k}\right)\\
& \mathwhere \tilde \rho_{\n\k} := \expect{\q(\beta, \nu \vert \eta)}
       {\log\p(\x_n \vert \beta_\k) + \log \pi_\k}.
\end{align*}
%
See \citet{bishop:2006:PRML} and \citet{blei:2017:vi_review} for details.
To satisfy this optimality condition,
we set the optimal $\etaopt_{\z_\n}$ to be
%
\begin{align*}
%
\etaopt_{\z_\n} = \left(\log\frac{\tilde\rho_{\n1}}{\tilde\rho_{\n\kmax}},
\log\frac{\tilde\rho_{\n2}}{\tilde\rho_{\n\kmax}}, \ldots,
\log\frac{\tilde\rho_{\n(\kmax-1)}}{\tilde\rho_{\n\kmax}}\right).
%
\end{align*}
%
Thus, as long as the expectations $\tilde\rho_{\n\k}$ can be provided
as a closed-form function of
$(\etabeta, \etanu)$, the optimal $\etaopt_{\z_\n}$ can be also be set in closed-form as
a function of $(\etabeta, \etanu)$.
%
\end{ex}


\subsection{More details on computing and inverting the Hessian}
\applabel{more_hessian}
We fill in more details for the efficient computation of the Hessian outlined in
\secref{computing_sensitivity}.

We start from our formula in \eqref{global_local_derivative_breakdown}.
%
\begin{align*}
%
\fracat{d \etaopt(\t)}{d \t}{t = 0} ={}&
-\left(
\begin{array}{cc}
   \hess{\gamma\gamma} & \hess{\gamma\ell} \\
   \hess{\ell\gamma}     & \hess{\ell\ell} \\
\end{array}
\right)^{-1}
\left( \begin{array}{c} \crosshessian_\gamma \\ 0 \end{array}\right),
%
\end{align*}
%
and an application of the Schur complement gives
%
\begin{align*}
%
\fracat{d \etaopt(\t)}{d \t}{t = 0} ={}&
-\left(\begin{array}{c}
I_{\gamma\gamma} \\
\hess{\ell\ell}^{-1} \hess{\ell\gamma}
\end{array}\right)
\left(\hess{\gamma\gamma} -
      \hess{\gamma\ell} \hess{\ell\ell}^{-1} \hess{\ell\gamma}\right)^{-1} \crosshessian_\gamma,
%
\end{align*}
%
where $I_{\gamma\gamma}$ is the identity matrix with
the same dimension as $\eta_\gamma$.
%
Specifically, observe that the sensitivity of the global parameters
is given by
%
\begin{align*}
  \fracat{d \etaopt_\gamma(\t)}{d \t}{t = 0} &=
  - \hessopt_\gamma^{-1}\crosshessian_\gamma
  \mathwhere
  \hessopt_\gamma := \left(\hess{\gamma\gamma} -
        \hess{\gamma\ell} \hess{\ell\ell}^{-1} \hess{\ell\gamma}\right),
\end{align*}
%
In our model, $\hess{\ell\ell}$ is sparse, and the size of $\hess{\gamma\gamma}$
does not grow with $\N$. Thus, each term of $\hessopt_\gamma$ can be tractably
computed, stored in  memory, and inverted, even on very large datasets.

One can derive the exact same identity using the optimality of
$\etaoptlocal(\eta_\gamma)$.  By applying the chain rule, one can
verify that
%
\begin{align}\eqlabel{global_kl_hess}
\hessopt_{\gamma} &=
    \frac{\partial^2}{\partial\eta_\gamma\partial\eta_\gamma^T}
    \KLglobal(\etaopt_\gamma, 0).
\end{align}
%
In practice, we evaluate $\hessopt_\gamma$ using automatic differentiation and
\eqref{global_kl_hess} rather than the Schur complement.


\subsection{Expressing $\g$ using global parameters only}
\applabel{vb_insample_nclusters_example}

Given a posterior quantity $\g$,
we again take advantage of the fact that the optimal
local parameters can be found in closed form given global parameters.
In general, $\g$ will be a function of the entire vector of variational parameters.
However, in the same way that $\KLglobal$ implicitly sets the local parameters at their optimum
and is a function of only global parameters and the prior parameter $\t$,
we can construct an analogous mapping for $\g$,
\begin{align}\eqlabel{g_as_global}
(\t, \etaglob) \mapsto g\Big(\big(\etaglob, \etaoptz(\etaglob, \t))\Big).
\end{align}

We illustrate this mapping
when our quantity of interest is the in-sample expected posterior number of clusters.

\begin{ex}\exlabel{vb_insample_nclusters_globallocal}
%
Let
$\gclustersabbr(\eta)$ denote our variational approximation to
$\expect{\p(\z\vert\x)}{\nclusters(\z)}$.   Using the fact that
$\p(\z_\n\vert \beta, \nu, \x) = \q(\z_\n \vert \etaopt_{\z_\n})$
is available in closed form, we can then take
%
\begin{align*}
%
\gclustersabbr(\etaopt) :={}&
    \expect{\q(\beta, \nu \vert\etaopt)}{
        \expect{\p(\z \vert \beta, \nu, \x)}{\nclusters(\z)}
    }
\\\approx{}&
    \expect{\p(\beta, \nu \vert \x)}{
        \expect{\p(\z \vert \beta, \nu, \x)}{\nclusters(\z)}
    }
    = \expect{\p(\z\vert\x)}{\nclusters(\z)} \Rightarrow \\
%
\gclustersabbr(\eta) ={}&
    \sumkm \left(1 -  \prod_{\n=1}^\N
        \left(1 - \expect{\q(\beta, \nu \vert \eta_\beta, \eta_\nu)}
                    {\expect{\p(\z_{\n} \vert \beta, \nu, \x)}{\z_{\n\k}}}
                    \right)\right).
%
\end{align*}
%
In this way, $\gclustersabbr(\eta)$ depends only on $\eta_\beta$ and $\eta_\nu$,
which are much lower-dimensional than $\eta_\z$, and retains nonlinearities in
the map
%
\begin{align*}
%
\eta_\beta, \eta_\nu \mapsto \expect{\q(\beta, \nu \vert \eta_\beta,
\eta_\nu)} {\expect{\p(\z_{\n} \vert \beta, \nu, \x)}{\z_{\n\k}}}.
%
\end{align*}
%
\end{ex}


The mapping \eqref{g_as_global} can be constructed for any posterior quantity $\g$.
Therefore, linearizing the global parameters using \eqref{global_sens, global_lin_approx} is sufficient:
we do not need to invert the full Hessian
and linearize the entire set of variational parameters, global and local.


\subsection{Evaluating stick expectations}\applabel{gh_quadrature}
We describe how to compute expectations with repsect to the stick-breaking
proportion $\nu_\k$. Let $f: \mathbb{R}\mapsto\mathbb{R}$ be a smooth function,
and we are interested in expectations of the form
\begin{align*}
  \expect{\q(\nuk \vert \eta)}{f(\nuk)}.
\end{align*}
For example, $f$ might be $f(\nu_\k) = \log \p(\nu_\k)$, whose
expectation appears in the $\mathrm{KL}$ divergence.

Recall that we chose the distribution on the logit-transformed
stick-breaking proportions $\lnu_\k$ to be normally distributed.
Let $\lnumean_\k$ and $\lnusd_\k$ be the location and scale, respectively,
of the Gaussian distribution on $\lnu_\k$.
Also let $\s$ be the sigmoid function, so that $\nu_\k = \s(\lnu_\k)$.

To compute expectations of a smooth function
$f(\nuk)$, the law of the unconscious statistician states that,
\begin{align*}
  \expect{\q(\nuk \vert \eta)}{f(\nuk)} ={}&
  \expect{\q(\lnu_\k \vert \eta)}
         {f\circ \s\left(\lnu_\k\right)}.
\end{align*}
By choosing $\q(\lnu_\k \vert \eta)$ to be Gaussian,
the right-hand side of is a Gaussian integral,
which we approximate
using GH quadrature with $\ngh$ knots,
located at $\xi_g$, weighted by $\omega_g$:
%
\begin{align}\eqlabel{gh_integral}
%
\expect{\q(\lnu_\k \vert \eta)}
       {f\circ \s\left(\lnu_\k\right)}
\approx{}&
    \sum_{g=1}^{\ngh} \omega_g f\circ \s \left(\lnusd_\k \xi_{g} + \lnumean_\k\right)
 \nonumber\\=:{}&
\expecthat{\q(\nuk \vert \eta)}{f(\nuk)}.
%
\end{align}
%
Using GH quadrature to approximate the expectation
is similar to the ``reparameterization trick,'' only using
GH points rather than standard normal draws.


\subsection{Unconstrained variational parameterizations}
\applabel{app_vb_unconstrained}
Recall from \thmref{etat_deriv} that we require the optimal variational parameters
$\etaopt$ to be in the interior of its domain. One way to achieve this
is to use only \textit{unconstrained} parameterizations for the
component distributions of $\q$. One such parameterization was
presented in \exref{qz_optimality}, where we let
$\eta_{\z_\n}$, which parameterize the cluster assignments, be allowed to
take any value in $\mathbb{R}^{\kmax - 1}$; the assignment probabilities
$m_{\n}\in\mathbb{R}^{\kmax}$, which are constrained to sum to one,
are then formed with an appropriate transform of the unconstrained parameters
$\eta_{\z_\n}$

Other variables require careful parameterization as well.
For instance, instead of parameterizing the normal distribution on
logit-sticks $\lnu_\k$ using a mean and variance, we let $\eta_{\nu_\k}\in\mathbb{R}^2$
be the mean and \textit{log} variance. The variance is constrained to be positive;
the log-variance is unconstrained on the real line.
In general, a real-valued parameter $\mu_i$ which must be constrained
$a \leq \mu_i \leq b$ can be transformed to its unconstrained parameterization
by letting
\begin{align*}
  \eta_i = \log(\mu_i - a) - \log(b - \mu_i).
\end{align*}

In the variational approximation to the GMM model,
we let the component variables $\beta_\k$ be Normal-Wishart.
In this case, the scale matrix of the Normal-Wishart, $W_\k\in\mathbb{R}^{d\times d}$,
is constrained be positive definite.
Because $W$ is symmetric, we only neeed $d(d + 1) / 2$ parameters to represent it.
To form an unconstrained parameterization, we factorize $W$ using the Cholesky decomposition,
\begin{align*}
W = L^T L,
\end{align*}
where $L$ is a lower-triangular matrix, with positive diagonal entries.
The unconstrained parameterization of $W$ is then taken to be
the strictly lower-diagonal entries of $L$,
along with the $\log$ of the diagonal entries of $L$.


\section{Additional experiment details}
\applabel{app_results}

First, we review the Beta prior on the stick-breaking proportions,
which are common to each model we considered.
Then, we give some additional modeling details for each experiment.

    \subsection{The Beta prior}
    \applabel{app_beta_prior}
    In the iris experiment, we considered $\alpha\in[0.1, 4.0]$.
Over this range, the shape of
the $\betadist{1, \alpha}$ stick-breaking density varies considerably, as shown in
\figref{beta_priors}.

\BetaPriorsEx

To help us understand the effect of the concentration parameter
$\alpha$ we often use the following fact. Under the $\gem$ prior, the {\em a priori} expected
number of distinct clusters in a dataset of size $N$ is given by
%
\begin{align}\eqlabel{prior_num_clusters}
\expect{\p(\z \vert \pi)\p(\pi \vert \alpha)}{\nclusters(\z)} =
\sum_{\n = 1}^\N \frac{\alpha}{\alpha + \n - 1}.
\end{align}
%
See \citep[Equation 75]{jordan:2015:gentleintrodp}. \textcolor{red}{There's a much, much earlier citation for this. E.g.\ the original Blackwell MacQueen paper?}


    \subsection{Gaussian mixture modeling on iris data}
    \applabel{app_iris}
    The observations are vectors $\x_\n \in \mathbb{R}^\d$, and we model each
component with a multivariate Gaussian. In this model, $\beta_\k = (\mu_k,
\Lambda_\k)$, where $\mu_\k \in \mathbb{R}^\d$, $\Lambda_\k$ is a $\d\times\d$
positive definite information matrix, and
%
\begin{align*}
%
\p(\x_\n \vert \beta_\k) ={}& \normdist{\x_n \vert \mu_\k, \Lambda_\k^{-1}} \\
\log\p(\x_\n \vert \beta_\k) ={}&
    -\frac{1}{2}(\x_n - \mu_k)^T \Lambda_\k (\x_n - \mu_k)
    + \frac{1}{2} \log |\Lambda_\k| + \const.\\
    & \constdesc{\beta_\k}
%
\end{align*}
We let $\pbetaprior(\beta_\k)$ be the conjugate prior, which in this case is normal-Wishart:
\begin{align*}
  \pbetaprior(\beta_\k) &= \normalwishart{\beta_\k \vert \tau_0, n_0, p_0, V_0}\\
  \log\pbetaprior(\beta_\k) &=
      -\frac{\tau_0}{2}(\mu_\k - \mu_0)^T \Lambda_\k (\mu_\k - \mu_0)\\
      &{} + \frac{n_0 - p_0 - 1}{2} \log |\Lambda_\k| -
      \frac{1}{2} \textrm{Tr}(V_0 \Lambda_\k) + \const,
\end{align*}
where $(\tau_0, n_0, p_0, V_0)$ are fixed prior parameters.
%

In this model, the conditionally conjugate variational distribution on $\beta_\k$ is
normal-Wishart, which we denote as
$\q(\beta_\k \vert \eta) = \normalwishart{\beta_k \vert \eta_{\beta_\k}}$,
with $\eta_{\beta_\k}$ the normal-Wishart parameters.
The conditionally conjugate variational distribution on $\z$ are multinomial.

\Figref{iris_fit} shows the
inferred clustering for $\alpha_0 = 2$,
which recovers that there are three iris species.

\IrisInitFit


    \subsection{Regression mixture modeling}
    \applabel{app_mice}
    \subsection{The data}
The data come from a publicly available data set of mice gene expression
\citep{shoemaker:2015:ultrasensitive}.
Our analysis focuses on mice treated with the ``A/California/04/2009'' strain.
We normalize the data as described in
\citet{shoemaker:2015:ultrasensitive} and then apply the differential
analysis tool EDGE \citep{Storey:2005:significance} to rank the genes from most to least significantly differentially expressed.
We run our analysis on the top $\ngenes = 1000$ genes.

The left plot of \figref{example_genes}
shows the measurements of a single gene over time.
We model each gene as belonging to a latent component,
where each component defines a smooth expression curve over time.
Then, observations are drawn by adding i.i.d.\ noise to the smoothed
curve along with a gene-specific offset.

\MiceExampleGenes

\subsection{The B-spline basis}
Notice from \figref{example_genes}, which shows an example time-course for a single gene,
that the time points are unevenly spaced, with more frequent observations at the beginning.
Following \citet{Luan:2003:clustering} we use cubic B-splines to smooth the time course expression data.
Specifically, we model the first 11 time points using
cubic B-splines with 7 degrees of freedom.
For the last three time points, $\timeindx = 72, 120, 168$ hours,
we use indicator functions.
That is, if $\tilde \regmatrix$ is the design
matrix where each column is a
B-spline basis vector evaluated at the $\ntimepoints$ measurement times,
we append to $\tilde \regmatrix$ three additional columns:
in these columns, entries are 1
if $\timeindx = 72, 120,$ or 168, receptively, and 0 otherwise.
The resulting matrix is the full design matrix $\regmatrix$.
We use indicators for the last three time points for numerical stability;
without the indicator columns,
the matrix $\tilde \regmatrix^T \tilde \regmatrix$ is nearly singular
because the later time points are more spread out.
The left column of \figref{example_genes} shows our basis functions.

\subsection{The generative model}
\eqref{mice_model} gives the per-component conditional likelihood.
We use a normal prior for the shifts $\b_\n$,
a multivariate normal prior for the coefficients $\mu_\k$,
and a gamma prior for the inverse variance $\tau_\k$.
The prior on the mixture weights $\pi$ are constructed using the stick-breaking
construction in the main-text, and the cluster assignments $\z_\n$
are drawn from a multinomial with wieghts $\pi$, as usual.



\subsection{The variational approximation}
The variational approximation, factorizes as
\begin{align*}
\q(\zeta \vert \eta) =
    \left( \prod_{\k=1}^{\kmax - 1} \q(\nuk \vert \eta) \right)
    \left( \prod_{\k=1}^{\kmax} \q(\beta_\k \vert \eta) \right)
    \left( \prod_{\n=1}^{\N} \q(\z_{\n} \vert \eta)
    \q(\b_{\n} \vert \z_{\n}, \eta)\right).
\end{align*}
Note that the variational distribution for $\b_\n$ conditions on $\z$.
We set $\q(\b_{\n} \vert \z_{\n} = k, \eta)$ to be Gaussian
with variational parameters dependent on $\k$.
For simplicity in this application,
we let $\q(\beta_\k \vert \eta) = \delta (\beta_k \vert \eta)$,
where $\delta(\cdot \vert \eta)$ denotes a point mass at a parameterized location.

As discussed in \exref{qz_optimality},
the optimal distribution $\q(\z_\n\vert\eta)$ is multinomial whose parameters
can be set in closed form as a function of the global variational parameters only.
We allow the distribution of $\b_\n$ to depend on $\z_{\n\k}$ so that
the its optimal distribution can also be set in closed form as a function of
global parameters.

The optimal distribution $q(\b_\n\vert \z_{\n\k} = 1, \eta)$ is Gaussian,
\begin{align*}
q(\b_\n\vert \z_{\n\k} = 1, \eta) = \normdist{\b_\n \vert \hat\mu_{\b_{\n\k}}, \hat\sigma^2_{\b_{\n\k}}}.
\end{align*}
To define the optimal parameters $\hat\mu_{\b_{\n\k}}, \hat\sigma^2_{\b_{\n\k}}$, let
\begin{align*}
  \rho^{(1)}_{\n\k} &= \expect{\q(\beta_k|\eta)}{\sum_{m=1}^\ntimepoints \tau_{k}(x_{nm} - \regmatrix_m\mu_\k)} +
  \tau_0 \mu_0 \\
  \rho^{(2)}_{\n\k} &= \ntimepoints \expect{\q(\beta_k|\eta)}{\tau_{k}} + \tau_0,
\end{align*}
where $\mu_0$ and $\tau_0$ are the prior mean and information on $\b_n$, respectively.

The optimal parameters for the Gaussian distribution on $\b_\n$ are given by
\begin{align*}
  \hat\mu_{\b_{\n\k}} &= \rho^{(1)}_{\n\k} / \rho^{(2)}_{\n\k}\\
  \hat\sigma^2_{\b_{\n\k}} &= 1 / \rho^{(2)}_{\n\k}.
\end{align*}

\figref{gene_centroids} shows the inferred smoothers
$\regmatrix \expect{\q}{\mu_\k}$ for selected clusters.
\MiceSmoothers

% with parameters $\hat\mu_{\b_{\n\k}}, \hat\mu_{\b_{\n\k}}:


    \subsection{fastSTRUCTURE}
    \applabel{app_structure}
    The generative process was described in the main text (\secref{results_structure}).
We detail here the variational approximation.
Like in all our examples, the variational distribution is mean-field:
\begin{align*}
\q(\zeta \vert \eta) =
    \left(
    \prod_{\n=1}^{\nindiv}\prod_{\k=1}^{\kmax - 1}
    \q(\nu_{nk} \vert \eta) \right)
    \left(\prod_{\k=1}^{\kmax}\prod_{l=1}^{\nloci}
    \q(\latentpop_{\k l} \vert \eta) \right)
    \left( \prod_{\n=1}^{\N} \prod_{l=1}^{\nloci} \prod_{i=1}^{2} \q(\z_{\n l i} \vert \eta) \right).
\end{align*}
We let all distributions be conditionally conjugate except for the sticks,
which are logit-normal.
Each membership indicator $\z_{\n l i}$ is categorical, and the
allele frequencies $\latentpop_{\k l}$ are Dirichlet distributed.

In this model, we still call $(\beta, \nu)$ the global latent variables, even though they scale
with the number of individuals $\N$;
they do not, however, scale with both the number of individuals and the number of loci
like $\z$ does. Thus, we call $\z$ the local latent variables.
The local variational parameters $\eta_\z$ can be set optimally in
an analagous way as \exref{qz_optimality}, except with the
indices $\n\k$ replaced with $\n\l\i\k$.

The posterior quantities of interest in this application are the admixtures
$\pi_\n$. \figref{stru_init_fit} plots the inferred admixtures
$\expect{\q(\pi_\n \vert \etaopt)}{\pi_\n}$ for all individuals $\n$.

In the approximate posterior with $\alpha_0 = 3$, there appear to be three dominant
latent populations, which we arbitrarily label as populations 1, 2, and 3
(\figref{stru_init_fit}). The inferred admixture proportions generally
correspond with geographic regions: Mbololo individuals are primarily population
1; Ngangao individuals are primarily population 2; and Chawia individuals are a
mixture of populations 1, 2, and 3.

\StructureInitialFit



\section{fastSTRUCTURE supplemental results}\applabel{app_structure_results}
Recall from \secref{results_structure} and \figref{stru_func_sens_admix}
that the linear approximation
failed to capture the change in the admixture proportion of an individual,
$n = 25$ after a worst-case functional perturbation.

\figref{stru_lin_bad_example} examines individual $n = 25$ more closely.
The bottom row plots this individual's
admixture proportions as $\t$ varies from 0 to 1 in the perturbed prior
$\p(\nu\vert \t) = \p_0(\nuk)\exp(\t\phiworstcase(\nuk))$.
The linearized parameters poorly captured the change in admixture proportions observed after refitting, particularly
for populations 1 and 2, for values of $\t$ close to 1.
Even though we retain non-linearities
in the mapping from variational parameters to the posterior statistic,
for this perturbation, the mapping from prior parameter
$\t$ to the relevant variational parameters
is highly non-linear.
This latter mapping is what we linearize
and what causes our approximation to fail in this case.
Specifically, the variational location parameter on the first stick-breaking proportion is concave as a function of $\t$ ---
the location parameter increases for small $\t$,
then decreases as $\t\rightarrow1$.
However,
$\etalin(\t)$ linearizes the relationship between the location parameter and $\t$.
Therefore, the corresponding admixture mixture proportion of
population 1 is over-estimated under the linearized variational parameters.
Furthermore, because our linearized variational parameters
over-estimated the length of the first stick,
and the second admixture proportion is a product of the
remaining stick times the second stick-breaking proportion,
the linearized variational parameters then under-estimates
the admixture proportion of population 2.

\StructureLimitationsA

\figref{stru_fully_lin_example} shows a similar situation for individual $n = 74$.
The linearized variational parameters grossly over-estimated the length of the first stick,
resulting in the later admixture proportions being under-estimated.
The third admixture proportion was particularly poorly approximated under the linearized variational parameters.
Given the recursive nature of the relationship between admixtures and stick-breaking proportions, errors at early sticks affect later admixture proportions.
Fully linearizing the mapping $\t\mapsto\g(\etaopt(\t))$ to form the approximation
$\glin(t)$ avoids this problem.
In this example, $\glin(t)$ outperforms $\g(\etalin(t))$, with $\g$ being the admixture proportion of population 3.
In our experience,
computing $\g(\etalin(t))$, and thus retaining non-linearities in the mapping from $\eta\mapsto\g(\eta)$,
is usually beneficial to the quality of the approximation.
It is likely that $\g(\etalin(t))$ outperforms $\glin(\t)$ for most posterior quantities,
though as we see in \figref{stru_fully_lin_example},
this is not guaranteed to always be true.

\StructureLimitationsB





\end{document}
