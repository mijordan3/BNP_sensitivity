We now consider two classes of prior perturbations: parametric and
non-parametric.  Recall from \eqref{sens_mixed_partial} that we
need to specify a map $\t \mapsto \log \pstick(\nuk \vert \t)$.

First, consider the parametric case where we take $\t = \alpha$ and
%
\begin{align*}
%
\pstick(\nuk \vert \alpha) ={}&
    \betadist{\nuk \vert 1, \alpha} \Rightarrow\\
\log \pstick(\nuk \vert \alpha) ={}&
    (\alpha - 1) \log(1 - \nuk) + \const. &
    \constdesc{\nuk}
%
\end{align*}
%
In this case,
%
\begin{align*}
%
\expect{\q(\nuk \vert \eta)}
       {\fracat{\log \pstick(\nuk \vert \alpha)}{\partial \alpha}{\alpha_0}
} ={}&
    \expect{\q(\nuk \vert \eta)}{\log(1 - \nuk)}.
%
\end{align*}
%
We can thus numerically compute the needed derivatives.

To consider more general prior perturbations, fix for the moment a base stick
distribution, $\pb(\nuk)$ and an alternative stick distribution
$\pa(\nuk)$.  Take $\t = \epsilon$ and define
%
\begin{align*}
%
\pstick(\nuk \vert \epsilon) ={}&
\frac{\pb(\nuk)^{1 - \epsilon} \pa(\nuk)^\epsilon}
     {\int_0^1 \pb(\nuk')^{1 - \epsilon} \pa(\nuk')^\epsilon d\nuk'}.
%
\end{align*}
%
We thus have that $\epsilon$ parameterizes a multiplicative path from
$\pb(\nuk)$ to $\pa(\nuk)$, with $\pstick(\nuk \vert \epsilon = 0) = \pb(\nuk)$
and $\pstick(\nuk \vert \epsilon=1) = \pa(\nuk)$.  We will take
$\t_0$ to be $\epsilon = 0$, so that we are computing $\etaopt$ with the
prior $\pb(\nuk)$ and approximating the optimum if we had used $\pa(\nuk)$.
%
Given this definition,
%
\begin{align*}
%
\fracat{\partial \log \pstick(\nuk \vert \epsilon) }{\partial \epsilon}{0} ={}&
    \log \frac{\pa(\nuk)}{\pb(\nuk)} + \const.
    & \constdesc{\nuk}
%
\end{align*}
%
Again, for a fixed $\pa(\nuk)$ we can compute the needed derivatives.

There are many (an infinite number!) of $\pa(\nuk)$ to choose from, and so it
can be useful to think of $\etaopt(\epsilon)$ as functional of the stick
breaking prior, following \citet{gustafson:1996:local}.  Let us fix $\pb(\nuk)$,
define $\phi(\nuk) := \epsilon \log \left(\pa(\nuk) / \pb(\nuk)\right)$, and
re-write $\pstick(\nuk \vert \epsilon)$ in the following equivalent form:
%
\begin{align}\eqlabel{phi_perturbation}
%
\pstick(\nuk \vert \phi) ={}&
\frac{\exp\left(\log \pb(\nuk) + \phi(\nuk)\right)}
     {\int_0^1 \exp\left(\log \pb(\nuk') + \phi(\nuk')\right) d\nuk'}.
%
\end{align}
%
The advantage of \eqref{phi_perturbation} is that $\pstick(\nuk \vert \phi)$ is
well-defined for any Lebesgue-integrable function $\phi(\nuk):
(0,1)\mapsto\mathbb{R}$, and a valid distribution for any $\phi$ such that the
denominator is positive and finite.  In fact, let us define the norm
%
\begin{align*}
%
\norminf{\phi} :={} \esssup_{\nuk} |\phi(\nuk)|,
%
\end{align*}
%
where $\esssup$ is defined with respect to the Lebesgue measure on $[0,1]$.
Then $\pstick(\nuk \vert \phi)$ is a valid distribution whenever $\phi$ lies
within a neighborhood of $\phiz$, as we state in the following proposition.

%%%%%%%%%%%%%%%%%%%%%%%%%%%%%%%%%%%%%%%%%%%%%%%%%%%%%%%%%%%%%%%%%%%%%%%%%%%%%%
%%%%%%%%%%%%%%%%%%%%%%%%%%%%%%%%%%%%%%%%%%%%%%%%%%%%%%%%%%%%%%%%%%%%%%%%%%%%%%
\begin{prop}\proplabel{pert_well_defined}
%
Let $\linf$ denote the Banach space of functions on $[0,1]$ with the norm
$\norminf{\cdot}$ (as in, e.g., \citet[Section 2.1, Example
5]{luenberger:1997:optimization} or \citet[Theorem 5.2.1]{dudley:2018:real}).
There exists a $\norminf{\cdot}$-neighborhood of $\phiz$, denoted $\ball_\phi
\subset \linf$, such that $\pstick(\nuk \vert \phi)$ as defined in
\eqref{phi_perturbation} is a valid probability density for all $\phi \in
\ball_\phi$\footnote{As pointed out by \citet{gustafson:1996:local}, the
converse of \propref{pert_well_defined} not true---there exist valid
distributions with infinte $\norminf{\phi}$.  We will discuss this and other
issues related to \citet{gustafson:1996:local} below.}.
%
\begin{proof}
%
The function $\pstick(\nuk \vert \phi)$ is a valid density whenever
$\pstick(\nuk \vert \phi) \ge 0$ with Lebesgue probability one, and $\int
\pstick(\nuk \vert \phi) = 1$. Since $\pb(\nuk) \ge 0$,
%
\begin{align*}
%
\essinf_{\nuk \in [0,1]} \exp\left(\log \pb(\nuk) + \phi(\nuk)\right)
    \ge{}&
\essinf_{\nuk \in [0,1]} \pb(\nuk)
\essinf_{\nuk \in [0,1]} \exp\left(\phi(\nuk)\right)
\ge{}0,
%
\end{align*}
%
so $\pstick(\nuk \vert \phi) \ge 0$ with Lebesgue probability one. Similarly,
since $\int \pb(\nuk) d\nuk = 1$,
%
\begin{align*}
%
\exp(-\norminf{\phi}) \le{}
\abs{\int_0^1 \exp\left(\log \pb(\nuk) + \phi(\nuk)\right) d\nuk}
\le{}
\exp(\norminf{\phi}).
%
\end{align*}
%
so that $\int \pstick(\nuk \vert \phi) = 1$ whenever $\norminf{\phi} < \infty$.
%
\end{proof}
%
\end{prop}
%%%%%%%%%%%%%%%%%%%%%%%%%%%%%%%%%%%%%%%%%%%%%%%%%%%%%%%%%%%%%%%%%%%%%%%%%%%%%%

Thanks to \propref{pert_well_defined}, we might expect to sensibly write
$\etaopt(\phi) := \argmin_{\eta \in \etadom} \KL{\eta, \phi}$ for the optimal
variational parameters when using the prior $\pstick(\nuk \vert \phi)$.  The
optimum $\etaopt(\phi): \linf \mapsto \mathbb{R}^\etadim$ is thus a functional
of $\phi$.  We might hope to form a Taylor series expansion of $\etaopt(\phi)$.

%
% %%%%%%%%%%%%%%%%%%%%%%%%%%%%%%%%%%%%%%%%%%%%%%%%%%%%%%%%%%%%%%%%%%%%%%%%%%%%
% %%%%%%%%%%%%%%%%%%%%%%%%%%%%%%%%%%%%%%%%%%%%%%%%%%%%%%%%%%%%%%%%%%%%%%%%%%%%
% \begin{assu}\assulabel{kl_stick_opt_ok}
% %
% Let the following assumptions hold for some neighborhood $\ball_\eta$
% of $\etaopt(\phiz)$, and for $\eta \in \ball_\eta$:
% %
% %%%%%%%%%%%%%%%%%%%%%%%%%%%%%%%%%%%%%%%%%%%%%%%%%%%%%%%%%%%%%%%%%%%%%%
% \begin{enumerate}
% %
%     \item \itemlabel{kl_diffable}
%     The map $\eta \mapsto \KL{\eta, \phiz}$ is twice
%     continuously differentiable.
% %
% %%%%%%%%%%%%%%%%%%%%%%%%%%%%%%%%%%%%%%%%%%%%%%%%%%%%%%%%%%%%%%%%%%%%%%
% \item\itemlabel{kl_hess}
%     The Hessian matrix $\fracat{\partial^2 \KL{\eta, \phiz}}
%                     {\partial \eta \partial \eta^T}
%                     {\eta}$ is positive definite.%
% %%%%%%%%%%%%%%%%%%%%%%%%%%%%%%%%%%%%%%%%%%%%%%%%%%%%%%%%%%%%%%%%%%%%%%
% \item\itemlabel{q_stick_regular} The variational approximations $\q(\nu
% \vert \etanuk)$ to the stick-breaking posteriors satisfy \assuref{q_regular}
% with $\eta_0 = \etaopt(\phiz)$.
% %
% \end{enumerate}
% %%%%%%%%%%%%%%%%%%%%%%%%%%%%%%%%%%%%%%%%%%%%%%%%%%%%%%%%%%%%%%%%%%%%%%
% %
% \end{assu}
% %%%%%%%%%%%%%%%%%%%%%%%%%%%%%%%%%%%%%%%%%%%%%%%%%%%%%%%%%%%%%%%%%%%%%%%%%%%%

%%%%%%%%%%%%%%%%%%%%%%%%%%%%%%%%%%%%%%%%%%%%%%%%%%%%%%%%%%%%%%%%%%%%%%%%%%%%
%%%%%%%%%%%%%%%%%%%%%%%%%%%%%%%%%%%%%%%%%%%%%%%%%%%%%%%%%%%%%%%%%%%%%%%%%%%%
\begin{thm}
%
Let \assuref{kl_opt_ok} hold. Then the map $\phi \mapsto \etaopt(\phi)$ is
well-defined and continuously Fr{\'e}chet differentiable in a neighborhood of
$\phiz$ as a map from $\linf$ to $\mathbb{R}^\etadim$. Furthermore, the
derivative in the direction $\phi - \phiz$ is of the formfassuref
%
\begin{align*}
%
\fracat{d \etaopt(\phi)}{d\phi}{\phiz}\left( \phi - \phiz\right) ={}&
    \int \infl(\nu) \phi(\nu) d\nu \mathtxt{where} \\
%
\infl(\nu) :={}&
-\left(\fracat{\partial^2 \KL{\eta, \phiz}}
                {\partial \eta \partial \eta^T}
                {\etaopt}\right)^{-1}
\left(
    \sumkm
    \q(\nu \vert \etaoptnuk)
    \nabla \log \q(\nu \vert \etaoptnuk)
\right).
%
\end{align*}
%

%
%%%%%%%%%%%%%%%%%%%%%%%%%%%%%%%%%%%%%%%%%%%%%%%%%%%%%%%%%%%%%%%%%%%%%%
%%%%%%%%%%%%%%%%%%%%%%%%%%%%%%%%%%%%%%%%%%%%%%%%%%%%%%%%%%%%%%%%%%%%%%

\begin{proof}
%
The map $\eta, \phi \mapsto \KL{\eta, \phi}$ is a map from the Banach space
$\mathbb{R}^\etadim \times \linf$ into the Banach space $\mathbb{R}$. Let us
take the L2 norm $\norm{\cdot}_2$ on $\mathbb{R}^{\etadim}$ and $\mathbb{R}$.
Let $\ball$ denote the ball $\ball_\eta \times \{ \phi: \norminf{\phi} <
\delta\}$ for some $\delta > 0$.  Let $\linop$ denote a linear operator from
$\ball$ to $\mathbb{R}^\etadim$, and define the dual norm
%
\begin{align*}
%
\norm{\linop}^* :=
    \sup_{\eta: \norm{\eta}_2 \le 1} \sup_{\phi: \norminf{\phi} \le 1}
     \norm{\linop(\eta, \phi)}_2.
%
\end{align*}

Recall that $\etaopt(\phi)$, if it exists, is defined as a solution to
%
\begin{align*}
%
\fracat{\partial \KL{\eta, \phi}}
                {\partial \eta}
                {\etaopt(\phi)} = 0_\etadim.
%\KLgrad{\etaopt(\phi), \phi} ={}& 0_\etadim.
%
\end{align*}
%
Using this estimating equation, the key to the proof will be the implicit
function theorem for Banach spaces. To use the implicit function theorem, we
need to show that $\KLgrad{\eta, \phi}$ is continuously Fr{\'e}chet
differentiable at $\etaopt(\phiz), \phiz$ and that $\KLhess{\eta, \phi}$ is
positive definite in $\ball$.

Let us define some compact notation for the duration of the proof.
Let
%
\begin{align*}
%
\KLgrad{\eta, \phi} :={}&
    \fracat{\partial \KL{\eta, \phi}}{\partial \eta}{\eta, \phi}
\mathtxt{and}
\KLhess{\eta, \phi} :={}&
    \fracat{\partial^2 \KL{\eta, \phi}}
           {\partial \eta \partial \eta^T}{\eta, \phi}.
%
\end{align*}
%
Additionally, define $\rho(\etanuk, \phi) :={} \expect{\q(\nu \vert
\etanuk)}{\phi(\nu)}$, with
%
\begin{align*}
%
\rho_\eta(\etanuk, \phi) :={}
    % \fracat{\partial \expect{\q(\nu \vert \etanuk)}{\phi(\nu)}}
    %        {\partial \etanuk}{\etanuk} \mathand
\fracat{\partial \rho(\etanuk, \phi)}
       {\partial \etanuk}{\etanuk} \mathand
\rho_{\eta\eta}(\etanuk, \phi) :={}
   \fracat{\partial^2 \rho(\etanuk, \phi)}
          {\partial \etanuk \partial \etanuk^T}{\etanuk}.
%
\end{align*}

Observe that the estimating equation can be written as
%
\begin{align*}
%
% \KL{\eta, \phi} ={}&
%     \KL{\eta, \phiz} + \sumkm \rho(\etanuk, \phi)
%     \Rightarrow \\
%
\KLgrad{\eta, \phi} ={}&
\KLgrad{\eta, \phiz} + \sumkm \rho_\eta(\etanuk, \phi)
= 0_\etadim.
%
\end{align*}

Let us first show that $\KLgrad{\eta, \phi}$ is continuously Fr{\'e}chet
differentiable.  By \assuitemref{kl_opt_ok}{kl_diffable}, $\eta \mapsto
\KLgrad{\eta, \phiz}$ is continuously differentiable and does not depend on
$\phi$, so we need only to show that $\rho_\eta(\etanuk, \phi)$ is continuously
Fr{\'e}chet differentiable. To show this, it will suffice to show that the
partial deritatives of the maps $\eta \mapsto \rho_\eta{\eta, \phi}$ and $\phi
\mapsto \rho_\eta{\eta, \phi}$ exist and are continuous in the dual norm
$\norm{\cdot}^*$ as a function of the location $\eta, \phi$ at which the
derivatives are evaluated.

We first show that $\phi \mapsto \rho_\eta(\etanuk, \phi)$ is Fr{\'e}chet
differentiable.  To do so it


It follows from \eqref{q_sens_is_cov} that the map $\phi \mapsto
\rho_\eta(\etanuk, \phi)$ is a linear functional of $\phi$.  Applying Jensen's
inequality to the $\norm{\cdot}_2$ norm and the triangle inequality twice gives:
%
\begin{align*}
%
%\sup_{\phi: \norminf{\phi} = 1}
\norm{\rho_\eta(\etanuk, \phi)}_2 \le
    4 \expect{\q(\nu \vert \etanuk)}
             {\norm{\nabla \log \q(\nu \vert \etanuk)}_2}
              \norminf{\phi(\nu)}.
%
\end{align*}
%
It follows from ... that $\rho_\eta(\etanuk,
\phi)$ is bounded as a function of $\phi$ for all $\eta \in \ball_\eta$.
Bounded linear functionals are Fr{\'e}chet differentiable (essentially by
definition), so $\phi \mapsto \rho_\eta(\etanuk, \phi)$ is Fr{\'e}chet
differentiable.

It remains to show that the derivative is continuous.

% so the linear operator is bounded as a function of $\phi$, and a continuous
% funciton of $\eta$ by assumption.  Therefore $\phi \mapsto \fracat{\partial
% \expect{\q(\nu \vert \etanuk)} {\phi(\nu)}}{\partial \eta}{\eta}$ is
% continuously Fr{\'e}chet differentiable.
%
% Again differentiating under the integral,
% %
% \begin{align*}
% %
% \fracat{\partial^2 \expect{\q(\nu \vert \etanuk)}
%               {\phi(\nu)}}{\partial \eta \partial \eta^2}{\eta} ={}&
% \expect{\q(\nu \vert \etanuk)}
%        {\left(\nabla \log \q(\nu \vert \etanuk)
%          - \expect{\q(\nu \vert \etanuk)}
%                   {\nabla \log \q(\nu \vert \etanuk)}
%        \right)^2
%        \left( \phi(\nu) - \expect{\q(\nu \vert \etanuk)}{\phi(\nu)} \right)} +
% \\&
% \expect{\q(\nu \vert \etanuk)}{
%        \left(                            \nabla^2 \log \q(\nu \vert \etanuk)
%         - \expect{\q(\nu \vert \etanuk)}{\nabla^2 \log \q(\nu \vert \etanuk)}
%        \right)
%        \left( \phi(\nu) - \expect{\q(\nu \vert \etanuk)}{\phi(\nu)} \right)
%        }.
% %
% \end{align*}
% %
Again this is a linear operator as a function of $\phi$, which is continuous by
assumption.  Consequently the Hessian is invertible in a neighborhood.

It follows by \citet[Proposition 4.14(c)]{zeidler:2013:functional} that
$\KLgrad{\eta, \phi}$ is continuosly Fr{\'e}chet differentiable.  Furthermore,
\citet[Chapter 4 Condition 21b]{zeidler:2013:functional} holds since
$\KLhess{\etaopt(\phiz), \phiz}$ is invertible.  So we satisfy conditions (i),
(ii), and (iii) of \citet[Theorem 4.B(c)]{zeidler:2013:functional}, giving that
the function $\etaopt(\phi)$ exists.  Moreover, since $\KLgrad{\eta, \phi}$ is
continuously Fre{\'e}chet differentiable in a neighborhood of $\etaopt(\phiz),
\phiz$, by \citet[Theorem 4.B(d)]{zeidler:2013:functional}, $\etaopt(\phi)$ is
also continuously Fr{\'e}chet differentiable.

The directional derivative

% Other references: We will use the
% formulation given in \citep[Theorem 3.4.10]{krantz:2012:implicit}, for which
% we must show that... actually that doesn't give differentiability.
% The result follows then from \citet[Proposition 4.8(c)]{zeidler:2013:functional}.
% See also \citet[Corollary 1.4]{averbukh:1967:theory} and \citep[Appendix
% A]{reeds:1976:thesis}).
%
\end{proof}
%
\end{thm}
%
%%%%%%%%%%%%%%%%%%%%%%%%%%%%%%%%%%%%%%%%%%%%%%%%%%%%%%%%%%%%%%%%%%%%%%%%%%%%
