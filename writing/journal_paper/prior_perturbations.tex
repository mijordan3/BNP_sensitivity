Recall from \exref{alpha_perturbation} and \corref{gem_approximation_ok} derive
sensitivity measures for perturbations that lie within the
$\mathrm{GEM}(\alpha)$ family.

However, there is typically no {\em a priori}
reason to believe that the stick breaking prior lies within this parametric
family.  In this section, we consider perturbing the functional form of the
prior, ultimatlely proving an analogue of \thmref{etat_deriv} for a particular
function space.

To define functional perturbations, fix for the moment a base stick
distribution, $\pbase(\nuk)$ and an alternative stick distribution $\palt(\nuk)$.
Take $\t = \epsilon$ and define
%
\begin{align}\eqlabel{epsilon_pert}
%
\pstick(\nuk \vert \epsilon) ={}&
\frac{\pbase(\nuk)^{1 - \epsilon} \palt(\nuk)^\epsilon}
     {\int_0^1 \pbase(\nuk')^{1 - \epsilon} \palt(\nuk')^\epsilon d\nuk'}.
%
\end{align}
%
We thus have that $\epsilon$ parameterizes a multiplicative path from
$\pbase(\nuk)$ to $\palt(\nuk)$, with $\pstick(\nuk \vert \epsilon = 0) = \pbase(\nuk)$
and $\pstick(\nuk \vert \epsilon=1) = \palt(\nuk)$.  We will take
$\t_0$ to be $\epsilon = 0$, so that we are computing $\etaopt$ with the
prior $\pbase(\nuk)$ and approximating the optimum if we had used $\palt(\nuk)$.
%
Given this definition,
%
\begin{align*}
%
\log \pstick(\nuk \vert \epsilon) ={}&
    \log \palt(\nuk) + \epsilon \log \frac{\palt(\nuk)}{\pbase(\nuk)} + \const.
    & \constdesc{\nuk}
%
\end{align*}
%
Identifying $\t$ with $\epsilon$, we can apply \thmref{etat_deriv} to
$\pstick(\nuk \vert \epsilon)$ as long as $\log (\palt(\nuk) / \pbase(\nuk))$
satisfies \assuref{q_stick_regular}.

%%%%%%%%%%%%%%%%%%%%%%%%%%%%%%%%%%%%%%%%%%%%%%%%%%%%%%%%%%%%%%%%%%%%%%
%%%%%%%%%%%%%%%%%%%%%%%%%%%%%%%%%%%%%%%%%%%%%%%%%%%%%%%%%%%%%%%%%%%%%%
\begin{ex}
%
We can use \eqref{epsilon_pert} to form an equivalent representation of the
parametric perturbation given in \exref{alpha_perturbation}.  For fixed
$\alpha_0$ and $\alpha_1$, take
%
\begin{align*}
%
\pbase(\nuk) :={}& \betadist{\nuk \vert 1, \alpha_0}\\
\palt(\nuk) :={}& \betadist{\nuk \vert 1, \alpha_1}.
%
\end{align*}
%
Then, up to a constant not depending on $\nuk$,
%
\begin{align}\eqlabel{gem_epsilon_pert}
%
\log \pstick(\nuk \vert \epsilon) =
    (\alpha_0 - 1) \log(1 - \nuk) +
    \epsilon  (\alpha_1 - \alpha_0) \log(1 - \nuk).
%
\end{align}
%
Comparing \eqref{gem_alpha_pert} and \eqref{gem_epsilon_pert}, we see by taking
$\alpha = \epsilon \alpha_1$ that the two parameterizations of the prior are
formally equivalent.
%
\end{ex}
%%%%%%%%%%%%%%%%%%%%%%%%%%%%%%%%%%%%%%%%%%%%%%%%%%%%%%%%%%%%%%%%%%%%%%

There are many (an infinite number!) of $\palt(\nuk)$ to choose from, so, rather
than fixing $\palt$, it can be useful to think of $\etaopt(\epsilon)$ as {\em
functional} of $\palt$ \citep{gustafson:1996:local}.  Let us fix $\pbase(\nuk)$,
define $\phi(\nuk) := \epsilon \log \left(\palt(\nuk) / \pbase(\nuk)\right)$,
and re-write $\pstick(\nuk \vert \epsilon)$ in the following equivalent form:
%
\begin{align}
%
\pstick(\nuk \vert \phi) ={}&
\frac{\exp\left(\log \pbase(\nuk) + \phi(\nuk)\right)}
     {\int_0^1 \exp\left(\log \pbase(\nuk') + \phi(\nuk')\right) d\nuk'}
        \Rightarrow \eqlabel{phi_perturbation}\\
\log \pstick(\nuk \vert \phi) ={}&
    \log \pbase(\nuk) + \log \phi(\nuk) + \const.
    & \constdesc{\nuk} \nonumber
%
\end{align}
%
The advantage of using $\phi$ in \eqref{phi_perturbation} rather than specifying
$\palt$ directly is that $\pstick(\nuk \vert \phi)$ is well-defined for any
Lebesgue-integrable function $\phi(\nuk): (0,1)\mapsto\mathbb{R}$, and a valid
distribution for any function $\phi$ such that the numerator is positive and the
denominator is nonzero.  We can thus safely define perturbations in terms of
general functions without worrying about whether the functions are valid
distributions. We will take $\phiz$ to be the identically zero perturbation so
that $\pstick(\nuk \vert \phiz) = \pbase(\nuk)$, and take $\etaopt$ with no
argument to notate $\etaopt(\phiz)$.

By analogy with \eqref{kl_shorthand}, we can write
%
\begin{align}\eqlabel{kl_fun_shorthand}
%
\KL{\eta, \phi} := \KL{\q(\zeta \vert \eta) || p(\x \vert \zeta, \phi)}
\mathand
\etaopt(\phi) := \argmin_{\zeta \in \etadom} \KL{\eta, \phi}.
%
\end{align}
%
Further, by analogy with \eqref{etalin_def}, if the map $\phi \mapsto
\etaopt(\phi)$ is sufficiently ``nice'', then we might hope to find a linear
functional $d\etaopt(\phi) / d\phi$ such that
%
\begin{align}\eqlabel{etalin_fun_def}
%
\etalin(\phi) := \etaopt + \fracat{d \etaopt(\phi)}{d \phi}{\t=0} \phi
%
\end{align}
%
and $\etaopt(\phi) \approx \etalin(\phi)$ when $\phi$ is ``small'' in some sense
(again, to be specified precisely below).  The compuationally intensive task of
finding $\etaopt(\phi)$ for infinitely many alternative prior distributions can
then be replaced by the analytically and computationally simple $\etalin(\phi)$.
In addition to approximating the effect of particular perturbations, we will see
below that the linearization will allow us to find worst-case perturbations as
well as convenient visual summaries of the effect of prior perturbation.

Our task, then is to find notions of ``nice'' and ''small'' such that
\eqref{etalin_fun_def} holds and $\etaopt(\phi) \approx \etalin(\phi)$. To that
end, we will embed $\phi$ in a complete normed vector space, that is, in a
Banach space.  We can then appeal to the Banach space version of the
implicit function theorem, in perfect analogy with \thmref{etat_deriv}.

Define the norm
%
\begin{align*}
%
\norminf{\phi} :={} \esssup_{\nuk} |\phi(\nuk)|,
%
\end{align*}
%
where $\esssup$ is defined with respect to the Lebesgue measure on $[0,1]$. As
pointed out by \citet{gustafson:1996:local}, $\norminf{\cdot}$ is attactive
because $\pstick(\nuk \vert \phi)$ is a valid distribution whenever $\phi$ lies
within a neighborhood of $\phiz$, as we state in the following
proposition.\footnote{\citet{gustafson:1996:local} also observes that the
converse of \propref{pert_well_defined} is not true---there exist valid
distributions with infinte $\norminf{\phi}$.  We will discuss this and other
issues related to \citet{gustafson:1996:local} below.}


Superficially \assuref{q_fun_stick_regular} seems weaker than
\assuref{q_stick_regular}.  However, the additional condition on $\log
\pstick(\nu \vert \t)$ in \assuref{q_stick_regular} will be replaced in
\thmref{eta_phi_deriv} by the stronger assumption that $\norminf{\phi} <
\delta$.
