In \corref{gem_approximation_ok} of \secref{local_sensitivity}, we showed that
we can form a Taylor series approximation to the dependence of a variational
optimum on the parameter $\alpha$ in a Beta prior. However, there is typically
no {\em a priori} reason to believe that the stick breaking prior lies within
the parametric Beta family.  In this section, we follow
\citet{gustafson:1996:local} and define a class of ways to perturb to the
functional form of the prior.  These classes of nonparametric perturbations will
allow us to employ \thmref{etat_deriv} for any given alternative  prior.  We
will take on the more difficult task of the existence and accuracy of the
derivative in nonparameteric spaces of priors in \secref{valid_priors,
differentiability} below.

Let us fix a base prior density, $\pbase(\theta)$, at which we have computed a
VB approximation, and suppose we wish to ask what the variational optimum would
have been had we used some alternative prior density, $\palt(\theta)$.  Let us
write $\etaopt(\pbase)$ and $\etaopt(\palt)$ for these two approximations,
respectively.  To approximately answer this question using the local sensitivity
approach of \secref{local_sensitivity}, we must somehow define a continuous path
from $\pbase(\theta)$ to $\palt(\theta)$ parameterized, say, by $\t \in [0, 1]$.

There are many ways to do so.  For example, one might form the mixture
distribution:
%
\begin{align*}
%
\p_{lin}(\theta \vert \t) =
    (1- \t) \pbase(\theta) + \t \palt(\theta).
%
\end{align*}
%
Then $\p_{lin}(\theta \vert \t=0) = \pbase(\theta)$, $\p_{lin}(\theta \vert \t=1) =
\palt(\theta)$, and $\p_{lin}(\theta \vert \t)$ interpolates smoothly between the
two.  We could then attempt to apply \thmref{etat_deriv} using $\pstick(\nu \vert
\t)$ to compute $d\etaopt(\t) / d\t$, and approximate
%
\begin{align*}
%
\etaopt(\palt) \approx \etaopt(\pbase) + \fracat{d \etaopt(\t)}{d\t}{\t=1}(1 - 0).
%
\end{align*}
%
However, we might alternatively have defined the mixture in the log densities:
%
\begin{align*}
%
\log \p_{mult}(\theta \vert \t) =
    (1- \t) \log\pbase(\theta) + \t \log\palt(\theta) -
    \const. \\ \constdesc{\theta}
%
\end{align*}
%
Again, $\p_{mult}(\theta \vert \t=0) = \pbase(\theta)$, $\p_{mult}(\theta \vert
\t=1) = \palt(\theta)$, and $\p_{mult}(\theta \vert \t)$ interpolates smoothly
between the two.

Indeed, one may define a family of prior perturbations by adding the densities
after transforming pointwise by any invertible transformation, and then
transforming back into the original space.  We will consider (a generalization
of) the family of ``nonlinear'' functional perturbations given by
\citep{gustafson:1996:local}, of which our examples $\p_{lin}$ and $\p_{mult}$
are the two extremes, corresponding to $p=1$ and $p=\infty$, respectively.

First, some necessary measure theory notation. Below, we will take $\lambda$ to
denote the Lebesgue measure on the Borel sets of $\thetadom \subseteq
\mathbb{R}^{\thetadim}$.  We will be interested in densities with respect to
$\lambda$, expressed as Radon-Nikodym derivatives, though it will be convenient
to use the same notation for a density and for the measure induced by the
density.  Specifically, for a $\lambda$-measurable set $S$, and a Radon-Nikodym
derivative $f$ defined with respect to $\lambda$, we will write $f(S) =
\int_{\theta \in S} f(\theta) \lambda(d\theta)$.  Similarly, for two densities
$f$ and $g$, we will write $f \ll g$ to mean that $g(S) = 0 \Rightarrow f(S) =
0$ for all $\lambda$-measurable sets $S$.

%%%%%%%%%%%%%%%%%%%%%%%%%%%%%%%%%%%%%%%%%%%%%%%%%%%%%%%%%%%%%%%%%%%%%%%%%
%%%%%%%%%%%%%%%%%%%%%%%%%%%%%%%%%%%%%%%%%%%%%%%%%%%%%%%%%%%%%%%%%%%%%%%%%
\begin{defn}\deflabel{prior_nl_pert}
%
Let $\mu$ denote a measure with $\mu \ll \lambda$, and fix $\pbase(\theta)$, a
density with respect to $\mu$.  Assume that $\pbase(\theta) > 0$ on $\thetadom$.
Let $p \in [1, \infty]$.  For any $\phi(\theta)$ for which the expressions are
well-defined, let
%
\begin{align*}
%
\rho(\theta \vert \phi) :={}& \begin{cases}
%
\pbase(\theta)^{1/p} + \frac{1}{p}\phi(\theta)
    & \textrm{when }p \in [1, \infty) \\
\pbase(\theta)\exp(\phi(\theta))
    & \textrm{when }p = \infty
%
\end{cases}\\
%
\tilde{\p}(\theta \vert \phi) :={}&
    \mathrm{sign}(\rho(\theta \vert \phi)) \abs{\rho(\theta \vert \phi)}^p.
%
\end{align*}
%
As usual, the normalized prior is given by
$\p(\theta \vert \phi) :=
    \tilde{\p}(\theta \vert \phi) /
         \int \tilde{\p}(\theta' \vert \phi) \mu(d\theta')$
when $0 < \int \tilde{\p}(\theta' \vert \phi) \mu(d\theta') < \infty$,
or is otherwise undefined.
%
\end{defn}
%
%%%%%%%%%%%%%%%%%%%%%%%%%%%%%%%%%%%%%%%%%%%%%%%%%%%%%%%%%%%%%%%%%%%%%%%%%

We have specified \defref{prior_nl_pert} in terms of the general function
$\phi(\theta)$ rather than an alternative prior density $\palt(\theta)$. Later,
this more general notation will allow us to embed our prior perturbations in a
vector space and analyze our linear approximations using tools from functional
analysis.  For the moment, however, we simply note that nothing has been lost,
since one can extrapolate to any alternative density $\palt(\theta)$ by taking
$\phi$ as given in the following definition.

%%%%%%%%%%%%%%%%%%%%%%%%%%%%%%%%%%%%%%%%%%%%%%%%%%%%%%%%%%%%%%%%%%%%%%%%%%%
%%%%%%%%%%%%%%%%%%%%%%%%%%%%%%%%%%%%%%%%%%%%%%%%%%%%%%%%%%%%%%%%%%%%%%%%%%%

\begin{defn}\deflabel{prior_pert_class}
%
Fix the quantities in \defref{prior_nl_pert}.  Fix a density $\palt(\theta)$
with respect to $\mu$, with $\palt \ll \pbase$. For a given $\beta > 0$, let
%
\begin{align}
%
\phi(\theta | \beta, \palt) :={}
\begin{cases}
\beta \palt(\theta)^{1/p} - p \pbase(\theta)^{1/p}
    & \textrm{when }p \in [1, \infty) \\
\log \palt(\theta) - \log \pbase(\theta) + \log \beta
    & \textrm{when }p = \infty.
\end{cases} \eqlabel{phi_for_palt}
%
\end{align}
%
Similarly, define the set of $\phi$ that can be constructed from
valid priors as
%
\begin{align*}
%
\pertset := \bigg\{&
    \phi:  \phi(\theta | \beta, \palt) %\\&
    \textrm{ for some }\beta > 0\textrm{ and some density }\palt \ll \pbase
\bigg\}.
%
\end{align*}
%
Note that $\p(\theta \vert \t \phi(\cdot \vert \beta, \palt))$ equals $\pbase$
at $\t = 0$ and $\palt$ at $\t = 1$, so every valid prior that is absolutely
continous with respect to $\pbase$ is of the form $\p(\theta \vert \phi)$ for
some $\phi \in \pertset$.
%
\end{defn}
%%%%%%%%%%%%%%%%%%%%%%%%%%%%%%%%%%%%%%%%%%%%%%%%%%%%%%%%%%%%%%%%%%%%%%%%%%%


%%%%%%%%%%%%%%%%%%%%%%%%%%%%%%%%%%%%%%%%%%%%%%%%%%%%%%%%%%%%%%%%%%%%%%%%%%%%%%%
%%%%%%%%%%%%%%%%%%%%%%%%%%%%%%%%%%%%%%%%%%%%%%%%%%%%%%%%%%%%%%%%%%%%%%%%%%%%%%%

\begin{ex}\exlabel{zero_phi}
%
Taking $\palt = \pbase$, we have that $\phi = \phiz$ if and only if $\beta = p$
for $p \in [1, \infty)$ or $\beta = 0$ for $p = \infty$. In this sense, one
might wish to restrict to $\beta = p$.  However, it will cost us little to keep
the notation general.
%
\end{ex}
%%%%%%%%%%%%%%%%%%%%%%%%%%%%%%%%%%%%%%%%%%%%%%%%%%%%%%%%%%%%%%%%%%%%%%%%%%%%%%%

%%%%%%%%%%%%%%%%%%%%%%%%%%%%%%%%%%%%%%%%%%%%%%%%%%%%%%%%%%%%%%%%%%%%%%%%%%%%%%%
%%%%%%%%%%%%%%%%%%%%%%%%%%%%%%%%%%%%%%%%%%%%%%%%%%%%%%%%%%%%%%%%%%%%%%%%%%%%%%%
\begin{ex}\exlabel{phi_negative}
%
The perturbation $\phi(\theta)$ as given by \defref{prior_pert_class} can be
negative. Take $\thetadom = [0,1]$ and let $\pbase(\theta) = 1$. Let us choose a
$\palt(\theta)$ that shifts mass away from a small region next to zero.
Specifically, for  $\delta \in (0, 1)$ and $\epsilon \in (0, 1)$, let
%
\begin{align*}
%
\palt(\theta) :={}&
    \left(\frac{1-\delta \epsilon}{1 - \epsilon} \right)
        \ind{\epsilon \le \nu \le 1} +
    \delta \ind{0 \le \nu \le \epsilon}.
%
\end{align*}
%
% where the final approximation is due to the smallness of $\epsilon$.
Then \eqref{phi_for_palt} gives, for $p \in [0, \infty)$ and any $\beta > 0$,
%
\begin{align*}
%
\phi(\theta) ={}&
    \left( p \beta\left(\frac{1-\delta \epsilon}{1-\epsilon} \right)^{1/p}
        - 1
    \right)
        \ind{\epsilon \le \theta \le 1} +
    \left(p \beta \delta^{1/p} - 1 \right) \ind{0 \le \theta \le \epsilon}.
%
\end{align*}
%
When $\beta < \frac{1}{p} \delta^{-1/p}$, then $\phi(\theta)$ is negative for
$\theta \in (0, \epsilon)$.  Similarly, when $p = \infty$,
%
\begin{align*}
%
\phi(\theta) ={}&
    \ind{\epsilon \le \nu \le 1}
        \log \left(\frac{1-\delta \epsilon}{1 - \epsilon} \right) +
    \ind{0 \le \nu \le \epsilon} \log \delta + \log \beta,
%
\end{align*}
%
so $\phi(\theta)$ is negative for $\theta \in (0, \epsilon)$ when
$\beta < \delta$.

Note that, in this example, one could choose $\beta$ sufficiently large that
$\phi(\theta)$ is non-negative everywhere, but at the cost of making
$\phi(\theta)$ large on $\theta \in (\epsilon, 1)$.   See \appref{positive_pert}
for more discussion of this point.

\end{ex}
%%%%%%%%%%%%%%%%%%%%%%%%%%%%%%%%%%%%%%%%%%%%%%%%%%%%%%%%%%%%%%%%%%%%%%%%%%%%%%%


%%%%%%%%%%%%%%%%%%%%%%%%%%%%%%%%%%%%%%%%%%%%%%%%%%%%%%%%%%%%%%%%%%%%%%%%%%%%%%%
%%%%%%%%%%%%%%%%%%%%%%%%%%%%%%%%%%%%%%%%%%%%%%%%%%%%%%%%%%%%%%%%%%%%%%%%%%%%%%%

\begin{ex}\exlabel{phi_necessarily_negative}
%
It is not always possible to choose $\beta$ large enough to guarantee positive
$\phi(\theta)$. For example, again take $\thetadom=[0,1]$, and let
%
\begin{align*}
%
\pbase(\theta) ={} 2 \theta^{-1/2} \mathand
\palt(\theta) ={} 1.
%
\end{align*}
%
Then
%
\begin{align*}
%
\phi(\theta) ={}&
\begin{cases}
        p\beta - 2 \theta^{-1/2} & \textrm{for }p \in [0, \infty) \\
        \frac{1}{2} \log \theta + \log 2 + \log \beta
            & \textrm{for }p = \infty
\end{cases}.
%
\end{align*}
%
For this perturbation, $\inf_\theta \phi(\theta) = -\infty$ irrespective
of $p$ or $\beta$, and no choice of $\beta$ can induce a change from
$\pbase$ to $\palt$ with a positive $\phi(\theta)$.
%
\end{ex}
%%%%%%%%%%%%%%%%%%%%%%%%%%%%%%%%%%%%%%%%%%%%%%%%%%%%%%%%%%%%%%%%%%%%%%%%%%%%%%%

% %%%%%%%%%%%%%%%%%%%%%%%%%%%%%%%%%%%%%%%%%%%%%%%%%%%%%%%%%%%%%%%%%%%%%%%%%%%%%%%
% %%%%%%%%%%%%%%%%%%%%%%%%%%%%%%%%%%%%%%%%%%%%%%%%%%%%%%%%%%%%%%%%%%%%%%%%%%%%%%%
% \begin{ex}
% %
% Take $\thetadom = [0,1]$.  The pair $\pbase(\theta)  = \frac{1}{2}\ind{\theta <
% 1/2}$ and $\palt(\theta) = \frac{1}{2} \ind{\theta > 1/2}$ are disallowed by
% \defref{prior_pert_class} because we do not have $\palt \ll \pbase$.
% %
% \end{ex}
%%%%%%%%%%%%%%%%%%%%%%%%%%%%%%%%%%%%%%%%%%%%%%%%%%%%%%%%%%%%%%%%%%%%%%%%%%%%%%%

In order to approximate the effect of replacing $\pbase$ with $\palt$, we can
choose a $p$, choose a $\beta$, compute the corresponding $\phi(\cdot \vert
\beta, \palt)$ using \defref{prior_nl_pert}, and apply \thmref{etat_deriv} with
$\p(\theta \vert \t) = \p(\theta \vert \t \phi(\cdot \vert \beta, \palt))$. To
do so, we will require that \assuref{dist_fun_nice} holds with $\psi(\theta, \t) =
\log \p(\theta \vert \t \phi(\cdot \vert \beta, \palt)) - \log \p(\theta \vert
\phiz)$.  Since we are taking the log, we will require $\p(\theta \vert \phi)$
to be $\mu$-almost everywhere positive, a point that will be important later in
\secref{valid_priors}. Assuming that $\p(\theta \vert \phi)$ is positive and
plugging in \defref{prior_nl_pert}, we see that
%
\begin{align}
%
\logp(\theta \vert \t \phi) - \log \pbase(\theta) ={}&
\begin{cases}
    p \log\left(1 + \t \frac{\phi(\theta)}{p \pbase(\theta)^{1/p}}\right)
    & \textrm{when }p \in [1, \infty) \\
    \t \phi(\theta)
    & \textrm{when }p = \infty
%
\end{cases}  \eqlabel{nl_vb_pert_p}\\
%
\fracat{\partial \log \p(\theta \vert \t)}{\partial \t}{\t=0} ={}&
\begin{cases}
   \frac{\phi(\theta)}{\pbase(\theta)^{1/p}}
   & \textrm{when }p < \infty \\
   \phi(\theta)
   & \textrm{when }p = \infty.
\end{cases}\eqlabel{nl_vb_pert_pinf}
%
\end{align}

When \thmref{etat_deriv} can be applied,  it turns out that the derivative takes
the form of an integral against $\phi$.  The integrand is known as the
``influence function,'' and characterizes the sensitivity of a function of
interest to nonparametric prior perturbations for all $\phi$ to which
\thmref{etat_deriv} can be applied.

%%%%%%%%%%%%%%%%%%%%%%%%%%%%%%%%%%%%%%%%%%%%%%%%%%%%%%%%%%%%%%%%%%%%%%%%%
%%%%%%%%%%%%%%%%%%%%%%%%%%%%%%%%%%%%%%%%%%%%%%%%%%%%%%%%%%%%%%%%%%%%%%%%%

\begin{cor}\corlabel{etafun_deriv_form}
%
Let \assuref{kl_opt_ok} hold at $\eta_0 = \etaopt$.
%
Fix the quantities given in \defref{prior_nl_pert}, and let $g(\eta): \etadom
\mapsto \mathbb{R}$ denote a continuously differentiable real-valued function of
interest.  Define the influence function
%
\begin{align}\eqlabel{infl_defn}
%
\infl_p(\theta) :={}&
\begin{cases}
    - \fracat{d g(\eta)}{ d \eta^T}{\etaopt} \hessopt^{-1}
        \lqgradbar{\theta \vert \etaopt}
        \frac{\q(\theta \vert \etaopt)}{\pbase(\theta)^{1/p}},
& \textrm{when }p \in [1, \infty) \\
    - \fracat{d g(\eta)}{ d \eta^T}{\etaopt} \hessopt^{-1}
        \lqgradbar{\theta \vert \etaopt}
        \q(\theta \vert \etaopt).
& \textrm{when }p = \infty.
%
\end{cases}
%
\end{align}
%
Let \assuref{dist_fun_nice} be satisfied at $\t_0 = 0$ with $\psi(\theta, \t) =
\log \p(\theta \vert \t \phi)$ as given in \eqref{nl_vb_pert_p,
nl_vb_pert_pinf}. Then the map $\t \mapsto g(\etaopt(\t \phi))$ is continuously
differentiable at $\t=0$ with derivative
%
\begin{align}\eqlabel{vb_eta_infl_sens}
%
\fracat{d g(\etaopt(\t \phi))}{d \t}{0} ={}&
    \int \infl_p(\theta) \phi(\theta) \mu(d\theta).
%
\end{align}
%
\begin{proof}
%
This follows immediately from \thmref{etat_deriv}, \eqref{nl_vb_pert_p}, and
\eqref{nl_vb_pert_pinf}.
%
\end{proof}
%
\end{cor}

%%%%%%%%%%%%%%%%%%%%%%%%%%%%%%%%%%%%%%%%%%%%%%%%%%%%%%%%%%%%%%%%%%%%%%%%%

We now apply the above concepts to the special case of stick-breaking priors.

%%%%%%%%%%%%%%%%%%%%%%%%%%%%%%%%%%%%%%%%%%%%%%%%%%%%%%%%%%%%%%%%%%%%%%%%%%%%%%%
%%%%%%%%%%%%%%%%%%%%%%%%%%%%%%%%%%%%%%%%%%%%%%%%%%%%%%%%%%%%%%%%%%%%%%%%%%%%%%%

\begin{ex}\exlabel{phi_for_beta}
%
Take $\pbase(\theta) = \betadist{\theta \vert 1, \alpha_0}$ and $\palt(\theta) =
\betadist{\theta \vert 1, \alpha_1}$.  Then $\thetadom=[0,1]$, and we can let
$\mu$ be the Lebesgue measure on $[0,1]$, so
%
\begin{align*}
%
\pbase(\theta) ={}&
    \frac{\Gamma(\alpha_0)}{\Gamma(1 + \alpha_0)} (1 - \theta)^{\alpha_0 - 1}.
%
\end{align*}
%
Then, for $p = 1$,

\begin{align}\eqlabel{phi_beta_p1}
%
\phi(\theta \vert \beta, \palt) ={}&
    p \beta \frac{\Gamma(\alpha_1)}{\Gamma(1 + \alpha_1)}
        (1 - \theta)^{\alpha_1 - 1} -
    \frac{\Gamma(\alpha_0)}{\Gamma(1 + \alpha_0)}
        (1 - \theta)^{\alpha_0 - 1}.
%
\end{align}
%
Note that, when $\t = 1$, the normalizing constant and $\beta$ cancel in the
normalization of $\p(\theta \vert \t \phi)$:
%
\begin{align*}
%
\p(\theta \vert \phi(\theta \vert \beta, \palt)) =
\frac{p \beta \frac{\Gamma(\alpha_1)}{\Gamma(1 + \alpha_1)}
        (1 - \theta)^{\alpha_1 - 1}}
     {p \beta \frac{\Gamma(\alpha_1)}{\Gamma(1 + \alpha_1)}
       \int_0^1 (1 - \theta')^{\alpha_1 - 1} \lambda(d\theta')}
=
\frac{(1 - \theta)^{\alpha_1 - 1}}
     {\int_0^1 (1 - \theta')^{\alpha_1 - 1} \lambda(d\theta')}.
%
\end{align*}
%
For this reason, we are free to choose $\beta > 0$ in \defref{prior_pert_class}
and still extrapolate to the same $\palt$ when $\t = 1$.

\end{ex}
%%%%%%%%%%%%%%%%%%%%%%%%%%%%%%%%%%%%%%%%%%%%%%%%%%%%%%%%%%%%%%%%%%%%%%%%%%%%%%%


%%%%%%%%%%%%%%%%%%%%%%%%%%%%%%%%%%%%%%%%%%%%%%%%%%%%%%%%%%%%%%%%%%%%%%%%%%%%%%%
%%%%%%%%%%%%%%%%%%%%%%%%%%%%%%%%%%%%%%%%%%%%%%%%%%%%%%%%%%%%%%%%%%%%%%%%%%%%%%%

\begin{ex}\exlabel{phi_for_beta_pinf}

When $p = \infty$,
%
\begin{align}
%
\phi(\theta \vert \beta, \palt) ={}&
    \log \left(
        \frac{\Gamma(\alpha_1) }{\Gamma(1 + \alpha_1)}
    \right)  + (\alpha_1 - 1) \log (1 - \theta) - \nonumber\\
{}&
    \log \left(
        \frac{\Gamma(\alpha_0)}{\Gamma(1 + \alpha_0)}
    \right) + (\alpha_0 - 1) \log (1 - \theta)  + \log \beta \nonumber\\
={}&
\log \left(
    \frac{\Gamma(\alpha_1) }{\Gamma(1 + \alpha_1)}
\right) -
\log \left(
    \frac{\Gamma(\alpha_0)}{\Gamma(1 + \alpha_0)}
\right) + \log \beta + (\alpha_1 - \alpha_0) \log(1 - \theta).
\eqlabel{phi_beta_pinf}
%
\end{align}
%
For $p = \infty$, the normalizing constant and $\beta$ cancel for all $\t$:
%
\begin{align*}
%
\p(\theta \vert \t \phi(\theta \vert \beta, \palt)) ={}&
\frac{  \beta \frac{\Gamma(\alpha_1) \Gamma(1 + \alpha_0) }
                   {\Gamma(1 + \alpha_1) \Gamma(\alpha_0)}
        (1-\theta)^{\alpha_0 (1 - \t) + \alpha_1 \t}
    }
    {
        \beta \frac{\Gamma(\alpha_1) \Gamma(1 + \alpha_0) }
                   {\Gamma(1 + \alpha_1) \Gamma(\alpha_0)}
             \int_0^1 (1-\theta')^{\alpha_0 (1 - \t) +
                                   \alpha_1 \t} \lambda(d\theta')
     }
\\={}&
\frac{(1-\theta)^{\alpha_0 (1 - \t) + \alpha_1 \t}}
     {\int_0^1 (1-\theta')^{\alpha_0 (1 - \t) + \alpha_1 \t} \lambda(d\theta')}.
%
\end{align*}
%
\end{ex}
%%%%%%%%%%%%%%%%%%%%%%%%%%%%%%%%%%%%%%%%%%%%%%%%%%%%%%%%%%%%%%%%%%%%%%%%%%%%%%%


%%%%%%%%%%%%%%%%%%%%%%%%%%%%%%%%%%%%%%%%%%%%%%%%%%%%%%%%%%%%%%%%%%%%%%%%%
%%%%%%%%%%%%%%%%%%%%%%%%%%%%%%%%%%%%%%%%%%%%%%%%%%%%%%%%%%%%%%%%%%%%%%%%%
\begin{ex}\exlabel{gem_fun_pert}
%
Let us apply \corref{etafun_deriv_form} to the GEM stick-breaking priors with
$p=1$, following \exref{phi_for_beta}.  In order  to use
\lemref{normal_q_is_regular}, it will be helpful to express the priors
in the logit stick space as in \exref{gem_pert_ok}.  We thus take
$\theta = \lnuk$, $\thetadom = \mathbb{R}$, and...
%
\end{ex}
%%%%%%%%%%%%%%%%%%%%%%%%%%%%%%%%%%%%%%%%%%%%%%%%%%%%%%%%%%%%%%%%%%%%%%%%%
