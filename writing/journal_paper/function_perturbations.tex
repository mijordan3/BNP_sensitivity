In \corref{gem_approximation_ok} of \secref{local_sensitivity}, we showed that
we can form a Taylor series approximation to the dependence of a variational
optimum on the parameter $\alpha$ in a Beta prior. However, there is typically
no {\em a priori} reason to believe that the stick breaking prior lies within
the parametric Beta family.  In this section, we follow
\citet{gustafson:1996:local} and define a class of ways to perturb to the
functional form of the prior, corresponding to the $\lp{p}$ classes of
integrable functions (which we will define and discuss below).

For any particular perturbation, \thmref{etat_deriv} can be applied directly by
verifying \assuref{q_stick_regular}.  However, we will argue that it is
typically more useful to examine the form of the derivative to find influential
perturbations, for which one wants a stronger result than can be provided
by \thmref{etat_deriv} alone---we will require that the derivatives give
{\em uniformly good linear approximations} within bounded sets.
We address this question in \secref{differentiability}.

Let us remain with the general problem of inference on a parameter $\theta$. Fix
a base prior density, $\pbase(\theta)$, at which we have computed a VB
approximation.  Suppose we wish to ask what the variational optimum would have
been had we used some alternative prior density, $\palt(\theta)$.  Let us write
$\etaopt(\pbase)$ and $\etaopt(\palt)$ for these two approximations,
respectively.  To approximately answer this question using the local sensitivity
approach of \secref{local_sensitivity}, we must somehow define a continuous path
from $\pbase(\theta)$ to $\palt(\theta)$ parameterized, say, by $\t \in [0, 1]$.

There are many ways to do so.  For example, one might form the mixture
distribution:
%
\begin{align*}
%
\p_{lin}(\theta \vert \t) =
    (1- \t) \pbase(\theta) + \t \palt(\theta).
%
\end{align*}
%
Then $\p_{lin}(\theta \vert \t=0) = \pbase(\theta)$, $\p_{lin}(\theta \vert \t=1) =
\palt(\theta)$, and $\p_{lin}(\theta \vert \t)$ interpolates smoothly between the
two.  We could then attempt to apply \thmref{etat_deriv} using $\pstick(\nu \vert
\t)$ to compute $d\etaopt(\t) / d\t$, and approximate
%
\begin{align*}
%
\etaopt(\palt) \approx \etaopt(\pbase) + \fracat{d \etaopt(\t)}{d\t}{\t=1}(1 - 0).
%
\end{align*}
%
However, we might alternatively have defined the mixture in the log densities:
%
\begin{align*}
%
\log \p_{mult}(\theta \vert \t) =
    (1- \t) \log\pbase(\theta) + \t \log\palt(\theta) -
    \const. \\ \constdesc{\theta}
%
\end{align*}
%
Again, $\p_{mult}(\theta \vert \t=0) = \pbase(\theta)$, $\p_{mult}(\theta \vert
\t=1) = \palt(\theta)$, and $\p_{mult}(\theta \vert \t)$ interpolates smoothly
between the two.

Indeed, one may define a family of prior perturbations by adding the densities
after transforming pointwise by any invertible transformation, and then
transforming back into the original space.  We will consider (a generalization
of) the family of ``nonlinear'' functional perturbations given by
\citep{gustafson:1996:local}, of which our examples $\p_{lin}$ and $\p_{mult}$
are the two extremes, corresponding to $p=1$ and $p=\infty$, respectively.

Below, we will take $\lambda$ to denote the Lebesgue measure on the Borel sets
of $\thetadom \subseteq \mathbb{R}^{\thetadim}$.  We will be interested in
densities with respect to $\lambda$, expressed as Radon-Nikodym derivatives,
though it will be convenient to use the same notation for a density and for the
measure induced by the density.  Specifically, for a $\lambda$-measurable set
$S$, and a Radon-Nikodym derivative $f$ defined with respect to $\lambda$, we
will write $f(S) = \int_{\theta \in S} f(\theta) \lambda(d\theta)$.  Similarly,
for two densities $f$ and $g$, we will write $f \ll g$ to mean that $g(S) = 0
\Rightarrow f(S) = 0$ for all $\lambda$-measurable sets $S$.

%%%%%%%%%%%%%%%%%%%%%%%%%%%%%%%%%%%%%%%%%%%%%%%%%%%%%%%%%%%%%%%%%%%%%%%%%
%%%%%%%%%%%%%%%%%%%%%%%%%%%%%%%%%%%%%%%%%%%%%%%%%%%%%%%%%%%%%%%%%%%%%%%%%
\begin{defn}\deflabel{prior_nl_pert}
%
Let $\mu$ denote a measure with $\mu \ll \lambda$, and fix $\pbase(\theta)$, a
density with respect to $\mu$.  Assume that $\pbase(\theta) > 0$ on $\thetadom$.
Let $p \in [1, \infty]$.  For any $\phi(\theta)$ for which the expressions are
well-defined, let
%
\begin{align*}
%
\rho(\theta \vert \phi) :={}& \begin{cases}
%
\pbase(\theta)^{1/p} + \frac{1}{p}\phi(\theta)
    & \textrm{when }p \in [1, \infty) \\
\pbase(\theta)\exp(\phi(\theta))
    & \textrm{when }p = \infty
%
\end{cases}\\
%
\tilde{\p}(\theta \vert \phi) :={}&
    \mathrm{sign}(\rho(\theta \vert \phi)) \abs{\rho(\theta \vert \phi)}^p\\
\p(\theta \vert \phi) :={}&
    \frac{\tilde{\p}(\theta \vert \phi)}
         {\int \tilde{\p}(\theta' \vert \phi) d\theta'}.
%
\end{align*}
%
\end{defn}
%
%%%%%%%%%%%%%%%%%%%%%%%%%%%%%%%%%%%%%%%%%%%%%%%%%%%%%%%%%%%%%%%%%%%%%%%%%

We have specified \defref{prior_nl_pert} in terms of the general function
$\phi(\theta)$ rather than an alternative $\palt(\theta)$.  The reason is so
that the perturbed prior $\p(\theta \vert \phi)$ is well-defined for $\phi$ that
may not be probabilities densities.  As we will soon show, this allows us to
embed $\phi$ in a vector space, in which $\p(\theta \vert \phi)$ is well-defined
in an open neighborhood of the zero function.

Nothing has been lost, however, since one can extrapolate to any alternative
density $\palt(\theta)$ by taking $\phi$ as given in the following definition.

%%%%%%%%%%%%%%%%%%%%%%%%%%%%%%%%%%%%%%%%%%%%%%%%%%%%%%%%%%%%%%%%%%%%%%%%%%%
%%%%%%%%%%%%%%%%%%%%%%%%%%%%%%%%%%%%%%%%%%%%%%%%%%%%%%%%%%%%%%%%%%%%%%%%%%%

\begin{defn}\deflabel{prior_pert_class}
%
Fix the quantities in \defref{prior_nl_pert}.  Fix a density $\palt(\theta)$
with respect to $\mu$, with $\palt \ll \pbase$. For a given $\beta > 0$, let
%
\begin{align}
%
\phi(\theta | \beta, \palt) :={}
\begin{cases}
\beta \palt(\theta)^{1/p} - p \pbase(\theta)^{1/p}
    & \textrm{when }p \in [1, \infty) \\
\log \palt(\theta) - \log \pbase(\theta) + \log \beta
    & \textrm{when }p = \infty.
\end{cases} \eqlabel{phi_for_palt}
%
\end{align}
%
Similarly, define
%
\begin{align*}
%
\pertset := \bigg\{&
    \phi:  \phi(\theta | \beta, \palt) %\\&
    \textrm{ for some }\beta > 0\textrm{ and some density }\palt \ll \pbase
\bigg\}.
%
\end{align*}
%
Then $\p(\theta \vert \t \phi(\cdot \vert \beta, \palt))$ equals $\pbase$
at $\t = 0$ and $\palt$ at $\t = 1$.
%
\end{defn}
%%%%%%%%%%%%%%%%%%%%%%%%%%%%%%%%%%%%%%%%%%%%%%%%%%%%%%%%%%%%%%%%%%%%%%%%%%%

%%%%%%%%%%%%%%%%%%%%%%%%%%%%%%%%%%%%%%%%%%%%%%%%%%%%%%%%%%%%%%%%%%%%%%%%%%%%%%%
%%%%%%%%%%%%%%%%%%%%%%%%%%%%%%%%%%%%%%%%%%%%%%%%%%%%%%%%%%%%%%%%%%%%%%%%%%%%%%%

\begin{ex}\exlabel{phi_for_beta}
%
Take $\pbase(\theta) = \betadist{\theta \vert 1, \alpha_0}$ and
$\palt(\theta) = \betadist{\theta \vert 1, \alpha_1}$.  Then
$\thetadom=[0,1]$ and $\lambda$ is the Lebesgue measure on $[0,1]$, so
%
\begin{align*}
%
\pbase(\theta) ={}&
    \frac{\Gamma(\alpha_0)}{\Gamma(1 + \alpha_0)} (1 - \theta)^{\alpha_0 - 1}.
%
\end{align*}
%
Then, for $p = 1$,

\begin{align}\eqlabel{phi_beta_p1}
%
\phi(\theta \vert \beta, \palt) ={}&
    p \beta \frac{\Gamma(\alpha_1)}{\Gamma(1 + \alpha_1)}
        (1 - \theta)^{\alpha_1 - 1} -
    \frac{\Gamma(\alpha_0)}{\Gamma(1 + \alpha_0)}
        (1 - \theta)^{\alpha_0 - 1}.
%
\end{align}
%
Note that, when $\t = 1$, the normalizing constant and $\beta$ cancel in the
normalization of $\p(\theta \vert \t \phi)$:
%
\begin{align*}
%
\p(\theta \vert \phi(\theta \vert \beta, \palt)) =
\frac{p \beta \frac{\Gamma(\alpha_1)}{\Gamma(1 + \alpha_1)}
        (1 - \theta)^{\alpha_1 - 1}}
     {p \beta \frac{\Gamma(\alpha_1)}{\Gamma(1 + \alpha_1)}
       \int_0^1 (1 - \theta')^{\alpha_1 - 1} \lambda(d\theta')}
=
\frac{(1 - \theta)^{\alpha_1 - 1}}
     {\int_0^1 (1 - \theta')^{\alpha_1 - 1} \lambda(d\theta')}.
%
\end{align*}
%
For this reason, we are free to choose $\beta > 0$ in \defref{prior_pert_class}
and still extrapolate to the same $\palt$ when $\t = 1$.

When $p = \infty$,
%
\begin{align}
%
\phi(\theta \vert \beta, \palt) ={}&
    \log \left(
        \frac{\Gamma(\alpha_1) }{\Gamma(1 + \alpha_1)}
    \right)  + (\alpha_1 - 1) \log (1 - \theta) - \nonumber\\
{}&
    \log \left(
        \frac{\Gamma(\alpha_0)}{\Gamma(1 + \alpha_0)}
    \right) + (\alpha_0 - 1) \log (1 - \theta)  + \log \beta \nonumber\\
={}&
\log \left(
    \frac{\Gamma(\alpha_1) }{\Gamma(1 + \alpha_1)}
\right) -
\log \left(
    \frac{\Gamma(\alpha_0)}{\Gamma(1 + \alpha_0)}
\right) + \log \beta + (\alpha_1 - \alpha_0) \log(1 - \theta).
\eqlabel{phi_beta_pinf}
%
\end{align}
%
For $p = \infty$, the normalizing constant and $\beta$ cancel for all $\t$:
%
\begin{align*}
%
\p(\theta \vert \t \phi(\theta \vert \beta, \palt)) ={}&
\frac{  \beta \frac{\Gamma(\alpha_1) \Gamma(1 + \alpha_0) }
                   {\Gamma(1 + \alpha_1) \Gamma(\alpha_0)}
        (1-\theta)^{\alpha_0 (1 - \t) + \alpha_1 \t}
    }
    {
        \beta \frac{\Gamma(\alpha_1) \Gamma(1 + \alpha_0) }
                   {\Gamma(1 + \alpha_1) \Gamma(\alpha_0)}
             \int_0^1 (1-\theta')^{\alpha_0 (1 - \t) +
                                   \alpha_1 \t} \lambda(d\theta')
     }
\\={}&
\frac{(1-\theta)^{\alpha_0 (1 - \t) + \alpha_1 \t}}
     {\int_0^1 (1-\theta')^{\alpha_0 (1 - \t) + \alpha_1 \t} \lambda(d\theta')}.
%
\end{align*}
%
\end{ex}


%%%%%%%%%%%%%%%%%%%%%%%%%%%%%%%%%%%%%%%%%%%%%%%%%%%%%%%%%%%%%%%%%%%%%%%%%%%%%%%
%%%%%%%%%%%%%%%%%%%%%%%%%%%%%%%%%%%%%%%%%%%%%%%%%%%%%%%%%%%%%%%%%%%%%%%%%%%%%%%
\begin{ex}\exlabel{phi_negative}
%
The perturbation $\phi(\theta)$ as given by \defref{prior_pert_class} can be
negative. Take $\thetadom = [0,1]$ and let $\pbase(\theta) = 1$. Let us choose a
$\palt(\theta)$ that shifts mass away from a small region next to zero.
Specifically, for  $\delta \in (0, 1)$ and $\epsilon \in (0, 1)$, let
%
\begin{align*}
%
\palt(\theta) :={}&
    \left(\frac{1-\delta \epsilon}{1 - \epsilon} \right)
        \ind{\epsilon \le \nu \le 1} +
    \delta \ind{0 \le \nu \le \epsilon}.
%
\end{align*}
%
% where the final approximation is due to the smallness of $\epsilon$.
Then \eqref{phi_for_palt} gives, for $p \in [0, \infty)$ and any $\beta > 0$,
%
\begin{align*}
%
\phi(\theta) ={}&
    \left( p \beta\left(\frac{1-\delta \epsilon}{1-\epsilon} \right)^{1/p}
        - 1
    \right)
        \ind{\epsilon \le \theta \le 1} +
    \left(p \beta \delta^{1/p} - 1 \right) \ind{0 \le \theta \le \epsilon}.
%
\end{align*}
%
When $\beta < \frac{1}{p} \delta^{-1/p}$, then $\phi(\theta)$ is negative for
$\theta \in (0, \epsilon)$.  Similarly, when $p = \infty$,
%
\begin{align*}
%
\phi(\theta) ={}&
    \ind{\epsilon \le \nu \le 1}
        \log \left(\frac{1-\delta \epsilon}{1 - \epsilon} \right) +
    \ind{0 \le \nu \le \epsilon} \log \delta + \log \beta,
%
\end{align*}
%
so $\phi(\theta)$ is negative for $\theta \in (0, \epsilon)$ when
$\beta < \delta$.

\end{ex}

\begin{ex}
%
In \exref{phi_negative}, one could choose $\beta$ sufficiently large that
$\phi(\theta)$ is non-negative everywhere, but at the cost of making
$\phi(\theta)$ large on $\theta \in (\epsilon, 1)$. However, it is not always
possible to choose $\beta$ large enough to guarantee positive $\phi(\theta)$.
For example, again take $\thetadom=[0,1]$, and let
%
\begin{align*}
%
\pbase(\theta) ={} 2 \theta^{-1/2} \mathand
\palt(\theta) ={} 1.
%
\end{align*}
%
Then
%
\begin{align*}
%
\phi(\theta) ={}&
\begin{cases}
        p\beta - 2 \theta^{-1/2} & \textrm{for }p \in [0, \infty) \\
        \frac{1}{2} \log \theta + \log 2 + \log \beta
            & \textrm{for }p = \infty
\end{cases}.
%
\end{align*}
%
For this perturbation, $\inf_\theta \phi(\theta) = -\infty$ irrespective
of $p$ or $\beta$, and no choice of $\beta$ can induce a change from
$\pbase$ to $\palt$ with a positive $\phi(\theta)$.
%
\end{ex}
%%%%%%%%%%%%%%%%%%%%%%%%%%%%%%%%%%%%%%%%%%%%%%%%%%%%%%%%%%%%%%%%%%%%%%%%%%%%%%%

%%%%%%%%%%%%%%%%%%%%%%%%%%%%%%%%%%%%%%%%%%%%%%%%%%%%%%%%%%%%%%%%%%%%%%%%%%%%%%%
%%%%%%%%%%%%%%%%%%%%%%%%%%%%%%%%%%%%%%%%%%%%%%%%%%%%%%%%%%%%%%%%%%%%%%%%%%%%%%%
\begin{ex}
%
Take $\thetadom = [0,1]$.  The pair $\pbase(\theta)  = \frac{1}{2}\ind{\theta < 1/2}$
and $\palt(\theta) = \frac{1}{2} \ind{\theta > 1/2}$ are disallowed
by \defref{prior_pert_class} because we do not have $\palt \ll \pbase$.
%
\end{ex}
%%%%%%%%%%%%%%%%%%%%%%%%%%%%%%%%%%%%%%%%%%%%%%%%%%%%%%%%%%%%%%%%%%%%%%%%%%%%%%%


The classes of perturbations given in \defref{prior_nl_pert} have a natural
correspondence with the $\lp{\lambda, p}$ spaces of functions, which we now
define.

%%%%%%%%%%%%%%%%%%%%%%%%%%%%%%%%%%%%%%%%%%%%%%%%%%%%%%%%%%%%%%%%%%%%%%%%%%%
%%%%%%%%%%%%%%%%%%%%%%%%%%%%%%%%%%%%%%%%%%%%%%%%%%%%%%%%%%%%%%%%%%%%%%%%%%%

\begin{defn}\deflabel{lp_spaces}
\citep[Sections 5.1-5.2]{dudley:2018:real}
%
For a measure $\mu$ and $p \in [1, \infty]$, let $\lp{\mu,p}$ define the
space of equivalence classes of real-valued $\mu$-measurable functions,
where two functions are equivalent if they disagree only on a set of
$\mu$-measure zero.

Let $\esssup_\theta^\mu$ denote the essential supremum over $\theta$ with
respect to the measure $\mu$. The norm on $\lp{\mu,p}$ is given by
%
\begin{align*}
%
\norm{\phi}_{\mu,p} :={}&
\begin{cases}
    \left(\int \abs{\phi(\theta)}^p \mu(d\theta)\right)^{1/p}
    & \textrm{when }p \in [1, \infty)\\
    \esssup_{\theta} \abs{\phi(\theta)}
    & \textrm{when }p = \infty
\end{cases}\\
%
\phi \in \lp{\lambda,p} \Leftrightarrow{}& \norm{\phi}_{\lambda,p} < \infty.
%
\end{align*}
%
When $\mu$ is the Lebesgue measure, we may simply write $\lp{\lambda,p} =
\lp{p}$ and $\norm{\cdot}_{p} = \norm{\cdot}_{\lambda,p}$.
%
Let $\ball_{\mu,p}(\epsilon) := \{\phi: \phi \in \lp{\mu, p},
\norm{\phi}_p < \epsilon \}$ denote the $\epsilon$-ball in $\lp{\mu, p}$.
%
\end{defn}

By \citep[Theorem 5.2.1]{dudley:2018:real}, $\lp{\mu,p}$ is a Banach
space (i.e., a complete, normed vector space).

The following lemma is due to \citep{gustafson:1996:local}, and provides
a key part of the motivation for the use of $\norm{\cdot}_{\mu,p}$ to measure
the size of prior perturbations.

%%%%%%%%%%%%%%%%%%%%%%%%%%%%%%%%%%%%%%%%%%%%%%%%%%%%%%%%%%%%%%%%%%%%%%%%%%%
%%%%%%%%%%%%%%%%%%%%%%%%%%%%%%%%%%%%%%%%%%%%%%%%%%%%%%%%%%%%%%%%%%%%%%%%%%%
\begin{lem}\lemlabel{pert_invariance}
%
(\citet{gustafson:1996:local})
%
Fix the quantities given in \defref{prior_nl_pert}.  For a fixed probability
measure $\p \ll \mu$, the map $\p \mapsto \norm{\phi(\cdot \vert \beta,
\p)}_p$ does not depend on $\mu$, and is invariant to invertible transformations
of $\theta$.
%
\seeproof{pert_invariance}
%
\end{lem}
%%%%%%%%%%%%%%%%%%%%%%%%%%%%%%%%%%%%%%%%%%%%%%%%%%%%%%%%%%%%%%%%%%%%%%%%%%%




%%%%%%%%%%%%%%%%%%%%%%%%%%%%%%%%%%%%%%%%%%%%%%%%%%%%%%%%%%%%%%%%%%%%%%%%%%%

\begin{lem}\lemlabel{pert_well_defined}
%
Fix the quantities given in \defref{prior_nl_pert}.  An unnormalized prior
$\ptil$ is valid if $\ptil \ll \pbase$ that is non-negative $\mu$-almost
everywhere and normalizable in the sense that $0 < \int \ptil(\theta)
\mu(d\theta) < \infty$.

Furthermore, the following table summarizes properties of priors derived from
$\phi \in \ball_{\mu,p}$ and from $\phi \in \pertset$.  The columns are
shorthand for the following properites:
%
\begin{align*}
%
\int \ptil < \infty \Rightarrow{}&
    \int \ptil(\theta \vert \phi) \mu(d\theta) < \infty\\
\int \ptil > \infty \Rightarrow{}&
    \int \ptil(\theta \vert \phi) \mu(d\theta) > 0\\
\ptil \ge 0 \Rightarrow{}&
    \essinf_\theta^\mu \ptil(\theta \vert \phi) \ge 0\\
% \norm{\phi} < \infty \Rightarrow{}&
%     \exists M < \infty \textrm{ independent of }\phi\textrm{ such that }
%     \sup_\phi \norm{\phi}_{\mu,p} < M.
\norm{\phi} < \infty \Rightarrow{}& \norm{\phi}_{\mu,p} < \infty.
%
\end{align*}

A ``Y'' indicates that the columns property is satisfied for all $\phi$ in the
corresponding set.  For example, the ``Y'' in the first row and first column
means that, for all $\phi \in \pertset$ with $p \in [1, \infty)$, $\int
\ptil(\theta \vert \phi) \mu(d\theta) < \infty$.  A ``N'' means the converse,
i.e., that there exists, in general, some $\phi$ in the corresponding set that
does not have the column's property.

\begin{table}[h!]
%\vspace{1em}
\begin{centering}
%\begin{tabular}{|c|c|c|c|c|c|}
\begin{tabular}{cccccc}
    %\hline
    && $\int \ptil < \infty$
    & $\int \ptil > 0$
    & $\ptil \ge 0$
    & $\norm{\phi} < \infty$\\[0.5em] \hline
$p \in [1, \infty)$   &     $\phi \in \pertset$ &
%    Y & Y & Y & Y, if $\beta < \infty$ \\ \hline
    Y & Y & Y & Y \\ \hline
$p \in [1, \infty)$   &     $\phi \in \ball_{\mu,p}(\delta)$ &
    Y & Y, if $\delta < p$ & N & Y (by defn) \\ \hline
$p = \infty$   &     $\phi \in \pertset[\infty]$ &
    Y & Y & Y & N \\ \hline
$p = \infty$   &     $\phi \in \ball_{\mu,\infty}(\delta)$ &
    Y & Y & Y & Y (by defn) \\ \hline
\end{tabular}
\caption{The relationships between $\pertset$, $\ball_{\mu,p}$, and valid priors.
The ball radius $\delta > 0$ can be any positive number.}
\tablabel{pert_well_defined}
\end{centering}
%\vspace{1em}
\end{table}

\Tabref{pert_well_defined} has the following implications:

\begin{enumerate}
%
\item For all $p \in [1, \infty]$, the set of priors that can be formed from
$\phi \in \pertset$ is identical to the set of all valid priors.
%
\item For $p \in [1, \infty)$, all valid priors can be formed from
some $\phi \in \lp{\mu,p}$.
%
\item For $p \in [1, \infty)$, one can form invalid (negative) priors from
some $\phi \in \ball_{\mu,p}(\delta)$ even for arbitrarily small $\delta$.
(\Exref{lp_negative}.)
%
\item For $p = \infty$, there exist valid priors not formed from any
$\phi \in \lp{\mu,\infty}$.  (\Exref{beta_inf_norm}.)
%
\end{enumerate}

\seeproof{pert_well_defined}
%
\end{lem}
%%%%%%%%%%%%%%%%%%%%%%%%%%%%%%%%%%%%%%%%%%%%%%%%%%%%%%%%%%%%%%%%%%%%%%%%%%%
%%%%%%%%%%%%%%%%%%%%%%%%%%%%%%%%%%%%%%%%%%%%%%%%%%%%%%%%%%%%%%%%%%%%%%%%%%%

%%%%%%%%%%%%%%%%%%%%%%%%%%%%%%%%%%%%%%%%%%%%%%%%%%%%%%%%%%%%%%%%%%%%%%%%%
% \/  \/  \/  \/  \/  \/  \/  \/  \/  \/  \/  \/  \/  \/  \/  \/  \/  \/
\begin{ex}\exlabel{lp_negative}
%
For $1 \le p < \infty$, it is possible for $\phi \in
\ball_{\mu,p}(\epsilon)$ for arbitrarily small $\epsilon$ and yet have
$\essinf^\mu_{\theta} \p(\theta \vert \phi) < 0$.

Since $\pbase \ll \mu \ll \lambda$, there exists a
sequence $\epsilon_n \rightarrow 0$ with $\epsilon_n > 0$ and a sequence of
corresponding sets such that $\pbase(S_n) = \epsilon_n$. (See
\lemref{continuity_partition} for a proof of this fact, which is a
straightforward consequence of \citet[Proposition 15.5]{nielsen:1997:measure}
and the continuity of the Lebesgue measure.)  Take
%
%\begin{align*}
%
$\phi_n(\theta) := - \frac{2}{p} \pbase(\theta)^{1/p} \ind{\theta \in S_n}$.
%
%\end{align*}
%
Then $\norm{\phi_n}_{\lambda, p} = \frac{2}{p} \epsilon_n^{1/p} \rightarrow 0$
and
%
\begin{align*}
%
\pbase(\theta)^{1/p} + p \phi(\theta) ={}
\pbase(\theta)^{1/p}
\left(\ind{\theta \notin S_n} - \ind{\theta \in S_n} \right)
%
\end{align*}
%
so $\essinf_\theta \p(\theta \vert \phi_n) < 0$ for all $n$.
%
\end{ex}
% /\    /\    /\    /\    /\    /\    /\    /\    /\    /\    /\    /\    /\
%%%%%%%%%%%%%%%%%%%%%%%%%%%%%%%%%%%%%%%%%%%%%%%%%%%%%%%%%%%%%%%%%%%%%%%%%%%%




%%%%%%%%%%%%%%%%%%%%%%%%%%%%%%%%%%%%%%%%%%%%%%%%%%%%%%%%%%%%%%%%%%%%%%%%%
% \/  \/  \/  \/  \/  \/  \/  \/  \/  \/  \/  \/  \/  \/  \/  \/  \/  \/
\begin{ex}\exlabel{beta_inf_norm}
%
It is possible for $\phi \in \pertset[\infty]$ to have $\norminf{\phi} =
\infty$, and so $\phi \notin \lp{\lambda,\infty}$.  Take $\mu$ to be the
Lebesgue measure on $[0,1]$, let $\pbase(\theta) = \betadist{\theta \vert 1,
\alpha_0}$ and $\palt(\theta) = \betadist{\theta \vert 1, \alpha_1}$ for
$\alpha_0 \ne \alpha_1$.  Then we can choose $\beta$ such that $\phi(\theta) =
(\alpha_1 - \alpha_0) \log(1 - \theta)$
%
and
%
\begin{align*}
%
\norminf{\phi} =
    \abs{\alpha_1 - \alpha_0} \sup_{\theta \in [0,1]} \abs{\log(1 - \theta)} =
    \infty.
%
\end{align*}
%
Consequently, $\phi \notin \lp{\mu,\infty}$.
%
\end{ex}
% /\    /\    /\    /\    /\    /\    /\    /\    /\    /\    /\    /\    /\
%%%%%%%%%%%%%%%%%%%%%%%%%%%%%%%%%%%%%%%%%%%%%%%%%%%%%%%%%%%%%%%%%%%%%%%%%%%%

-----------------------------
-----------------------------


For any $p$, the transformation given in \defref{prior_nl_pert} and the norm
defined in \defref{lp_spaces} jointly specify a ``size'' of a particular
perturbation.   It turns out that this notion of ``size'' has a number of
attractive properties: \citep[Result 2]{gustafson:1996:local} observes that this
notion of ``size'' is a norm, invariant to changes in the base measure, and
invariant under measurable one-to-one re-parameterizations (where the measure
$\lambda$ is transformed to the push-forward measure of the reparameterization).

It is also desirable that perturbations which are bounded in
$\norm{\cdot}_{\lambda,p}$ lead to proper priors.  On this point we deviate from
\citep{gustafson:1996:local}, who requires $\phi$ to be non-negative
$\lambda$-almost everywhere.  In contrast, we allow the perturbations $\phi$ to
be negative. There are two reasons for doing so.  First, if $\phi$ must be
pointwise positive, then $\p(\theta \vert \phi)$ is not defined in an open ball
containing $\phiz$, and standard results from functional analysis cannot be
directly applied to establish properties like Fr{\'e}chet differentiability, a
central concern of our \secref{differentiability} below.  There are alternative
priors that cannot be achieved by considering only positive perturbations.
Additionally, when $\phi$ must be positive, the norm $\norm{\cdot}_{\lambda, p}$
treats ablating and adding prior mass very asymmetrically when $p < \infty$,
arguably violating an intutive notions of the ``size'' of perturbation, though
we defer a detailed discussion of this point to \appref{positive_pert}.


-----------------------------
-----------------------------

For any particular $\phi$, we can apply \thmref{etat_deriv}.
%  Note that the
% perturbations are given, for $1 \le p < \infty$, by
% %
% \begin{align}
% %
% \logp(\theta \vert \t \phi) ={}&
%    % p \log\left(\pbase^{1/p} + \t \phi(\theta) \right) \Rightarrow \nonumber\\
% % \\={}&
%    \log \pbase(\theta) +
%        p \log\left(1 + \t \frac{\phi(\theta)}{\pbase(\theta)^{1/p}}\right)
% \Rightarrow \nonumber\\
% \fracat{\partial \log \p(\theta \vert \t)}{\partial \t}{\t=0} ={}&
%    p \frac{\phi(\theta)}{\pbase(\theta)^{1/p}},
%    \eqlabel{nl_vb_pert_p}
% %
% \end{align}
% %
% and, for $p = \infty$,
% %
% \begin{align}
% %
% \logp(\theta \vert \t) ={}&
%    \log \pbase(\theta) + \t \phi(\theta)
% \Rightarrow \nonumber\\
% \fracat{\partial \log \p(\theta \vert \t)}{\partial \t}{\t=0} ={}&
%    \phi(\theta).
% \eqlabel{nl_vb_pert_pinf}
% %
% \end{align}
%

%%%%%%%%%%%%%%%%%%%%%%%%%%%%%%%%%%%%%%%%%%%%%%%%%%%%%%%%%%%%%%%%%%%%%%%%%
%%%%%%%%%%%%%%%%%%%%%%%%%%%%%%%%%%%%%%%%%%%%%%%%%%%%%%%%%%%%%%%%%%%%%%%%%

\begin{cor}\corlabel{etafun_deriv_form}
%
Let \assuref{kl_opt_ok} hold at $\eta_0 = \etaopt$.
%
Fix $\phi$ and $p$ in \defref{prior_nl_pert}, with $\norm{\phi}_{\lambda,p} <
\infty$.  Define the influence function
%
\begin{align}\eqlabel{infl_defn}
%
\infl_p(\theta) :={}&
\begin{cases}
    - \hessopt^{-1}
        \lqgradbar{\theta \vert \etaopt}
        \frac{\q(\theta \vert \etaopt)}{\pbase(\theta)^{1/p}}
& \textrm{when }1 \le p < \infty \\
%
    - \hessopt^{-1}
        \lqgradbar{\theta \vert \etaopt}
        \q(\theta \vert \etaopt).
& \textrm{when }p = \infty
%
\end{cases}
%
\end{align}
%
Let the variational densities $\q(\theta \vert \eta)$ satisfy
\assuref{dist_fun_nice} for $\t_0 = 0$ with both $\psi(\theta, \t) \equiv 1$ (no
$\theta$ dependence) and with and $\psi(\theta, \t) = \log \p(\theta \vert \t
\phi)$, where
%
\begin{align}
%
\logp(\theta \vert \t \phi) ={}&
\begin{cases}
    \log \pbase(\theta) +
        \log\left(1 + \t \frac{\phi(\theta)}{\pbase(\theta)^{1/p}}\right)
    & \textrm{when }p < \infty \\
    \log \pbase(\theta) + \t \phi(\theta)
    & \textrm{when }p = \infty
%
\end{cases}  \eqlabel{nl_vb_pert_p}
%
\end{align}
%
Then the map $\t \mapsto \etaopt(\t \phi)$ is continuously
differentiable at $\t=0$ with derivative
%
\begin{align}\eqlabel{vb_eta_infl_sens}
%
\fracat{d \etaopt(\t \phi)}{d \t}{0} ={}&
    \int \infl_p(\theta) \phi(\theta) \lambda(d\theta).
%
\end{align}
%
\begin{proof}
%
The result follows immediately from \thmref{etat_deriv}, \eqref{nl_vb_pert_p},
and \eqref{nl_vb_pert_pinf}.
%
\end{proof}
%
\end{cor}

%%%%%%%%%%%%%%%%%%%%%%%%%%%%%%%%%%%%%%%%%%%%%%%%%%%%%%%%%%%%%%%%%%%%%%%%%


Note that, under \eqref{nl_vb_pert_p}, the derivatives of the
log perturbation are given by

\begin{align}
%
\fracat{\partial \log \p(\theta \vert \t)}{\partial \t}{\t=0} ={}&
\begin{cases}
   \frac{\phi(\theta)}{\pbase(\theta)^{1/p}}
   & \textrm{when }p < \infty \\
   \phi(\theta)
   & \textrm{when }p = \infty.
\end{cases}\eqlabel{nl_vb_pert_pinf}
%
\end{align}
