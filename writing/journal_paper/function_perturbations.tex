In \corref{gem_approximation_ok} of \secref{local_sensitivity}, we showed that
we can form a Taylor series approximation to the dependence of a variational
optimum on the parameter $\alpha$ in a Beta prior. However, there is typically
no {\em a priori} reason to believe that the stick breaking prior lies within
the parametric Beta family.  In this section, we follow
\citet{gustafson:1996:local} and define a class of ways to perturb to the
functional form of the prior, corresponding to the $\lp{p}$ classes of
integrable functions (which we will define and discuss below).

For any particular perturbation, \thmref{etat_deriv} can be applied directly by
verifying \assuref{q_stick_regular}.  However, we will argue that it is
typically more useful to examine the form of the derivative to find influential
perturbations, for which one wants a stronger result than can be provided
by \thmref{etat_deriv} alone---we will require that the derivatives give
{\em uniformly good linear approximations} within bounded sets.
We address this question in \secref{differentiability}.

Let us return again to the general problem of inference on a parameter
$\theta$>

Fix for the moment a base stick distribution, $\pbase(\nuk)$.  Suppose we have
found $\etaopt$ using the prior $\pbase(\nuk)$, wish to ask what the variational
optimum would have been had we used some alternative stick distribution,
$\palt(\nuk)$.  Let us write $\etaopt(\pbase)$ and $\etaopt(\palt)$ for these
two approximations, respectively.  To approximately answer this question using
the local sensitivity approach of \secref{local_sensitivity}, we must somehow
define a continuous path from $\pbase(\nuk)$ to $\palt(\nuk)$ parameterized,
say, by $\t \in [0, 1]$.  There are, in fact, many ways to do so.  For example,
one might form the mixture distribution:
%
\begin{align*}
%
\pstick(\nu \vert \t) =
    (1- \t) \prod_{k=1}^{\kmax - 1} \pbase(\nuk) +
    \t \prod_{k=1}^{\kmax - 1} \sumkm \palt(\nuk).
%
\end{align*}
%
Then $\pstick(\nu \vert 0)$
We could then attempt to apply \thmref{etat_deriv} using $\pstick(\nu \vert \t)$
to compute $d\etaopt(\t) / d\t$, and approximate
%
\begin{align*}
%
\etaopt(\palt) \approx \etaopt(\pbase) + \frac{d \etaopt(\t)}{d\t}{\t=1}(1 - 0).
%
\end{align*}
%
However, we might just as well have defined
%
\begin{align*}
%
\log\pstick(\nu \vert \t) =
    (1- \t) \sumkm \log\pbase(\nuk) + \t \sumkm \log\palt(\nuk).
%
\end{align*}
%
