Typically, many different choices for $\pstick$ may be {\em a priori}
reasonable.  A particularly common choice for $\pstick$ is the
$\mathrm{Beta}(\nuk \vert 1, \alpha)$ density, which we write as
%
\begin{align*}
%
\pstick(\nuk \vert \alpha) :=
\mathrm{Beta}(\nuk \vert 1, \alpha) =
    \frac{\Gamma(1 + \alpha) (1 - \nuk)^{\alpha - 1}}
         {\Gamma(\alpha)},
%
\end{align*}
%
When $\pstick$ is $\mathrm{Beta}(\nuk \vert 1, \alpha)$, the resulting
distribution on $\pi$ is known as the $\textit{GEM distribution}$, and we write
$\pi \sim \mathrm{GEM}(\alpha)$.
%
The GEM distribution is closely related to the Dirichlet process (DP).
Define a measure on $\betadom$ as
%
\begin{align*}
  \mathcal{M} = \sum_{\k = 1}^\infty \pi_\k\delta_{\beta_\k},
\end{align*}
%
which places atoms at points $\beta_k$ with weight $\pi_\k$. When $\pi \sim
\mathrm{GEM}(\alpha)$ and $\beta_\k \iid \pbetaprior(\beta_\k)$, $\mathcal{M}$
is a random measure is distributed according to Dirichlet process with
concentration parameter $\alpha$ and base measure $\pbetaprior$
\citep{ferguson:1973:bayesian, sethuraman:1994:constructivedp}.

Typically, the concentration parameter $\alpha$ is not known in advance.
Rather, $\alpha$ may {\em a priori} plausibly lie within some reasonable range.
Since the prior $\pstick(\nuk \vert \alpha)$ depends on $\alpha$, the posterior
the posterior expectation $\expect{\p(\z \vert \x, \alpha)}{\nclusters_0(\z)}$
depends on $\alpha$ as well.  If $\expect{\p(\z \vert \x,
\alpha)}{\nclusters_0(\z)}$ varies meaningfully as $\alpha$ varies over its
plausible values, then the quantity of interest $\expect{\p(\z \vert \x,
\alpha)}{\nclusters_0(\z)}$ is not robust to the choice of $\alpha$.

In practice, one chooses some ``base value,'' $\alpha_0$, and runs a
computationally expensive posterior approximation procedure such variational
Bayes (VB), giving an approximate value for $\expect{\p(\z \vert \x,
\alpha_0)}{\nclusters_0(\z)}$.


More generally, there may be no {\em a priori} reason to believe that $\pstick$
lies in the Beta family at all (other than computational convenience). By
$\pbase$ and $\palt$ denote two candidate stick-breaking densities, we can
. Suppose that
parameterizing one-dimensional paths in the space of prior densities.  Let
$\pbase$


%
%
%
%
% Even if one is willing to restrict However, there is typically no {\em a priori}
% reason to assume that $\pi \sim \mathrm{GEM}(\alpha)$ is a realistic summary of
% our prior beliefs for any particular $\alpha$.
%
%
% We keep the generic notation $\pstick$ for stick-breaking distributions because
% in our sensitivity analysis, we will consider stick-breaking distributions that
% are outside the family of Beta distributions.
