Suppose we have committed to using a Beta density for $\pstick$ as in
\eqref{beta_density}. Since $\pstick(\nuk \vert \alpha)$ depends on $\alpha$,
the posterior depends on $\alpha$ through Bayes rule; in turn, the KL divergence
depends on $\alpha$; in turn the optimal VB parameters depend on $\alpha$; in
turn, the quantity of interest depends on $\alpha$ as well. Typically, the
$\alpha$ may {\em a priori} plausibly lie within some reasonable range.

Let us notate the dependence of the VB objective and optimal parameters on
$\alpha$ as
%
\begin{align*}
%
\etaopt(\alpha) := \argmin_{\eta \in \etadom} \KL{\eta, \alpha}.
%
\end{align*}
%
If $\g(\etaopt(\alpha))$ varies meaningfully as $\alpha$ ranges over its
plausible values, we say that a the quantity of interest $\g$ is {\em not
robust} to changes of $\alpha$.  In any particular problem, some posterior
quantities of interest may be robust while others are not.

Note that non-robustness is to some extent subjective, in that it depends on a
decision about what a ``reasonable'' range of $\alpha$ might be, as well as how
much variation in $\g(\etaopt(\alpha))$ is ``acceptable.''  In the present paper
we will primarily focus on the task of approximating the computation of
$\g(\etaopt(\alpha))$, leaving the context-specific decision of what priors are
reasonable up to the the reader.  However, when considering non-parametric
perturbations to the prior, we will explicitly consider whether the
nonparametric neighborhoods correspond to intuitively reasonable sets of prior
perturbations.

In a world with no computational costs, we could simply evaluate
$\g(\etaopt(\alpha))$ for a large number of candidate $\alpha$ and directly
check for robustness.  Unfortunately, the evaluation of $\etaopt(\alpha)$
requires solving an optimization problem, which is often computationally
expensive for even a single $\alpha$, much less many different $\alpha$.

We propose avoiding the need to repeatedly re-optimize by forming a {\em linear
approximation} to the map $\alpha \mapsto \etaopt(\alpha)$ as follows. Supposing
for the moment that the map $\alpha \mapsto \etaopt(\alpha)$ is continuously
differentiable (we will establish precise conditions below), and that we have
found $\etaopt(\alpha_0)$ for some ``initial value,'' $\alpha_0$.  If we can
compute the derivative $d \etaopt(\alpha) / d\alpha$ (which is a
$\etadim$-length vector), we can form the first-order Taylor series
approximation
%
\begin{align*}
%
\etaopt(\alpha) \approx \etalin(\alpha) :={}
    \etaopt(\alpha_0) +
    \fracat{d \etaopt(\alpha)}{d\alpha}{\alpha_0} (\alpha - \alpha_0).
%
\end{align*}
%
Given the derivative $d \etaopt(\alpha) / d\alpha$, the cost of evaluating
$\etalin(\alpha)$ for any $\alpha$ is only that of vector multiplication, not
that of solving a new optimization problem.

We can then use the approximation $\etalin(\alpha)$ to approximate our quantity
of interest, $\g(\etaopt(\alpha))$.  For this, we can either use the
approximation
%
\begin{align*}
%
\g(\etaopt(\alpha)) \approx{} \g(\etalin(\alpha))
%
\end{align*}
%
or, when $\eta \mapsto \g(\eta)$ is continuously differentiable, using the chain
rule to form
%
\begin{align*}
%
%\g(\etaopt(\alpha)) \approx{}& \g(\etaopt(\alpha)) \quad \textrm{or}\\
\g(\etaopt(\alpha)) \approx{}&
    \glin(\alpha) :=
        \g(\etaopt(\alpha_0)) +
            \fracat{d \g(\eta)}{ d\eta^T}{\etaopt(\alpha_0)}
            \fracat{d \etaopt(\alpha)}{d\alpha}{\alpha_0} (\alpha - \alpha_0).
%
\end{align*}
%
We will discuss the relative advantages of $\g(\etalin(\alpha))$ and
$\glin(\alpha)$ below.  In short, the advantage of $\g(\etalin(\alpha))$ is that
the approximation retains nonlinearities in the map from $\eta \mapsto
\g(\eta)$, but the full linear approximation $\glin(\alpha)$ allows the
computation of helpful quantities such worst-case nonparametric prior
perturbations.

In order for the approximation $\etalin(\alpha)$ to be useful, three necessary
conditions must be met.  First, the map $\alpha \mapsto \etaopt(\alpha)$ must be
continuously differentiable; if it is not differentiable, then the derivative
does not even exist, and if it is not continuously differentiable, then small
$\abs{\alpha - \alpha_0}$ may not guarantee a good approximation to
$\etaopt(\alpha)$. Second, the derivative $d \etaopt(\alpha) / d\alpha$ must be
relatively easy to compute---in particular, it must be easier to compute than
solving a new optimization problem.  Finally, the derivative must {\em
extrapolate} meaningfully over a meaningful range of alternative values of
$\alpha$, i.e., the map $\alpha \mapsto \etaopt(\alpha)$ must be shown to be
smooth enough in practical problems to be useful.

The remainder of the paper addresses these desiderata in turn, both for
parametric perturbations to the concentration parameter as well as a particular
family of nonparametric perturbations.  First, we provide theoretical conditions
under which we have continuously differentiability of the VB optimum in
\secref{local_sensitivity}, and show that they are satisfied for our BNP model.
In \secref{computing_sensitivity}, we use the theoretical results of
\secref{local_sensitivity} to show that the derivative can be computed
efficiently in the BNP problems we are considering, with an emphasis on how
modern automatic differentiation tools render many of the computations automatic
\citep{jax2018github}.  Finally, in \secref{results}, we demonstrate the
practical usefulness of the approximation on a set of three real-world BNP
problems of increasing complexity.
