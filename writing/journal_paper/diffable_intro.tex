In this section we state general conditions under which VB optima, as defined by
\eqref{vb_optimization}, are differentiable functions of both parametric and
nonparametric prior perturbations.  The desired results for the BNP model will
follow as special cases of these general results.

We first define our general setup in \defref{prior_t}, and then connect
the general definition to the our BNP problem.

%%%%%%%%%%%%%%%%%%%%%%%%%%%%%%%%%%%%%%%%%%%%%%%%%%%%%%%%%%%%%%%%%%%%%%%%%%%
%%%%%%%%%%%%%%%%%%%%%%%%%%%%%%%%%%%%%%%%%%%%%%%%%%%%%%%%%%%%%%%%%%%%%%%%%%%%
\begin{defn}\deflabel{prior_t}
%
For some parameter $\theta \in \thetadom \subseteq \mathbb{R}^{\thetadim}$, let
$\p(\theta \vert \t)$ denote a class of probability densities relative to
a sigma-finite measure $\mu$, defined for $\t$ in an open set $\ball_\t
\subseteq \mathbb{R}$ containing $0$.  Let $\q(\theta \vert \eta)$ be a
family of approximating densities, also defined relative to $\mu$.

Define
%
\begin{align*}
%
\psi(\theta, \t) :={}
    \expect{\q(\theta \vert \eta)}
       {\log \p(\theta \vert \t) - \log \p(\theta \vert \t=0)},
%
\end{align*}
%
and let the variational objective factorize as
%
\begin{align}
%
\KL{\eta, \t} :={}& \KL{\eta} + \psi(\theta, \t)
           \eqlabel{perturbed_objective}\\
\etaopt(\t) :={}& \argmin_{\eta \in \etadom} \KL{\eta, \t}.
    \eqlabel{perturbed_optimum}
%
\end{align}
%
Let $\etaopt$ with no argument refer to $\etaopt(0)$, the minimizer
of $\KL{\eta}$.
%
\end{defn}
%%%%%%%%%%%%%%%%%%%%%%%%%%%%%%%%%%%%%%%%%%%%%%%%%%%%%%%%%%%%%%%%%%%%%%%%%%%%

%%%%%%%%%%%%%%%%%%%%%%%%%%%%%%%%%%%%%%%%%%%%%%%%%%%%%%%%%%%%%%%%%%%%%%%%%%%%%%%%
%%%%%%%%%%%%%%%%%%%%%%%%%%%%%%%%%%%%%%%%%%%%%%%%%%%%%%%%%%%%%%%%%%%%%%%%%%%%%%%%
\begin{ex}\exlabel{alpha_perturbation}
%
For the BNP model with the GEM prior, take $\theta = (\nu_1, \ldots,
\nu_{\kmax-1})$, $\mu$ to be the Lebesgue measure on $[0,1]^{\kmax-1}$.
Let $\alpha_0$ be some base value of the concentration parameter, and
let $\t$ be $\alpha - \alpha_0$, so that deviations of $\t$ away from
$0$ represent deviations of $\alpha$ away from $\alpha_0$.

Expanding the KL divergence in \eqref{kl_def}, we see that the prior
$\p(\nuk \vert \alpha)$ enters the VB objective in a term of the form
$\sum_{\k=1}^\infty \expect{\q(\nuk \vert \eta)}{\log \p(\nuk \vert \alpha)}$.
Adding and subtracting the this term evaluated at $\alpha_0$ gives
% %
% \begin{align*}
% %
% \MoveEqLeft
% \KL{\q(\zeta \vert \eta) || \p(\zeta \vert \x, \alpha)}
% \\={}&    \expect{\q(\zeta \vert \eta)}{
%         \log \q(\zeta \vert \eta) - \logp(\x, \zeta)} + \logp(\x)
% \\={}&    \expect{\q(\zeta \vert \eta)}{\log \q(\zeta \vert \eta)} -
%         \expect{\q(\zeta \vert \eta)}
%                {\log \p(\x, \beta, \z \vert \nu )}
%                +\logp(\x)
% \\{}& -\sum_{\k=1}^{\kmax - 1}
%             \left(
%                 \expect{\q(\nuk \vert \eta)}{\log \p(\nuk \vert \alpha)} -
%                 \expect{\q(\nuk \vert \eta)}{\log \p(\nuk \vert \alpha_0)} +
%                 \expect{\q(\nuk \vert \eta)}{\log \p(\nuk \vert \alpha_0)}
%              \right)
% \\{}&
% - \sum_{\k = \kmax}^\infty
%     \expect{\q(\zeta \vert \eta)}{\log \p(\nuk \vert \alpha)}.
% %
% \end{align*}
% %
% Recall that the truncated VB approximation ignores the terms
% $\expect{\q(\zeta \vert \eta)}{\log \p(\nuk \vert \alpha)}$ for $\k \ge \kmax$,
% so that we can write the truncated objective as
%
\begin{align*}
%
\KL{\eta, \alpha} = \KL{\eta, \alpha_0}
-\sum_{\k=1}^{\kmax - 1}
            \left(
                \expect{\q(\nuk \vert \eta)}{\log \p(\nuk \vert \alpha)} -
                \expect{\q(\nuk \vert \eta)}{\log \p(\nuk \vert \alpha_0)}
             \right).
%
\end{align*}
%
By the definition of $\p(\nuk \vert \alpha)$,
%
\begin{align*}
%
\log \p(\nu_1, ..., \nu_\kmax \vert \alpha)
    ={}& (\alpha - 1) \sum_{k=1}^\kmax \log (1 - \nuk)
    + \kmax \log \frac{\Gamma(1 + \alpha)}{\Gamma(\alpha)}.
%
\end{align*}
%
The normalizing constant does not depend on $\nuk$, and so its expectation
does not depend on $\eta$, and so it can be neglected in the VB objective,
and we can write
%
\begin{align*}
%
\KL{\eta, \alpha} = \KL{\eta, \alpha_0}
-(\alpha - \alpha_0) \sum_{\k=1}^{\kmax - 1}
    \expect{\q(\nuk \vert \eta)}{\log (1 - \nuk)}.
%
\end{align*}
%
Identifying $\KL{\eta}$ of \defref{prior_t} with $\KL{\eta, \alpha_0}$ evaluated
at the base concentration parameter shows that the BNP problem with the GEM
prior is of the form \eqref{perturbed_objective}, and that
%
\begin{align*}
%
\psi(\nu, \t) = -\t \sum_{\k=1}^{\kmax - 1}
    \expect{\q(\nuk \vert \eta)}{\log (1 - \nuk)}.
%
\end{align*}
%
\end{ex}
%%%%%%%%%%%%%%%%%%%%%%%%%%%%%%%%%%%%%%%%%%%%%%%%%%%%%%%%%%%%%%%%%%%%%%%%%%%%%%%%
