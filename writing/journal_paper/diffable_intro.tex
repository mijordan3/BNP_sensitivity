In this section we state general conditions under which VB optima, as defined by
\eqref{vb_optimization}, are differentiable functions of both parametric and
nonparametric prior perturbations.  The desired results for the BNP model will
follow as special cases of these general results.

In particular, we consider a generic unknown parameter, $\theta \in \thetadom
\subseteq \mathbb{R}^{\thetadim}$, with corresponding likelihood $\p(\x \vert
\theta)$ and approximating variational family $\q(\theta \vert \eta)$, with
$\eta \in \etadom \subseteq \mathbb{R}^\etadim$.  We let the prior densities for
$\theta$ live in a one-dimensional parametric family, as stated in
\defref{prior_t}.

%%%%%%%%%%%%%%%%%%%%%%%%%%%%%%%%%%%%%%%%%%%%%%%%%%%%%%%%%%%%%%%%%%%%%%%%%%%
%%%%%%%%%%%%%%%%%%%%%%%%%%%%%%%%%%%%%%%%%%%%%%%%%%%%%%%%%%%%%%%%%%%%%%%%%%%%
\begin{defn}\deflabel{prior_t}
%
Let $\q(\theta \vert \eta)$ and the prior $\p(\theta)$ be densities defined with
respect to the  Let the prior density Let $\mu$ denote a sigma finite measure,
and let $\p(\theta \vert \t)$ denote a class of probability densities relative
$\mu$.  Let $\p(\theta \vert \t)$ be defined for $\t$ in an open set $\ball_\t
\subseteq \mathbb{R}$ containing $0$. Assume that the variational densities
$\q(\theta \vert \eta)$ are also defined relative to $\mu$.

Let $\qtil$ and $\ptil$ refer to potentially
unnormalized (but normalizable) versions of the respectively corresponding $\q$
and $\p$, so that,
%
\begin{align*}
%
\q(\theta \vert \eta) :={}
    \frac{\qtil(\theta \vert \eta)}
    {\int \qtil(\theta' \vert \eta) \mu(d\theta')} \mathand
\p(\theta \vert \t) :={}
    \frac{\ptil(\theta \vert \t)}
    {\int \ptil(\theta' \vert \t) \mu(d\theta')}.
%
\end{align*}
%
\end{defn}
%%%%%%%%%%%%%%%%%%%%%%%%%%%%%%%%%%%%%%%%%%%%%%%%%%%%%%%%%%%%%%%%%%%%%%%%%%%%

We allow that the full model may involve parameters other than $\theta$. Suppose
that these parameters are $\thetatil$, with $\p(\x, \theta, \thetatil) = \p(\x
\thetatil \vert \theta) \p(\theta \vert \t)$,
so that
%
\begin{align*}
%
\MoveEqLeft
\KL{\q(\theta, \thetatil \vert \eta) || \p(\theta, \thetatil \vert \x, \t)}\\
    ={}& \expect{\q(\theta, \thetatil \vert \eta)}
                {\log \q(\theta, \thetatil \vert \eta)} -
        \expect{\q(\theta, \thetatil \vert \eta)}
               {\log \p(\x, \thetatil \vert \theta)} -
       \expect{\q(\theta \vert \eta)}
              {\p(\theta \vert \t)},
%
\end{align*}
%
which we can write as
%
\begin{align*}
%
\etaopt :={} \argmax_{\eta \in \etadom} \left( \KL{\eta} + \KL{\eta, \t}\right).
%
\end{align*}
%



The prior depends on $\t$; in turn, the posterior $\p(\theta \vert \x, \t)$
depends on $\t$; in turn, the variatoinal objective depends on $\t$; in turn, the
optimal variational parameters depend on $\t$.  Define the shorthand notation
%
\begin{align}\eqlabel{kl_shorthand}
%
\KL{\eta, \t} := \KL{\q(\theta \vert \eta) || \p(\x \vert \theta, \t)}
\mathand
\etaopt(\t) := \argmin_{\eta \in \etadom} \KL{\eta, \t},
%
\end{align}
%
where we write $\etaopt(\t)$ to emphasize the dependence of the optimum on $\t$.
In \defref{prior_t}, we take $\t = 0$ at the ``original'' problem,
\eqref{vb_optimization}, without loss of generality.  We will thus continue to
use $\etaopt$ with no argument to refer to $\etaopt(0)$.


%%%%%%%%%%%%%%%%%%%%%%%%%%%%%%%%%%%%%%%%%%%%%%%%%%%%%%%%%%%%%%%%%%%%%%%%%%%%%%%%
%%%%%%%%%%%%%%%%%%%%%%%%%%%%%%%%%%%%%%%%%%%%%%%%%%%%%%%%%%%%%%%%%%%%%%%%%%%%%%%%
\begin{ex}\exlabel{alpha_perturbation}
%
When drawing from the classical $\mathrm{GEM}(\alpha)$ distribution, we take
$\mu$ in \defref{prior_t} to be the Lebesgue measure on $[0,1]$ and model
%
\begin{align*}
%
\pstick(\nuk \vert \alpha) ={}&
    \betadist{\nuk \vert 1, \alpha} \Rightarrow\\
\log \pstick(\nuk \vert \alpha) ={}&
    (\alpha - 1) \log(1 - \nuk) + \const. &
    \constdesc{\nuk}
%
\end{align*}
%
Fix some ``original'' $\alpha_0$.  In this case, we represent deviations from
the choice $\alpha_0$ by identifying $\t$ with $\alpha - \alpha_0$:
%
\begin{align*}
%
\log \pstick(\nuk \vert \t) ={}&
(\t + \alpha_0 - 1) \log(1 - \nuk) + \const.
\end{align*}
%
The prior on the full $\nu$ vector is given by
%
\begin{align*}
%
\log \pstick(\nu \vert \t) ={}&
    \sumkm (t + \alpha_0 - 1) \log(1 - \nuk)+ \const.
%
\end{align*}
%
\end{ex}
%%%%%%%%%%%%%%%%%%%%%%%%%%%%%%%%%%%%%%%%%%%%%%%%%%%%%%%%%%%%%%%%%%%%%%%%%%%%%%%%
