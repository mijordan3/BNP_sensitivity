In this section we state general conditions under which VB optima, as defined by
\eqref{vb_optimization}, are differentiable functions of both parametric and
nonparametric prior perturbations.  The desired results for the BNP model will
follow as special cases of these general results.

We first define our general setup in \defref{prior_t}, and then connect
the general definition to the our BNP problem.

%%%%%%%%%%%%%%%%%%%%%%%%%%%%%%%%%%%%%%%%%%%%%%%%%%%%%%%%%%%%%%%%%%%%%%%%%%%
%%%%%%%%%%%%%%%%%%%%%%%%%%%%%%%%%%%%%%%%%%%%%%%%%%%%%%%%%%%%%%%%%%%%%%%%%%%%
\begin{defn}\deflabel{prior_t}
%
For some parameter $\theta \in \thetadom \subseteq \mathbb{R}^{\thetadim}$, let
$\p(\theta \vert \t)$ denote a class of probability densities relative to
a sigma-finite measure $\mu$, defined for $\t$ in an open set $\ball_\t
\subseteq \mathbb{R}$ containing $0$.  Let $\q(\theta \vert \eta)$ be a
family of approximating densities, also defined relative to $\mu$.

Let the variational objective factorize as
%
\begin{align}
%
\KL{\eta, \t} :={}&
    \KL{\eta} -
    \expect{\q(\theta \vert \eta)}
       {\left(\log \p(\theta \vert \t) - \log \p(\theta \vert \t=0)\right)}           \eqlabel{perturbed_objective}\\
\etaopt(\t) :={}& \argmin_{\eta \in \etadom} \KL{\eta, \t}.
    \eqlabel{perturbed_optimum}
%
\end{align}
%
Let $\etaopt$ with no argument refer to $\etaopt(0)$, the minimizer
of $\KL{\eta}$.
%
\end{defn}
%%%%%%%%%%%%%%%%%%%%%%%%%%%%%%%%%%%%%%%%%%%%%%%%%%%%%%%%%%%%%%%%%%%%%%%%%%%%

The decomposition \eqref{perturbed_objective} is always possible, in the sense
that one could always take $\theta = \zeta$ and $\KL{\eta} = 0$.  We decompose
the objective in this way in order to state strict regularity assumptions only
on the part of the KL divergence that is being perturbed.  Indeed, we will
require little from the $\KL{\eta}$ part of the decomposition other than that it
can be differentiated and optimized.

By identfiying $\t$ with some hyperparameter (e.g. the concentration parameter,
as in \exref{alpha_perturbation} below), we can use \defref{prior_t} to study
parametric perturbations.  Furthermore, by parameterizing a path through the
space of general densities, \defref{prior_t} will allow us to study
nonparametric perturbations (e.g. \exref{gem_mult_perturbation} below and the
detailed analysis of \secref{diffable_nonparametric, diffable_lp}).  We
can thus study VB prior robustness in general by studying problems of the
form \defref{prior_t}.
%
% As $\etaopt(\t)$ varies, so does our quantity of interest, $\g(\etaopt(\t))$. If
% $\g(\etaopt(\t))$ varies meaningfully as $\t$ ranges over its plausible values,
% we say that a the quantity of interest $\g$ is {\em not robust} to changes of
% $\t$.  In any particular problem, some posterior quantities of interest may be
% robust while others are not.
% %
% Note that non-robustness is to some extent subjective, in that it depends on a
% decision about what a ``reasonable'' range of $\alpha$ might be, as well as how
% much variation in $\g(\etaopt(\alpha))$ is ``acceptable.''  In the present paper
% we will primarily focus on the task of approximating the computation of
% $\g(\etaopt(\alpha))$, leaving the context-specific decision of what priors are
% reasonable up to the the reader.  However, when considering non-parametric
% perturbations to the prior, we will explicitly consider whether the
% nonparametric neighborhoods correspond to intuitively reasonable sets of prior
% perturbations.
%
% In a world with no computational costs, we could simply evaluate
% $\g(\etaopt(\t))$ for a large number of candidate $\t$ and directly
% check for robustness.  Unfortunately, the evaluation of $\etaopt(\t)$
% requires solving an optimization problem, which is often computationally
% expensive for even a single $\t$, much less many different $\t$.  For
% nonparametric perturbations, the problem is even greater, since each $\t$
% explores only a single alternative $\palt(\theta)$.
%
% We propose avoiding the need to repeatedly re-optimize by forming a {\em linear
% approximation} to the map $\alpha \mapsto \etaopt(\alpha)$ as follows. Supposing
% for the moment that the map $\alpha \mapsto \etaopt(\alpha)$ is continuously
% differentiable (we will establish precise conditions below), and that we have
% found $\etaopt(\alpha_0)$ for some ``base value,'' $\alpha_0$.  If we can
% compute the derivative $d \etaopt(\alpha) / d\alpha$ (which is a
% $\etadim$-length vector), we can form the first-order Taylor series
% approximation
% %
% \begin{align*}
% %
% \etaopt(\alpha) \approx \etalin(\alpha) :={}
%     \etaopt(\alpha_0) +
%     \fracat{d \etaopt(\alpha)}{d\alpha}{\alpha_0} (\alpha - \alpha_0).
% %
% \end{align*}
% %
% Given the derivative $d \etaopt(\alpha) / d\alpha$, the cost of evaluating
% $\etalin(\alpha)$ for any $\alpha$ is only that of vector multiplication, not
% that of solving a new optimization problem.
%
% We can then use the approximation $\etalin(\alpha)$ to approximate our quantity
% of interest, $\g(\etaopt(\alpha))$.  For this, we can either use the
% approximation
% %
% \begin{align*}
% %
% \g(\etaopt(\alpha)) \approx{} \g(\etalin(\alpha))
% %
% \end{align*}
% %
% or, when $\eta \mapsto \g(\eta)$ is continuously differentiable, using the chain
% rule to form
% %
% \begin{align*}
% %
% %\g(\etaopt(\alpha)) \approx{}& \g(\etaopt(\alpha)) \quad \textrm{or}\\
% \g(\etaopt(\alpha)) \approx{}&
%     \glin(\alpha) :=
%         \g(\etaopt(\alpha_0)) +
%             \fracat{d \g(\eta)}{ d\eta^T}{\etaopt(\alpha_0)}
%             \fracat{d \etaopt(\alpha)}{d\alpha}{\alpha_0} (\alpha - \alpha_0).
% %
% \end{align*}
% %
% We will discuss the relative advantages of $\g(\etalin(\alpha))$ and
% $\glin(\alpha)$ below.  In short, the advantage of $\g(\etalin(\alpha))$ is that
% the approximation retains nonlinearities in the map from $\eta \mapsto
% \g(\eta)$, but the full linear approximation $\glin(\alpha)$ allows the
% computation of helpful quantities such worst-case nonparametric prior
% perturbations.
%
% In order for the approximation $\etalin(\alpha)$ to be useful, three necessary
% conditions must be met.  First, the map $\alpha \mapsto \etaopt(\alpha)$ must be
% continuously differentiable; if it is not differentiable, then the derivative
% does not even exist, and if it is not continuously differentiable, then small
% $\abs{\alpha - \alpha_0}$ may not guarantee a good approximation to
% $\etaopt(\alpha)$. Second, the derivative $d \etaopt(\alpha) / d\alpha$ must be
% relatively easy to compute---in particular, it must be easier to compute than
% solving a new optimization problem.  Finally, the derivative must {\em
% extrapolate} meaningfully over a meaningful range of alternative values of
% $\alpha$, i.e., the map $\alpha \mapsto \etaopt(\alpha)$ must be shown to be
% smooth enough in practical problems to be useful.
%
% The remainder of the paper addresses these desiderata in turn, both for
% parametric perturbations to the concentration parameter as well as a particular
% family of nonparametric perturbations.  First, we provide theoretical conditions
% under which we have continuously differentiability of the VB optimum in
% \secref{local_sensitivity}, and show that they are satisfied for our BNP model.
% In \secref{computing_sensitivity}, we use the theoretical results of
% \secref{local_sensitivity} to show that the derivative can be computed
% efficiently in the BNP problems we are considering, with an emphasis on how
% modern automatic differentiation tools render many of the computations automatic
% \citep{jax2018github}.  Finally, in \secref{results}, we demonstrate the
% practical usefulness of the approximation on a set of three real-world BNP
% problems of increasing complexity.

%%%%%%%%%%%%%%%%%%%%%%%%%%%%%%%%%%%%%%%%%%%%%%%%%%%%%%%%%%%%%%%%%%%%%%%%%%%%%%%%
%%%%%%%%%%%%%%%%%%%%%%%%%%%%%%%%%%%%%%%%%%%%%%%%%%%%%%%%%%%%%%%%%%%%%%%%%%%%%%%%
\begin{ex}\exlabel{alpha_perturbation}
%
For the BNP model with the $\gem$ prior, take $\theta = (\nu_1, \ldots,
\nu_{\kmax-1})$, and take $\mu$ to be the Lebesgue measure on $[0,1]^{\kmax-1}$.
Let $\alpha_0$ be some base value of the concentration parameter, and
let $\t$ be $\alpha - \alpha_0$, so that deviations of $\t$ away from
$0$ represent deviations of $\alpha$ away from $\alpha_0$.

Expanding the KL divergence in \eqref{kl_def}, we see that the prior
$\p(\nuk \vert \alpha)$ enters the VB objective in a term of the form
$\sum_{\k=1}^\infty \expect{\q(\nuk \vert \eta)}{\log \p(\nuk \vert \alpha)}$.
Adding and subtracting the this term evaluated at $\alpha_0$ gives
% %
% \begin{align*}
% %
% \MoveEqLeft
% \KL{\q(\zeta \vert \eta) || \p(\zeta \vert \x, \alpha)}
% \\={}&    \expect{\q(\zeta \vert \eta)}{
%         \log \q(\zeta \vert \eta) - \logp(\x, \zeta)} + \logp(\x)
% \\={}&    \expect{\q(\zeta \vert \eta)}{\log \q(\zeta \vert \eta)} -
%         \expect{\q(\zeta \vert \eta)}
%                {\log \p(\x, \beta, \z \vert \nu )}
%                +\logp(\x)
% \\{}& -\sum_{\k=1}^{\kmax - 1}
%             \left(
%                 \expect{\q(\nuk \vert \eta)}{\log \p(\nuk \vert \alpha)} -
%                 \expect{\q(\nuk \vert \eta)}{\log \p(\nuk \vert \alpha_0)} +
%                 \expect{\q(\nuk \vert \eta)}{\log \p(\nuk \vert \alpha_0)}
%              \right)
% \\{}&
% - \sum_{\k = \kmax}^\infty
%     \expect{\q(\zeta \vert \eta)}{\log \p(\nuk \vert \alpha)}.
% %
% \end{align*}
% %
% Recall that the truncated VB approximation ignores the terms
% $\expect{\q(\zeta \vert \eta)}{\log \p(\nuk \vert \alpha)}$ for $\k \ge \kmax$,
% so that we can write the truncated objective as
%
\begin{align*}
%
\KL{\eta, \alpha} = \KL{\eta, \alpha_0}
-\sum_{\k=1}^{\kmax - 1}
            \left(
                \expect{\q(\nuk \vert \eta)}{\log \p(\nuk \vert \alpha)} -
                \expect{\q(\nuk \vert \eta)}{\log \p(\nuk \vert \alpha_0)}
             \right).
%
\end{align*}
%
Plugging in the definition of $\p(\nuk \vert \alpha)$, recognizing that the
normalizing constant does not depend on $\nuk$ and so can be neglected in the
optimization, letting $\KL{\eta} := \KL{\eta, \alpha_0}$, and substituting $\t =
\alpha - \alpha_0$ gives
%
\begin{align*}
%
\KL{\eta, \t} = \KL{\eta, \alpha_0}
-\t \sum_{\k=1}^{\kmax - 1}
    \expect{\q(\nuk \vert \eta)}{\log (1 - \nuk)}.
%
\end{align*}
%
\end{ex}
%%%%%%%%%%%%%%%%%%%%%%%%%%%%%%%%%%%%%%%%%%%%%%%%%%%%%%%%%%%%%%%%%%%%%%%%%%%%%%%%



%%%%%%%%%%%%%%%%%%%%%%%%%%%%%%%%%%%%%%%%%%%%%%%%%%%%%%%%%%%%%%%%%%%%%%%%%%%%%%%%
%%%%%%%%%%%%%%%%%%%%%%%%%%%%%%%%%%%%%%%%%%%%%%%%%%%%%%%%%%%%%%%%%%%%%%%%%%%%%%%%
\begin{ex}\exlabel{gem_mult_perturbation}
%
As in \exref{alpha_perturbation}, take $\theta = (\nu_1, \ldots, \nu_{\kmax-1})$
and $\mu$ to be the Lebesgue measure on $[0,1]^{\kmax-1}$. Let $\pbase(\nuk) :=
\betadist{\nuk \vert 1, \alpha_0}$, and let $\palt(\nuk)$ be a density, not
in the Beta family, that shifts mass towards zero:
%
\begin{align*}
%
\palt(\nuk) :=
    \frac{\exp(-\nuk)\pbase(\nuk)}{\int \exp(-\nuk')\pbase(\nuk') d\nuk'}.
%
\end{align*}
%
For $\t \in [0,1]$ define the multiplicatively perturbed prior
%
\begin{align*}
%
\p(\nuk \vert \t) :=
    \frac{\palt(\nuk)^{\t} \pbase(\nuk)^{1-\t}}
         {\int \palt(\nuk')^{\t} \pbase(\nuk')^{1-\t} d\nuk'}.
%
\end{align*}
%
When $\t = 0$, $\p(\nuk \vert \t) = \pbase(\nuk)$, when $\t = 1$,
$\p(\nuk \vert \t)  = \palt(\nuk)$, and $\p(\nuk \vert \t)$ varies smoothly
between the two for intermediate values of $\t$.

As in \exref{alpha_perturbation}, up to constants not depending on
$\lnuk$ we can write
%
\begin{align*}
%
\log \p(\nuk \vert \t) - \log \p(\nuk \vert \t=0) ={}&
    -\t \log \pbase(\nuk) + \t \log \palt(\nuk) + \const
\\={}& -\t \nuk + \const \Rightarrow
\\
\KL{\eta, \t} ={}& \KL{\eta} -\t \nuk + \const.
%
\end{align*}
%
Different choices for $\palt(\nuk)$ would give different additive
perturbations to the KL divergence.
%
\end{ex}
%
%%%%%%%%%%%%%%%%%%%%%%%%%%%%%%%%%%%%%%%%%%%%%%%%%%%%%%%%%%%%%%%%%%%%%%%%%%%%%%%%
