In this section, we return to the BNP problem and prove systematically that the
map $\alpha \mapsto \etaopt(\alpha)$ satisfies \assuref{kl_opt_ok,
exchange_order}, and so the conditions of \thmref{etat_deriv}.

Differentiability of $\KL{\eta}$ (\assuitemref{kl_opt_ok}{kl_diffable}) is
immediately satisfied for the $\eta$ that parameterize $\q(\beta \vert \eta)$
and $\q(\z \vert \eta)$ by our use of conjugate approximating families and
standard parameterizations.  For $\q(\nu \vert \eta)$, we will prove
\assuitemref{kl_opt_ok}{kl_diffable} on the way to proving that
\assuref{exchange_order} is satisfied.  Indeed,
\assuitemref{kl_opt_ok}{kl_diffable} is typically satisfied in VB problems; when
it is not, many black-box optimization methods also do not apply.

Non-singularity of the Hessian matrix $\hessopt$
(\assuitemref{kl_opt_ok}{kl_hess}) is satisfied whenever $\etaopt$ is at a local
optimum of $\KL{\eta}$.  In practice, we compute $\etaopt$ and (approximately)
check \assuitemref{kl_opt_ok}{kl_hess} numerically as part of computing the
sensitivity $\hessopt^{-1} \crosshessian$.  As with
\assuitemref{kl_opt_ok}{kl_diffable}, if \assuitemref{kl_opt_ok}{kl_hess} is
violated, then the user will probably have difficulty optimizing $\KL{\eta}$.

\assuitemref{kl_opt_ok}{kl_opt_interior} essentially requires that $\KL{\eta}$
be well-defined in an $\mathbb{R}^\etadim$ neighborhood of $\etaopt$, and can
require some care in choosing the parameterization $\eta$.  As an example of a
parameterization that would violate \assuitemref{kl_opt_ok}{kl_opt_interior},
consider parametrizing $\q(\z_{\n} \vert \eta)$ by the $\kmax$ expectations
$m_\k := \expect{\q(\z_{\n} \vert \eta)}{\z_{\n\k}}$.  The set $(m_1, \ldots,
m_\kmax)$ completely specify $\q(\z_{\n} \vert \eta)$, but violate
\assuitemref{kl_opt_ok}{kl_opt_interior}, since any valid parametization
satisfies $\sum_{\k=1}^\kmax m_\k = 1$, and so no open ball in
$\mathbb{R}^\etadim$ can be contained in $\etadom$.  However,
\assuitemref{kl_opt_ok}{kl_opt_interior} is satisfied we use an {\em
unconstrained parameterization} for $\q(\zeta \vert \eta)$.   Unconstrained
parameterizations of variational distributions allow the use of unconstrained
optimization for variational inference and are a good practice when available
\citep{kucukelbir:2016:advi}.  For details on our parameterizations, see
\appref{app_vb_details}.

Verifying \assuref{exchange_order} is the principal technical challenge of
satisfying the conditions of \thmref{etat_deriv}. Recall from \secref{model_vb}
that we take $\q(\nuk \vert \eta)$ to be a normal distribution on the
logit-transformed sticks, $\lnu_\k$.  For the duration of this section, write
$\q(\lnuk \vert \eta) = \normdist{\lnuk \vert \mu_\k, \sigma^2_\k}$, so that the
subvector of $\eta$ parameterizing $\q(\lnuk \vert \eta)$ is $\etanuk = (\mu_\k,
\sigma_\k)$.  Derivatives with respect to any components of $\eta$ are zero and
so \assuref{exchange_order} is trivially satisfied.

By the formula for transformation of probability densities,
%
\begin{align*}
%
\q(\nuk \vert \etanuk) =
    \normdist{\log\left(\frac{\nu_\k}{1 - \nu_\k} \right)
        \Big\vert  \mu_\k, \sigma^2_\k}
    \frac{1}{\nuk (1 - \nuk)},
%
\end{align*}
%
where we have used the fact that $\fracat{d \lnu_\k}{ d\nuk}{\nuk} =
\frac{1}{\nuk (1 - \nuk)}$.  Similarly, for any function $f(\nuk)$ of the stick
lengths, we can transform the expecations as $\expect{\q(\nuk \vert
\etanuk)}{f(\nuk)} = \expect{\q(\lnuk \vert \etanuk)}{f\left(
\frac{\exp(\lnuk)}{1 + \exp(\lnuk)}  \right))}$, using the fact that
$\nuk = \frac{\exp(\lnuk)}{1 + \exp(\lnuk)}$.

Recall from \exref{alpha_perturbation} that $\log \ptil(\nuk \vert \t) = t \log
(1 - \nuk)$, so we need to establish \assuref{exchange_order} for
%
\begin{align*}
%
-\expect{\q(\nuk \vert \etanuk)}{t \log (1 - \nuk)} =
% -\expect{\q(\lnuk \vert \etanuk)}
%        {t \log (1 - \frac{\exp(\lnuk)}{1 + \exp(\lnuk)})} =
\expect{\q(\lnuk \vert \etanuk)}
      {t \log (1 + \exp(\lnuk))}.
%
\end{align*}
%
Since the preceding equality holds for all $\t$ and $\etanuk$, it suffices to
establish that we can exchange the order of integration and differentiation
for the right hand side.

Furthermore, to establish \assuitemref{kl_opt_ok}{kl_diffable}, we must show
that the entropy $\expect{\q(\nuk \vert \etanuk)}{\log \q(\nuk \vert \etanuk)}$
is continuously differentiable; since $\q(\nuk \vert \etanuk)$ is not a standard
exponential family, this is not trivial.

% \begin{align}\eqlabel{lnuk_derivatives}
% %
% \fracat{d \lnu_\k}{ d\nuk}{\nuk} ={}
% %     \frac{1-\nuk}{\nuk}
% %     \left(\frac{1}{1 - \nuk} + \frac{\nuk}{(1 - \nuk)^2} \right)
% % \\={}& \frac{1}{\nuk} + \frac{1}{1 - \nuk}
% % \\={}&
%     \frac{1}{\nuk (1 - \nuk)} \mathand
% %
% \fracat{d \nuk}{ d\lnuk}{\lnuk} ={}
%     \frac{\exp(\lnuk)}{(1 + \exp(\lnuk))^2}.
% %
% \end{align}



%%%%%%%%%%%%%%%%%%%%%%%%%%%%%%%%%%%%%%%%%%%%%%%%%%%%%%%%%%%%%%%%%%%%%%%%%%%%
%%%%%%%%%%%%%%%%%%%%%%%%%%%%%%%%%%%%%%%%%%%%%%%%%%%%%%%%%%%%%%%%%%%%%%%%%%%%
\begin{lem}\lemlabel{normal_q_is_regular}
%
In the setting of \assuref{dist_fun_nice}, let $\theta \in \mathbb{R}$, let
$\mu$ be the Lebesgue measure, and let $\q(\theta \vert \eta) = \normdist{\theta
\vert \eta}$ be a normal distribution parameterized by its natural exponential
family parameters.

Let $\sigma(\eta)$ denote the standard deviation of the normal distribution.
Fix $\eta_0$ such that $\sigma(\eta_0) > 0$, and let $\ball_\eta$ be an open set
containing $\eta_0$ such a that $\sigma_{max} := \sup_{\eta \in \ball_\eta}
\sigma(\eta) < \infty$.

If there exists a neighborhood $\ball_\t$ of $\t_0$ such that $\abs{\psi(\theta,
\t)}$ and $\abs{\psigrad{\theta, \t}}$ are uniformly bounded by some multiple of
$\exp(-\abs{\theta})$ for all $\t \in \ball_\t$, then $\q$ and $\psi$ satisfy
\assuref{dist_fun_nice}.

\begin{proof}
%
By properties of the exponential family,
%
\begin{align*}
%
\lqgrad{\theta, \eta} ={} (\theta, \theta^2)^T \mathand&
\lqhess{\theta, \eta} ={} 0_{2\times2} \Rightarrow\\
%
\norm{\lqgrad{\theta, \eta}}_2^2 ={} \theta^2 + \theta^4 \mathand&
\norm{\lqhess{\theta, \eta}}_2 ={} 0.
%
\end{align*}
%
Let $\ballclosed_\eta$ denote the closure of $\ball_\eta$, and let
%
\begin{align*}
%
\eta^* := \argmax_{\eta \in \ballclosed_\eta}
    \expect{\q(\theta \vert \eta)}{\exp(\abs{\theta})}.
%
\end{align*}
%
By standard properties of the normal and the boundedness of $\sigma(\eta)$, the
right hand side of the preceding display is finite.
%
Then
%
\begin{align*}
\int \q(\theta \vert \eta) \psi(\theta, \t) \mu(d \theta) \le{}&
    \left( \sup_{\theta} \sup_{\t \in \ball_\t}
        \abs{\psi(\theta, \t)} \exp(-\abs{\theta}) \right)
    \int \q(\theta \vert \eta) \exp(\theta) \mu(d \theta)
%
\\\le{}&
    \const
    \expect{\q(\theta \vert \eta^*)}{\exp(\abs{\theta})}.
    \quad\constdesc{\eta, \t}
%
\end{align*}
%
Therefore, for \assuitemref{dist_fun_nice}{fundom}, we can take $M(\theta)
\propto \q(\theta \vert \eta^*) \exp(\abs{\theta})$. The other terms follow
similarly, since each multiplier of $\q(\theta \vert \eta)$ is dominated by
$\exp(-\abs{\theta})$.  The final $M(\theta)$ simply takes the largest
of the five constants.
%
\end{proof}
%
\end{lem}
%%%%%%%%%%%%%%%%%%%%%%%%%%%%%%%%%%%%%%%%%%%%%%%%%%%%%%%%%%%%%%%%%%%%%%%%%%%%


%%%%%%%%%%%%%%%%%%%%%%%%%%%%%%%%%%%%%%%%%%%%%%%%%%%%%%%%%%%%%%%%%%%%%%%%%%%%
%%%%%%%%%%%%%%%%%%%%%%%%%%%%%%%%%%%%%%%%%%%%%%%%%%%%%%%%%%%%%%%%%%%%%%%%%%%%
\begin{ex}\exlabel{gem_pert_ok}
%
To analyze \exref{alpha_perturbation}, we take $\theta$
in \lemref{normal_q_is_regular}
be the unconstrained stick-breaking proportion $\lnuk$, which
recall from \secref{stick_expectations}
was normally distributed under $\q$.
Let $\mu$ be the Lebesgue measure on $\mathbb{R}$.

In \exref{alpha_perturbation},
the parameterization of the stick-breaking distribution was given by
\begin{align*}
  \log \pstick(\nuk \vert \t) - \log \pstick(\nuk \vert \t=0) ={}&
  \t \log(1 - \nuk).
\end{align*}
%
Expressing the perturbation in terms of
$\lnuk$,
%
\begin{align*}
%
\log \pstick(\lnuk \vert \t) - \log \pstick(\lnuk \vert \t=0) ={}&
\t \log\left(1 - \frac{\exp(\lnuk)}{1 + \exp(\lnuk)}\right)
\\={}&
-\t \log\left(1 + \exp(\lnuk)\right).
%
\end{align*}
%
So, by \lemref{normal_q_is_regular}, \assuref{q_stick_regular} is satisfied with
the VB approximation given in \secref{stick_expectations} and the parametric
perturbation given in \exref{alpha_perturbation}.
%
\end{ex}
%%%%%%%%%%%%%%%%%%%%%%%%%%%%%%%%%%%%%%%%%%%%%%%%%%%%%%%%%%%%%%%%%%%%%%%%%%%%



%%%%%%%%%%%%%%%%%%%%%%%%%%%%%%%%%%%%%%%%%%%%%%%%%%%%%%%%%%%%%%%%%%%%%%%%%%%%
%%%%%%%%%%%%%%%%%%%%%%%%%%%%%%%%%%%%%%%%%%%%%%%%%%%%%%%%%%%%%%%%%%%%%%%%%%%%

\begin{cor}\corlabel{gem_approximation_ok}
%
For the variational approximation of \secref{model_vb} and perturbation
given in \exref{alpha_perturbation}, $\alpha \mapsto \etaopt(\alpha)$
is continuously differentiable.
%
\begin{proof}
%
We have already shown in \exref{gem_pert_ok} that \assuref{q_stick_regular} is
satisfied.  Given that the variational approximations to $\p(\z \vert \x, \beta,
\nu)$ and $\p(\beta \vert \x, \nu, \z)$ are conjugate exponential family
approximations, $\eta \mapsto \KL{\eta, 0}$ is continuous.  It remains only to
numerically find $\etaopt$ and verify \assuitemref{kl_opt_ok}{kl_hess}, i.e.
that the Hessian is positive definite at the optimum.
%
\end{proof}
%
\end{cor}
