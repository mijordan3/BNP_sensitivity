The generative process was described in the main text (\secref{results_structure}).
We detail here the variational approximation.
Like in all our examples, the variational distribution is mean-field:
\begin{align*}
\q(\zeta \vert \eta) =
    \left(
    \prod_{\n=1}^{\nindiv}\prod_{\k=1}^{\kmax - 1}
    \q(\nu_{nk} \vert \eta) \right)
    \left(\prod_{\k=1}^{\kmax}\prod_{l=1}^{\nloci}
    \q(\latentpop_{\k l} \vert \eta) \right)
    \left( \prod_{\n=1}^{\N} \prod_{l=1}^{\nloci} \prod_{i=1}^{2} \q(\z_{\n l i} \vert \eta) \right).
\end{align*}
We let all distributions be conditionally conjugate except for the sticks,
which are logit-normal.
Each membership indicator $\z_{\n l i}$ is categorical, and the
allele frequencies $\latentpop_{\k l}$ are Dirichlet distributed.

In this model, we still call $(\beta, \nu)$ the global latent variables, even though they scale
with the number of individuals $\N$;
they do not, however, scale with both the number of individuals and the number of loci
like $\z$ does. Thus, we call $\z$ the local latent variables.
The local variational parameters $\eta_\z$ can be set optimally in
an analagous way as \exref{qz_optimality}, except with the
indices $\n\k$ replaced with $\n\l\i\k$.

The posterior quantities of interest in this application are the admixtures
$\pi_\n$. \figref{stru_init_fit} plots the inferred admixtures
$\expect{\q(\pi_\n \vert \etaopt)}{\pi_\n}$ for all individuals $\n$.

In the approximate posterior with $\alpha_0 = 3$, there appear to be three dominant
latent populations, which we arbitrarily label as populations 1, 2, and 3
(\figref{stru_init_fit}). The inferred admixture proportions generally
correspond with geographic regions: Mbololo individuals are primarily population
1; Ngangao individuals are primarily population 2; and Chawia individuals are a
mixture of populations 1, 2, and 3.

\StructureInitialFit
