\todo{This abstract needs to be re-done to match the new outline}
A Bayesian nonparametric approach to clustering treats the number of distinct
clusters in a data set as a random quantity by modeling an infinite number of
components.  It therefore avoids specification of the number clusters \textit{a
priori} and allows for inference about the number clusters to be done through
the posterior distribution. However, a Bayesian nonparametic model requires the
specification of a prior density, and it is important in practice to
establish the the sensitivity of any posterior inferences to possibly arbitrary
prior choices. We derive local sensitivity measures based on Taylor series
approximations for a truncated variational Bayes (VB) approximation based on the
Kullback-Leibler divergence. In addition, we show that local sensitivity can be
expressed as a functional derivative in an $L_p$ vector space of integrable
functions. The linear derivative operator, in this context called an
\textit{influence function}, can guide our search for prior perturbations with
high senesitivity. To safely search through functional space however, the VB
posterior requires more careful consideration than the exact posterior: while
the exact posterior is Fr{\'e}chet differentiable for all $p$, the VB posterior
is Fr{\'e}chet differentiable only with multiplicative perturbations and
$p=\infty$. We test our local approximation on real-world datasets and show that
it can be an accurate way to quickly assess robustness to both parametric and
non-parametric prior perturbations.










% Two central questions in many probabilistic clustering problems is how many
% distinct clusters are present in a particular dataset, and which observations
% cluster together.  Bayesian nonparametrics (BNP) addresses this question by
% placing a generative process on cluster assignment, making the number of
% distinct clusters present amenable to Bayesian inference. However, like all
% Bayesian approaches, BNP requires the specification of a prior, and this prior
% may favor a greater or fewer number of distinct clusters.  In practice, it is
% important to quantitatively establish that the prior is not too informative,
% particularly when---as is often the case in BNP---the particular form of the
% prior is chosen for mathematical convenience rather than because of a considered
% subjective belief.
%
% We derive local sensitivity measures based on Taylor series approximations for a
% truncated variational Bayes approximation based on the Kullback-Leibler
% divergence.  Using a stick-breaking representation of a Dirichlet process, we
% consider perturbations both to the scalar concentration parameter and to the
% functional form of the stick-breaking distribution by embedding the
% stick-breaking density in the $L_p$ vector spaces of integrable functions.  We
% prove that, though the variational optimum is directional differentiable for all
% $1 \le p \le \infty$, the derivative provides a uniformly good approximation in
% a neighborhood of the original prior only with multiplicative perturbations and
% $p=\infty$.
%
% We apply our methods to several real-world datasets, estimating sensitivity to
% the BNP prior specification of key posterior quantities, and evaluating the
% accuracy of our approximations by comparing to the much more expensive process
% of re-fitting the model.  In the design and evaluation of our local sensitivity
% measures we pay special attention to our ability to accurately extrapolate to
% different priors, rather than treating the sensitivity as a measure of
% robustness {\em per se}.  We show the accuracy of our approximation both for
% particular parametric and non-parametric perturbations, and demonstrate the
% usefulness of the influence function to find maximally influential alternative
% priors.
