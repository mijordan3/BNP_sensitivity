Two central goals of many clustering problems are inferring how many distinct
clusters are present in a particular dataset and which observations cluster together.
A Bayesian nonparametric (BNP) approach to clustering assumes an infinite
number of \textit{components},
of which a finite number are present in the data as \textit{clusters}.
Being a Bayesian approach, BNP places a generative process
on cluster assignment,
and inference about any quantity of interest,
such as number of distinct clusters present in a data set,
is entirely defined by the posterior distribution.
However, like all Bayesian approaches, BNP requires the specification
of a prior, and this prior may favor a greater or fewer number of distinct clusters.
In practice, it is important to establish that the prior is not too informative, particularly
when---as is often the case in BNP---the particular form of the prior is chosen for
mathematical convenience rather than because of a considered subjective belief.

The posterior in a BNP model cannot be calcluated analytically,
and thus approximate methods are required in practice.
We do inference using variational Bayes (VB), which posits a constrained
family of distributions, and uses optimization to find
the member of the family that is closest to the true posterior in
Kullback-Leibler ($\mathrm{KL}$) divergence.
Typically, the family of distributions is parameterized by a real-valued vector,
in which case minimizing distribution can be found using
standard numerical optimization techniques.
VB is a popular method on large-scale data sets
because solving the numerical optimization problem can be much faster than
running Markov chain Monte Carlo (MCMC).
VB is particularly apt in the applications we consider,
as they are primarily exploratory in nature, and quick, approximate solutions
suffice.
For these applications, we thus would like a similarly quick, possibly approximate,
method to assess
the sensitivity of the VB posterior to prior choices.
Assessing sensitivity is especially important
when the results of these exploratory analyses will be guiding further
downstream investigation.

A simple way to assess sensitivity would be to refit the VB posterior for
several different prior choices.
However, repeatedly solving for variational optima after each prior choice
may be unnecessarily expensive, particularly for exploratory settings where
approximate optima might suffice.
Moreover, it is impossible to refit the model for every possible prior
choice. 




In VB, the approximate posterior is characterized by a real-valued vector which
is the solution to a numerical optimization problem.
The optimization objective is defined as the Kullback-Leibler ($\mathrm{KL}$) divergence
between .
Thus, the variational posterior depends on prior parameters through the $\mathrm{KL}$
optimization,
and assessing sensitivity amounts to characterizing the dependence of
the VB optimum on the prior choices.
We derive local sensitivity measures to approximate this dependence,
which may be nonlinear,
with a local Taylor series approximation
\citep{gustafson:1996:local, giordano:2018:covariances}.

Using a stick-breaking representation of a Dirichlet process (DP), we
let the stick-breaking density be parameterized by a scalar hyperparameter.
For example, this hyperparameter might be the scalar concentration parameter in a DP.
We will also consider sensitivity to the functional form stick-breaking density
by embedding the
stick-breaking density in the $L_p$ vector spaces of integrable functions.
In this case, the hyperparameter parameterizes a path between
an initial stick-breaking density and a perturbed density.
Local sensitivity in a VB approximation is defined as the derivative
of the VB optimum with respect to the hyperparameter.
We prove that, though the variational optimum is directional differentiable for all
$1 \le p \le \infty$, the derivative provides a uniformly good approximation in
a neighborhood of the original prior only with multiplicative perturbations and
$p=\infty$.




The variational approach is particularly suitable for the
sensitivity calculations we derive.
Because the approximate posterior is characterized by a real-valued vector,
which in turn is characterized as a fixed point to an optimization
objective, the derivative of the VB optimum with respect to any prior
parameter can be computed in closed form.
Using modern automatic differentiation tools, computing local sensitivity
in practice requires almost no additional code beyond
implementing the optimization objective itself.


We apply our methods to several real-world datasets, estimating the sensitivity
of key posterior quantities to
the BNP prior specification.
Notably, we go beyond previous work on local Bayesian sensitivity (e.g. \citet{basu:1996:local})
which treated sensitivity as a measure of robustness \textit{per se}.
Rather, in the design and evaluation of our local sensitivity
measures we pay special attention to our ability
to accurately extrapolate posterior inferences to different priors.
We show the accuracy of our local approximation both for
parametric and nonparametric perturbations by comparing
against the much more expensive process of refitting the variational posterior.
The speed of the local approximation allows rapid exploration of a wide range of
possible perturbations.

Even with the speed of our approximation,
testing sensitivity for \textit{all} possible prior perturbations
is impossible.
However, we also show that the local sensitivity takes the form of an inner product,
in an appropriate Hilbert space, between an \textit{influence function}
and a prior perturbation;
we demonstrate how to use the influence function to guide our
search for prior perturbations that result in high sensitivity.
In particular, we can use the influence function to find maximally influential alternative
priors.

\secref{model_bnp} details the stick-breaking construction of Dirichlet process priors.
\secref{model_vb} outlines our truncated variational approximation.
\secref{local_sensitivity} and \secref{functional_perturbations} presents our
local approximation for parametric
and functional perturbations, respectively.
\secref{computing_sensitivity} discusses how computing the sensitivity is done in practice.

In our results (\secref{results}),
we first demonstrate our local sensitivity methodology on
a toy example, a Gaussian mixture model
of Fisher's iris data set.
Then, we apply local sensitivity to real data analysis problems, which include a
regression model of time-course gene expression data,
and a topic model for studying population structure in a genetic database.
We find that posterior quantities can be sensitive or non-sensitive,
depending on both the application and the quantity of interest.
In most cases, the local approximation well-approximates the results found under
refitting.
We empirically observe that computing the local sensitivity
can be an magnitude faster than refitting.
We also discuss some limitations of local sensitivity
and present scenarios where it fails to be a good approximation to refitting.
\secref{bnp_conclusion} concludes.
