% This file is for tex not in the paper but which I don't want to delete
% just yet in case we want it later.%
We consider a few cases, depending on how $\KL{\eta, \t}$ is evaluated.
For illustration, we consider the following simple example.

\begin{ex}\exlabel{q_normal}
%
Let $\x \in \mathbb{R}$, $\theta \in \mathbb{R}$, $p(\x \vert \theta) = \normdist{\x
\vert \theta, 1}$, and $p(\theta \vert \t) = \normdist(\theta \vert \t, 1)$.  Then
%
\begin{align*}
%
\logp(\x \vert \theta) ={}&
    -\frac{1}{2} \left(\theta^2 - 2 \theta \x \right) + \const
    & \constdesc{\theta} \\
\logp(\theta \vert \t)={}&
    -\frac{1}{2} \left(\theta^2 - 2 \theta \t \right) + \const
    & \constdesc{\theta} \Rightarrow  \\
\logp(\x \vert \theta) + \logp(\theta \vert \t)={}&
    \theta(\x - \t) - \theta^2 + \const.
    & \constdesc{\theta}.
%
\end{align*}
%
Let $\eta = (\mu, \sigma)^T$ and $\q(\theta \vert \eta) = \normdist{\theta \vert \mu,
\sigma^2}$, so that
%
\begin{align*}
%
\log \q(\theta \vert \eta) &=
    -\frac{1}{2} \sigma^{-2} (\theta - \mu)^2 -\frac{1}{2} \log \sigma^2 +
    \const.&\constdesc{\theta, \eta}
%
\end{align*}
%
\end{ex}

In the simplest case, we can evaluate $\expect{\q(\theta \vert \eta)}{ f(\theta,
\t) }$ explicitly as a funciton of $\eta$, in which case we can compute
the needed derivatives exactly.

\begin{ex}\exlabel{q_normal_exact}
%
In \exref{q_normal}, we can compute
TODO(fix this)
%
\begin{align*}
%
\expect{\q(\theta \vert \eta)}
       {-\logp(\x \vert \theta) - \logp(\theta \vert \t)} ={}&
\expect{\q(\theta \vert \eta)}{\theta(\t - \x) + \theta^2 + \const}
    %& \constdesc{\theta}
\\={}&
\mu (\x - \t) - \mu^2 - \sigma^2  + \const %& \constdesc{\eta}
%
\end{align*}
%
and
%
\begin{align*}
%
\expect{\q(\theta \vert \eta)}{\log \q(\theta \vert \eta)} ={}&
\expect{\q(\theta \vert \eta)}
       {-\frac{1}{2} \sigma^{-2} (\theta - \mu)^2 -\frac{1}{2} \log \sigma^2 +
        \const} %&\constdesc{\theta, \eta}
\\={}& \frac{1}{2} \log \sigma^2 + \const. %&\constdesc{\eta}.
%
\end{align*}
%
So
%
\begin{align*}
%
\KL{\eta, \t} ={}&
    \mu (\x - \t) - \mu^2 - \sigma^2 + \frac{1}{2} \log \sigma^2 +
    \const,
    &\constdesc{\eta}
%
\end{align*}
%
and we can directly compute
%
\begin{align*}
%
\fracat{\partial \KL{\eta, \t}}{\partial\eta}{\eta, \t} ={}&
    \left(\begin{array}{c}
    \x - \t - 2\mu\\
    -2\sigma + \sigma^{-1}
    \end{array}\right)\\
\fracat{\partial^2 \KL{\eta, \t}}{\partial\eta \partial \eta^T}{\eta, \t} ={}&
    \left(\begin{array}{cc}
    -2          &           0\\
    0           &           -2 - \sigma^{-2}
    \end{array}\right)\\
%
\end{align*}
%


%
\end{ex}






Observe that
%
\begin{align*}
%
\log \pstick(\nuk \vert \phi) ={}&
    \log \pb(\nuk) + \phi(\nuk) + \const
    & \constdesc{\nuk} \Rightarrow\\
\KL{\eta, \phi} ={}&
    \KL{\eta, 0} + \sumkm \expect{\q(\nuk \vert \eta)}{\phi(\nuk)}.
%
\end{align*}
%
Let $\phiz(\cdot)$ denote the zero function.  Then, using \eqref{vb_eta_sens}
gives that the directional (Gateaux) derivative in the direction $\phi$
evaluated at $\phiz$ is given by
%
\begin{align*}
%
\fracat{d \etaopt(\phi)}{d \phi}{\phiz} ={}&
    - \hess{\zeta\zeta}^{-1}
    \evalat{
        \sumkm \frac{\partial}{\partial \eta}
            \expect{\q(\nuk \vert \eta)}{\phi(\nuk)}}
           {\etaopt(\phiz), \phiz}.
%
\end{align*}
%
Differentiating under the integral in $\expect{\q(\nuk \vert
\eta)}{\phi(\nuk)}$ gives
%
\begin{align*}
%
\evalat{
\frac{\partial}{\partial \eta}
    \expect{\q(\nuk \vert \eta)}{\phi(\nuk)}
}{\etaopt} ={}&
\expect{\q(\nuk \vert \etaoptnuk)}
       {\evalat{\frac{\partial}{\partial \etanuk}
                  \log\q(\nuk \vert \etanuk)}
                {\etaoptnuk}
        \phi(\nuk)} \\
={}&
\int_0^1
    \q(\nu \vert \etaoptnuk)
    \evalat{\frac{\partial}{\partial \etanuk}
               \log\q(\nu \vert \etanuk)}
             {\etaoptnuk}
    \phi(\nu) d\nu.
%
\end{align*}
%
Plugging in gives
%
\begin{align*}
%
\fracat{d \etaopt(\phi)}{d \phi}{\phiz} =&{}
    \int_0^1
    -\hess{\zeta\zeta}^{-1}
    \left(
        \sumkm
        \q(\nu \vert \etaoptnuk)
        \fracat{\partial \log\q(\nu \vert \etanuk)}
               {\partial \etanuk}
               {\etaoptnuk}
    \right) \phi(\nu) d\nu.
%
\end{align*}
%
Thus, the influence function for $\etaopt(\phi)$ at $\phiz$ is given by
%
\begin{align*}
%
\infl(\nu) :={}&
-\hess{\zeta\zeta}^{-1}
\left(
    \sumkm
    \q(\nu \vert \etaoptnuk)
    \fracat{\partial \log\q(\nu \vert \etanuk)}
           {\partial \etanuk}
           {\etaoptnuk}
\right) \\
\fracat{d \etaopt(\phi)}{d \phi}{\phiz} =&{}
    \int_0^1 \infl(\nu) \phi(\nu) d\nu.
%
\end{align*}
%
Analogously, for a differentiable function of interest $g(\eta)$,
the influence function is given via \eqref{vb_g_sens} to be
%
\begin{align*}
%
\inflg(\nu) :={}&
-\fracat{\partial g(\eta)}{\partial \eta^T}{\etaopt(\t_0)}
    \hess{\zeta\zeta}^{-1}
\left(
    \sumkm
    \q(\nu \vert \etaoptnuk)
    \fracat{\partial \log\q(\nu \vert \etanuk)}
           {\partial \etanuk}
           {\etaoptnuk}
\right) \\
\fracat{d g(\etaopt(\phi))}{d \phi}{\phiz} =&{}
    \int_0^1 \inflg(\nu) \phi(\nu) d\nu.
%
\end{align*}
%
The influence funciton is a convenient summary of the effect of functional prior
perturbations on a differnetiable summary statistic, as we will show below.

An additional benefit of the influence function is that it admits a closed-form
expression for the ``worst-case'' perturbation in an $\norminf{\cdot}$ ball
via Holder's inequality, since
%
\begin{align*}
%
\sup_{\phi: \norminf{\phi} \le \delta}
    \fracat{d g(\etaopt(\phi))}{d \phi}{\phiz} =&{}
\sup_{\phi: \norminf{\phi} \le \delta}
    \int_0^1 \inflg(\nu) \phi(\nu) d\nu \\
\le&{} \delta \int_0^1 \abs{\inflg(\nu) }d\nu,
%
\end{align*}
%
with equality when $\phi(\nu) = \delta \, \mathrm{sign}(\inflg(\nu))$.

In order for the supremum over a bounded set $\phi: \norminf{\phi} < \delta$
to be meaningful, a minimal requirement is that the function $\etaopt(\phi)$
be Fr{\'e}chet differentiable.





















Let us define some compact notation for the duration of the proof.
Let
%
\begin{align*}
%
\KLgrad{\eta, \phi} :={}&
    \fracat{\partial \KL{\eta, \phi}}{\partial \eta}{\eta, \phi}
\mathtxt{and}
\KLhess{\eta, \phi} :={}&
    \fracat{\partial^2 \KL{\eta, \phi}}
           {\partial \eta \partial \eta^T}{\eta, \phi}.
%
\end{align*}
%
Additionally, define $\rho(\etanuk, \phi) :={} \expect{\q(\nu \vert
\etanuk)}{\phi(\nu)}$, with
%
\begin{align*}
%
\rho_\eta(\etanuk, \phi) :={}
    % \fracat{\partial \expect{\q(\nu \vert \etanuk)}{\phi(\nu)}}
    %        {\partial \etanuk}{\etanuk} \mathand
\fracat{\partial \rho(\etanuk, \phi)}
       {\partial \etanuk}{\etanuk} \mathand
\rho_{\eta\eta}(\etanuk, \phi) :={}
   \fracat{\partial^2 \rho(\etanuk, \phi)}
          {\partial \etanuk \partial \etanuk^T}{\etanuk}.
%
\end{align*}

Observe that the estimating equation can be written as
%
\begin{align*}
%
% \KL{\eta, \phi} ={}&
%     \KL{\eta, \phiz} + \sumkm \rho(\etanuk, \phi)
%     \Rightarrow \\
%
\KLgrad{\eta, \phi} ={}&
\KLgrad{\eta, \phiz} + \sumkm \rho_\eta(\etanuk, \phi)
= 0_\etadim.
%
\end{align*}

Let us first show that $\KLgrad{\eta, \phi}$ is continuously Fr{\'e}chet
differentiable.  By \assuitemref{kl_opt_ok}{kl_diffable}, $\eta \mapsto
\KLgrad{\eta, \phiz}$ is continuously differentiable and does not depend on
$\phi$, so we need only to show that $\rho_\eta(\etanuk, \phi)$ is continuously
Fr{\'e}chet differentiable. To show this, it will suffice to show that the
partial deritatives of the maps $\eta \mapsto \rho_\eta{\eta, \phi}$ and $\phi
\mapsto \rho_\eta{\eta, \phi}$ exist and are continuous in the dual norm
$\norm{\cdot}^*$ as a function of the location $\eta, \phi$ at which the
derivatives are evaluated.

We first show that $\phi \mapsto \rho_\eta(\etanuk, \phi)$ is Fr{\'e}chet
differentiable.  To do so it


It follows from \eqref{q_sens_is_cov} that the map $\phi \mapsto
\rho_\eta(\etanuk, \phi)$ is a linear functional of $\phi$.  Applying Jensen's
inequality to the $\norm{\cdot}_2$ norm and the triangle inequality twice gives:
%
\begin{align*}
%
%\sup_{\phi: \norminf{\phi} = 1}
\norm{\rho_\eta(\etanuk, \phi)}_2 \le
    4 \expect{\q(\nu \vert \etanuk)}
             {\norm{\nabla \log \q(\nu \vert \etanuk)}_2}
              \norminf{\phi(\nu)}.
%
\end{align*}
%
It follows from ... that $\rho_\eta(\etanuk,
\phi)$ is bounded as a function of $\phi$ for all $\eta \in \ball_\eta$.
Bounded linear functionals are Fr{\'e}chet differentiable (essentially by
definition), so $\phi \mapsto \rho_\eta(\etanuk, \phi)$ is Fr{\'e}chet
differentiable.

It remains to show that the derivative is continuous.

% so the linear operator is bounded as a function of $\phi$, and a continuous
% funciton of $\eta$ by assumption.  Therefore $\phi \mapsto \fracat{\partial
% \expect{\q(\nu \vert \etanuk)} {\phi(\nu)}}{\partial \eta}{\eta}$ is
% continuously Fr{\'e}chet differentiable.
%
% Again differentiating under the integral,
% %
% \begin{align*}
% %
% \fracat{\partial^2 \expect{\q(\nu \vert \etanuk)}
%               {\phi(\nu)}}{\partial \eta \partial \eta^2}{\eta} ={}&
% \expect{\q(\nu \vert \etanuk)}
%        {\left(\nabla \log \q(\nu \vert \etanuk)
%          - \expect{\q(\nu \vert \etanuk)}
%                   {\nabla \log \q(\nu \vert \etanuk)}
%        \right)^2
%        \left( \phi(\nu) - \expect{\q(\nu \vert \etanuk)}{\phi(\nu)} \right)} +
% \\&
% \expect{\q(\nu \vert \etanuk)}{
%        \left(                            \nabla^2 \log \q(\nu \vert \etanuk)
%         - \expect{\q(\nu \vert \etanuk)}{\nabla^2 \log \q(\nu \vert \etanuk)}
%        \right)
%        \left( \phi(\nu) - \expect{\q(\nu \vert \etanuk)}{\phi(\nu)} \right)
%        }.
% %
% \end{align*}
% %
Again this is a linear operator as a function of $\phi$, which is continuous by
assumption.  Consequently the Hessian is invertible in a neighborhood.

It follows by \citet[Proposition 4.14(c)]{zeidler:2013:functional} that
$\KLgrad{\eta, \phi}$ is continuosly Fr{\'e}chet differentiable.  Furthermore,
\citet[Chapter 4 Condition 21b]{zeidler:2013:functional} holds since
$\KLhess{\etaopt(\phiz), \phiz}$ is invertible.  So we satisfy conditions (i),
(ii), and (iii) of \citet[Theorem 4.B(c)]{zeidler:2013:functional}, giving that
the function $\etaopt(\phi)$ exists.  Moreover, since $\KLgrad{\eta, \phi}$ is
continuously Fre{\'e}chet differentiable in a neighborhood of $\etaopt(\phiz),
\phiz$, by \citet[Theorem 4.B(d)]{zeidler:2013:functional}, $\etaopt(\phi)$ is
also continuously Fr{\'e}chet differentiable.

The directional derivative

% Other references: We will use the
% formulation given in \citep[Theorem 3.4.10]{krantz:2012:implicit}, for which
% we must show that... actually that doesn't give differentiability.
% The result follows then from \citet[Proposition 4.8(c)]{zeidler:2013:functional}.
% See also \citet[Corollary 1.4]{averbukh:1967:theory} and \citep[Appendix
% A]{reeds:1976:thesis}).
%
