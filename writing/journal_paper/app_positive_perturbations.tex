
The following example illustrates how, by requiring pertubations to be positive,
one can induce counterintuitive notions of the ``size'' of a perturbation that
ablates prior mass.

%%%%%%%%%%%%%%%%%%%%%%%%%%%%%%%%%%%%%%%%%%%%%%%%%%%%%%%%%%%%%%%%%%%%%%%%%
\begin{ex}\exlabel{positive_pert_large}
%
\SimPositivePertFig

Take $\mu$ to be the Lebesgue measure on $[0,1]$. Let $\pbase(\theta) = \ind{0
\le \theta \le 1}$.  For some $\delta > 0$ and $0 < \epsilon \ll 1$, let
%
\begin{align*}
%
\palt(\theta) :={}&
    \left(\frac{1-\delta \epsilon}{1 - \epsilon} \right)
        \ind{\epsilon \le \theta \le 1} +
    \delta \ind{0 \le \theta \le \epsilon}.
%
\end{align*}
%

We can use \eqref{p_pert_simple} to give $\ptil(\theta \vert \tp=1) = \palt(\theta)$ by using, for any $\alpha > 0$,
%
\begin{align*}
%
\phi(\theta) ={}&
    \left( \alpha\left(\frac{1-\delta \epsilon}{1-\epsilon} \right)^{1/p}
        - 1
    \right)
        \ind{\epsilon \le \theta \le 1} +
    \left(\alpha \delta^{1/p} - 1 \right) \ind{0 \le \theta \le \epsilon}.
%
%
\end{align*}
%
It follows that
%
\begin{align*}
%
\norm{\phi}_p ={}&
    \left( \alpha\left(\frac{1-\delta \epsilon}{1-\epsilon} \right)^{1/p} - 1
    \right) (1- \epsilon) +
    \left(\alpha \delta^{1/p} - 1 \right) \epsilon.
%
\end{align*}
%
For $\phi$ to be positive, we require
%
\begin{align*}
%
\alpha^p \ge \frac{1 - \epsilon}{1 - \delta \epsilon}
    \mathtxt{and}
\alpha^p \ge \frac{1}{\delta}.
%
\end{align*}

First, let us consider adding a small amount of prior mass, taking $\delta = 2 -
\epsilon$; let the corresponding perturbation be $\phi^+$.  For $\delta > 1$,
then we achieve $\phi \ge 0$ by taking $\alpha^p = \frac{1 - \epsilon}{1 -
\delta \epsilon}$.  Using the fact that $\epsilon \ll 1$ and keeping only
leading-order terms,
%
\begin{align*}
%
\frac{1-\epsilon}{1 - \delta \epsilon} \approx{}&
    (1- \epsilon)(1 + \delta \epsilon)
\\\approx{}& 1 + (\delta - 1) \epsilon
\\\approx{}& 1 + \epsilon,
%
\end{align*}
%
so
%
\begin{align*}
%
\norm{\phi^+}_p  ={}&
    \left(\alpha \delta^{1/p} - 1 \right) \epsilon
\\\approx{}&
    \left(
        \left( \left(1 + \epsilon\right) \left(2 - \epsilon \right)\right)^{1/p}
        - 1 \right) \epsilon
\\\approx{}&
%
\left( 2^{1/p} - 1 \right) \epsilon.
%
\end{align*}
%

Next, consider removing the same amount of mass with the symmetric change
$\delta = \epsilon$, letting $\phi^-$ be the corresponding perturbation. Then we
can ensure that $\phi(\theta) \ge 0$ with $\alpha^p \ge \epsilon^{-1}$, and
$\epsilon \ll 1$ gives
%
\begin{align*}
%
\frac{1-\delta\epsilon}{1 - \epsilon} \approx{}& 1- \epsilon,
%
\end{align*}
%
and
%
\begin{align*}
%
\norm{\phi^-}_p  ={}&
    \left( \alpha\left(\frac{1-\delta \epsilon}{1-\epsilon} \right)^{1/p} - 1
    \right) (1- \epsilon)
\\\approx{}&
\left(\left(\frac{1- \epsilon}{\epsilon}  \right)^{1/p} - 1\right)(1 - \epsilon)
%
\\\approx{}&
    \left( \frac{1}{\epsilon}\right)^{1/p}.
%
\end{align*}

Since $\epsilon$ is small, $\norm{\phi^-}_p \approx \left(
\frac{1}{\epsilon}\right)^{1/p} \gg \norm{\phi^+}_p \approx \left( 2^{1/p} - 1
\right) \epsilon$, despite the two perturbations respectively removing and
adding the same amount of arbitrarily small probability mass.

\end{ex}
%%%%%%%%%%%%%%%%%%%%%%%%%%%%%%%%%%%%%%%%%%%%%%%%%%%%%%%%%%%%%%%%%%%%%%%%%
