\subsection{The data}
The data come from a publicly available data set of mice gene expression
\citep{shoemaker:2015:ultrasensitive}.
Our analysis focuses on mice treated with the ``A/California/04/2009'' strain.
We normalize the data as described in
\citet{shoemaker:2015:ultrasensitive} and then apply the differential
analysis tool EDGE \citep{Storey:2005:significance} to rank the genes from most to least significantly differentially expressed.
We run our analysis on the top $\ngenes = 1000$ genes.

\subsection{The B-spline basis}
Notice from \figref{example_genes}, which shows an example time-course for a single gene,
that the time points are unevenly spaced, with more frequent observations at the beginning.
Following \citet{Luan:2003:clustering} we use cubic B-splines to smooth the time course expression data.
Specifically, we model the first 11 time points using
cubic B-splines with 7 degrees of freedom.
For the last three time points, $\timeindx = 72, 120, 168$ hours,
we use indicator functions.
That is, if $\tilde \regmatrix$ is the design
matrix where each column is a
B-spline basis vector evaluated at the $\ntimepoints$ measurement times,
we append to $\tilde \regmatrix$ three additional columns:
in these columns, entries are 1
if $\timeindx = 72, 120,$ or 168, receptively, and 0 otherwise.
The resulting matrix is the full design matrix $\regmatrix$.
We use indicators for the last three time points for numerical stability;
without the indicator columns,
the matrix $\tilde \regmatrix^T \tilde \regmatrix$ is nearly singular
because the later time points are more spread out.
The left column of \figref{example_genes} shows our basis functions.

\subsection{The generative model}
\eqref{mice_model} gives the per-component conditional likelihood.
We use a normal prior for the shifts $\b_\n$,
a multivariate normal prior for the coefficients $\mu_\k$,
and a gamma prior for the inverse variance $\tau_\k$.
The prior on the mixture weights $\pi$ are constructed using the stick-breaking
construction in the main-text, and the cluster assignments $\z_\n$
are drawn from a multinomial with wieghts $\pi$, as usual.



\subsection{The variational approximation}
The variational approximation, factorizes as
\begin{align*}
\q(\zeta \vert \eta) =
    \left( \prod_{\k=1}^{\kmax - 1} \q(\nuk \vert \eta) \right)
    \left( \prod_{\k=1}^{\kmax} \q(\beta_\k \vert \eta) \right)
    \left( \prod_{\n=1}^{\N} \q(\z_{\n} \vert \eta)
    \q(\b_{\n} \vert \z_{\n}, \eta)\right).
\end{align*}
Note that the variational distribution for $\b_\n$ conditions on $\z$.
We set $\q(\b_{\n} \vert \z_{\n} = k, \eta)$ to be Gaussian
with variational parameters dependent on $\k$.
For simplicity in this application,
we let $\q(\beta_\k \vert \eta) = \delta (\beta_k \vert \eta)$,
where $\delta(\cdot \vert \eta)$ denotes a point mass at a parameterized location.

As discussed in \exref{qz_optimality},
the optimal distribution $\q(\z_\n\vert\eta)$ is multinomial whose parameters
can be set in closed form as a function of the global variational parameters only.
We allow the distribution of $\b_\n$ to depend on $\z_{\n\k}$ so that
the its optimal distribution can also be set in closed form as a function of
global parameters.

The optimal distribution $q(\b_\n\vert \z_{\n\k} = 1, \eta)$ is Gaussian,
\begin{align*}
q(\b_\n\vert \z_{\n\k} = 1, \eta) = \normdist{\b_\n \vert \hat\mu_{\b_{\n\k}}, \hat\sigma^2_{\b_{\n\k}}}.
\end{align*}
Let
\begin{align*}
  \rho^{(1)}_{\n\k} &= \expect{\q(\beta_k|\eta)}{\sum_{m=1}^\ntimepoints \tau_{k}(x_{nm} - \regmatrix_m\mu_\k)} +
  \tau_0 \mu_0 \\
  \rho^{(2)}_{\n\k} &= \ntimepoints \expect{\q(\beta_k|\eta)}{\tau_{k}} + \tau_0,
\end{align*}
where $\mu_0$ and $\tau_0$ are the prior mean and information on $\b_n$, respectively.

The optimal parameters for the Gaussian distribution on $\b_\n$ is given by
\begin{align*}
  \hat\mu_{\b_{\n\k}} &= \rho^{(1)}_{\n\k} / \rho^{(2)}_{\n\k}\\
  \hat\sigma^2_{\b_{\n\k}} &= 1 / \rho^{(2)}_{\n\k}.
\end{align*}

% with parameters $\hat\mu_{\b_{\n\k}}, \hat\mu_{\b_{\n\k}}:
