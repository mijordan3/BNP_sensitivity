In \secref{diffable_nonparametric} we considered multiplicative perturbations to
the prior density.  One might ask whether one could consider other paths through
the space of priors, such as additive perturbations.  In this section, we
briefly consider a broader class of nonlinear perturbations investigated by
\citet{gustafson:1996:local}, of which additive and multiplicative perturbations
are special cases, and show that, within this class, only multiplicative
perturbations are compatible with KL divergence.

As in \secref{diffable_nonparametric}, suppose we have a base prior $\pbase$ and
an alternative $\palt$, and that we wish to parameterize a continuous path
between them.  We will do so as follows. For some $p \in [1, \infty)$, let
%
\begin{align}\eqlabel{p_pert_simple}
%
\ptil(\theta \vert \tp) :=
    \left((1 - \tp)\pbase(\theta)^{1/p} +
    \tp \frac{1}{p}\palt(\theta)^{1/p} \right)^{p}.
%
\end{align}
%
As with \eqref{mult_pert_simple}, $\p(\theta \vert \tp = 0) = \pbase(\theta)$,
$\p(\theta \vert \tp = 1) = \palt(\theta)$, and $\p(\theta \vert \tp)$ moves
continously between the two in $\tp \in (0, 1)$.  When $p = 1$,
$\ptil(\theta \vert \tp)$ defines an ``additive perturbation,'' and
the limit as $p \rightarrow \infty$ gives the multiplicative perturbation
of \eqref{mult_pert_simple}.
%
\citet{gustafson:1996:local} proves that a result analogous to
\eqref{p_pert_simple} for \eqref{p_pert_simple}, where the
$\norminf{\cdot}$ norm is repalaced by
%
\begin{align}\eqlabel{phi_lp_norm}
%
\phi(\theta \vert \palt, p) :={}
    \palt(\theta)^{1/p} - \pbase(\theta)^{1/p} \mathand
\norm{\phi}_p :={} \left(\int \abs{\phi(\theta)}^p \right)^{1/p}.
%
\end{align}
%
We refer the reader to \citet{gustafson:1996:local} for details.  For our
present discussion, what matters is that the use of the perturbation in
\eqref{p_pert_simple} strongly motivates the use of the norm $\norm{\phi(\theta
\vert \palt, p)}_p$ when forming, for example, worst-case perturbations as in
\corref{etafun_worst_case}.

Though the $\norm{\phi(\theta \vert \palt, p)}_p$ norm does not appear to cause
major difficulties for the full Bayesian posterior, this norm is not compatible
with KL divergence, in the sense that KL divergence is {\em discontinuous} in
this norm.  Prior changes that are arbitrarily small according to
$\norm{\phi(\theta \vert \palt, p)}_p$ can induce arbitrarily large changes in
the KL divergence \eqref{kl_def}, and so (in genreal) arbitrarily large
changes in its optimum.  The problem is best illustrated with an
example.

%%%%%%%%%%%%%%%%%%%%%%%%%%%%%%%%%%%%%%%%%%%%%%%%%%%%%%%%%%%%%%%%%%%%%%%%%
%%%%%%%%%%%%%%%%%%%%%%%%%%%%%%%%%%%%%%%%%%%%%%%%%%%%%%%%%%%%%%%%%%%%%%%%%
\begin{thm}\thmlabel{kl_discontinuous}
%
Let $\mu$ denote a measure on $\thetadom$ that is absolutely continous
with respect to the Lebesgue measure, and let $\q(\theta)$ and
$\pbase(\theta)$ denote densities with respect to $\mu$.  Without loss of
generality, assume that $\q(\theta) > 0$ on $\thetadom$.  Assume that
$\KL{q(\theta) || \pbase(\theta)}$ is well-defined and finite.

Then, for any $\epsilon > 0$ and any $M > 0$, we can find a density
$\palt(\theta)$ such that $\norm{\phi(\theta \vert \palt, p)}_p < \epsilon$ but
$\abs{\KL{q(\theta) || \palt(\theta)} - \KL{q(\theta) || \pbase(\theta)}} > M$.

\begin{proof}
%
The proof will follow by perturbing $\pbase(\theta)$ to zero in a small
interval.  By making the interval narrow, we can make $\norm{\phi(\theta \vert
\palt, p)}_p$ small, but by making the $\palt(\theta)$ sufficiently close to
zero, we can make the KL divergence difference large irresective of how narrow
the interval is.

First, observe that
%
\begin{align*}
%
\KL{q(\theta) || \palt(\theta)} -
\KL{q(\theta) || \pbase(\theta)} ={}&
\expect{\q(\theta)}{\log \frac{\palt(\theta)}{\pbase(\theta)}}.
%
\end{align*}

For any set $S$ with $\pbase(S) = \epsilon$, define
%
\begin{align*}
%
\palt(\theta \vert S, \delta) :=
    \frac{\delta^{\ind{\theta \in S}}}{1 + \epsilon(1 - \delta)}.
%
\end{align*}
%
Then $\palt(\theta \vert S, \delta)$ is a valid density, and
%
\begin{align*}
%
\KL{q(\theta) || \palt(\theta)} - \KL{q(\theta) || \pbase(\theta)}
    ={}& \q(S) \log \delta - \log\left( 1 + \epsilon(1 - \delta) \right).
%
\end{align*}
%
By \eqref{phi_lp_norm},
%
\begin{align*}
%
\phi(\theta \vert \palt, p) ={}&
    \pbase(\theta)^{1/p} \left(
        \frac{\left(\delta^{1/p}\right)^{\ind{\theta \in S}}}
             {\left( 1 + \epsilon(1 - \delta) \right)^{1/p}} - 1 \right) \mathand\\
\norm{\phi(\theta \vert \palt, p)}_p^p ={}&
\epsilon \left(
   \frac{\left(\delta^{1/p}\right)}
        {\left( 1 + \epsilon(1 - \delta) \right)^{1/p}} - 1 \right) +
(1 - \epsilon) \left(
   \frac{1}
        {\left( 1 + \epsilon(1 - \delta) \right)^{1/p}} - 1 \right).
%
\end{align*}

Since $\mu$ is absolutely continuous with respect to the Lebesgue measure, there
exists a sequence $\epsilon_n \rightarrow 0$ with $\epsilon_n > 0$ and a
sequence of corresponding sets $S_n$ such that $\pbase(S_n) = \epsilon_n$. (See
\lemref{continuity_partition} for a proof of this fact, which is a
straightforward consequence of \citet[Proposition 15.5]{nielsen:1997:measure}
and the continuity of the Lebesgue measure.) Since $\q(\theta) > 0$ on
$\thetadom$, $\q(S_n) > 0$ for all $n$.  Since $\KL{\q(\theta) ||
\pbase(\theta)}$ is finite, we must have $\lim_{n \rightarrow} \q(S_n) = 0$.

Take $\delta_n  = \exp(-1 / (\q(S_n)^2))$, and take $\palt(\theta) =
\palt(\theta \vert S_n, \delta_n)$.  Then $\epsilon_n (1 - \delta_n) \rightarrow
0$, and $\q(S_n)\log \delta_n = -1 / \q(S_n)$, so
%
\begin{align*}
%
\abs{\KL{q(\theta) || \palt(\theta \vert S_n, \delta_n)} -
    \KL{q(\theta) || \pbase(\theta)}} \rightarrow{}& \infty, \quad \textrm{but}\\
%
\norm{\phi(\theta \vert \palt(\cdot \vert S_n, \delta_n), p)}_p^p
    \rightarrow{}& 0.
%
\end{align*}
%
Thus, for sufficiently large $n$, the conclusion follows.
%
\end{proof}
%
\end{thm}
%%%%%%%%%%%%%%%%%%%%%%%%%%%%%%%%%%%%%%%%%%%%%%%%%%%%%%%%%%%%%%%%%%%%%%%%%

Since Fr{\'e}chet differentiability implies continuity,
\thmref{kl_discontinuous} shows that it is impossible to derive an analogue of
\thmref{eta_phi_deriv} for perturbations of the form \eqref{p_pert_simple}
with the norms \eqref{phi_lp_norm}.

Recall \exref{beta_inf_norm}, where we showed that there exist valid
priors for which $\norminf{\phi} < \infty$.  Viewed in light of the
proof of \thmref{kl_discontinuous}, the limited expressiveness of the
$\norminf{\cdot}$ norm begins to look like a feature, rather than a bug.
% The KL divergence that defines a variational objective cannot handle
% prior densities that are too close to zero.  The $\norminf{\cdot}$ norm
% considers such densities to be ``distant'' from $\pbase$, whereas the
% more permissive $\norm{\cdot}_p$ norms do not.
The situation is illustrated in  \figref{func_dist}.  The two shown densities
are far from one another according to KL divergence (from blue to red) since the
red density has nearly zero mass where the blue does not.  They are also distant
from one another in $\norminf{\cdot}$, since it takes a large multiplicative
change to turn a nonzero number into a nearly zero number.  However, the two
densities are close in $\norm{\cdot}_{p}$, since the region where the red
density is nearly zero has a small measure. In order for VB approximations to be
continuous (a necessary condition for Fr{\'e}chet differentiability), one must
consider a topology on priors that is no coarser than the topology induced by KL
divergence.  But since valid priors can take values close to zero, a sacrifice
in expressiveness of the neighborhood of zero must be made in order to induce a
topology that works with KL divergence.  Multiplicative changes and the
$\norminf{\cdot}$ norm make such a tradeoff in a natural, easy-to-understand
way.  If one desires a more expressive prior space, it seems to the authors more
fruitful to investigate variational divergences other than KL rather than
attempt to adapt the space of prior perturbations.

\FunctionDistFig{}
