Though every directional derivative exists, the quality of the linear
approximation can be arbitrarily poor; in the $\lp{\mu,p}$ space, the role of
$\theta$ near integer multiples of $\pi$ is played by prior perturbations with
density near zero in small sets of nonzero measure.  To carry the analogy
futher, one might be tempted to further restrict $\phi$ so that $\p(\theta \vert
\phi)$ is bounded away from zero.

Indeed, the tension between differentiability and expressiveness appears
fundamental.  Consider \figref{func_dist}.  These two densities are far from one
another according to KL divergence (from blue to red) since the red density has
nearly zero mass where the blue does not.  They are also distant from one
another in $\norminf{\cdot}$, since it takes a large multiplicative change to
turn a nonzero number into a nearly zero number.  However, the two densities are
close in $\norm{\cdot}_{p}$, since the region where the red density is nearly
zero has a small measure. In order for VB approximations to be continuous (a
necessary condition for Fr{\'e}chet differentiability), one must consider a
topology on priors that is no coarser than the topology induced by KL
divergence.  But since valid priors can take values close to zero, a sacrifice
in expressiveness of the neighborhood of zero must be made in order to induce a
topology that works with KL divergence.  Multiplicative changes and the
$\norminf{\cdot}$ norm make such a tradeoff in a natural, easy-to-understand
way.  If one desires a more expressive prior space, it seems to the authors more
fruitful to investigate variational divergences other than KL rather than
attempt to adapt the space of prior perturbations.

\FunctionDistFig{}
