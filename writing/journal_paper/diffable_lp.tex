In \secref{local_sensitivity, influence_function} we considered
multiplicative perturbations to the prior density.  One might ask whether one
could consider other paths through the space of priors, such as additive
perturbations.  In this section, we briefly consider a broader class of
nonlinear perturbations investigated by \citet{gustafson:1996:local}, of which
additive and multiplicative perturbations are special cases, and show that,
within this class, only multiplicative perturbations lead to Fr{\'e}chet
differentiable VB optima.

As in \secref{local_sensitivity}, suppose we have an initial prior $\pbase$
and an alternative $\palt$, and that we wish to parameterize a continuous path
between them.  Deviating from multiplicative perturbations, for some $p \in [1,
\infty)$, let
%
\begin{align}\eqlabel{p_pert_simple}
%
\ptil(\theta \vert \tp) :=
    \left((1 - \tp)\pbase(\theta)^{1/p} +
    \tp \frac{1}{p}\palt(\theta)^{1/p} \right)^{p}.
%
\end{align}
%
As with \eqref{mult_pert_simple}, $\p(\theta \vert \tp = 0) = \pbase(\theta)$,
$\p(\theta \vert \tp = 1) = \palt(\theta)$, and $\p(\theta \vert \tp)$ moves
continuously between the two in $\tp \in (0, 1)$.  When $p = 1$,
$\ptil(\theta \vert \tp)$ defines an ``additive perturbation,'' and
the limit as $p \rightarrow \infty$ gives the multiplicative perturbation
of \eqref{mult_pert_simple}.

\citet[Result 2]{gustafson:1996:local} states a result analogous to
\lemref{pert_invariance} for \eqref{p_pert_simple}, where the
$\norminf{\cdot}$ norm is replaced by
%
\begin{align}\eqlabel{phi_lp_norm}
%
\phi(\theta \vert \palt, p) :={}
    \palt(\theta)^{1/p} - \pbase(\theta)^{1/p} \mathand
\norm{\phi}_p :={} \left(\int \abs{\phi(\theta)}^p \right)^{1/p}.
%
\end{align}
%
We refer the reader to \citet{gustafson:1996:local} for details
%
\footnote{\citet{gustafson:1996:local} in fact considers only pointwise positive
perturbations $\phi(\theta \vert \palt, p) > 0$, $\mu$-almost everywhere.  It is
not hard to extend \lemref{pert_invariance} \citet[Result
2]{gustafson:1996:local} to permit negative perturbations, except for the fact
that $\norm{\cdot}_p$-neighborhoods of the zero function will always contain
pointwise negative ``priors.''
%
We allow for negative $\phi(\theta \vert \palt, p)$ because otherwise
$\norm{\cdot}_p$ leads to counter intuitive notions of the ``size'' of prior
perturbations, as we discuss in \appref{positive_pert}, and because standard
results in functional analysis used in the proof of \thmref{eta_phi_deriv}
require open neighborhoods.

However, we must acknowledge that the main result of this section,
\thmref{kl_discontinuous} below, relies on the possibility that $\phi(\theta
\vert \palt, p)$ can be negative.   In light of this, one might reasonably
wonder whether we should in fact restrict to positive perturbations in an
attempt to avoid the consequences of \thmref{kl_discontinuous}. In the view of
the authors, restricting to pointwise positive perturbations is a somewhat
artificial solution to a fundamental disconnect between the $\norm{\cdot}_p$
norm and KL divergence which we dicuss at the end of the present section. We
believe that the disconnect is resolved more transparently and naturally through
the use of the $\norminf{\cdot}$ norm and multiplicative perturbations which are
allowed to be negative.}
%
.  For our present discussion, what matters is that the use of the perturbation
in \eqref{p_pert_simple} strongly motivates the use of the norm
$\norm{\phi(\theta \vert \palt, p)}_p$ when forming, for example, worst-case
perturbations as in \coryref{etafun_worst_case}.

Though the $\norm{\phi(\theta \vert \palt, p)}_p$ norm does not appear to cause
major difficulties for the full Bayesian posterior,
%
\footnote{Other than the fact that there exist pointwise negative priors
induced by $\phi(\theta \vert \palt, p)$ in every neighborhood of the
zero function.}
%
the $\norm{\phi(\theta \vert \palt, p)}_p$ norm is not compatible with KL
divergence, in the sense that KL divergence is {\em discontinuous} in this norm.
Prior changes that are arbitrarily small according to $\norm{\phi(\theta \vert
\palt, p)}_p$ can induce arbitrarily large changes in the KL divergence, and so
(in general) arbitrarily large changes in its optimum.

%%%%%%%%%%%%%%%%%%%%%%%%%%%%%%%%%%%%%%%%%%%%%%%%%%%%%%%%%%%%%%%%%%%%%%%%%
%%%%%%%%%%%%%%%%%%%%%%%%%%%%%%%%%%%%%%%%%%%%%%%%%%%%%%%%%%%%%%%%%%%%%%%%%
\begin{thm}\thmlabel{kl_discontinuous}
%
Let $\mu$ denote a measure on $\thetadom$ that is absolutely continuous
with respect to the Lebesgue measure, and let $\q(\theta)$ and
$\pbase(\theta)$ denote densities with respect to $\mu$.  Without loss of
generality, assume that $\q(\theta) > 0$ on $\thetadom$.  Assume that
$\KL{q(\theta) || \pbase(\theta)}$ is well-defined and finite.

Then, for any $\epsilon > 0$ and any $M > 0$, we can find a density
$\palt(\theta)$ such that $\norm{\phi(\theta \vert \palt, p)}_p < \epsilon$ but
$\abs{\KL{q(\theta) || \palt(\theta)} - \KL{q(\theta) || \pbase(\theta)}} > M$.

\begin{proof}
%
For the duration of the proof, we will use the shorthand that a density
applied to a set represents the integral of the density over the set.
For example, for a set $S$, $\p(S) = \int_S \p(\theta)\mu(d\theta)$.

The proof will be constructive, based on an alternative $\palt(\theta)$ formed
by driving $\pbase(\theta)$ to zero in a small interval.  By making the interval
narrow, we can make $\norm{\phi(\theta \vert \palt, p)}_p$ small, but by making
the $\palt(\theta)$ sufficiently close to zero, we can make the KL divergence
difference large irrespective of how narrow the interval is.

First, observe that
%
\begin{align*}
%
\KL{q(\theta) || \palt(\theta)} -
\KL{q(\theta) || \pbase(\theta)} ={}&
\expect{\q(\theta)}{\log \frac{\palt(\theta)}{\pbase(\theta)}}.
%
\end{align*}

For any set $S$ with $\pbase(S) = \epsilon$, define
%
\begin{align*}
%
\palt(\theta \vert S, \delta) :=
    \frac{\delta^{\ind{\theta \in S}}}{1 + \epsilon(1 - \delta)} \pbase(\theta).
%
\end{align*}
%
Then $\palt(\theta \vert S, \delta)$ is a valid density, and
%
\begin{align*}
%
\KL{q(\theta) || \palt(\theta)} - \KL{q(\theta) || \pbase(\theta)}
    ={}& \q(S) \log \delta - \log\left( 1 + \epsilon(1 - \delta) \right).
%
\end{align*}
%
By \eqref{phi_lp_norm},
%
\begin{align*}
%
\phi(\theta \vert \palt, p) ={}&
    \pbase(\theta)^{1/p} \left(
        \frac{\left(\delta^{1/p}\right)^{\ind{\theta \in S}}}
             {\left( 1 + \epsilon(1 - \delta) \right)^{1/p}} - 1 \right) \mathand\\
\norm{\phi(\theta \vert \palt, p)}_p^p ={}&
\epsilon \left(
   \frac{\left(\delta^{1/p}\right)}
        {\left( 1 + \epsilon(1 - \delta) \right)^{1/p}} - 1 \right) +
(1 - \epsilon) \left(
   \frac{1}
        {\left( 1 + \epsilon(1 - \delta) \right)^{1/p}} - 1 \right).
%
\end{align*}

Since $\mu$ is absolutely continuous with respect to the Lebesgue measure, there
exists a sequence $\epsilon_n \rightarrow 0$ with $\epsilon_n > 0$ and a
sequence of corresponding sets $S_n$ such that $\pbase(S_n) = \epsilon_n$. (See
\lemref{continuity_partition} for a proof of this fact, which is a
straightforward consequence of \citet[Proposition 15.5]{nielsen:1997:measure}
and the continuity of the Lebesgue measure.) Since $\q(\theta) > 0$ on
$\thetadom$, $\q(S_n) > 0$ for all $n$.  Since $\KL{\q(\theta) ||
\pbase(\theta)}$ is finite, we must have $\lim_{n \rightarrow} \q(S_n) = 0$.

Take $\delta_n  = \exp(-1 / (\q(S_n)^2))$, and take $\palt(\theta) =
\palt(\theta \vert S_n, \delta_n)$.  Then $\epsilon_n (1 - \delta_n) \rightarrow
0$, and $\q(S_n)\log \delta_n = -1 / \q(S_n)$, so
%
\begin{align*}
%
\abs{\KL{q(\theta) || \palt(\theta \vert S_n, \delta_n)} -
    \KL{q(\theta) || \pbase(\theta)}} \rightarrow{}& \infty, \quad \textrm{but}\\
%
\norm{\phi(\theta \vert \palt(\cdot \vert S_n, \delta_n), p)}_p^p
    \rightarrow{}& 0.
%
\end{align*}
%
Thus, for sufficiently large $n$, the conclusion follows.
%
\end{proof}
%
\end{thm}
%%%%%%%%%%%%%%%%%%%%%%%%%%%%%%%%%%%%%%%%%%%%%%%%%%%%%%%%%%%%%%%%%%%%%%%%%

% Since Fr{\'e}chet differentiability implies continuity \citep[Proposition 4.8
% (d)]{zeidler:2013:functional}, \thmref{kl_discontinuous} shows that it is
% impossible to derive an analogue of \thmref{eta_phi_deriv} for perturbations of
% the form \eqref{p_pert_simple} with the norms \eqref{phi_lp_norm}.
%
% Viewed in light of the proof of \thmref{kl_discontinuous}, the limited
% expressiveness of the $\norminf{\cdot}$ norm, as demonstrated in
% \exref{beta_inf_norm}, looks like a feature, rather than a bug.
% % The KL divergence that defines a variational objective cannot handle
% % prior densities that are too close to zero.  The $\norminf{\cdot}$ norm
% % considers such densities to be ``distant'' from $\pbase$, whereas the
% % more permissive $\norm{\cdot}_p$ norms do not.
% Consider \figref{func_dist}, which succinctly summarizes the tradeoffs between
% the various norms.  The two shown densities are far from one another according
% to KL divergence (from blue to red) since the red density takes values that are
% nearly zero where the blue density has nonzero mass. They are also distant from
% one another in $\norminf{\cdot}$, since it takes a large multiplicative change
% to turn the nonzero blue density into the nearly zero red density. However, the
% two densities are close in $\norm{\cdot}_{p}$, since the region where the red
% density is nearly zero has a small measure. In order for VB approximations to be
% continuous (a necessary condition for Fr{\'e}chet differentiability), one must
% consider a topology on priors that is no coarser than the topology induced by KL
% divergence.  But since valid priors can take values close to zero, a sacrifice
% in expressiveness of the neighborhood of zero must be made in order to induce a
% topology that works with KL divergence. Multiplicative changes and the
% $\norminf{\cdot}$ norm make such a tradeoff in a natural, easy-to-understand
% way.
%
% In this sense, VB approximations based on KL divergence are inherently
% non-robust to priors that ablate mass nearly to zero.  No parameterization of
% the space of priors will relieve this non-robustness.  Only by basing
% variational approximations on divergences other than KL will this non-robustness
% be alleviated.
%
% \FunctionDistFig{}
