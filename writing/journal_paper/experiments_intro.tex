We evaluate the prior sensitivity in BNP models applied to three distinct data
analysis examples. We first fit a Gaussian mixture model
(\exref{iris_bnp_process}) to the canonical iris data set. Secondly, we cluster
time-course gene expression data using a regression model and study the
resulting co-clustering matrix. Finally, we fit a topic model on a data set of
sampled genotypes in an endangered bird species, inferring  ancestral migration
patterns from the posterior population structure.

In each data example, we first specify our quantities of interest and fit the
variational approximation to a model with a $\gem$ prior at some chosen
parameter $\alpha = \alpha_0$. We then evaluate sensitivity to the chosen
$\alpha$ parameter by refitting the variational approximation for each $\alpha$
in a set $\{\alpha_1, ..., \alpha_m\}$ of plausible values. Next, we examine the
effects of changing the functional form the stick-breaking prior, using the
influence function to guide our choice of prior perturbation, either
heuristically or using \coryref{etafun_worst_case}.  For each prior perturbation,
we validate the performance of the linear approximation against re-fitting the
model and comare the relative compuatational cost of the linear approximation
and re-fitting.
