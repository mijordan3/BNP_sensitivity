





















The classes of perturbations given in \defref{prior_nl_pert} have a natural
correspondence with the $\lp{\lambda, p}$ spaces of functions, which we now
define.

%%%%%%%%%%%%%%%%%%%%%%%%%%%%%%%%%%%%%%%%%%%%%%%%%%%%%%%%%%%%%%%%%%%%%%%%%%%
%%%%%%%%%%%%%%%%%%%%%%%%%%%%%%%%%%%%%%%%%%%%%%%%%%%%%%%%%%%%%%%%%%%%%%%%%%%

\begin{defn}\deflabel{lp_spaces}
\citep[Sections 5.1-5.2]{dudley:2018:real}
%
For a measure $\mu$ and $p \in [1, \infty]$, let $\lp{\mu,p}$ define the
space of equivalence classes of real-valued $\mu$-measurable functions,
where two functions are equivalent if they disagree only on a set of
$\mu$-measure zero.

Let $\esssup_\theta^\mu$ denote the essential supremum over $\theta$ with
respect to the measure $\mu$. The norm on $\lp{\mu,p}$ is given by
%
\begin{align*}
%
\norm{\phi}_{\mu,p} :={}&
\begin{cases}
    \left(\int \abs{\phi(\theta)}^p \mu(d\theta)\right)^{1/p}
    & \textrm{when }p \in [1, \infty)\\
    \esssup_{\theta} \abs{\phi(\theta)}
    & \textrm{when }p = \infty
\end{cases}\\
%
\phi \in \lp{\lambda,p} \Leftrightarrow{}& \norm{\phi}_{\lambda,p} < \infty.
%
\end{align*}
%
When $\mu$ is the Lebesgue measure, we may simply write $\lp{\lambda,p} =
\lp{p}$ and $\norm{\cdot}_{p} = \norm{\cdot}_{\lambda,p}$.
%
Let $\ball_{\mu,p}(\epsilon) := \{\phi: \phi \in \lp{\mu, p},
\norm{\phi}_p < \epsilon \}$ denote the $\epsilon$-ball in $\lp{\mu, p}$.
%
\end{defn}

By \citep[Theorem 5.2.1]{dudley:2018:real}, $\lp{\mu,p}$ is a Banach
space (i.e., a complete, normed vector space).

The following lemma is due to \citep{gustafson:1996:local}, and provides
a key part of the motivation for the use of $\norm{\cdot}_{\mu,p}$ to measure
the size of prior perturbations.

%%%%%%%%%%%%%%%%%%%%%%%%%%%%%%%%%%%%%%%%%%%%%%%%%%%%%%%%%%%%%%%%%%%%%%%%%%%
%%%%%%%%%%%%%%%%%%%%%%%%%%%%%%%%%%%%%%%%%%%%%%%%%%%%%%%%%%%%%%%%%%%%%%%%%%%
\begin{lem}\lemlabel{pert_invariance}
%
(\citet{gustafson:1996:local})
%
Fix the quantities given in \defref{prior_nl_pert}.  For a fixed probability
measure $\p \ll \mu$, the map $\p \mapsto \norm{\phi(\cdot \vert \beta,
\p)}_p$ does not depend on $\mu$, and is invariant to invertible transformations
of $\theta$.
%
\seeproof{pert_invariance}
%
\end{lem}
%%%%%%%%%%%%%%%%%%%%%%%%%%%%%%%%%%%%%%%%%%%%%%%%%%%%%%%%%%%%%%%%%%%%%%%%%%%




%%%%%%%%%%%%%%%%%%%%%%%%%%%%%%%%%%%%%%%%%%%%%%%%%%%%%%%%%%%%%%%%%%%%%%%%%%%

\begin{lem}\lemlabel{pert_well_defined}
%
Fix the quantities given in \defref{prior_nl_pert}.  Say that an unnormalized
prior $\ptil$ is ``valid'' if $\ptil \ll \pbase$ is non-negative
$\mu$-almost everywhere and normalizable in the sense that $0 < \int
\ptil(\theta) \mu(d\theta) < \infty$.

The following table summarizes properties of priors derived from
$\phi \in \ball_{\mu,p}$ and from $\phi \in \pertset$.  The columns are
shorthand for the following properties:
%
\begin{align*}
%
\int \ptil < \infty \Rightarrow{}&
    \int \ptil(\theta \vert \phi) \mu(d\theta) < \infty   &\quad
%
\int \ptil > \infty \Rightarrow{}&
    \int \ptil(\theta \vert \phi) \mu(d\theta) > 0\\
%
\ptil \ge 0 \Rightarrow{}&
    \essinf_{\theta \sim \mu} \ptil(\theta \vert \phi) \ge 0  &\quad
% \norm{\phi} < \infty \Rightarrow{}&
%     \exists M < \infty \textrm{ independent of }\phi\textrm{ such that }
%     \sup_\phi \norm{\phi}_{\mu,p} < M.
\norm{\phi} < \infty \Rightarrow{}& \norm{\phi}_{\mu,p} < \infty.
%
\end{align*}

A ``Y'' indicates that the columns property is satisfied for all $\phi$ in the
corresponding set.  For example, the ``Y'' in the first row and first column
means that, for all $\phi \in \pertset$ with $p \in [1, \infty)$, $\int
\ptil(\theta \vert \phi) \mu(d\theta) < \infty$.  A ``N'' means the converse,
i.e., that there exists, in general, some $\phi$ in the corresponding set that
does not have the column's property.

\begin{table}[h!]
%\vspace{1em}
\begin{centering}
%\begin{tabular}{|c|c|c|c|c|c|}
\begin{tabular}{cccccc}
    %\hline
    && $\int \ptil < \infty$
    & $\int \ptil > 0$
    & $\ptil \ge 0$
    & $\norm{\phi} < \infty$\\[0.5em] \hline
$p \in [1, \infty)$   &     $\phi \in \pertset$ &
%    Y & Y & Y & Y, if $\beta < \infty$ \\ \hline
    Y & Y & Y & Y \\ \hline
$p \in [1, \infty)$   &     $\phi \in \ball_{\mu,p}(\delta)$ &
    Y & Y, if $\delta < p$ & N & Y (by defn) \\ \hline
$p = \infty$   &     $\phi \in \pertset[\infty]$ &
    Y & Y & Y & N \\ \hline
$p = \infty$   &     $\phi \in \ball_{\mu,\infty}(\delta)$ &
    Y & Y & Y & Y (by defn) \\ \hline
\end{tabular}
\caption{The relationships between $\pertset$, $\ball_{\mu,p}$, and valid priors.
The properties hold for any ball radius $\delta > 0$.}
\tablabel{pert_well_defined}
\end{centering}
%\vspace{1em}
\end{table}

\Tabref{pert_well_defined} has the following implications:

\begin{enumerate}
%
\item \itemlabel{pertset_is_valid}
For all $p \in [1, \infty]$, the set of priors that can be formed from
$\phi \in \pertset$ is identical to the set of all valid priors.
%
\item \itemlabel{pball_is_valid}
For $p \in [1, \infty)$, all valid priors can be formed from
some $\phi \in \lp{\mu,p}$.
%
\item \itemlabel{pball_is_invalid}
For $p \in [1, \infty)$, one can form invalid (negative) priors from
some $\phi \in \ball_{\mu,p}(\delta)$ even for arbitrarily small $\delta$.
(\Exref{lp_negative}.)
%
\item \itemlabel{pinfball_is_invalid}
For $p = \infty$, there exist valid priors not formed from any
$\phi \in \lp{\mu,\infty}$.  (\Exref{beta_inf_norm}.)
%
\end{enumerate}

\seeproof{pert_well_defined}
%
\end{lem}
%%%%%%%%%%%%%%%%%%%%%%%%%%%%%%%%%%%%%%%%%%%%%%%%%%%%%%%%%%%%%%%%%%%%%%%%%%%
%%%%%%%%%%%%%%%%%%%%%%%%%%%%%%%%%%%%%%%%%%%%%%%%%%%%%%%%%%%%%%%%%%%%%%%%%%%

%%%%%%%%%%%%%%%%%%%%%%%%%%%%%%%%%%%%%%%%%%%%%%%%%%%%%%%%%%%%%%%%%%%%%%%%%
% \/  \/  \/  \/  \/  \/  \/  \/  \/  \/  \/  \/  \/  \/  \/  \/  \/  \/
\begin{ex}\exlabel{lp_negative}
%
For $1 \le p < \infty$, it is possible for $\phi \in
\ball_{\mu,p}(\epsilon)$ for arbitrarily small $\epsilon$ and yet have
$\essinf^\mu_{\theta} \p(\theta \vert \phi) < 0$.
%
Since $\pbase \ll \mu \ll \lambda$, there exists a
sequence $\epsilon_n \rightarrow 0$ with $\epsilon_n > 0$ and a sequence of
corresponding sets such that $\pbase(S_n) = \epsilon_n$. (See
\lemref{continuity_partition} for a proof of this fact, which is a
straightforward consequence of \citet[Proposition 15.5]{nielsen:1997:measure}
and the continuity of the Lebesgue measure.)  Take
%
%\begin{align*}
%
$\phi_n(\theta) := - \frac{2}{p} \pbase(\theta)^{1/p} \ind{\theta \in S_n}$.
%
%\end{align*}
%
Then $\norm{\phi_n}_{\lambda, p} = \frac{2}{p} \epsilon_n^{1/p} \rightarrow 0$
and
%
\begin{align*}
%
\pbase(\theta)^{1/p} + p \phi(\theta) ={}
\pbase(\theta)^{1/p}
\left(\ind{\theta \notin S_n} - \ind{\theta \in S_n} \right)
%
\end{align*}
%
so $\essinf_\theta \p(\theta \vert \phi_n) < 0$ for all $n$.
%
\end{ex}
% /\    /\    /\    /\    /\    /\    /\    /\    /\    /\    /\    /\    /\
%%%%%%%%%%%%%%%%%%%%%%%%%%%%%%%%%%%%%%%%%%%%%%%%%%%%%%%%%%%%%%%%%%%%%%%%%%%%




Note that \citep{gustafson:1996:local} avoids the difficulty of
\exref{lp_negative} by restricting to positive $\phi(\theta)$, i.e. $\phi$ such
that $\essinf_\theta^\mu \phi(\theta) \ge 0$.  Of course, \exref{lp_negative}
implies that there exist negative $\phi$ in any neighborhood of $\phiz$,
prohibiting the use of standard functional analysis results requiring open
neighborhoods, such as the implicit function theorem in Banach spaces, (our key
tool proving \thmref{eta_phi_deriv} below). Furthermore, \exref{phi_negative,
phi_necessarily_negative} shows that restricting $\phi$ to be positive
sacrifices \lemref{pert_well_defined} \itemref{pertset_is_valid}.  Perhaps more
importantly, restricting to positive $\phi$ induces counterintutive notions of
the ``size'' of perturbations that ablate mass, detailed discussion of which we
provid in \appref{positive_pert}.



%%%%%%%%%%%%%%%%%%%%%%%%%%%%%%%%%%%%%%%%%%%%%%%%%%%%%%%%%%%%%%%%%%%%%%%%%
% \/  \/  \/  \/  \/  \/  \/  \/  \/  \/  \/  \/  \/  \/  \/  \/  \/  \/
\begin{ex}\exlabel{beta_inf_norm}
%
It is possible for $\phi \in \pertset[\infty]$ to have $\norminf{\phi} =
\infty$, and so $\phi \notin \lp{\lambda,\infty}$.  Take $\mu$ to be the
Lebesgue measure on $[0,1]$, let $\pbase(\theta) = \betadist{\theta \vert 1,
\alpha_0}$ and $\palt(\theta) = \betadist{\theta \vert 1, \alpha_1}$ for
$\alpha_0 \ne \alpha_1$.  Then we can choose $\beta$ such that $\phi(\theta) =
(\alpha_1 - \alpha_0) \log(1 - \theta)$
%
and
%
\begin{align*}
%
\norminf{\phi} =
    \abs{\alpha_1 - \alpha_0} \sup_{\theta \in [0,1]} \abs{\log(1 - \theta)} =
    \infty.
%
\end{align*}
%
Consequently, $\phi \notin \lp{\mu,\infty}$.
%
\end{ex}
% /\    /\    /\    /\    /\    /\    /\    /\    /\    /\    /\    /\    /\
%%%%%%%%%%%%%%%%%%%%%%%%%%%%%%%%%%%%%%%%%%%%%%%%%%%%%%%%%%%%%%%%%%%%%%%%%%%%
