We describe how to compute expectations with repsect to the stick-breaking
proportion $\nu_\k$. Let $f: \mathbb{R}\mapsto\mathbb{R}$ be a smooth function,
and we are interested in expectations of the form
\begin{align*}
  \expect{\q(\nuk \vert \eta)}{f(\nuk)}.
\end{align*}
For example, $f$ might be $f(\nu_\k) = \log \pstick(\nu_\k)$, whose
expectation appears in the $\mathrm{KL}$ divergence.

Recall that we chose the distribution on the logit-transformed
stick-breaking proportions $\lnu_\k$ to be normally distributed.
Let $\lnumean_\k$ and $\lnusd_\k$ be the location and scale, respectively,
of the Gaussian distribution on $\lnu_\k$.
Also let $\s$ be the sigmoid function, so that $\nu_\k = \s(\lnu_\k)$.

To compute expectations of a smooth function
$f(\nuk)$, the law of the unconscious statistician states that,
\begin{align*}
  \expect{\q(\nuk \vert \eta)}{f(\nuk)} ={}&
  \expect{\q(\lnu_\k \vert \eta)}
         {f\circ \s\left(\lnu_\k\right)}.
\end{align*}
By choosing $\q(\lnu_\k \vert \eta)$ to be Gaussian,
the right-hand side of is a Gaussian integral,
which we approximate
using GH quadrature with $\ngh$ knots,
located at $\xi_g$, weighted by $\omega_g$:
%
\begin{align}\eqlabel{gh_integral}
%
\expect{\q(\lnu_\k \vert \eta)}
       {f\circ \s\left(\lnu_\k\right)}
\approx{}&
    \sum_{g=1}^{\ngh} \omega_g f\circ \s \left(\lnusd_\k \xi_{g} + \lnumean_\k\right)
 \nonumber\\=:{}&
\expecthat{\q(\nuk \vert \eta)}{f(\nuk)}.
%
\end{align}
%
Using GH quadrature to approximate the expectation
is similar to the ``reparameterization trick,'' only using
GH points rather than standard normal draws.
