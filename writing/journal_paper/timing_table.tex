%%%%%%%%%%%%%%%%%%%%%%%%%%%%%%%%%%%%%%
%%%%%%%%%%%%%%%%%%%%%%%%%%%%%%%%%%%%%%
% Do not edit the TeX file your work
% will be overwritten.  Edit the RnW
% file instead.
%%%%%%%%%%%%%%%%%%%%%%%%%%%%%%%%%%%%%%
%%%%%%%%%%%%%%%%%%%%%%%%%%%%%%%%%%%%%%



The data sets we considered in our experiments had varying degrees
of complexity,
and the computational of cost of fitting the variational approximation
thus also varies accordingly.
Consistent across each experiment, the cost of forming the linear approximation
is roughly an order of magnitude faster than refitting (\tabref{timing_table})

For example, in the thrush data and fastSTRUCTURE model,
the initial fit took seven seconds, with subsequent refits
(which recall were warm-started with the initial fit) taking
between five and ten seconds.
Inverting the Hessian to form the linear approximation takes less than a second.
In all the examples, after forming the linear approximation, 
extrapolating to a new $\alpha$ or a 
perturbation $\phi$ takes \sfrac{1}{1000}th of a second. 


In each data set and model we consider, the the speed of the linear
approximation allows us to quickly explore a wide range of possible prior perturbations. 
However, even with the speed of our approximation, refitting for all possible perturbations is impossible.
The influence function, which is also cheap to compute, 
is able to provide a guide to uncover influential perturbations, 
at which we can then refit the model. 


% For example, in the mice data and regression model,
% the initial fit took 30 seconds, with subsequent refits
% (which recall were warm-started with the initial fit) taking
% between 20 and 30 seconds.
% Inverting the Hessian to form the linear approximation three seconds.
% Computing the influence function also takes less than a second.





\begin{table}[tb]
\centering
\caption{Compute time in seconds of various quantities on each data set.
Reported times for $\etaopt(\alpha)$ and $\etalin(\alpha)$ are
median times over the set of considered $\alpha$'s.
The reported influence function time is the time required to
evaluate the influence function on a grid of 1000 points. }
\tablabel{timing_table}
\begin{tabular}{|r|r|r|r|}
    \hline
    & iris & mice  & thrush \\
    \hline
    %%%%%%%%%%%%%%%%%%%%%%%%%%%%%%%%
    % initial fit
    %%%%%%%%%%%%%%%%%%%%%%%%%%%%%%%%
    Initial fit &
    1 &
    30 &
    7 \\
    \hline
    %%%%%%%%%%%%%%%%%%%%%%%%%%%%%%%%
    % hessian solve for alpha
    %%%%%%%%%%%%%%%%%%%%%%%%%%%%%%%%
    Hessian solve for $\alpha$ sensitivity &
    0.02 &
    3 &
    0.3 \\
    %%%%%%%%%%%%%%%%%%%%%%%%%%%%%%%%
    % linear approx time for alpha
    %%%%%%%%%%%%%%%%%%%%%%%%%%%%%%%%
    Linear approx. $\etalin(\alpha)$ &
    0.0008 &
    0.001 &
    0.0008 \\
    %%%%%%%%%%%%%%%%%%%%%%%%%%%%%%%%
    % refit time for alpha
    %%%%%%%%%%%%%%%%%%%%%%%%%%%%%%%%
    Refits $\etaopt(\alpha)$ &
    0.5 &
    30 &
    5 \\
    \hline
    %%%%%%%%%%%%%%%%%%%%%%%%%%%%%%%%
    % influence function
    %%%%%%%%%%%%%%%%%%%%%%%%%%%%%%%%
    The influence function &
    0.09 &
    3 &
    0.6 \\
    %%%%%%%%%%%%%%%%%%%%%%%%%%%%%%%%
    % hessian solve for functional perturbation
    %%%%%%%%%%%%%%%%%%%%%%%%%%%%%%%%
    Hessian solve for $\phi$ &
    0.02 &
    3 &
    0.4\\
    %%%%%%%%%%%%%%%%%%%%%%%%%%%%%%%%
    % linear approx for functional perturbation
    %%%%%%%%%%%%%%%%%%%%%%%%%%%%%%%%
    Linear approx. $\etalin(\phi)$ &
    0.001 &
    0.001 &
    0.0008 \\
    %%%%%%%%%%%%%%%%%%%%%%%%%%%%%%%%
    % refit time for functional perturbation
    %%%%%%%%%%%%%%%%%%%%%%%%%%%%%%%%
    Refit $\etalin(\phi)$ &
    0.6 &
    20 &
    10 \\
    \hline
\end{tabular}
\end{table}
