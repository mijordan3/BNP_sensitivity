A principal challenge of computing the sensitivity efficiently is the
high-dimensionality of the parameter $\zeta$ and hence variational parameters
$\eta$. In particular, we have seen that, in our BNP stick-breaking model,
$\zeta$ and $\eta$ both grow linearly with the number of data points $N$. This
growth leads to two potential major computational challenges: (1) a
high-dimensional optimization problem to extremize the VB objective and (2)
computing and inverting the Hessian $\hessopt$. Here we show how we can use
special structure in the model to reduce to low-dimensional problems and thereby
enjoy efficient computation.

%%
\noindent \textbf{Global and local parameters.} In both cases, the key to
reducing to a lower-dimensional problem is separating \emph{global} and
\emph{local} parameters. Global variables are common to all data points. Local
variables are unique to each data point. For instance, in a Gaussian (or other
typical) mixture model, the stick-breaking proportions $\nu$ and component
parameters $\beta$ are global. But the cluster assignment parameters $z$ are
local.

Let $\gamma$ denote the collection of global parameters. Since we use mean-field
VB, these parameters have their own variational parameters, which we denote
$\etaglob$. Similarly, let $\ell$ denote the local parameters and let
$\etalocal$ be the corresponding local variational parameters.

%%
\noindent \textbf{Reducing to optimization over the global variational
parameters.} We next show how to reduce the potentially high-dimensional
optimization problem over all of $\eta$ to optimizing over just the global
variational parameters $\etaglob$.

In all models we will consider, the conditional posterior $\p(\z \vert
\gamma,\x)$ has a tractable closed form.  Since we choose a conjugate mean field
approximating family for $\q(\z \vert \eta)$, the optimal local variational
parameters $\etaoptlocal$ can be written as a closed-form function of the global
variational parameters $\etaglob$. Let $\etaoptlocal(\eta_\gamma; \t)$ denote
this mapping, so that
%
\begin{align}\eqlabel{local_eta_optim}
\etaoptlocal(\etaglob; \t) :=
    \argmin_{\etalocal} \KL{(\eta_\gamma, \etalocal), \t}.
\end{align}
%
%When $\t=0$, we will write $\etaoptlocal(\etaglob; \t=0) =
%\etaoptlocal(\etaglob)$.
In \exref{qz_optimality} (\appref{gmm_global_local_vb}), we illustrate with a
Gaussian mixture model example.
%
Using \eqref{local_eta_optim}, we can rewrite our objective as a
function of the global parameters.  Define
%
\begin{align*}
\KLglobal(\etaglob, \t) :=
    \mathrm{KL}\Big((\etaglob, \etaoptlocal(\etaglob; \t)), \t\Big).
\end{align*}
%
The $\etaoptglob(\t)$ that minimizes $\KLglobal(\etaglob, \t)$ is the same as
the corresponding sub-vector of the $\etaopt(\t)$ that minimizes $\KL{\eta,
\t}$.  %Therefore we can use the objective function $\KLglobal(\etaglob, \t)$ as
%a numerical surrogate for $\KL{\eta, \t}$ when optimizing or computing
%derivatives.

Rather than optimizing the $\KL{\eta}$ over all variational parameters then, we
numerically optimize $\KLglobal$, which is a function only of the relatively
low-dimensional global parameters.  To minimize $\KLglobal(\etaglob)$ in
practice, we run the BFGS algorithm with a loose convergence tolerance followed
by trust-region Newton conjugate gradient
\citep[Chapter~7]{nocedal:2006:numerical} to find a high-quality optimum. After
the optimization terminates at an optimal $\etaoptglob$, the optimal local
parameters $\etaoptlocal$ can be set in closed form to produce the entire vector
of optimal variational parameters $\etaopt = (\etaoptglob, \etaoptlocal)$.

\subsection{Computing and inverting the Hessian} Since the dimension $\etadim$
of $\eta$ scales with $N$, we can quickly reach cases where inverting or even
instantiating a dense matrix of size $\etadim \times \etadim$ in memory would be
prohibitive. The key to efficient computation is that $\hessopt$ is not dense;
we will again exploit structure inherent in the global/local decomposition.

For generic variables $a$ and $b$, let $\hess{ab}$ denote the sub-matrix
$\evalat{\partial^2 \KL{\eta} / \partial \eta_a \eta_b^T}{\etaopt}$, the Hessian
with respect to the variational parameters governing $a$ and $b$. We decompose
the Hessian matrix $\hessopt$ into four blocks according to the global/local
decomposition:
%
\begin{align*}
%
\hessopt =
\fracat{\partial^2 \KL{\eta}}
       {\partial \eta \partial \eta^T}
       {\etaopt} ={}&
\left(
\begin{array}{cc}
   \hess{\gamma\gamma} & \hess{\gamma\ell} \\
   \hess{\ell\gamma}     & \hess{\ell\ell} \\
\end{array}
\right).
%
\end{align*}
%
Similarly, let $\crosshessian_\gamma$ be the components of $\crosshessian$
corresponding to the variational parameters $\etaglob$.  The local components,
$\crosshessian_\ell$, are zero since no local variables enter the expectation in
\eqref{bnp_vb_crosshessian} when we are perturbing the stick-breaking
distribution.
%
%We can thus write
%\begin{align*}
%  \crosshessian = \left( \begin{array}{c} \crosshessian_\gamma \\ 0 \end{array}\right).
  %
%\end{align*}

In this notation,
%
\begin{align} \eqlabel{global_local_derivative_breakdown}
%
\fracat{d \etaopt(\t)}{d \t}{t = 0} ={}&
-\left(
\begin{array}{cc}
   \hess{\gamma\gamma} & \hess{\gamma\ell} \\
   \hess{\ell\gamma}     & \hess{\ell\ell} \\
\end{array}
\right)^{-1}
\left( \begin{array}{c} \crosshessian_\gamma \\ 0 \end{array}\right),
%
\end{align}
%
Applying the Schur complement and focusing on the global parameters (see
\appref{more_hessian} for more details), we find
%
\begin{align}\eqlabel{global_sens}
  \fracat{d \etaopt_\gamma(\t)}{d \t}{t = 0} &=
  - \hessopt_\gamma^{-1}\crosshessian_\gamma
  \mathwhere
  \hessopt_\gamma := \left(\hess{\gamma\gamma} -
        \hess{\gamma\ell} \hess{\ell\ell}^{-1} \hess{\ell\gamma}\right),
\end{align}
%
In our model, $\hess{\ell\ell}$ is block diagonal, and the size of
$\hess{\gamma\gamma}$ is relatively small. Thus, each term of $\hessopt_\gamma$
can be tractably computed, stored in memory, and inverted, even on very large
datasets. While the Schur complement calculation is illustrative, we can get the
same benefits directly from automatic differentiation; see \appref{more_hessian}
for details.

In our BNP applications, it is not cost-effective to form and factorize
$\hessopt$ in memory.  Instead, we numerically solve linear systems of the form
$\hessopt^{-1} v$ using the conjugate gradient (CG) algorithm \citep[Chapter
5]{nocedal:2006:numerical}, which requires only Hessian-vector products that are
readily available through automatic differentiation.

%%
\noindent \textbf{A linear approximation only in the global variational parameters}
%
With the tools above, we can separate out the linear approximation in the global
parameters and then directly compute the local parameters. In particular, we
compute
%
\begin{align}\eqlabel{global_lin_approx}
  \etalin_\gamma(\t) := \etaopt_\gamma +
  \fracat{d \etaopt_\gamma(\t)}{d \t}{\t=0} \t .
\end{align}
%
and then use $\etaoptlocal(\etaglob)$ e.g.\ in computing our quantity of
interest. We give an example for the expected number of clusters in
\appref{vb_insample_nclusters_example}.  In all our experiments, we use
\eqref{global_lin_approx} in this way.
