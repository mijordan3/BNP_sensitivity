
\subsubsection*{the cross-hessian}

\todo{Move the discussion about the cross-hessian earlier: maybe after corollary 1.}

Define
\begin{align*}
\crosshessian :=   \expect{\q(\theta \vert \etaopt)}{
      \lqgradbar{\theta \vert \etaopt}
      \fracat{\partial \log \p(\theta \vert \t)}{\partial \t}{\t=0}
  }
\end{align*}

\todo{maybe define $\crosshessian$ in theorem 1?}

We assume that $\q(\theta \vert \eta)$ is normalized, so
$\lqgradbar{\theta \vert \etaopt} = \lqgrad{\theta \vert \etaopt}$.

In our case, only the prior on stick-breaking proportions
$\nu = (\nu_1, ..., \nu_{\kmax - 1})$ depends on $t$.
Because $\nu$ fully factorizes under both the prior and the variational distributions,
$\crosshessian$ decomposes as
\begin{align}
  \crosshessian &=
  \sum_{\k=1}^{\kmax - 1}
          \expect{\q(\nuk \vert \eta)}
                 {
                 \lqgrad{\nuk \vert \etaopt}
                 \fracat{\log \pstick(\nuk \vert \t)}{\partial \t}{\t = 0}
                 } \notag\\
  &= \sum_{\k=1}^{\kmax - 1}
         \evalat{\nabla_\eta \expect{\q(\nuk \vert \eta)}
                {
                \fracat{\log \pstick(\nuk \vert \t)}{\partial \t}{\t = 0}
                }}{\eta = \etaopt(0)}
\eqlabel{sens_mixed_partial}
\end{align}
We approximate the expectation using GH quadrature (\eqref{gh_integral}),
with
% $f(\nuk) = \frac{\partial}{\partial t}[\log \pstick(\nuk \vert \t)]\vert_{t =0}$.
$f(\nu_k) = \fracat{\log \pstick(\nuk \vert \t)}{\partial \t}{\t = 0}$.
In all the functional forms for
$\t \mapsto \pstick(\nuk \vert \t)$ considered below,
$f(\nu_k)$ can be provided in either closed-form or computed with automatic differentiation.
The resulting GH approximation is a deteriminstic function of $\eta$,
and another application of automatic differentiation with respect to $\eta$
provides the derivative in \eqref{sens_mixed_partial}.
Observe that $\crosshessian$ is sparse:
it is zero for all entries of
$\eta$ other than those that parameterize the sticks.

\subsubsection*{The hessian}

Typically, it is the computation and inversion of the Hessian matrix that is
the most computationally intensive part of \eqref{vb_eta_sens}, especially when
$\etadim$ is large.
In particular, though the dimensions of $\etanu$ and
$\etabeta$ are of order $\kmax$, the dimension of $\etaz$ is of order $\kmax
\times\N$.
For large $\N$, it may not be possible to even instantiate the Hessian matrix in memory.

Fortunately, the latent variables in our BNP model (\eqref{bnp_model}) factorize
into a set of ``global" and ``local" latent variables.
The global latent variables are the cluster latent variables $\beta$ and
the stick-breaking proportions $\nu$;
these variables are ``global" because they are shared across the entire data set.
The local variables are the cluster assignments $z$, which are unique to each data point.
We take advantage of this structure to more efficiently compute the Hessian,
as we now describe.

Let $\gamma := (\beta, \nu)$ denote the global latent variables, and
$\eta_\gamma := (\etabeta, \etanu)$ be the variational parameters for the global latent variables.
Note that the dimension of $\eta_\gamma$ is of order $\kmax$ and does not scale with
the number of data points $\N$.

For generic parameters $a$ and $b$
let $\hess{ab}$ denote $\evalat{\partial^2 \KL{\eta, \t} / \partial \eta_a
\eta_b^T}{\etaopt(0), t=0}$, the Hessian with respect to the variational
parameters governing $a$ and $b$.  Specifically:
%
\begin{align*}
%
\fracat{\partial^2 \KL{\eta, \t}}
       {\partial \eta \partial \eta^T}
       {\etaopt(0), t= 0} ={}&
\left(
\begin{array}{cc}
   \hess{\gamma\gamma} & \hess{\gamma\z} \\
   \hess{\z\gamma}     & \hess{\z\z} \\
\end{array}
\right).
%
\end{align*}

Next, let $\crosshessian_\gamma$ be the components of
$\crosshessian$ corresponding to the variational parameters of
$\gamma$---that is, $\crosshessian_\gamma$ is given by replacing
the operator $\nabla_\eta$ with $\nabla_{\eta_\gamma}$ in \eqref{sens_mixed_partial}.
The analagous quantity for $z$, $\crosshessian_\z$, is zero, since $\z$ does not enter the expectation in \eqref{sens_mixed_partial}.
We can thus write
\begin{align*}
  \crosshessian = \left( \begin{array}{c} \crosshessian_\gamma \\ 0 \end{array}\right).
  %
\end{align*}

In this notation,
%
\begin{align*}
%
\fracat{d \etaopt(\t)}{d \t}{t = 0} ={}&
-\left(
\begin{array}{cc}
   \hess{\gamma\gamma} & \hess{\gamma\z} \\
   \hess{\z\gamma}     & \hess{\z\z} \\
\end{array}
\right)^{-1}
\left( \begin{array}{c} \crosshessian_\gamma \\ 0 \end{array}\right)
%
\end{align*}
%
And an application of the Schur complement gives
%
\begin{align*}
%
\fracat{d \etaopt(\t)}{d \t}{t = 0} ={}&
-\left(\begin{array}{c}
I_{\gamma\gamma} \\
\hess{\z\z}^{-1} \hess{\z\gamma}
\end{array}\right)
\left(\hess{\gamma\gamma} -
      \hess{\gamma\z} \hess{\z\z}^{-1} \hess{\z\gamma}\right)^{-1} \crosshessian_\gamma,
%
\end{align*}
where $I_{\gamma\gamma}$ is the identity matrix with
the same dimension as $\eta_\gamma$.
Specifically, observe that the sensitivity of the global parameters
is given by
\begin{align}\eqlabel{global_sens}
  \fracat{d \etaopt_\gamma(\t)}{d \t}{t = 0} &=
  - \hessopt_\gamma^{-1}\crosshessian_\gamma
  \mathwhere
  \hessopt_\gamma := \left(\hess{\gamma\gamma} -
        \hess{\gamma\z} \hess{\z\z}^{-1} \hess{\z\gamma}\right).
\end{align}

$\hessopt_\gamma$ is of dimension $D_\gamma \times D_\gamma$,
with $D_\gamma$ the dimension of $\eta_\gamma$.
$D_\gamma$ is of order $\kmax$ and
does not scale with the number of data points $\N$.
Hence, $\hessopt_\gamma$ is more memory efficient to store
and faster to invert
than the full $\etadim\times\etadim$ Hessian of all the variational parameters.

Each term in \eqref{global_sens} can be easily computed using automatic differentiation.
Moreover, the $\hess{\z\z}$ term is block-diagonal with $N$ blocks of size $\kmax\times\kmax$;
the $n$-th block corresponds to the variational parameters governing $z_n$.
Each block, and resulting matrix $\hess{\z\z}$, has a closed form inverse.

Alternatively, recall that the optimal local variational parameters $\etaoptz$ can be written
as a closed-form function of the global variational parameters $\eta_\gamma$.
We write $\etaoptz(\eta_\gamma; \t)$ to denote the optimal local variational parameters given the
global variational parameters at prior parameter $\t$; that is,
\begin{align*}
  \etaoptz(\eta_\gamma; \t) := \argmin_{\eta_\z} \KL{(\eta_\gamma, \eta_z), \t}
\end{align*}


Using \eqref{global_sens}, we can linearize the global parameters:
\begin{align}\eqlabel{etalin_global_def}
  \etalin_\gamma(\t) := \etaopt_\gamma +
  \fracat{d \etaopt_\gamma(\t)}{d \t}{\t=0} \t .
\end{align}



\subsection*{old text}







% However, this Hessian matrix does not depend on $\t$, and
% so can be computed once and re-used for many different values of $\t$ or many
% different classes of prior perturbations. In our case, w

  Fortunately, the Hessian of the KL divergence
is sparse and highly structured in $\etaz$.  In particular, it is
block-diagonal, with
%
\begin{align*}
%
\fracat{\partial^2 \KL{\eta, \t}}
       {\partial \eta_{\z_{n}} \partial \eta_{\z_{m}}^T}
       {\etaopt(t = 0), \t_0} ={}& 0 \mathtxt{for}n\ne m.
%
\end{align*}
%
Recall from \exref{qz_unconstrained} that $\eta_{\z_\n}$ is of dimension $\kmax -
1$, and that the Hessian of $\expect{\q(\zeta \vert \eta)}{\logp(\x \vert
\zeta)}$ with respect to $\etaz$ may be nonzero despite $\logp(\x \vert \zeta)$
being linear in $\z$ due to the unconstrained parameterization of $\etaz$.

We propose the following techniques for computing the inverse Hessian when the
fully matrix is prohibitively large.  To describe the techniques it will be
useful to use the following compact notation.  Let $\gamma = (\beta, \nu)$
denote all the parameters besides $\z$, and for generic parameters $a$ and $b$
let $\hess{ab}$ denote $\evalat{\partial^2 \KL{\eta, \t} / \partial \eta_a
\eta_b^T}{\etaopt(\t_0), \t_0}$, the Hessian with respect to the variational
parameters governing $a$ and $b$.  Specifically:
%
\begin{align*}
%
\fracat{\partial^2 \KL{\eta, \t}}
       {\partial \eta \partial \eta^T}
       {\etaopt(\t_0), \t_0} ={}&
\hess{\zeta\zeta} =
\left(
\begin{array}{cc}
   \hess{\gamma\gamma} & \hess{\gamma\z} \\
   \hess{\z\gamma}     & \hess{\z\z} \\
\end{array}
\right).
%
\end{align*}
%
The terms $\hess{\gamma\gamma}$ and $\hess{\z\gamma} = \hess{\gamma\z}^T$ are
typically dense and easily computed using atuomatic differentiation, and the
term $\hess{\z\z}$ is block diagonal with a closed-form inverse.
%
Analogously, let
%
\begin{align*}
%
\hess{\zeta\t} :=
\fracat{\partial^2 \KL{\eta, \t}}
       {\partial \eta \partial \t}
       {\etaopt(\t_0), \t_0},
%
\end{align*}
%
and observe that
%
\begin{align*}
%
\hess{\zeta\t} ={}& \left( \begin{array}{c} \hess{\gamma\t} \\ 0 \end{array}\right).
%
\end{align*}

In this notation,
%
\begin{align*}
%
\fracat{d \etaopt(\t)}{d \t}{\t_0} ={}&
-\left(
\begin{array}{cc}
   \hess{\gamma\gamma} & \hess{\gamma\z} \\
   \hess{\z\gamma}     & \hess{\z\z} \\
\end{array}
\right)^{-1}
\left( \begin{array}{c} \hess{\gamma\t} \\ 0 \end{array}\right)
%
\end{align*}
%
We can then use the Schur complement to get the computationally tractable
expression:
%
\begin{align*}
%
\fracat{d \etaopt(\t)}{d \t}{\t_0} ={}&
-\left(\begin{array}{c}
I_{\gamma\gamma} \\
\hess{\z\z}^{-1} \hess{\z\gamma}
\end{array}\right)
\left(\hess{\gamma\gamma} -
      \hess{\gamma\z} \hess{\z\z}^{-1} \hess{\z\gamma}\right)^{-1} \hess{\gamma\t}.
%
\end{align*}

In fact, the former expression can be computed entirely using automatic
differentiation using the fact that $\etaoptz$ has an explicit closed form given
$\etaoptgamma$, as dicussed in \exref{qz_form}.  TODO: is this worth writing
here?

When even $\hess{\gamma\gamma}$ is too large to form in memory, one can
use the conjugate gradient algorithm.  TODO: fill in details.



%
%  we can evaluate the mixed partial using the derivative of
% \eqref{gh_integral}.  We will consider particular functional forms for
% $\t \mapsto \pstick(\nuk \vert \t)$ in below.
%
%
% In order to evaluate $d\etaopt(\t) / d\t$ as given by \eqref{vb_eta_sens}, we
% need to solve a linear system involving the $\etadim \times \etadim$ Hessian of
% the objective function $\KL{\eta, 0}$ and the $\etadim \times 1$ mixed
% second-order partial derivative of $\KL{\eta, \t}$ with respect to $\eta$ and
% $\t$.
% \todo{Theorem 1 isn't in the form of a mixed partial; maybe this needs more explaining.
% Perhaps we could put \eqref{sens_mixed_partial} and a discussion about this earlier?
% e.g. after corollary 1?}
% We will discuss these two tasks in turn.
%
% First, consider the mixed partial derivative $\partial^2 \KL{\eta, \t} /
% \partial \eta \partial \t$.  In our case, only the priors $\log \pstick(\nuk
% \vert \t)$ depend on $\t$, and so
% %
% \begin{align}\eqlabel{sens_mixed_partial}
% %
% \fracat{\partial^2 \KL{\eta, \t}}
%        {\partial \eta \partial \t}
%        {\etaopt(\t_0), \t_0} ={}&
% \sum_{\k=1}^{\kmax}
%     \evalat{
%         \frac{\partial}{\partial\eta}
%         \expect{\q(\nuk \vert \eta)}
%                {\fracat{\log \pstick(\nuk \vert \t)}{\partial \t}{\t_0}
%                }
%         }
%         {\etaopt(\t_0)}.
% %
% \end{align}
% %
% Consequently, as long as we can explicitly evaluate $\fracat{\log \pstick(\nuk
% \vert \t)}{\partial \t}{\t_0} $, either in closed form or with automatic
% differentiation, we can evaluate the mixed partial using the derivative of
% \eqref{gh_integral}.  We will consider particular functional forms for
% $\t \mapsto \pstick(\nuk \vert \t)$ in below.
%
% Observe that \eqref{sens_mixed_partial} will be zero for all entries of
% $\eta$ other than those that parameterize the sticks.
