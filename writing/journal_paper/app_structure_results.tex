Recall from \secref{results_structure} and \figref{stru_func_sens_admix}
that the linear approximation
failed to capture the change in the admixture proportion of an individual,
$n = 25$ after a worst-case functional perturbation.

\figref{stru_lin_bad_example} examines individual $n = 25$ more closely.
The bottom row plots this individual's
admixture proportions as $\t$ varies from 0 to 1 in the perturbed prior
$\p(\nu\vert \t) = \p_0(\nuk)\exp(\t\phiworstcase(\nuk))$.
The linearized parameters poorly captured the change in admixture proportions observed after refitting, particularly
for populations 1 and 2, for values of $\t$ close to 1.
Even though we retain non-linearities
in the mapping from variational parameters to the posterior statistic,
for this perturbation, the mapping from prior parameter
$\t$ to the relevant variational parameters
is highly non-linear.
This latter mapping is what we linearize
and what causes our approximation to fail in this case.
Specifically, the variational location parameter on the first stick-breaking proportion is concave as a function of $\t$ ---
the location parameter increases for small $\t$,
then decreases as $\t\rightarrow1$.
However,
$\etalin(\t)$ linearizes the relationship between the location parameter and $\t$.
Therefore, the corresponding admixture mixture proportion of
population 1 is over-estimated under the linearized variational parameters.
Furthermore, because our linearized variational parameters
over-estimated the length of the first stick,
and the second admixture proportion is a product of the
remaining stick times the second stick-breaking proportion,
the linearized variational parameters then under-estimates
the admixture proportion of population 2.

\StructureLimitationsA

\figref{stru_fully_lin_example} shows a similar situation for individual $n = 74$.
The linearized variational parameters grossly over-estimated the length of the first stick,
resulting in the later admixture proportions being under-estimated.
The third admixture proportion was particularly poorly approximated under the linearized variational parameters.
Given the recursive nature of the relationship between admixtures and stick-breaking proportions, errors at early sticks affect later admixture proportions.
Fully linearizing the mapping $\t\mapsto\g(\etaopt(\t))$ to form the approximation
$\glin(t)$ avoids this problem.
In this example, $\glin(t)$ outperforms $\g(\etalin(t))$, with $\g$ being the admixture proportion of population 3.
In our experience,
computing $\g(\etalin(t))$, and thus retaining non-linearities in the mapping from $\eta\mapsto\g(\eta)$,
is usually beneficial to the quality of the approximation.
It is likely that $\g(\etalin(t))$ outperforms $\glin(\t)$ for most posterior quantities,
though as we see in \figref{stru_fully_lin_example},
this is not guaranteed to always be true.

\StructureLimitationsB
