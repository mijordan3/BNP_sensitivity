Recall from \thmref{etat_deriv} that we require the optimal variational parameters
$\etaopt$ to be in the interior of its domain. One way to achieve this
is to use only \textit{unconstrained} parameterizations for the
component distributions of $\q$. One such parameterization was
presented in \exref{qz_optimality}, where we let
$\eta_{\z_\n}$, which parameterize the cluster assignments, be allowed to
take any value in $\mathbb{R}^{\kmax - 1}$; the assignment probabilities
$m_{\n}\in\mathbb{R}^{\kmax}$, which are constrained to sum to one,
are then formed with an appropriate transform of the unconstrained parameters
$\eta_{\z_\n}$

Other variables require careful parameterization as well.
For instance, instead of parameterizing the normal distribution on
logit-sticks $\lnu_\k$ using a mean and variance, we let $\eta_{\nu_\k}\in\mathbb{R}^2$
be the mean and \textit{log} variance. The variance is constrained to be positive;
the log-variance is unconstrained on the real line.
In general, a real-valued parameter $\mu_i$ which must be constrained
$a \leq \mu_i \leq b$ can be transformed to its unconstrained parameterization
by letting
\begin{align*}
  \eta_i = \log(\mu_i - a) - \log(b - \mu_i).
\end{align*}

In the variational approximation to the GMM model,
we let the component variables $\beta_\k$ be Normal-Wishart.
In this case, the scale matrix of the Normal-Wishart, $W_\k\in\mathbb{R}^{d\times d}$,
is constrained be positive definite.
Because $W$ is symmetric, we only neeed $d(d + 1) / 2$ parameters to represent it.
To form an unconstrained parameterization, we factorize $W$ using the Cholesky decomposition,
\begin{align*}
W = L^T L,
\end{align*}
where $L$ is a lower-triangular matrix, with positive diagonal entries.
The unconstrained parameterization of $W$ is then taken to be
the strictly lower-diagonal entries of $L$,
along with the $\log$ of the diagonal entries of $L$.
