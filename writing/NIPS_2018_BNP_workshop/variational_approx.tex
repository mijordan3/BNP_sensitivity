%
A true BNP representation would take $K = \infty$, but in order to make a
computationally feasible algorithm we truncate $K$ at a value large enough so
that many clusters are essentially unoccupied in the approximate posterior
\citep{blei:2006:dirichletbnp}. The variational approximation is
%
\begin{align}
q(\nu, \mu, \Sigma, z) & =
\Big\{\prod_{k=1}^{K}q\left(\nu_{k}\right)\delta\left(\mu_{k}\right)
    \delta\left(\Sigma_{k}\right)\Big\} \prod_{n=1}^{150}q\left(z_{n}\right),
\end{align}
%
where $\delta\left(\cdot\right)$ denotes a point mass at a parameterized
location, $q\left(\nu_{k}\right)$ is a logitnormal distribution, and
$q\left(z_{n}\right)$ is a multinomial distribution.

With this variational approximation, we seek
%
\begin{align}
\eta^* = \argmin_{\eta} KL\left(
    q_\eta(\nu, \mu, \Sigma, z) \big\| p(\nu, \mu, \Sigma, z | y)
    \right) \label{eq:kl_objective}
\end{align}
%
where we use $\eta$ to represent the variational parameters be the parameters of
the variational distribution (the mean variance of the logitnormal, the
locations of the point masses, etc.).

We use this variational distribution to define our posterior quantitiy of
interest: for us, it will be the expected number of clusters,
%
\begin{align}
  E_{q(\pi)}\Big[E\Big[\#\{\text{distinct clusters}\}|\pi\Big]\Big] &=
  E_{q(\pi)}\Big[\sum_{i=1}^K P[\text{cluster $k$ is in the dataset}|\pi_k]\Big] \\
    &= E_{q(\pi)} \Big[\sum_{i=1}^K (1 - (1 - \pi_k)^N)\Big]
    \label{eq:expected_num_clusters}
\end{align}
%
For a given variational distribtion, this quanitity can be computed with
Monte-Carlo draws of sticks $\nu_k$ from the $q(\nu_k)$, and the weights $\pi_k$
are functions of the sticks $\nu_k$ (recall eq. \ref{eq:stick_breaking}).

In the next section we will outline our method for quickly evaluating the
sensitivity of this quantity with respect to various prior perturbations.

% Note that to compute the sensitivity using equation \ref{eq:func_sensitivity},
% we require the derivative
% of this expectation with respect to the variational parameters in $q$. We can compute
% the expectation by sampling the sticks $\nu_k$ from the variational distribution, and
% the cluster weights $\pi_k$'s are functions of the sticks $\nu_k$. Derivatives
% are computed by applying the reparameterization trick to the sticks $\nu_k$.
