%%%%%%%%%%%%%%%%%%%%%%%%%%%%%%%%%%%%%%
%%%%%%%%%%%%%%%%%%%%%%%%%%%%%%%%%%%%%%
% Do not edit the TeX file your work
% will be overwritten.  Edit the RnW
% file instead.
%%%%%%%%%%%%%%%%%%%%%%%%%%%%%%%%%%%%%%
%%%%%%%%%%%%%%%%%%%%%%%%%%%%%%%%%%%%%%



On the Iris data, we chose two different functional perturbations: for the first
we let $p_1(\nu_k)$ be a logit normal with parameters $\mu = -2, \sigma = 1$;
for the second, we let $p_1(\nu_k)$ be a logit normal with parameters $\mu = 2,
\sigma = 1$. In both cases, we then chose $\phi(\nu_k) = p_1(\nu_k) /
p_{0k}(\nu_k)$.



\begin{knitrout}
\definecolor{shadecolor}{rgb}{0.969, 0.969, 0.969}\color{fgcolor}\begin{kframe}


{\ttfamily\noindent\bfseries\color{errorcolor}{\#\# Error in file(file, "{}rt"{}): cannot open the connection}}\end{kframe}
\end{knitrout}
%
We find that the linear approximation in this case was able to capture the
direction of the perturbation, (the expected number of clusters increased under
the first pertubation, decreased under the second), although as $\delta
\rightarrow 1$ the quality of the approximation degraded.
