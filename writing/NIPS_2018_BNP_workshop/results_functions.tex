%%%%%%%%%%%%%%%%%%%%%%%%%%%%%%%%%%%%%%
%%%%%%%%%%%%%%%%%%%%%%%%%%%%%%%%%%%%%%
% Do not edit the TeX file your work
% will be overwritten.  Edit the RnW
% file instead.
%%%%%%%%%%%%%%%%%%%%%%%%%%%%%%%%%%%%%%
%%%%%%%%%%%%%%%%%%%%%%%%%%%%%%%%%%%%%%




% TODO: define these in knitr instead.
% \global\long\def\splinedegree{3}
% \global\long\def\ntime{14}
% \global\long\def\ngenes{1000}
% \global\long\def\nclusters{18}
% \global\long\def\fullparamdim{66199}
% \global\long\def\covregularization{0.1}
%
% \global\long\def\regmat{X_{df}}
% \global\long\def\mbe{\mathbb{E}}
% \global\long\def\cov{\mathrm{Cov}}
% \global\long\def\thetareg{\theta_{r}}
% \global\long\def\thetaclust{\theta_{c}}

\paragraph{Experimental Results for Functional Perturbations.}
%
We now use the functional perturbation described in
\prettyref{eq:expon_perturb} to perturb the prior on the stick distribution. The
results are shown in \prettyref{fig:func_sens_e_num_clusters}.

We chose two different functional perturbations: for the first we let
$p_1(\nu_k)$ be a logit normal with parameters $\mu = -2, \sigma = 1$; for the
second, we let $p_1(\nu_k)$ be a logit normal with parameters $\mu = 2, \sigma =
1$. In both cases, we then chose $\phi(\nu_k) = p_1(\nu_k) / p_{0k}(\nu_k)$.



\begin{knitrout}
\definecolor{shadecolor}{rgb}{0.969, 0.969, 0.969}\color{fgcolor}\begin{figure}[!h]

{\centering \includegraphics[width=0.98\linewidth,height=0.686\linewidth]{figure/functional_sens_plot-1} 

}

\caption{\label{fig:func_sens_e_num_clusters}
Left column: the original prior p0 in purple, the perturbed prior pc in red. Right: linearly approximated vs.
re-fitted expected number of clusters after the purtubation.  }\label{fig:functional_sens_plot}
\end{figure}


\end{knitrout}



% \begin{figure}[h!]
% 	\centering
% 	\begin{subfigure}[t]{0.32\textwidth}
% 		\includegraphics[width = \textwidth]{./functional_sens_results/perturbed_prior1_init3_5.png}
% 		\subcaption{}
% 	\end{subfigure}
%   \begin{subfigure}[t]{0.32\textwidth}
%     \includegraphics[width = \textwidth]{./functional_sens_results/pred_num_clusters1_init3_5.png}
%     \subcaption{}
%   \end{subfigure}\\
%   \centering
%   \begin{subfigure}[t]{0.32\textwidth}
%     \includegraphics[width = \textwidth]{./functional_sens_results/perturbed_prior2_init3_5.png}
%     \subcaption{}
%   \end{subfigure}
%   \begin{subfigure}[t]{0.32\textwidth}
%     \includegraphics[width = \textwidth]{./functional_sens_results/pred_num_clusters2_init3_5.png}
%     \subcaption{}
%   \end{subfigure}
% 	\caption{Left column: the original prior in purple, the perturbed prior $p_1$ in red. Right: linearly approximated vs.
%   re-fitted expected number of clusters after purturbing each stick by $\phi(\nu_k)^\epsilon =
%   [p_1(\nu_k) / p_{0k}(\nu_k)]^\epsilon$.  }
% 	\label{fig:func_sens_e_num_clusters}
% \end{figure}
%
We find that the linear approximation in this case was able to capture the
direction of the perturbation, (expected number of clusters increased under the
first pertubation, decreased under the second), but did not provide a good
approximation at $\epsilon = 1$.
