

%%%%%%%%%%%%%%%%%%%%%%%%%%%%%%%%%%%%%%%%%%%%%%%%%%%%%%%%%%%%%%%%%%%%%%%%%%%%%
%%%%%%%%%%%%%%%%%%%%%%%%%%%%%%%%%%%%%%%%%%%%%%%%%%%%%%%%%%%%%%%%%%%%%%%%%%%%%

\begin{proof}
%\prooflabel{pert_well_defined}
%\proofof{\thmref{pert_well_defined}}
%
First, consider the fact that perterubations are priors. It is clear that
$\mathscr{P}_{\mathrm{valid}} \subseteq \mathscr{P}_{\pbase,p}$, since one can
simply take $\t = 1$ for any $\palt \in \mathscr{P}_{\mathrm{valid}}$.  To show
that $\mathscr{P}_{\pbase,p} \subseteq \mathscr{P}_{\mathrm{valid}}$, it will
suffice to show that any element of $\mathscr{P}_{\pbase,p}$ is non-negative and
can be normalized.

Take $p \in [1, \infty)$ with $\phi \in \pertset$.  By definition, for $\phi \in
\pertset$, there exist $\beta > 0$, $\t \in [0,1]$, and $\palt(\theta)$ such
that
%
\begin{align*}
%
\pbase(\theta)^{1/p} + \frac{1}{p} \phi(\theta) ={}&
    \t \beta \palt(\theta)^{1/p} + (1- \t) \pbase(\theta)^{1/p}.
%
\end{align*}
%
From this it follows that $\ptil(\theta \vert \t \phi) \ge 0$, since $\t \in
[0,1]$.  Furthermore, for the same $\phi$,
%
\begin{align*}
%
\int \ptil(\theta \vert \phi) \mu(d\theta) ={}&
\int \left(\t \beta \palt(\theta)^{1/p} +
           (1- \t) \pbase(\theta)^{1/p} \right)^{p} \mu(d\theta)
\\\ge{}&
\t^p \beta^p \int \palt(\theta) \mu(d\theta) +
    (1- \t)^p \int \pbase(\theta) \mu(d\theta)
\\={}& \t^p \beta^p + (1- \t)^p > 0,
%
\end{align*}
%
and, by Jensen's inequality,
%
\begin{align*}
%
\int \ptil(\theta \vert \phi) \mu(d\theta) ={}&
2^p \int \left(\frac{1}{2} \t \beta \palt(\theta)^{1/p} +
           \frac{1}{2} (1- \t) \pbase(\theta)^{1/p} \right)^{p} \mu(d\theta)
\\\le{}&
%
2^{p-1} \left(
    \t^p \beta^p \int \palt(\theta) \mu(d\theta) +
    (1- \t)^p \int  \pbase(\theta)\mu(d\theta)
\right)
\\={}& 2^{p-1} \left( \t^p \beta^p + (1- \t)^p\right) < \infty.
%
\end{align*}

For $p = \infty$ and $\phi \in \pertset[\infty]$, it is clear that $\ptil(\theta
\vert \phi) = \pbase(\theta) \exp(\phi(\theta)) \ge 0$. As above, since $\phi
\in \pertset[\infty]$, there exist $\palt$, $\beta > 0$, and $\t \in [0,1]$ such
that
%
\begin{align*}
%
\phi(\theta) ={}&
    \t \log \palt(\theta) - (1-\t)\log \pbase(\theta) + \log \beta \Rightarrow\\
%
\int \ptil(\theta \vert \phi)\mu(d\theta) ={}&
    \beta \int \palt(\theta)^\t \pbase(\theta)^{1 - \t}\mu(d\theta)
\\={}&
\beta \int \pbase(\theta)
    \left( \frac{\palt(\theta)}{\pbase(\theta)}\right)^\t \mu(d\theta)
\\\le{}&
\beta \int \pbase(\theta)
    \frac{\palt(\theta)}{\pbase(\theta)}
    \ind{\palt(\theta) \ge \pbase(\theta)} \mu(d\theta) +
\\&
  \beta \int \pbase(\theta)
    \ind{\palt(\theta) < \pbase(\theta)} \mu(d\theta)
\\\le{}& 2\beta.
%
\end{align*}
%
This concludes the proof of PERT ARE PRIORS.

For PERTS VS BALL P, we show that, when $p \in [1, \infty)$ and
$\norm{\phi}_{\mu,p} < p$, then $\p(\theta \vert \phi)$ is normalizable. Um but
that is not necessary.  All we need to do is show that if $\phi \in \pertset$
then $\norm{\phi}_{\mu,p} < \infty$.

First, consider the case of general $\phi$ with $p \in [1, \infty)$. By Jensen's
inequality applied pointwise to the convex function $x \mapsto x^p$,
%
\begin{align*}
%
\abs{\pbase(\theta)^{1/p} + \frac{1}{p}\phi(\theta) }^{p} \le{}&
    \left(\pbase(\theta)^{1/p} + \frac{1}{p}\abs{\phi(\theta)} \right)^{p}
\\={}&
    2^p \left(\frac{1}{2}\pbase(\theta)^{1/p} +
              \frac{1}{2} \frac{1}{p}\abs{\phi(\theta)} \right)^{p}
\\\le{}&
    2^{p-1} \left(\pbase(\theta) + \frac{1}{p^p}\abs{\phi(\theta)}^p \right).
%
\end{align*}
%
Consequently,
%
\begin{align*}
%
\int \abs{\pbase(\theta)^{1/p} + \frac{1}{p}\phi(\theta) }^{p}
    \lambda(d\theta) \le{}&
2^{p-1} \int \left(\pbase(\theta) + \frac{1}{p}\abs{\phi(\theta)}^p \right)
    \lambda(d\theta)
\\={}&
    2^{p-1} \left(1 + \frac{1}{p^p}\norm{\phi}_p^p\right).
%
\end{align*}
%
So, as in \citep[Result 2]{gustafson:1996:local}, $\norm{\phi}_p < \infty$
implies that the prior can be normalized.

Continuing the case of general $\phi$ with $1 \le p < \infty$, by
convexity,\footnote{Apply the definition of convexity to the points $0$, $x$,
and $x + y$, and again to the points $0$, $y$, and $x+y$, then add the results.}
for any $x \ge y \ge 0$,
%
\begin{align*}
%
(x + y)^p \ge{} x^p + y^p \mathand
(x - y)^p \le{} x^p - y^p.
%
\end{align*}
%
Also note that, since $\pbase(\theta) \ge 0$,
%
\begin{align*}
%
\pbase(\theta)^{1/p} + \frac{1}{p}\phi(\theta) \le 0
\quad\Rightarrow\quad
\phi(\theta) \le 0 \mathand
\frac{1}{p} \abs{\phi(\theta)} - \pbase(\theta)^{1/p} \ge 0.
%
\end{align*}
%
We can thus write
%
\begin{align*}
%
\MoveEqLeft
\int \mathrm{sign}\left(\pbase(\theta)^{1/p} + \frac{1}{p}\phi(\theta)\right)
    \abs{\pbase(\theta)^{1/p} + \frac{1}{p}\abs{\phi(\theta)}}^{p} d\theta
\\={}&
    \int \left(\pbase(\theta)^{1/p} + \frac{1}{p}\phi(\theta)\right)^{p}
        \ind{\pbase(\theta)^{1/p} + \frac{1}{p}\phi(\theta) \ge 0}
        d\theta - \\&
    \int \left(\frac{1}{p}\abs{\phi(\theta)} - \pbase(\theta)^{1/p}\right)^{p}
        \ind{\pbase(\theta)^{1/p} + \frac{1}{p}\phi(\theta) < 0}
        d\theta
\\\ge{}&
    \int \left(\pbase(\theta) - \frac{1}{p^p}\abs{\phi(\theta)}^{p}\right)
        \ind{\pbase(\theta)^{1/p} + \frac{1}{p}\phi(\theta) \ge 0}
        d\theta - \\&
    \int \left(\frac{1}{p^p}\abs{\phi(\theta)}^p - \pbase(\theta)\right)
        \ind{\pbase(\theta)^{1/p} + \frac{1}{p}\phi(\theta) < 0}
        d\theta
\\={}&
    \int \pbase(\theta) d\theta - \frac{1}{p^p}\int \abs{\phi(\theta)}^p d\theta
\\={}&
    1 - \frac{1}{p^p} \norm{\phi}_p^p.
%
\end{align*}
%
The final line is non-negative when $\norm{\phi}_p \le p$.





Finally, consider $p = \infty$.  Since $\int \pbase(\theta) \lambda(d\theta) = 1$,
%
\begin{align*}
%
\exp(-\norminf{\phi}) \le{}
\abs{\int_0^1 \exp\left(\log \pbase(\theta) + \phi(\theta)\right) \lambda(d\theta)}
\le{}
\exp(\norminf{\phi}).
%
\end{align*}
%
so that $0 < \int \tilde{\p}(\theta \vert \phi) \lambda(d\theta) < \infty$
whenever $\norminf{\phi} < \infty$.  Furthermore,
%
\begin{align*}
%
\exp\left(\log \pbase(\theta) + \phi(\theta)\right) \ge 0.
%
\end{align*}
%
\end{proof}
%%%%%%%%%%%%%%%%%%%%%%%%%%%%%%%%%%%%%%%%%%%%%%%%%%%%%%%%%%%%%%%%%%%%%%%%%%%%%







%%%%%%%%%%%%%%%%%%%%%%%%%%%%%%%%%%%%%%%%%%%%%%%%%%%%%%%%%%%%%%%%%%%%%%%%%%%%%
%%%%%%%%%%%%%%%%%%%%%%%%%%%%%%%%%%%%%%%%%%%%%%%%%%%%%%%%%%%%%%%%%%%%%%%%%%%%%
\begin{lem}\lemlabel{lp_integral_bound}
%
Under \assuref{lp_regular}, if $\norm{\gamma}_{\pbase, p} < \infty$
and $\inf_\theta \gamma(\theta) > -\infty$, then there
exists an $M(\theta)$ with $\int M(\theta) \mu(d\theta) < \infty$
such that, for all $0 \le \t \le (2 \abs{\inf_\theta \gamma(\theta)})^{-1}$,
%
\begin{align}
%
\abs{\psi(\theta)}\abs{\log\left(1 + \t \gamma(\theta)\right)}\q(\theta)
    <{}& M(\theta) \mathand \eqlabel{lp_integral_bound}\\
%
\abs{\psi(\theta)}\abs{\frac{\t \gamma(\theta)}{1 + \t \gamma(\theta)}}\q(\theta)
    <{}& M(\theta). \eqlabel{lp_integral_deriv_bound}
%
\end{align}
%
\begin{proof}

Let $\t_{max} := (2 \abs{\inf_\theta \gamma(\theta)})^{-1}$.  Then
%
\begin{align*}
%
1 + \t \gamma(\theta) \ge{}& 1 - \t \abs{\inf_\theta \gamma(\theta)}
\\={}
    1 - \frac{\t}{2 \t_{max}}.
%
\end{align*}
%
So $\t \le \t_{max} \Rightarrow 1 + \t \gamma(\theta) \ge 1/2$ for all $\theta$.
Expanding the integral,
%
\begin{align*}
%
\MoveEqLeft
\int \abs{\psi(\theta)}
    \abs{\log\left(1 + \t \gamma(\theta)\right)}\q(\theta)
    \mu(d\theta) ={}
\\&
\int \abs{\psi(\theta)}
    \abs{\log\left(1 + \t \gamma(\theta)\right)}
    \ind{\gamma(\theta) \le 0}
    \q(\theta)\mu(d\theta)  +
\\&
\int \abs{\psi(\theta)}
    \abs{\log\left(1 + \t \gamma(\theta)\right)}
    \ind{\gamma(\theta) > 0}
    \q(\theta)\mu(d\theta).
%
\end{align*}
%
For all $0 \le \t \le \t_{max}$, $0 \le \abs{\log\left(1 + \t
\gamma(\theta)\right)} \ind{\gamma(\theta) \le 0} \le \log(1/2)$,
so the first term in the preceding display is dominated by the integrable
function $\log(1/2) \abs{\psi(\theta)}\q(\theta)$.

For the second term, by the fact that $x \ge 0 \Rightarrow \log (1 + x) \le x$
and Holder's inequality with respect to $\pbase(\theta)$,
%
\begin{align*}
%
\MoveEqLeft
\int \abs{\psi(\theta)}
    \abs{\log\left(1 + \t \gamma(\theta)\right)}
    \ind{\gamma(\theta) > 0}
    \q(\theta)\mu(d\theta)
\\\le{}&
\t
\int \abs{\psi(\theta)}
    \abs{\gamma(\theta)}
    \ind{\gamma(\theta) > 0}
    \q(\theta)\mu(d\theta)
\\={}&
\t
\int \abs{\psi(\theta)}
    \abs{\gamma(\theta)}
    \ind{\gamma(\theta) > 0}
    \frac{\q(\theta)}{\pbase(\theta)} \pbase(\theta) \mu(d\theta)
\\\le{}&
\t
\int \abs{\psi(\theta)}
    \abs{\gamma(\theta)}
    \frac{\q(\theta)}{\pbase(\theta)} \pbase(\theta) \mu(d\theta)
\\\le{}&
\t_{max}
\left(
\int
    \abs{\frac{\psi(\theta)\q(\theta)}{\pbase(\theta)}}^q
    \pbase(\theta) \mu(d\theta)
\right)^{1/q}
\left(
\int\abs{\gamma(\theta)}^p \pbase(\theta) \mu(d\theta)
\right)^{1/p}
\\\le{}&
\t_{max}
\left(
\int
    \abs{\frac{\psi(\theta)\q(\theta)}{\pbase(\theta)}}^q
    \pbase(\theta) \mu(d\theta)
\right)^{1/q}
\norm{\gamma}_{\pbase,p}.
%
\end{align*}
%
By assumption, the final equation in the preceding display is finite, and the
second line does not depend on $\t$, so we have proven the existence of an
integrable bounding function.  \Eqref{lp_integral_bound} follows.

For \eqref{lp_integral_deriv_bound} observe that $\t \le \t_{max}$ implies
%
\begin{align*}
%
\abs{\psi(\theta)}
    \abs{\frac{\t \gamma(\theta)}{1 + \t \gamma(\theta)}} \q(\theta)
\le{}&
2 \t \abs{\psi(\theta)} \abs{\gamma(\theta)} \q(\theta).
%
\end{align*}
%
Thus \eqref{lp_integral_deriv_bound} was proven along the way to
proving \eqref{lp_integral_bound}.
%
\end{proof}
%
\end{lem}
%%%%%%%%%%%%%%%%%%%%%%%%%%%%%%%%%%%%%%%%%%%%%%%%%%%%%%%%%%%%%%%%%%%%%%%%%%%%%







%%%%%%%%%%%%%%%%%%%%%%%%%%%%%%%%%%%%%%%%%%%%%%%%%%%%%%%%%%%%%%%%%%%%%%%%%%%%%
%%%%%%%%%%%%%%%%%%%%%%%%%%%%%%%%%%%%%%%%%%%%%%%%%%%%%%%%%%%%%%%%%%%%%%%%%%%%%
\begin{proof}\prooflabel{pert_invariance}\proofof{\lemref{pert_invariance}}
%
\todo{Put the proof of Linfty validity in here too.}

Let $\mu$ and $\mu'$ denote two mutually absolutely continuous candidate
dominating measures for $\pbase$, with respective densities (Radon-Nikodym
derivatives) $\pbase(\theta)$ and $\pbase'(\theta)$.  Let the respective
densities of the measure $\p$ be denoted $\palt(\theta)$ and $\palt'(\theta)$ as
well.  Let $R(\theta) = \fracat{d\mu}{d\mu'}{\theta}$ denote the Radon-Nikodym
derivative of $\mu$ with respect to $\mu'$, and note that $\pbase'(\theta) =
R(\theta) \pbase(\theta)$ and $\palt'(\theta) = R(\theta) \palt(\theta)$.

First, take $p \in [1, \infty)$.  By \defref{prior_nl_pert}, the perturbations
for $\mu$ and $\mu'$ are given respectively by
%
\begin{align*}
%
\phi(\theta \vert \beta, \palt) ={}&
    \beta p \palt(\theta)^{1/p} - \pbase(\theta)^{1/p}\\
\phi'(\theta \vert \beta, \palt') ={}&
    \beta p \palt'(\theta)^{1/p} - \pbase'(\theta)^{1/p},
%
\end{align*}
%
so
%
\begin{align*}
%
\norm{\phi(\theta \vert \beta, \palt)}_{\mu,p}^p ={}&
\int \abs{\beta p \palt(\theta)^{1/p} - \pbase(\theta)^{1/p}}^p
    \mu(d\theta)
%
\\={}&
\int \abs{\beta p \left(R(\theta) \palt(\theta)\right)^{1/p} -
          \left(R(\theta) \pbase(\theta)\right)^{1/p}}^p
     \mu'(d\theta)
%
\\={}&
\norm{\phi'(\theta \vert \beta, \palt')}_{\mu',p}^p.
%
\end{align*}
%
Next, for $p = \infty$, we have that the perturbations for $\mu$ and $\mu'$ are
given respectively by
%
\begin{align*}
%
\phi(\theta \vert \beta, \palt) ={}&
  \log \palt(\theta) - \log \pbase(\theta) + \log \beta \\
\phi'(\theta \vert \beta, \palt') ={}&
    \log \palt'(\theta) - \log \pbase'(\theta) + \log \beta
\\={}&
\log \palt(\theta) - \log R(\theta)
    - \log \pbase(\theta) + \log R(\theta)+ \log \beta
\\={}&
\phi(\theta \vert \beta, \palt).
%
\end{align*}
%
It follows that $\norminf{\phi(\cdot \vert \beta, \palt)} = \norminf{\phi'(\cdot
\vert \beta, \palt')}$.

Next, let $\tau := \tau(\theta)$ be an invertible transformation with Jacobian
$J(\theta) := \mathrm{det}\left(\fracat{d\tau}{d\theta^T}{\theta}\right)$. For
the dominating measure $\mu$, let $\pbase(\theta)$ and $\palt(\theta)$ denote
the densities of $\theta$ and $\pbase'(\tau)$ and $\palt'(\tau)$ denote the
densities of $\tau$.  The desired result follows by the exact same formal
argument as for the change of measure, except with $J(\theta) \mu(d\theta)$
and $\mu(d\tau)$ taking the place of $R(\theta) \mu(d\theta)$ and
$\mu'(d\theta)$, respectively.
%
\end{proof}
%%%%%%%%%%%%%%%%%%%%%%%%%%%%%%%%%%%%%%%%%%%%%%%%%%%%%%%%%%%%%%%%%%%%%%%%%%%%%
