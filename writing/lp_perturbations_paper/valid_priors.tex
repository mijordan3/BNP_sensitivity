\Corref{etafun_deriv_form} is explicitly about a particular $\phi$ derived from
some fixed alternative $\palt$.  As stated, each perturbation must be checked
individually, as in \exref{gem_fun_pert}.  However, the form of the influence
function motivates {\em searching} for $\phi$ that have large influence. In
particular, if we can define a notion of ``size'' of a perturbation, we might
use \eqref{vb_eta_infl_sens} to find the most influential prior perturbation of
a given ``size''.


To formally justify using \corref{etafun_deriv_form} in this way, we must at
least establish that \corref{etafun_deriv_form} applies to ``small'' $\phi$. We
now follow \citet{gustafson:1996:local} and define ``size'' in terms of the
$\lp{\mu, p}$ spaces of measurable functions, which we now define.  It turns
out that each class of perturbations defined in \defref{prior_nl_pert}
pairs naturally with the corresponding $\lp{\mu,p}$ space.

%%%%%%%%%%%%%%%%%%%%%%%%%%%%%%%%%%%%%%%%%%%%%%%%%%%%%%%%%%%%%%%%%%%%%%%%%%%
%%%%%%%%%%%%%%%%%%%%%%%%%%%%%%%%%%%%%%%%%%%%%%%%%%%%%%%%%%%%%%%%%%%%%%%%%%%

\begin{defn}\deflabel{lp_spaces}
\citep[Sections 5.1-5.2]{dudley:2018:real}
%
For a measure $\mu$ and $p \in [1, \infty]$, let $\lp{\mu,p}$ define the
space of equivalence classes of real-valued $\mu$-measurable functions,
where two functions are equivalent if they disagree only on a set of
$\mu$-measure zero.

Let $\esssup_{\theta\sim\mu}$ denote the essential supremum over $\theta$ with
respect to the measure $\mu$. The norm of $\phi \in \lp{\mu,p}$ is given by
%
\begin{align*}
%
\norm{\phi}_{\mu,p} :={}&
\begin{cases}
    \left(\int \abs{\phi(\theta)}^p \mu(d\theta)\right)^{1/p}
    & \textrm{when }p \in [1, \infty)\\
    \esssup_{\theta\sim\mu} \abs{\phi(\theta)}
    & \textrm{when }p = \infty
\end{cases}.
%
\end{align*}
%
By defintiion, $\phi \in \lp{\lambda,p} \Leftrightarrow{}
\norm{\phi}_{\lambda,p} < \infty$.
%
Let $\ball_{\mu,p}(\epsilon) := \{\phi: \phi \in \lp{\mu, p},
\norm{\phi}_p < \epsilon \}$ denote the $\epsilon$-ball in $\lp{\mu, p}$.

When $\mu$ is the Lebesgue measure, we may
simply write $\lp{\lambda,p} = \lp{p}$ and $\norm{\cdot}_{p} =
\norm{\cdot}_{\lambda,p}$.
%
\end{defn}

By \citep[Theorem 5.2.1]{dudley:2018:real}, $\lp{\mu,p}$ is a Banach
space (i.e., a complete, normed vector space).

The following lemma is due to \citep{gustafson:1996:local}, and provides
a key part of the motivation for the use of $\norm{\cdot}_{\mu,p}$ to measure
the size of prior perturbations.

%%%%%%%%%%%%%%%%%%%%%%%%%%%%%%%%%%%%%%%%%%%%%%%%%%%%%%%%%%%%%%%%%%%%%%%%%%%
%%%%%%%%%%%%%%%%%%%%%%%%%%%%%%%%%%%%%%%%%%%%%%%%%%%%%%%%%%%%%%%%%%%%%%%%%%%
\begin{lem}\lemlabel{pert_invariance_old}
%
(\citet{gustafson:1996:local})
%
Fix the quantities given in \defref{prior_nl_pert}.  For a fixed probability
measure $\p \ll \mu$, the map $\p \mapsto \norm{\phi(\cdot \vert \p)}_p$ with
$\beta = 1$ is a norm, does not depend on $\mu$, and is invariant to invertible
transformations of $\theta$.
%
\seeproof{pert_invariance}
%
\end{lem}
%%%%%%%%%%%%%%%%%%%%%%%%%%%%%%%%%%%%%%%%%%%%%%%%%%%%%%%%%%%%%%%%%%%%%%%%%%%

\Corref{etafun_deriv_form} suggests an intriguing result if it is taken to hold
for all $\phi$ in some ball, $\ball_p(\delta) := \{\phi: \norm{\phi}_{\lambda,p}
\le \delta \}$. Specifically, let $q = (1 - p^{-1})^{-1}$ so that $q^{-1} +
p^{-1} = 1$ (or let $\q=\infty$ when $p=1$) and observe that, by H{\"o}lder's
inequality \citep[Theorem 5.1.2 and subsequent disscussion]{dudley:2018:real},
%
\begin{align*}
%
\sup_{\phi \in \ball_p(\delta)} \fracat{d g(\etaopt(\t \phi))}{d \t}{0} ={}&
    \sup_{\phi \in \ball_p(\delta)}
        \int \infl_p(\theta) \phi(\theta) \mu(d\theta)
% \\\le{}&
%     \sup_{\phi \in \ball_p(\delta)}
%         \left( \int \abs{\infl_p(\theta)}^{1/q} \mu(d\theta) \right)^q
%         \left( \int \abs{\phi(\theta)}^{1/p} \mu(d\theta)\right)^p
\\={}&
\begin{cases}
\delta \left( \int \abs{\infl_p(\theta)}^{1/q} \mu(d\theta) \right)^q
    & \textrm{ when }p \in [1, \infty)\\
\delta \int \abs{\infl_p(\theta)} \mu(d\theta)
    & \textrm{ when }p = \infty,
\end{cases}
%
\end{align*}
%
with equality at the ``worst-case'' perturbation
%
\begin{align*}
%
\phi^*(\theta) \propto
\begin{cases}
\mathrm{sign}(\infl_p(\theta)) \abs{\infl_p(\theta)}^{p/q}.
& \textrm{ when }p \in [1, \infty)\\
\mathrm{sign}(\infl_p(\theta))
& \textrm{ when }p = \infty.
\end{cases}
%
\end{align*}
%
(For the most negative ``worst-case'', simply apply the preceding result to
$-g$.) The constant of proportionality in the preceding display must be adjusted
so that $\norm{\phi^*(\theta)}_{\lambda,p} = \delta$.  These ``worst-case''
perturbations are the variational Bayes analogues of the corresponding
``worst-case'' for exact Bayesian posteriors in \citet{gustafson:1996:local}.

Before proceeding to use the influence function in this way, however, we must
ask: do all $\phi \in \ball_{\mu,p}(\delta)$ correspond to valid priors?  Are
there valid priors that cannot be produced by $\phi \in \ball_{\mu,p}(\delta)$
for any $\delta$?  More generally, what is the relationship between
$\ball_{\mu,p}(\delta)$, the class of perturbations given in $\pertset$, and the
set of all valid priors?  The following theorem answers these questions.

%%%%%%%%%%%%%%%%%%%%%%%%%%%%%%%%%%%%%%%%%%%%%%%%%%%%%%%%%%%%%%%%%%%%%%%%%%%

\begin{thm}\thmlabel{pert_well_defined}
%
Fix the quantities given in \defref{prior_nl_pert}.  Say that an unnormalized
prior $\ptil$ is ``valid'' if $\ptil \ll \pbase$, $\ptil$ is non-negative
$\mu$-almost everywhere, and $\ptil$ is normalizable in the sense that $0 < \int
\ptil(\theta) \mu(d\theta) < \infty$.

The following table summarizes properties of densitites derived from $\phi \in
\ball_{\mu,p}$ and from $\phi \in \pertset$.  The columns are shorthand for the
following properties:
%
\begin{align*}
%
\int \ptil < \infty \Rightarrow{}&
    \int \ptil(\theta \vert \phi) \mu(d\theta) < \infty   &\quad
%
\int \ptil > \infty \Rightarrow{}&
    \int \ptil(\theta \vert \phi) \mu(d\theta) > 0\\
%
\ptil \ge 0 \Rightarrow{}&
    \essinf_{\theta \sim \mu} \ptil(\theta \vert \phi) \ge 0  &\quad
% \norm{\phi} < \infty \Rightarrow{}&
%     \exists M < \infty \textrm{ independent of }\phi\textrm{ such that }
%     \sup_\phi \norm{\phi}_{\mu,p} < M.
\norm{\phi} < \infty \Rightarrow{}& \norm{\phi}_{\mu,p} < \infty.
%
\end{align*}

A ``Y'' indicates that the columns property is satisfied for all $\phi$ in the
corresponding set.  For example, the ``Y'' in the first row and first column
means that, for all $\phi \in \pertset$ with $p \in [1, \infty)$, $\int
\ptil(\theta \vert \phi) \mu(d\theta) < \infty$.  A ``N'' means the converse,
i.e., that there exists, in general, some $\phi$ in the corresponding set that
does not have the column's property.

\begin{table}[h!]
%\vspace{1em}
\begin{centering}
%\begin{tabular}{|c|c|c|c|c|c|}
\begin{tabular}{cccccc}
    %\hline
    && $\int \ptil < \infty$
    & $\int \ptil > 0$
    & $\ptil \ge 0$
    & $\norm{\phi} < \infty$\\[0.5em] \hline
$p \in [1, \infty)$   &     $\phi \in \pertset$ &
%    Y & Y & Y & Y, if $\beta < \infty$ \\ \hline
    Y & Y & Y & Y \\ \hline
$p \in [1, \infty)$   &     $\phi \in \ball_{\mu,p}(\delta)$ &
    Y & Y, if $\delta < p$ & N & Y (by defn) \\ \hline
$p = \infty$   &     $\phi \in \pertset[\infty]$ &
    Y & Y & Y & N \\ \hline
$p = \infty$   &     $\phi \in \ball_{\mu,\infty}(\delta)$ &
    Y & Y & Y & Y (by defn) \\ \hline
\end{tabular}
\caption{The relationships between $\pertset$, $\ball_{\mu,p}$, and valid priors.
The properties hold for any ball radius $\delta > 0$.}
\tablabel{pert_well_defined}
\end{centering}
%\vspace{1em}
\end{table}

\Tabref{pert_well_defined} has the following implications:

\begin{enumerate}
%
\item \itemlabel{pertset_is_valid}
For all $p \in [1, \infty]$, the set of densitites that can be formed from
$\phi \in \pertset$ is identical to the set of all valid priors.
%
\item \itemlabel{pball_is_valid}
For $p \in [1, \infty)$, all valid priors can be formed from
some $\phi \in \lp{\mu,p}$.
%
\item \itemlabel{pball_is_invalid}
For $p \in [1, \infty)$, one can form invalid (negative) densities from
some $\phi \in \ball_{\mu,p}(\delta)$ even for arbitrarily small $\delta$.
(See \exref{lp_negative}.)
%
\item \itemlabel{pinfball_is_valid}
For $p = \infty$, all densities formed from
$\phi \in \lp{\mu,\infty}$ are valid.
%
\item \itemlabel{pinfball_is_invalid}
For $p = \infty$, there exist valid priors not formed from any
$\phi \in \lp{\mu,\infty}$.  (See \exref{beta_inf_norm}.)
%
\end{enumerate}

\seeproof{pert_well_defined}
%
\end{thm}
%%%%%%%%%%%%%%%%%%%%%%%%%%%%%%%%%%%%%%%%%%%%%%%%%%%%%%%%%%%%%%%%%%%%%%%%%%%
%%%%%%%%%%%%%%%%%%%%%%%%%%%%%%%%%%%%%%%%%%%%%%%%%%%%%%%%%%%%%%%%%%%%%%%%%%%

%%%%%%%%%%%%%%%%%%%%%%%%%%%%%%%%%%%%%%%%%%%%%%%%%%%%%%%%%%%%%%%%%%%%%%%%%
% \/  \/  \/  \/  \/  \/  \/  \/  \/  \/  \/  \/  \/  \/  \/  \/  \/  \/
\begin{ex}\exlabel{lp_negative}
%
For $1 \le p < \infty$, it is possible for $\phi \in
\ball_{\mu,p}(\epsilon)$ for arbitrarily small $\epsilon$ and yet have
$\essinf^\mu_{\theta} \p(\theta \vert \phi) < 0$.
%
Since $\pbase \ll \mu \ll \lambda$, there exists a
sequence $\epsilon_n \rightarrow 0$ with $\epsilon_n > 0$ and a sequence of
corresponding sets such that $\pbase(S_n) = \epsilon_n$. (See
\lemref{continuity_partition} for a proof of this fact, which is a
straightforward consequence of \citet[Proposition 15.5]{nielsen:1997:measure}
and the continuity of the Lebesgue measure.)  Take
%
%\begin{align*}
%
$\phi_n(\theta) := - \frac{2}{p} \pbase(\theta)^{1/p} \ind{\theta \in S_n}$.
%
%\end{align*}
%
Then $\norm{\phi_n}_{\lambda, p} = \frac{2}{p} \epsilon_n^{1/p} \rightarrow 0$
and
%
\begin{align*}
%
\pbase(\theta)^{1/p} + p \phi(\theta) ={}
\pbase(\theta)^{1/p}
\left(\ind{\theta \notin S_n} - \ind{\theta \in S_n} \right)
%
\end{align*}
%
so $\essinf_\theta \p(\theta \vert \phi_n) < 0$ for all $n$.
%
\end{ex}
% /\    /\    /\    /\    /\    /\    /\    /\    /\    /\    /\    /\    /\
%%%%%%%%%%%%%%%%%%%%%%%%%%%%%%%%%%%%%%%%%%%%%%%%%%%%%%%%%%%%%%%%%%%%%%%%%%%%


Note that \citep{gustafson:1996:local} avoids the difficulty of
\exref{lp_negative} by restricting to positive $\phi(\theta)$, i.e. $\phi$ such
that $\essinf_\theta^\mu \phi(\theta) \ge 0$.  Of course, \exref{lp_negative}
implies that there exist negative $\phi$ in any neighborhood of $\phiz$,
prohibiting the use of standard functional analysis results requiring open
neighborhoods, such as the implicit function theorem in Banach spaces, (our key
tool proving \thmref{eta_phi_deriv} below). Furthermore, \exref{phi_negative,
phi_necessarily_negative} shows that restricting $\phi$ to be positive
sacrifices \thmref{pert_well_defined} \itemref{pertset_is_valid}.  Perhaps more
importantly, restricting to positive $\phi$ induces counterintutive notions of
the ``size'' of perturbations that ablate mass, detailed discussion of which we
provide in \appref{positive_pert}.


%%%%%%%%%%%%%%%%%%%%%%%%%%%%%%%%%%%%%%%%%%%%%%%%%%%%%%%%%%%%%%%%%%%%%%%%%
% \/  \/  \/  \/  \/  \/  \/  \/  \/  \/  \/  \/  \/  \/  \/  \/  \/  \/
\begin{ex}\exlabel{beta_inf_norm_old}
%
It is possible for $\phi \in \pertset[\infty]$ to have $\norminf{\phi} =
\infty$, and so $\phi \notin \lp{\lambda,\infty}$.  Take $\mu$ to be the
Lebesgue measure on $[0,1]$, let $\pbase(\theta) = \betadist{\theta \vert 1,
\alpha_0}$ and $\palt(\theta) = \betadist{\theta \vert 1, \alpha_1}$ for
$\alpha_0 \ne \alpha_1$.  Then we can choose $\beta$ such that $\phi(\theta) =
(\alpha_1 - \alpha_0) \log(1 - \theta)$
%
and
%
\begin{align*}
%
\norminf{\phi} =
    \abs{\alpha_1 - \alpha_0} \sup_{\theta \in [0,1]} \abs{\log(1 - \theta)} =
    \infty.
%
\end{align*}
%
Consequently, $\phi \notin \lp{\mu,\infty}$.
%
\end{ex}
% /\    /\    /\    /\    /\    /\    /\    /\    /\    /\    /\    /\    /\
%%%%%%%%%%%%%%%%%%%%%%%%%%%%%%%%%%%%%%%%%%%%%%%%%%%%%%%%%%%%%%%%%%%%%%%%%%%%

\Thmref{pert_well_defined}, together with \exref{lp_negative, beta_inf_norm},
illustrate very different phenomena for $p \in [1, \infty)$ and $p = \infty$.
For all $p \in [1, \infty]$, the priors are normalizable within sufficiently
small balls, but there exist invalid, negative priors in
$\ball_{\mu,p}(\delta)$, no matter how small $\delta$ is taken to be.   For
$p=\infty$, the opposite situation obtains: all priors arising from $\phi \in
\ball_{\mu,\infty}$ are valid, but there exist valid priors which are not
contained in $\ball_{\mu,\infty}(\delta)$, no matter how large $\delta$ is.

The fact that $\phi \in \ball_{\mu,p}(\delta)$ for $p \in [1, \infty)$ can be
negative no matter how small $\delta$ is (arguably) not necessarily a problem
for the formal definition of a full Bayesian posterior, but it is fatal for
variational approximations, as we now discuss.  This difference will lead below
to very different results for the differentiability of $\t \mapsto \etaopt(\t
\phi)$ for $p \in [1, \infty)$ and for $p = \infty$.
%
For the remainder of this section, we will discuss only the more difficult case
of $p \in [1, \infty)$.  For $p=\infty$, we will prove a stronger result, using
a more general set of tools, in \secref{differentiability} below.

Negative priors are anathema to VB approximations because the term
$\expect{\q(\theta \vert \eta)}{\log \p(\theta \vert \t \phi)}$ enters the KL
divergence $\KL{\theta, \t}$, and if $\p(\theta \vert \t \phi) \le 0$ on any set
with nonzero $\q(\theta \vert \eta)$-measure, then the KL divergence will be
infinite.  Equivalently, observe that, by \eqref{nl_vb_pert_p}, for a fixed
$\phi$ and $p \in [1, \infty)$
%
\begin{align*}
%
\KL{\eta, \t} = \KL{\eta, 0} +
\expect{\q(\theta \vert \eta)}
       {\log\left(1 + \t \frac{\phi(\theta)}{p \pbase(\theta)^{1/p}}\right)}.
%
\end{align*}
%
Again, if $1 + \t \frac{\phi(\theta)}{p \pbase(\theta)^{1/p}} \le 0$ on any set
with nonzero $\q(\theta \vert \eta)$-measure, then the KL divergence is
infinite.  If $\essinf_{\theta \sim \mu}\frac{\phi(\theta)}{p
\pbase(\theta)^{1/p}} = -\infty$, then $\KL{\eta, \t}$ is undefined for all $\t >
0$.  The following example compactly illustrates the problem, and will be useful
in our subsequent discussion in \secref{differentiability}.

%%%%%%%%%%%%%%%%%%%%%%%%%%%%%%%%%%%%%%%%%%%%%%%%%%%%%%%%%%%%%%%%%%%%%%%%%
%%%%%%%%%%%%%%%%%%%%%%%%%%%%%%%%%%%%%%%%%%%%%%%%%%%%%%%%%%%%%%%%%%%%%%%%%
\begin{ex}\exlabel{e_log_disocontinuous_v1}
%
Let $\q(\theta)$ be a density relative to a measure $\mu$. For a given
$\mu$-measurable function $\gamma$, define
%
\begin{align*}
%
f(\gamma) = \begin{cases}
\expect{\q(\theta)}{\log\left(1 + \gamma(\theta)\right)}
    & \textrm{when }\essinf_{\theta \sim \mu} \gamma(\theta) > -1 \\
-\infty & \textrm{otherwise}.
\end{cases}
%
\end{align*}
%
Then the map $\t \mapsto f(\t \gamma)$ is not differentiable at $\t=0$ if
$\essinf_{\theta \sim \mu} \gamma(\theta) = -\infty$ since we then have $f (\t
\gamma) = -\infty$ for all $\t > 0$, but $f(0) = 0$.  Therefore $\t \mapsto f(\t
\gamma)$ is discontinuous and so not differentiable.
%
\end{ex}
%%%%%%%%%%%%%%%%%%%%%%%%%%%%%%%%%%%%%%%%%%%%%%%%%%%%%%%%%%%%%%%%%%%%%%%%%

In constrast, if $\essinf_{\theta \sim \mu}\frac{\phi(\theta)}{p
\pbase(\theta)^{1/p}} > -\infty$, then there exists a $\t$ sufficiently small
that $\KL{\eta, \t}$ will be defined in a neighborhood of $\t = 0$. Indeed, as
the following theorem states, $\essinf_{\theta \sim
\mu}\frac{\phi(\theta)}{p \pbase(\theta)^{1/p}} > -\infty$ is a sufficient
condition for the derivative to exist under some additional conditions on the
variational distribution.

%%%%%%%%%%%%%%%%%%%%%%%%%%%%%%%%%%%%%%%%%%%%%%%%%%%%%%%%%%%%%%%%%%%%%%%%%%%%%
%%%%%%%%%%%%%%%%%%%%%%%%%%%%%%%%%%%%%%%%%%%%%%%%%%%%%%%%%%%%%%%%%%%%%%%%%%%%%
\begin{assu}\assulabel{lp_regular}
%
Fix the quantites in \defref{prior_nl_pert} with $p \in [1, \infty)$, and let
$\q(\theta)$ be a density defined relative to $\mu$.  For a given function
$\psi(\theta): \thetadom \mapsto \mathbb{R}$ with
$\expect{\q(\theta)}{\abs{\psi(\theta)}} < \infty$, assume that
%
\begin{align*}
%
\int \abs{\frac{\psi(\theta) \q(\theta)}{\pbase(\theta)}}^q
\pbase(\theta) \mu(d\theta) <& \infty
    & \textrm{for }q^{-1} + p^{-1} = 1\textrm{ if }p > 1 \\
\esssup_{\theta \sim \pbase} \abs{\frac{\psi(\theta) \q(\theta)}{\pbase(\theta)}}
 <& \infty
    & \textrm{ if }p = 1.
%
\end{align*}
%
\end{assu}
%%%%%%%%%%%%%%%%%%%%%%%%%%%%%%%%%%%%%%%%%%%%%%%%%%%%%%%%%%%%%%%%%%%%%%%%%%%%%


%%%%%%%%%%%%%%%%%%%%%%%%%%%%%%%%%%%%%%%%%%%%%%%%%%%%%%%%%%%%%%%%%%%%%%%%%%%%%
%%%%%%%%%%%%%%%%%%%%%%%%%%%%%%%%%%%%%%%%%%%%%%%%%%%%%%%%%%%%%%%%%%%%%%%%%%%%%
\begin{assu}\assulabel{q_regular_lp}
%
For each $\eta$ in an open set $\ball_\eta$, let \assuref{lp_regular} hold with
$\psi(\theta)$ equal to each of the following functions:
%
\begin{align*}
%
 \norm{\lqgrad{\theta, \eta}}_2 \textrm{, }
 \norm{\lqgrad{\theta, \eta}}^2_2 \textrm{ and }
 \norm{\lqhess{\theta, \eta}}^2.
%
\end{align*}
%
\end{assu}
%%%%%%%%%%%%%%%%%%%%%%%%%%%%%%%%%%%%%%%%%%%%%%%%%%%%%%%%%%%%%%%%%%%%%%%%%%%%%


%%%%%%%%%%%%%%%%%%%%%%%%%%%%%%%%%%%%%%%%%%%%%%%%%%%%%%%%%%%%%%%%%%%%%%%%%%%%%
%%%%%%%%%%%%%%%%%%%%%%%%%%%%%%%%%%%%%%%%%%%%%%%%%%%%%%%%%%%%%%%%%%%%%%%%%%%%%

\begin{thm}\thmlabel{lp_derivative}
%
Fix the quantities \defref{prior_nl_pert}, let $p \in [1, \infty)$,
and let \assuref{kl_opt_ok, q_regular_lp} hold.
% Let $\q(\theta \vert \eta)$ define a class of densities relative to $\mu$
% parameterized by $\eta$ in an open ball $\ball_\eta$ satisfying
% \assuref{q_regular_lp}.  Let $\etaopt$ be a variational optimum satisfying
% \assuref{kl_opt_ok}.
Then, for each $\phi \in \lp{p}$ with $\norm{\phi}_p \le \infty$ and
$\essinf_{\theta\sim\mu} \phi(\theta) / (p \pbase(\theta)^{1/p})> -\infty$, the
function $\t \mapsto \etaopt(\t \phi)$ is differentiable at $\t = 0$ with the
derivative given in \corref{etafun_deriv_form}.
%
\begin{proof}
%
When $p \in [1, \infty)$, the KL divergence is given by
%
\begin{align*}
%
\KL{\eta, \t} ={}&
    \KL{\eta, 0} +
    \expect{\q(\theta \vert \eta)}
           {\log \left(1 + \t \frac{\phi(\theta)}{\pbase(\theta)^{1/p}} \right)}.
%
\end{align*}
%
Define $\gamma(\theta) := \frac{\phi(\theta)}{\pbase(\theta)^{1/p}}$, and note
that $\norm{\gamma}_{\pbase, p} = \norm{\phi}_{\mu,p}$, so
$\norm{\gamma}_{\pbase, p} \le \infty$ and $\inf_\theta \gamma(\theta) >
-\infty$.  By \assuref{q_regular_lp} and \lemref{lp_integral_bound}, we have
that that \assuref{dist_fun_nice} holds with $\psi(\theta, \t) = \log \left(1 +
\t \frac{\phi(\theta)}{\pbase(\theta)^{1/p}}\right)$, and \thmref{etat_deriv}
holds.

% The $p=\infty$ case is dealt with below by the stronger result
% \thmref{eta_phi_deriv}.
%
\end{proof}
%
\end{thm}
%%%%%%%%%%%%%%%%%%%%%%%%%%%%%%%%%%%%%%%%%%%%%%%%%%%%%%%%%%%%%%%%%%%%%%%%%%%%%


%%%%%%%%%%%%%%%%%%%%%%%%%%%%%%%%%%%%%%%%%%%%%%%%%%%%%%%%%%%%%%%%%%%%%%%%%%%%%
%%%%%%%%%%%%%%%%%%%%%%%%%%%%%%%%%%%%%%%%%%%%%%%%%%%%%%%%%%%%%%%%%%%%%%%%%%%%%
\begin{cor}\corlabel{pertset_bounded_below}
%
If $p \in [1, \infty)$ and  $\phi \in \pertset$, then $\norm{\phi}_p < \infty$
$\essinf_{\theta \sim \mu}\frac{\phi(\theta)}{\pbase(\theta)^{1/p}} > -\infty$.
Under the other regularity conditions given in \thmref{lp_derivative}, the
function $\t \mapsto \etaopt(\t \phi)$ is differentiable at $\t = 0$ with the
derivative given in \corref{etafun_deriv_form}.
%
\begin{proof}
%
The fact that $\norm{\phi}_p < \infty$ follows from \thmref{pert_well_defined}.
If $\phi \in \pertset$, there exists a valid prior $\palt$, a $\beta > 0$,
and a $\t \in [0,1]$ such that
%
\begin{align*}
%
\essinf_{\theta \sim \mu}  \frac{\phi(\theta)}{\pbase(\theta)^{1/p}}
 ={}&
 \essinf_{\theta \sim \mu} \left(
    \t p\beta \frac{\palt(\theta)^{1/p}}{\pbase(\theta)^{1/p}} - (1-\t) p
    \right)
    \ge{} - (1-\t) p > -\infty.
%
\end{align*}
%
\end{proof}
%
\end{cor}
%%%%%%%%%%%%%%%%%%%%%%%%%%%%%%%%%%%%%%%%%%%%%%%%%%%%%%%%%%%%%%%%%%%%%%%%%%%%%

\Corref{pertset_bounded_below}, tells us that we can at least form a linear
approximation to $\t \mapsto \etaopt(\t \phi)$ for $\phi \in \pertset$ formed
from particular priors.  In a sense, we only care about $\phi \in \pertset$ in
the sense that we only care about real priors.  In this sense, one might be
tempted to employ \thmref{lp_derivative} and somehow ignore $\phi \notin
\pertset$.  As we discuss in the next section, however, although the derivative
exists for any particular $\phi \in \pertset$, it provides an arbitrarily poor
linear approximation to the actual function $\etaopt(\t \phi)$ as $\phi$ ranges
over $\pertset$.  In contrast, for the restrictive $p=\infty$ case we can prove
a strong result for the uniform quality of the linear approxiation to the  map
$\phi \mapsto \etaopt(\phi)$ for $p = \infty$ that is unavailable for $p \in [1,
\infty)$, though, of course, at the cost of expressiveness of the set of prior
perturbations.
